\chapter{Teoria}
\label{cap:teoria}

\section{Variables i constants}

Quan escrivim a la pissarra una expressi\'{o} com $x+y=y+x$, a molta gent se
li acut de seguida la propietat commutativa de la suma. \'{E}s clar que s%
\'{\i}, i a m\'{e}s, d'aquesta manera no hem d'escriure a la pissarra que
l'ordre dels sumands no altera el resultat de la suma. Hem aconseguit
sintetitzar una idea amb una expressi\'{o} simb\`{o}lica fent \'{u}s de dues
variables $x$ i $y$. Si despr\'{e}s escrivim $x+2=2+x$, tothom s'adona que
aquesta igualtat \'{e}s un cas particular de l'anterior substituint $y$ per
la constant el nombre $2$. Podr\'{\i}em afegir molts m\'{e}s exemples, per%
\`{o} no cal, tots recordeu el plantejament de problemes per equacions i la
seva resoluci\'{o}. Tot aix\`{o} seria molt m\'{e}s dif\'{\i}cil de fer
sense l'ajut de variables i constants. En resum, heu de tenir molt present
que en el llenguatge de les matem\`{a}tiques \'{e}s imprescindible l'\'{u}s
de variables i constants.

Com hem dit, la formalitzaci\'{o} d'enunciats amb l'ajut de variables i
constants es fa indispensable per a continuar endavant. En l'exemple seg\"{u}%
ent es veu com es fa molt m\'{e}s senzill i alhora m\'{e}s rigor\'{o}s
treballar amb enunciats matem\`{a}tics si introdu\"{\i}m variables i
constants. \'{E}s habitual prendre com a variables les \'{u}ltimes lletres
de l'abecedari i com a constants les primeres, per\`{o} aix\`{o} \'{e}s
irrellevant.

\begin{exemple}
Amb l'ajut de variables i constants expressa els seg\"{u}ents enunciats:

\begin{enumerate}
\item La mitja geom\`{e}trica de dos nombres positius \'{e}s menor o igual
que la mitja aritm\`{e}tica.

Si $x,y$ s\'{o}n dos nombres positius, aleshores es compleix $\sqrt{x\cdot y}%
\leq\dfrac{x+y}{2}$.

\item El quadrat d'un nombre senar \'{e}s senar.

Si $x$ \'{e}s un nombre senar, aleshores \thinspace$x^{2}$ \'{e}s senar.

\item Donat un nombre real positiu, busqueu els nombres reals el quadrat
dels quals \'{e}s m\'{e}s petit o igual que el nombre donat.

Sigui $a>0$, aleshores hem de resoldre la inequaci\'{o} $x^{2}\leq a$.

\item Trobar tres nombres parells consecutius que sumen $12$.

Busqueu un nombre parell $x$ tal que $x+x+2+x+4=12$.
\end{enumerate}
\end{exemple}

\section{Proposicions i predicats{}}

\'{E}s evident que l'enunciat `si els elefants volen, existeix el nombre $%
\pi $' no \'{e}s matem\`{a}tic perqu\`{e} hi apareixen termes que no estan
definits dins de les matem\`{a}tiques; en canvi, `$161$ \'{e}s m\'{u}ltiple
de $7$' s\'{\i} que \'{e}s un enunciat matem\`{a}tic. Aqu\'{\i} no volem
definir en rigor qu\`{e} \'{e}s un enunciat matem\`{a}tic, nom\'{e}s ens
interessa que tingueu una idea intu\"{\i}tiva clara sobre com distingir un
enunciat matem\`{a}tic d'aquell que no ho \'{e}s, i aix\`{o} creiem que ho
podem assumir sense cap problema.

Hi han enunciats de les matem\`{a}tiques que s\'{o}n proposicions i altres,
predicats. Una \textbf{proposici\'{o}} \'{e}s un enunciat que \'{e}s o
vertader o fals. Per exemple, els enunciats seg\"{u}ents s\'{o}n
proposicions vertaderes:

\begin{itemize}
\item Si un cercle t\'{e} un radi de $r$ unitats, aleshores la seva \`{a}rea
\'{e}s $\pi r^{2}$ unitats quadrades.

\item Tot nombre parell \'{e}s divisible per $2$.

\item $2\in\mathbb{Z}$, que es llegeix `$2$ pertany als nombres enters'.

\item $\sqrt{2}\notin\mathbb{Q}$, que es llegeix `$\sqrt{2}$ no pertany als
nombres irracionals'.

\item $\mathbb{N}\subseteq\mathbb{Z}$, que es llegeix `els nombres naturals
\'{e}s un subconjunt dels nombres enters'.

\item El conjunt $\left\{ a,b,c\right\} $ t\'{e} 3 elements.

\item Alguns triangles rectangles s\'{o}n is\`{o}sceles.
\end{itemize}

En canvi, les proposicions seg\"{u}ents s\'{o}n falses:

\begin{itemize}
\item $1=2$.

\item $\sqrt{2}\notin\mathbb{R}$.

\item Per a tot nombre real $x$, es t\'{e} que $x^{2}+1\leq0$.

\item $\left\{ 1,2,3\right\} \cap\mathbb{N}=\varnothing$, que es llegeix `la
intersecci\'{o} dels conjunts $\left\{ 1,2,3\right\} $ i $\mathbb{N}$ \'{e}s
el conjunt vuit'.

\item La suma dels angles de qualsevol quadril\`{a}ter val 180%
%TCIMACRO{\U{ba}}%
%BeginExpansion
${{}^o}$%
%EndExpansion
.

\item Si $x=2$, aleshores $x^{2}=9$.

\item 6 \'{e}s un nombre primer.
\end{itemize}

Volem observar que hi ha proposicions que tenen variables. Per exemple,
l'enunciat `Si $x$ \'{e}s un nombre enter, $2x+1$ \'{e}s senar' \'{e}s una
proposici\'{o} vertadera. L'enunciat `Per a tots els nombres reals $x,y$ es
compleix $(x-y)(x+y)=x^{2}-y^{2}$' \'{e}s una proposici\'{o} verdadera. Tamb%
\'{e} hi ha proposicions falses amb variables com, per exemple, `Per a tot
nombre real $x$ es t\'{e} $\sqrt{x^{2}}=x$'.

Tamb\'{e} hi han enunciats que encara no sabem si s\'{o}n certs o no, que s%
\'{o}n anomenats \textbf{conjectures} i que no s\'{o}n pr\`{o}piament
proposicions. Per exemple, la conjectura de Goldbach diu: Tot enter parell
superior a 2 \'{e}s una suma de dos nombres primers. Tamb\'{e} tenim la
conjectura de Fermat segons la qual per a tots els nombres naturals $a,b,c,n$
i $n>2$ es t\'{e} $a^{n}+b^{n}\neq c^{n}$, per\`{o} en aquest cas, a 1993,
el matem\`{a}tic Andrew Wiles va demostrar que la conjectura era certa.

Finalment, els enunciats seg\"{u}ents no s\'{o}n proposicions:

\begin{itemize}
\item $x$ \'{e}s un nombre parell.

\item Una recta paral\textperiodcentered lela a la recta d'equaci\'{o} $%
x+y=1 $.

\item Els nombres enters parells m\'{e}s gran que un nombre irracional donat.

\item La funci\'{o} $f$ \'{e}s la funci\'{o} inversa de la funci\'{o} $g$.

\item $2x+1$.
\end{itemize}

Els \textbf{predicats} s\'{o}n enunciats oberts perqu\`{e} contenen
variables i s\'{o}n vertaders o falsos depenent dels valors assignats a les
variables. Aqu\'{\i} denotem els predicats per lletres maj\'{u}scules, com
per exemple $P(x)$ o $Q(m,n)$, afegint les variables que estan presents
entre par\`{e}ntesis. Per exemple, l'enunciat `$n$ \'{e}s un nombre primer'
\'{e}s un predicat que representem per $P(n)$, i l'enunciat $\ $`$m$ \'{e}s m%
\'{e}s gran o igual que $n$', o abreujadament, `$m\geq n$' per $Q(m,n)$.
Llavors, $P(2)$ \'{e}s la proposici\'{o} `$2$ \'{e}s un nombre primer' que
\'{e}s vertadera, i $Q(3,4)$ \'{e}s la proposici\'{o} `$3\geq4$' que \'{e}s
falsa.

Dels cinc \'{u}ltims enunciats anteriors que no s\'{o}n proposicions els
quatre primers s\'{o}n predicats. El primer \'{e}s el predicat `ser un
nombre parell', que si el denotem per $P$, aleshores l'enunciat \'{e}s $P(x)$%
. El segon \'{e}s el predicat `ser una recta paral\textperiodcentered lela a
$x+y=1$', que si el simbolitzem per $Q$ i escrivim una recta del pla com $%
ax+by=c$, aleshores l'enunciat \'{e}s $Q(a,b,c)$; per exemple, $Q(2,2,-1)$
\'{e}s la proposici\'{o} `$2x+2y=-1$ \'{e}s una recta
paral\textperiodcentered lela a $x+y=1$' que \'{e}s verdadera. De fet el
tercer \'{e}s una fam\'{\i}lia de predicats que dep\`{e}n del par\`{a}metre $%
a$ que \'{e}s un nombre irracional. En concret, si el designem per $P_{a}$ i
considerem que la variable $x$ pren valors a $\mathbb{Z}$, aleshores
l'enunciat \'{e}s $P_{a}(x)$; per exemple, $P_{\sqrt{2}}(4)$ \'{e}s la
proposici\'{o} `$4$ \'{e}s un nombre enter parell m\'{e}s gran que el nombre
irracional $\sqrt{2}$' que \'{e}s falsa. Finalment, l'\'{u}ltim \'{e}s el
predicat `$f$ \'{e}s la funci\'{o} inversa de $g$', que si el denotem per $S$
i $f,g$ s\'{o}n variables que denoten funcions, aleshores l'enunciat \'{e}s $%
S(f,g)$; per exemple, $S(\sqrt[3]{x},x^{3})$ \'{e}s la proposici\'{o} `$\sqrt%
[3]{x}$ \'{e}s la funci\'{o} inversa de $x^{3}$' que \'{e}s vertadera. En
canvi, l'\'{u}ltim no \'{e}s un predicat perqu\`{e} al substituir la
variable $x$ per un n\'{u}mero s'obt\'{e} un altre n\'{u}mero, per\`{o} en
cap cas una proposici\'{o}; diem que \'{e}s una expressi\'{o} algebraica.
Aquestes expressions estan compostes per variables, constants num\`{e}riques
i els s\'{\i}mbols de les quatre operacions fonamentals de l'aritm\`{e}tica.

Para acabar aquesta secci\'{o} volem advertir que hem d'escriure%
\begin{equation*}
\text{`}3\geq4\text{' \'{e}s falsa}
\end{equation*}
i no%
\begin{equation*}
3\geq4\text{ \'{e}s falsa}
\end{equation*}
perqu\`{e} els termes `vertadera' i `falsa' no formen part del llenguatge de
les matem\`{a}tiques (mireu l'observaci\'{o} seg\"{u}ent). De fet haur\'{\i}%
em d'escriure \textquotedblleft$3\geq4$' es falsa perqu\`{e} el nombre enter
$3$ no \'{e}s m\'{e}s gran o igual que el nombre enter $4$'. No \'{e}s el
nostre objectiu ser el m\`{a}xim de rigorosos, per\`{o} \'{e}s important
aquesta distinci\'{o} perqu\`{e} si no la fem, surten paradoxes, o sigui,
enunciats que condueixen a fets contradictoris, que no es poden donar
alhora. Aqu\'{\i} usem el catal\`{a} ampliat amb alguns s\'{\i}mbols (per
exemple, abans hem vist: $\in,\notin$ i $\subseteq$) que serveixen
d'abreujadors per descriure el llenguatge de les matem\`{a}tiques; de fet aix%
\`{o} no \'{e}s nou, ja que per aprendre la gram\`{a}tica anglesa podem fer
\'{u}s del catal\`{a}.

\begin{observacio}
Considerem una proposici\'{o} qualsevol que simbolitzem per $p$, aleshores
l'enunciat\ `$p$ \'{e}s falsa' \'{e}s una proposici\'{o}? Suposem que $p$
\'{e}s una proposici\'{o}, llavors $p$ \'{e}s vertadera o falsa. Si $p$ \'{e}%
s vertadera, llavors el que diu $p$ \'{e}s vertader i aix\`{o} contradiu el
que diu l'enunciat, que $p$ \'{e}s falsa. Si $p$ \'{e}s falsa, llavors el
que diu $p$ \'{e}s fals, per\`{o} com el que diu l'enunciat \'{e}s que $p$
\'{e}s falsa, llavors el que diu $p$ \'{e}s vertader que \'{e}s
contradictori amb sup\`{o}sit que $p$ \'{e}s falsa. Per tant, l'enunciat `$p$
\'{e}s falsa' no \'{e}s una proposici\'{o}.

Aquesta paradoxa tamb\'{e} es pot presentar d'una manera m\'{e}s quotidiana
com a paradoxa del mentider: Un home afirma que est\`{a} mentint. El que diu
\'{e}s vertader o fals? Suposem que l'home es diu Joan. D'aquesta manera
l'enunciat \'{e}s "Joan diu: Ell est\`{a} mentint." Destaquem l'enunciat
"Joan est\`{a} mentint". Aleshores, quan en Joan diu mentida, vol dir que
l'enunciat "Joan est\`{a} mentint" \'{e}s falsa, i quan diu veritat, "Joan
est\`{a} mentint" \'{e}s vertader. Tornem a trobar el cas d'abans on $p$
\'{e}s "Joan est\`{a} mentint".

Aquesta paradoxa es dona tamb\'{e} en el context dels conjunts, coneguda com
a paradoxa de Russell, i diu: El conjunt de les coses que no s\'{o}n
elements de si mateixes, \'{e}s un conjunt? Denotem simb\`{o}licament aquest
conjunt per $A$, aleshores es t\'{e}%
\begin{equation*}
A=\left\{ x:x\notin x\right\} .
\end{equation*}
Segons la l\`{o}gica cl\`{a}ssica, $A\in A$ o b\'{e} $A\notin A$. Com $A$
\'{e}s un conjunt, \'{e}s una cosa, i per tant, (1) si $A\in A$, aleshores $%
A\notin A$ i aix\`{o} \'{e}s una contradicci\'{o}; (2) si $A\notin A$,
aleshores $A\in A$ que tamb\'{e} \'{e}s una contradicci\'{o}. En conseq\"{u}%
\`{e}ncia, $A$ no \'{e}s conjunt.

La base d'aquestes paradoxes est\`{a} en el fet que no \'{e}s correcte
l'autorefer\`{e}ncia des d'un mateix llenguatge. El mentider no pot dir que
menteix si no vol contradir-se. Per evitar aquests problemes distingim entre
llenguatge-objecte i metallenguatge. Es diu metallenguatge al llenguatge que
usem per estudiar a un altre llenguatge, anomenat en aquest cas
llenguatge-objecte.
\end{observacio}

\section{Connectives l\`{o}giques}

En teoria de conjunts combinem o modifiquem els conjunts amb operacions com
la uni\'{o} o la intersecci\'{o}. De la mateixa manera, en aritm\`{e}tica
modifiquem nombres amb operacions com la suma o la multiplicaci\'{o}. Aix%
\`{o} tamb\'{e} passa en l\`{o}gica. Tenim algunes operacions per combinar o
modificar proposicions o predicats; aquestes operacions es tradueixen en el
llenguatge de les matem\`{a}tiques per les paraules `i', `o', `no', `si ...
llavors ...' i `... si i nom\'{e}s si ... '. \'{E}s habitual simbolitzar
aquestes operacions amb els s\'{\i}mbols caracter\'{\i}stics del llenguatge
de la l\`{o}gica. Aqu\'{\i} ho farem per una q\"{u}esti\'{o} d'economia sint%
\`{a}ctica. Cal tenir present que a matem\`{a}tiques, aquestes paraules
tenen significats precisos, que donarem a l'apartat seg\"{u}ent. En alguns
casos, el significat matem\`{a}tic d'aquestes paraules difereix lleugerament
o s\'{o}n m\'{e}s precisos que l'\'{u}s com\'{u} que fem en la vida
quotidiana.

En general, la regla sint\`{a}ctica \'{e}s que a partir de dues proposicions
$p$ i $q$ podem formar altres proposicions mitjan\c{c}ant les operacions l%
\`{o}giques, tamb\'{e} anomenades \textbf{connectives l\`{o}giques}. La
regla sem\`{a}ntica \'{e}s que la veritat o falsedat d'aquestes noves
proposicions dependr\`{a} de la veritat o falsedat de $p$ i $q$, i ser\`{a}
definida per les anomenades \textbf{taules de veritat}. A una proposici\'{o}
se l'assigna el valor de veritat que simbolitzem per V quan la proposici\'{o}
\'{e}s vertadera, i F quan \'{e}s falsa.

\subsection{Conjunci\'{o}}

Si $p$ i $q$ s\'{o}n dues proposicions, llavors la \textbf{conjunci\'{o}}
d'aquestes dues proposicions \'{e}s la proposici\'{o} `$p$ i $q$',
simbolitzada per $p\wedge q$, i que es defineix de la seg\"{u}ent manera:

\begin{itemize}
\item \'{E}s vertadera, quan $p$ i $q$ s\'{o}n ambdues vertaderes;

\item \'{E}s falsa, quan $p$ \'{e}s falsa o $q$ \'{e}s falsa o ambdues s\'{o}%
n falses.
\end{itemize}

Per exemple, la proposici\'{o} `$1<\sqrt{2}<2$' \'{e}s la conjunci\'{o} de
les proposicions `$\sqrt{2}>1$' i `$\sqrt{2}<2$', que s\'{o}n totes dues
vertaderes i, per tant, \'{e}s vertadera.

D'acord amb la definici\'{o} anterior, s'obt\'{e} la taula de veritat de la
conjunci\'{o} l\`{o}gica:%
\begin{equation*}
\begin{tabular}{ll|l}
$p$ & $q$ & $p\wedge q$ \\ \hline
\multicolumn{1}{c}{V} & \multicolumn{1}{c|}{V} & \multicolumn{1}{|c}{V} \\
\multicolumn{1}{c}{V} & \multicolumn{1}{c|}{F} & \multicolumn{1}{|c}{F} \\
\multicolumn{1}{c}{F} & \multicolumn{1}{c|}{V} & \multicolumn{1}{|c}{F} \\
\multicolumn{1}{c}{F} & \multicolumn{1}{c|}{F} & \multicolumn{1}{|c}{F}%
\end{tabular}
\text{.}
\end{equation*}
Observem que la conjunci\'{o} de dues proposicions nom\'{e}s \'{e}s
vertadera quan les dues proposicions s\'{o}n vertaderes.

\subsection{Disjunci\'{o}}

Si $p$ i $q$ s\'{o}n dues proposicions, llavors la \textbf{disjunci\'{o}}
d'aquestes dues proposicions \'{e}s la proposici\'{o} `$p$ o $q$',
simbolitzada per $p\vee q$, i que es defineix de la seg\"{u}ent manera:

\begin{itemize}
\item \'{E}s vertadera, quan almenys una de les dues proposicions \'{e}s
vertadera;

\item \'{E}s falsa, quan $p$ i $q$ s\'{o}n ambdues falses.
\end{itemize}

Per exemple, la proposici\'{o} `$\pi\in\left( -\infty,-1\right) \cup\left(
1,+\infty\right) $' \'{e}s la disjunci\'{o} de les proposicions `$\pi<-1$' o
`$\pi>1$', la primera \'{e}s falsa i la segona \'{e}s vertadera i, per tant,
\'{e}s vertadera. La taula de veritat d'aquesta connectiva \'{e}s%
\begin{equation*}
\begin{tabular}{ll|l}
$p$ & $q$ & $p\vee q$ \\ \hline
\multicolumn{1}{c}{V} & \multicolumn{1}{c|}{V} & \multicolumn{1}{|c}{V} \\
\multicolumn{1}{c}{V} & \multicolumn{1}{c|}{F} & \multicolumn{1}{|c}{V} \\
\multicolumn{1}{c}{F} & \multicolumn{1}{c|}{V} & \multicolumn{1}{|c}{V} \\
\multicolumn{1}{c}{F} & \multicolumn{1}{c|}{F} & \multicolumn{1}{|c}{F}%
\end{tabular}
\text{.}
\end{equation*}
Observem que la disjunci\'{o} de dues proposicions \'{e}s falsa quan les
dues proposicions s\'{o}n falses. Destaquem l'\'{u}s \textbf{inclusiu} que
fem d'aquesta connectiva respecte de l'exclusiu que fem servir habitualment
a la vida quotidiana i que fa que la proposici\'{o} sigui vertadera quan nom%
\'{e}s una de les proposicions \'{e}s vertadera.

\subsection{Negaci\'{o}}

Si $p$ \'{e}s una proposici\'{o}, llavors la \textbf{negaci\'{o}} d'aquesta
proposici\'{o} \'{e}s la proposici\'{o} `no $p$', simbolitzada per $\lnot p$%
, que es defineix de la seg\"{u}ent manera:

\begin{itemize}
\item \'{E}s vertadera, quan $p$ \'{e}s falsa;

\item \'{E}s falsa, quan $p$ \'{e}s vertadera.
\end{itemize}

Per exemple, la proposici\'{o} `$\sqrt{3}\notin\left( 0,1\right) $' \'{e}s
la negaci\'{o} de la proposici\'{o} `$\sqrt{3}\in\left( 0,1\right) $', que
\'{e}s falsa i, per tant, \'{e}s vertadera. La taula de veritat d'aquesta
connectiva \'{e}s%
\begin{equation*}
\begin{tabular}{l|l}
$p$ & $\lnot p$ \\ \hline
\multicolumn{1}{c|}{V} & \multicolumn{1}{|c}{F} \\
\multicolumn{1}{c|}{F} & \multicolumn{1}{|c}{V}%
\end{tabular}
\text{.}
\end{equation*}

\subsection{Condicional}

Si $p$ i $q$ s\'{o}n dues proposicions, llavors el \textbf{condicional}
d'aquestes dues proposicions \'{e}s la proposici\'{o} `si $p$, llavors $q$',
simbolitzada per $p\longrightarrow q$, i que es defineix de la seg\"{u}ent
manera:

\begin{itemize}
\item \'{E}s vertadera, quan $p$ i $q$ s\'{o}n ambdues vertaderes o $p$ \'{e}%
s falsa;

\item \'{E}s falsa, quan $p$ \'{e}s vertadera i $q$ \'{e}s falsa.
\end{itemize}

Una proposici\'{o} condicional `si $p$, llavors $q$' tamb\'{e} se'n diu
\textbf{implicaci\'{o}} i diem que "$p$ implica $q$", i que $p$ \'{e}s l'%
\textbf{antecedent} i $q$ \'{e}s el \textbf{conseq\"{u}ent} de la implicaci%
\'{o}. La interpretaci\'{o} que fem d'aquesta connectiva sorpr\`{e}n una
mica en el seu \'{u}s doncs, per exemple, en l'enunciat `Si hi ha dues
rectes en el pla que s\'{o}n paral\textperiodcentered leles, llavors 2 \'{e}%
s un nombre primer' no hi ha cap relaci\'{o} entre l'antecedent `hi ha dues
rectes en el pla que s\'{o}n paral\textperiodcentered leles' i el conseq\"{u}%
ent `$2$ \'{e}s un nombre primer', i, \'{e}s matem\`{a}ticament correcte,
independentment de si l'antecedent \'{e}s vertader o fals.

Segons la definici\'{o} que hem donat, la taula de veritat d'aquesta
connectiva \'{e}s%
\begin{equation*}
\begin{tabular}{ll|l}
$p$ & $q$ & $p\longrightarrow q$ \\ \hline
\multicolumn{1}{c}{V} & \multicolumn{1}{c|}{V} & \multicolumn{1}{|c}{V} \\
\multicolumn{1}{c}{V} & \multicolumn{1}{c|}{F} & \multicolumn{1}{|c}{F} \\
\multicolumn{1}{c}{F} & \multicolumn{1}{c|}{V} & \multicolumn{1}{|c}{V} \\
\multicolumn{1}{c}{F} & \multicolumn{1}{c|}{F} & \multicolumn{1}{|c}{V}%
\end{tabular}
\text{.}
\end{equation*}

Observem que nom\'{e}s \'{e}s falsa quan l'antecedent \'{e}s vertader i el
conseq\"{u}ent \'{e}s fals. El condicional de dues proposicions $p$ i $q$
podem trobar-lo en proposicions escrites al catal\`{a} de moltes maneres:
(1) $q$ si $p$; (2) si $p$, $q$; (3) $p$ nom\'{e}s si $q$; o (4) $q$ sempre
que $p$.

Per exemple, la proposici\'{o} `700 \'{e}s divisible per 4 perqu\`{e} el
nombre acaba amb dos zeros' \'{e}s el condicional: `$P(700)$ implica $Q(700)$%
, on $P$ \'{e}s el predicat `tenir zero a les decenes i tamb\'{e} a les
unitats' i $Q$ \'{e}s `ser divisible per 4'.

\subsection{Bicondicional}

Si $p$ i $q$ s\'{o}n dues proposicions, llavors el \textbf{bicondicional}
d'aquestes dues proposicions \'{e}s la proposici\'{o} `$p$ si i nom\'{e}s si
$q$', simbolitzada per $p\longleftrightarrow q$, i que es defineix de la seg%
\"{u}ent manera:

\begin{itemize}
\item \'{E}s vertadera, quan $p$ i $q$ s\'{o}n ambdues vertaderes o falses;

\item \'{E}s falsa, quan $p$ \'{e}s vertadera i $q$ \'{e}s falsa, o $p$ \'{e}%
s falsa i $q$ \'{e}s vertadera.
\end{itemize}

Una proposici\'{o} bicondicional `$p$ si i nom\'{e}s si $q$' tamb\'{e} se'n
diu \textbf{doble implicaci\'{o}} i diem que "$p$ implica $q$ i $q$ implica $%
p$". La taula de veritat d'aquesta connectiva \'{e}s%
\begin{equation*}
\begin{tabular}{ll|l}
$p$ & $q$ & $p\longleftrightarrow q$ \\ \hline
\multicolumn{1}{c}{V} & \multicolumn{1}{c|}{V} & \multicolumn{1}{|c}{V} \\
\multicolumn{1}{c}{V} & \multicolumn{1}{c|}{F} & \multicolumn{1}{|c}{F} \\
\multicolumn{1}{c}{F} & \multicolumn{1}{c|}{V} & \multicolumn{1}{|c}{F} \\
\multicolumn{1}{c}{F} & \multicolumn{1}{c|}{F} & \multicolumn{1}{|c}{V}%
\end{tabular}
\text{.}
\end{equation*}

Per exemple, la proposici\'{o} `98 \'{e}s m\'{u}ltiple de 6 si i nom\'{e}s
si 98 \'{e}s divisible per 2 i tamb\'{e} per 3' \'{e}s el bicondicional `$%
P(98)$ si i nomes si $Q(2)$ i $R(3)$, on $P$ \'{e}s el predicat `ser m\'{u}%
ltiple de 6', $Q$ \'{e}s `ser divisible per 2' i $R$ \'{e}s `ser divisible
per 3'.

\begin{exemple}
Escriu fent \'{u}s de les connectives les proposicions seg\"{u}ents: (1) $%
-4\leq-1$; (2) $\left\vert -3\right\vert =\left\vert 2\right\vert $; (3) $%
-1\in\left[ 0,1\right] $; (4) $2\notin\left[ 0,1\right] $; (5) $\sqrt {2}<2<2%
\sqrt{2}$ si $1<\sqrt{2}<2$; (6) $\sqrt[3]{1331}=11$ si i nom\'{e}s si $%
11^{3}=1331$.
\end{exemple}

\begin{solucio}
(1) La proposici\'{o} $-4\leq-1$ es pot expressar com la disjunci\'{o}: `$%
-4<-1$' o `$-4=-1$'. (2) La proposici\'{o} $\left\vert -3\right\vert
=\left\vert 2\right\vert $ \'{e}s la disjunci\'{o}: `$-3=2$' o'$-3=-2$'. (3)
La proposici\'{o} $-1\in\left[ 0,1\right] $ \'{e}s la conjunci\'{o}: `$%
-1\geq0$' i `$-1\leq1$'. (4) La proposici\'{o} $2\notin\left[ 0,1\right] $
\'{e}s la negaci\'{o}: `no $2\in\left[ 0,1\right] $'. (5) La proposici\'{o} `%
$\sqrt{2}<2<2\sqrt{2}$ si $1<\sqrt{2}<2$' \'{e}s el condicional `$1<\sqrt {2}%
<2$ implica $\sqrt{2}<2<2\sqrt{2}$'. (6) La proposici\'{o} `$\sqrt[3]{1331}%
=11$ si i nom\'{e}s si $11^{3}=1331$' \'{e}s el bicondicional ` $11^{3}=1331$
implica $\sqrt[3]{1331}=11$ i $\sqrt[3]{1331}=11$ implica $11^{3}=1331$'.
\end{solucio}

\section{Proposicions elementals i compostes}

Proposicions \textbf{elementals} s\'{o}n aquelles que no posseeixen cap
operador l\`{o}gic. Les proposicions \textbf{compostes} estan formades per
altres proposicions i operadors l\`{o}gics. Per exemple, a partir de dues
proposicions elementals $p$ i $q$, podem formar com hem vist $\lnot q$ i $%
p\vee q$. Per\`{o} tamb\'{e} $p\wedge\lnot q$, i finalment, $\left( p\vee
q\right) \longrightarrow\left( p\wedge\lnot q\right) $. Hem utilitzat els par%
\`{e}ntesis per evitar ambig\"{u}etats. De fet, tamb\'{e} hem utilitzat la
convenci\'{o} est\`{a}ndard segons el qual el s\'{\i}mbol de negaci\'{o} $%
\lnot$ t\'{e} prioritat sobre les altres quatre connectives, per\`{o} cap
d'aquestes t\'{e} prioritat sobre les altres. Per altra banda, \'{e}s clar
que $p\wedge\longrightarrow\lnot q$ no t\'{e} cap sentit perqu\`{e} no est%
\`{a} ben formada. La ra\'{o}, per exemple, est\`{a} en que $\wedge$
connecta dues proposicions i, en el nostre cas, $\longrightarrow\lnot q$ no
ho \'{e}s.

Volem ara formalitzar l'enunciat seg\"{u}ent: `Si no \'{e}s cert que 34043
sigui una suma de dos quadrats, llavors 34043 \'{e}s senar o \'{e}s
divisible per 3'. Identifiquem les seg\"{u}ents proposicions simples: $p=$
`34043 sigui una suma de dos quadrats', $q=$ `34043 \'{e}s senar', i $r=$
`34043 \'{e}s divisible per 3'. Llavors l'enunciat s'escriu simb\`{o}%
licament d'aquesta manera tenint en compte les connectives l\`{o}giques: $%
p\longrightarrow\left( q\vee r\right) $. Un altre exemple \'{e}s la proposici%
\'{o} `273 \'{e}s divisible per 91, si 273 \'{e}s m\'{u}ltiple de 7 i m\'{u}%
ltiple de 13' \'{e}s el condicional: `$P(273)\longrightarrow\left(
Q(273)\wedge R(273)\right) $', on $P$ \'{e}s el predicat `ser divisible per
91', $Q$ \'{e}s `ser m\'{u}tiple de 7' i $R$ \'{e}s `ser m\'{u}ltiple de 13'.

\bigskip

Donada una proposici\'{o} composta, volem determinar el seu valor de veritat
coneixent el valor de veritat de les proposicions simples que la conformen.
Suposem la interpretaci\'{o} segons la qual les proposicions simples $p$ i $%
q $ s\'{o}n vertaderes i $r$ i $s$ s\'{o}n falses. Llavors, quin \'{e}s el
significat de la proposici\'{o} composta $\lnot\left( p\vee q\right)
\longrightarrow\left( r\wedge\lnot s\right) $?

\begin{equation*}
\begin{tabular}{cccccccc|c}
$p$ & $q$ & $r$ & $s$ & $\lnot s$ & $p\vee q$ & $\alpha=\lnot\left( p\vee
q\right) $ & $\beta=\left( r\wedge\lnot s\right) $ & $\alpha\longrightarrow
\beta$ \\ \hline
V & V & F & F & V & V & F & F & V%
\end{tabular}
\end{equation*}

La taula dedu\"{\i}m que la propocisi\'{o} \'{e}s falsa.

\bigskip

Donat el valor de veritat d'una proposici\'{o} composta, ara volem
determinar el valor de veritat de les proposicions simples que la conformen.
Suposem que la proposici\'{o} composta $\left( p\wedge\lnot q\right)
\longrightarrow r$ \'{e}s falsa. Llavors, com la connectiva principal
d'aquesta proposici\'{o} \'{e}s el condicional. At\`{e}s que aquesta
implicaci\'{o} t\'{e} un valor de veritat fals \'{u}nicament quan
l'antecedent \'{e}s vertader i el conseq\"{u}ent \'{e}s fals, s'obt\'{e} que
$p\wedge\lnot q$ \'{e}s vertadera, i $r$ ha de ser falsa. Ara b\'{e} $%
p\wedge\lnot q$ \'{e}s falsa nom\'{e}s si $p$ i $q$ s\'{o}n vertadera i
falsa, respectivament. Per tant, $p$ \'{e}s vertadera i, $q$ i $r$, false

\section{Formes proposicionals}

Com hem fet fins ara les lletres $p,q,r,...$ s'usen com a variables que
representan a proposicions. El valor de veritat de $p$, per exemple, \'{e}s
desconegut mentre no s'especifica la seva interpretaci\'{o}, \'{e}s a dir,
no \'{e}s coneix el significat de la proposici\'{o} que representa.

Anomenem \textbf{formes proposicionals} a les expressions formals ben
formades constitu\"{\i}des per variables proposicionals i les connectives l%
\`{o}giques que les relacionen. Per exemple, $A(p,q,r)=\left( \left( p\wedge
q\right) \longrightarrow\left( r\vee\lnot p\right) \right) \wedge r$ \'{e}s
una forma proposional que cont\'{e} tres variables $p,q$ i $r$. Les formes
proposicionals no s\'{o}n proposicions, per\`{o} si cada variable
proposicional \'{e}s reempla\c{c}ada per una proposici\'{o} simple o
composta, la forma proposicional es converteix en una proposici\'{o}. Si
reemplacem les variables proposicionals per proposicions vertaderes o
falses, el nombre de proposicions que es generen \'{e}s $2^{n}$, sent $n$ el
nombre de variables proposicionals. A la taula de veritat que es genera se
l'anomena a vegades \textbf{matriu} de la forma proposicional. La matriu de
la forma $A(p,q,r)$ t\'{e} $2^{3}=8$ interpretacions posibles:%
\begin{equation*}
\begin{tabular}{cc|c|c|c|ccc}
$p$ & $q$ & $r$ & $\lnot p$ & $\alpha=p\wedge q$ & $\beta=r\vee\lnot p$ & $%
\alpha\longrightarrow\beta$ & $\left( \alpha\longrightarrow\beta\right)
\wedge r$ \\ \hline
V & V & V & F & V & V & V & V \\
V & V & F & F & V & F & F & F \\
V & F & V & F & F & V & V & V \\
V & F & F & F & F & F & V & F \\
F & V & V & V & F & V & V & V \\
F & V & F & V & F & V & V & F \\
F & F & V & V & F & V & V & V \\
F & F & F & V & F & V & V & F%
\end{tabular}
\
\end{equation*}
Per exemple, la forma es converteix en una proposici\'{o} vertadera quan les
variables proposicionals $p,q$ i $r$ s'interpreten respectivament com a
proposicions que s\'{o}n vertadera, falsa i vertadera (fila 3 de la matriu).

\begin{exemple}
Construeix la matriu de la forma $A(p,q)=\left( p\vee q\right)
\longrightarrow\left( p\wedge\lnot q\right) $. Quina interpretaci\'{o}
permet assegurar que la forma es converteix en una proposici\'{o} vertadera?
\end{exemple}

\begin{solucio}
La forma t\'{e} dues variables proposicionals i, per tant, $2^{2}=4$
interpretacions possibles. La matriu de la forma \'{e}s:
\begin{equation*}
\begin{tabular}{cc|c|c|cc}
$p$ & $q$ & $\lnot q$ & $p\vee q$ & $p\wedge\lnot q$ & $\left( p\vee
q\right) \longrightarrow\left( p\wedge\lnot q\right) $ \\ \hline
V & V & F & V & F & F \\
V & F & V & V & V & V \\
F & V & F & V & F & F \\
F & F & V & F & F & V%
\end{tabular}
\
\end{equation*}
Observem que una proposici\'{o} d'aquesta forma nom\'{e}s ser\`{a} vertadera
quan substitu\"{\i}m $q$ per una proposici\'{o} falsa.
\end{solucio}

\section{Tautologies i contradiccions}

Una \textbf{tautologia} \'{e}s una proposici\'{o} que \'{e}s vertadera per
necessitat l\`{o}gica. Una tautologia caracter\'{\i}stica de la l\`{o}gica cl%
\`{a}ssica \'{e}s la proposici\'{o} `$p$ o no $p$', sent $p$ qualsevol
proposici\'{o}. Aquesta tautologia se la coneix tamb\'{e} amb el nom de
\textbf{principi del tercer excl\`{o}s}, segons el qual la disjunci\'{o}
d'una proposici\'{o} i la seva negaci\'{o} \'{e}s sempre vertadera. En
efecte, la taula de veritat d'aquesta proposici\'{o} \'{e}s%
\begin{equation*}
\begin{tabular}{ll|l}
$p$ & $\lnot p$ & $p\vee\lnot p$ \\ \hline
\multicolumn{1}{c}{V} & \multicolumn{1}{c|}{F} & \multicolumn{1}{|c}{V} \\
\multicolumn{1}{c}{F} & \multicolumn{1}{c|}{V} & \multicolumn{1}{|c}{V}%
\end{tabular}
\text{,}
\end{equation*}
on veiem que l'\'{u}ltima columna hi ha nom\'{e}s el valor V com era
d'esperar. Es evident que la tautologia no t\'{e} cap inter\'{e}s matem\`{a}%
tic perqu\`{e} no informa de res, per\`{o} permet simplificar a vegades les
demostracions com veurem m\'{e}s endavant.

\bigskip

Ara b\'{e}, el concepte de tautologia t\'{e} m\'{e}s sentit quan en lloc de
tractar amb proposicions treballem en formes. Una forma \'{e}s tautologia
quan la matriu de la forma nom\'{e}s genera proposicions vertaderes; en unes
altres paraules, quan \'{e}s certa en totes les circumst\`{a}ncies
possibles. Per exemple, la forma $A(p,q)=(p\wedge q)\longrightarrow(p\vee q)
$ \'{e}s tautologia perqu\`{e} la seva taula \'{e}s:%
\begin{equation*}
\begin{tabular}{ll|l|l|l}
$p$ & $q$ & $p\wedge q$ & $p\vee q$ & $(p\wedge q)\longrightarrow(p\vee q)$
\\ \hline
\multicolumn{1}{c}{V} & \multicolumn{1}{c|}{V} & \multicolumn{1}{|c|}{V} &
\multicolumn{1}{|c|}{V} & \multicolumn{1}{|c}{V} \\
\multicolumn{1}{c}{V} & \multicolumn{1}{c|}{F} & \multicolumn{1}{|c|}{F} &
\multicolumn{1}{|c|}{V} & \multicolumn{1}{|c}{V} \\
\multicolumn{1}{c}{F} & \multicolumn{1}{c|}{V} & \multicolumn{1}{|c|}{F} &
\multicolumn{1}{|c|}{V} & \multicolumn{1}{|c}{V} \\
\multicolumn{1}{c}{F} & \multicolumn{1}{c|}{F} & \multicolumn{1}{|c|}{F} &
\multicolumn{1}{|c|}{F} & \multicolumn{1}{|c}{V}%
\end{tabular}
\end{equation*}
D'aqu\'{\i} surt el fet seg\"{u}ent: qualsevol proposici\'{o} composta de la
forma `$(p\wedge q)\longrightarrow(p\vee q)$' \'{e}s una tautologia,
independentment del que siguin les proposicions $p$ i $q$.

\bigskip

Una \textbf{contradicci\'{o}} \'{e}s una proposici\'{o} que \'{e}s falsa per
necessitat l\`{o}gica. De fet, tota contradicci\'{o} \'{e}s la negaci\'{o}
d'una tautologia i, al contrari, tota tautologia \'{e}s la negaci\'{o} d'una
contradicci\'{o}. La negaci\'{o} del principi del tercer excl\`{o}s \'{e}s
una contradicci\'{o} coneguda com el \textbf{principi de contradicci\'{o}},
segons el qual la conjunci\'{o} d'una proposici\'{o} i la seva negaci\'{o}
\'{e}s una proposici\'{o} sempre falsa. \'{E}s la proposici\'{o} `$p$ i no$%
~p $', sent $p$ qualsevol proposici\'{o}. La taula de veritat d'aquesta
proposici\'{o} \'{e}s%
\begin{equation*}
\begin{tabular}{ll|l}
$p$ & $\lnot p$ & $p\wedge\lnot p$ \\ \hline
\multicolumn{1}{c}{V} & \multicolumn{1}{c|}{F} & \multicolumn{1}{|c}{F} \\
\multicolumn{1}{c}{F} & \multicolumn{1}{c|}{V} & \multicolumn{1}{|c}{F}%
\end{tabular}
\text{,}
\end{equation*}
i veiem que l'\'{u}ltima columna nom\'{e}s t\'{e} el valor F. De la mateixa
manera que la tautologia, la contradicci\'{o} t\'{e} m\'{e}s sentit quan
s'aplica a formes proposicionals. Una forma \'{e}s contradicci\'{o} quan la
seva matriu nom\'{e}s genera proposicions falses. Per exemple, segons el
principi de contradicci\'{o} la forma $\left( p\longrightarrow q\right)
\wedge\lnot\left( p\longrightarrow q\right) $ \'{e}s una contradicci\'{o}.
Podem comprovar-ho fent la seva matriu:%
\begin{equation*}
\begin{tabular}{ll|l|l|l}
$p$ & $q$ & $p\longrightarrow q$ & $\lnot\left( p\longrightarrow q\right) $
& $(p\longrightarrow q)\wedge\lnot\left( p\longrightarrow q\right) $ \\
\hline
\multicolumn{1}{c}{V} & \multicolumn{1}{c|}{V} & \multicolumn{1}{|c|}{V} &
\multicolumn{1}{|c|}{F} & \multicolumn{1}{|c}{F} \\
\multicolumn{1}{c}{V} & \multicolumn{1}{c|}{F} & \multicolumn{1}{|c|}{F} &
\multicolumn{1}{|c|}{V} & \multicolumn{1}{|c}{F} \\
\multicolumn{1}{c}{F} & \multicolumn{1}{c|}{V} & \multicolumn{1}{|c|}{V} &
\multicolumn{1}{|c|}{F} & \multicolumn{1}{|c}{F} \\
\multicolumn{1}{c}{F} & \multicolumn{1}{c|}{F} & \multicolumn{1}{|c|}{V} &
\multicolumn{1}{|c|}{F} & \multicolumn{1}{|c}{F}%
\end{tabular}
\text{.}
\end{equation*}

Per acabar, quan una forma proposicional no \'{e}s tautologia ni contradicci%
\'{o} es diu que \'{e}s \textbf{conting\`{e}ncia}; en unes altres \
paraules, quan admet algunes proposicions vertaderes i altres falses per als
valors de veritat de les variables proposicionals. Un exemple de conting\`{e}%
ncia \'{e}s la forma $(p\vee q)\longrightarrow(p\wedge q)$ perqu\`{e} la
seva matriu \'{e}s:%
\begin{equation*}
\begin{tabular}{cc|c|c|c|}
$p$ & $q$ & $p\vee q$ & $p\wedge q$ & $(p\vee q)\longrightarrow(p\wedge q)$
\\ \hline
V & V & V & V & V \\
V & F & V & F & F \\
F & V & V & F & F \\
F & F & F & F & V%
\end{tabular}
\text{.}
\end{equation*}

\bigskip

Les formes proposicionals poden ser connectades amb operadors l\`{o}gics per
a formar noves formes proposicionals. D'aquesta manera, si $A$ i $B$ s\'{o}n
formes, llavors $\lnot A$, $A\wedge B$, $A\vee B$, $A\longrightarrow B$ i $%
A\longleftrightarrow B$ representen noves formes proposicionals. Per
exemple, si $A(p,q)=\lnot p\longrightarrow q$ i $B(p,r,s)=s\longrightarrow%
\left( p\vee\lnot r\right) $, aleshores $C(p,q,r,s)=A(p,q)\longrightarrow
B(p,r,s)$, \'{e}s a dir, $C(p,q,r,s)=\left( \lnot p\longrightarrow q\right)
\longrightarrow\left( s\longrightarrow\left( p\vee\lnot r\right) \right) $.
Observem que $A$ t\'{e} $2^{2}=4$ interpretacions, $B$ en t\'{e} $2^{3}=8$
interpretacions, i $C$ en t\'{e} $2^{4}=16$ interpretacions.

\begin{exemple}
Si $p,q$ i $r$ s\'{o}n variables proposicionals, llavors $\left( \left(
\lnot p\vee q\right) \wedge\left( \lnot r\longrightarrow q\right) \right)
\longrightarrow\left( p\longrightarrow r\right) $ \'{e}s una forma
proposicional tautol\`{o}gica?
\end{exemple}

\begin{solucio}
La forma cont\'{e} 3 variables i per tant la matriu de la forma t\'{e} $%
2^{3}=8$ interpretacions possibles. En lloc de fer la taula, podem raonar
suposant que la forma s'interpret\'{e}s com a contradicci\'{o}. Aleshores
existeix una valoraci\'{o} de les variables segons la qual l'implicaci\'{o}n
\'{e}s falsa i, per tant, l'antecedent $\left( \lnot p\vee q\right)
\wedge\left( \lnot r\longrightarrow q\right) $ \'{e}s vertader i el conseq%
\"{u}ent $p\longrightarrow r$ \'{e}s fals. Ara b\'{e}, l'antecedent \'{e}s
vertader quan $\lnot p\vee q$ i $\lnot r\longrightarrow q$ s\'{o}n ambdues
vertaderes, i el conseq\"{u}ent \'{e}s fals quan $p$ \'{e}s vertader i $r$
\'{e}s fals. Per\`{o} si prenem aquesta interpretaci\'{o}, aleshores $\lnot
p\vee q$ i $\lnot r\longrightarrow q$ no poden ser ambdues vertaderes perqu%
\`{e} $\lnot p\vee q$ \'{e}s vertadera nom\'{e}s si $q$ \'{e}s vertadera, per%
\`{o} $\lnot r\longrightarrow q$ \'{e}s aleshores falsa i, per tant, aix\`{o}
no \'{e}s possible. Com a conseq\"{u}\`{e}ncia, conclu\"{\i}m que la forma
no pot prendre el valor fals i, per tant, \'{e}s tautologia.

Alternativament, si constru\"{\i}m la matriu de la forma es t\'{e}:%
\begin{equation*}
\begin{tabular}{cccccccccc}
$p$ & $q$ & $r$ & $\lnot p$ & $\lnot r$ & $\alpha=\lnot p\vee q$ & $%
\beta=\lnot r\vee q$ & $\gamma=p\longrightarrow r$ & $\alpha\wedge\beta$ & $%
\left( \alpha\wedge\beta\right) \longrightarrow\gamma$ \\ \hline
V & V & V & F & F & V & V & V & V & V \\
V & V & F & F & V & V & V & F & F & V \\
V & F & V & F & F & F & F & V & F & V \\
V & F & F & F & V & F & V & F & F & V \\
F & V & V & V & F & V & V & V & V & V \\
F & V & F & V & V & V & V & V & V & V \\
F & F & V & V & F & V & F & V & F & V \\
F & F & F & V & V & V & V & V & V & V%
\end{tabular}
\end{equation*}
i surt que \'{e}s tautologia com era d'esperar.
\end{solucio}

\section{Implicaci\'{o} l\`{o}gica\label{1}}

Si $A$ i $B$ s\'{o}n dues formes proposicionals, llavors es diu que $A$
\textbf{implica l\`{o}gicament} $B$ quan la forma $A\longrightarrow B$ \'{e}%
s tautologia. Cal destacar que la implicaci\'{o} l\`{o}gica \'{e}s una relaci%
\'{o} entre formes proposicionals, que denotem per $\Longrightarrow$.
D'aquesta manera, quan escrivim $A\Longrightarrow B$, que llegim
abreujadament com "$A$ implica $B$", significa que $A\longrightarrow B$ \'{e}%
s tautologia. Observa que hem usat en aquest document la paraula "implica"
de dues maneres diferents. La primera, al parlar del condicional de dues
proposicions $p$ i $q$, $p\rightarrow q$, qu\`{e} no necess\`{a}riament es
una tautologia, mentre que la segona, fa un moment, quan la implicaci\'{o} s%
\'{\i} \'{e}s necess\`{a}riament una tautologia.

A la pr\`{a}ctica, el fet de tenir la implicaci\'{o} l\`{o}gica $%
A\Longrightarrow B$ permet fer el seg\"{u}ent argument: si $A\Longrightarrow
B$ i $A$ \'{e}s tautologia, aleshores necessari\`{a}ment $B$ \'{e}s
tautologia. D'aqu\'{\i} surt una interpretaci\'{o} molt com\'{u} en matem%
\`{a}tiques: quan $A\Longrightarrow B$, aleshores $B$ es diu que \'{e}s
\textbf{condici\'{o} necess\`{a}ria} per $A$, i tamb\'{e}, que $A$ \'{e}s
\textbf{condici\'{o} suficient} per $B$.

Per exemple, suposem que $n$ \'{e}s un enter positiu, aleshores l'enunciat
`si $n$ \'{e}s divisible per $4$, llavors $n$ \'{e}s divisible per $2$',
\'{e}s cert i pot escriure's com $P(n)\longrightarrow Q(n)$, on $P=$ `\'{e}s
divisible per $2$' i $Q=$ `\'{e}s divisible per $4$'. La condici\'{o} `$n$
\'{e}s divisible per $4$' \'{e}s suficient perqu\`{e} `$n$ sigui divisible
per $2$', \'{e}s a dir, nom\'{e}s cal que $n$ sigui divisible per $4$ perqu%
\`{e} sigui divisible per $2$. En canvi, la condici\'{o} `$n$ \'{e}s
divisible per $2$' \'{e}s necess\`{a}ria perqu\`{e} `$n$ \'{e}s divisible
per $4$', o sigui, si $n$ no fos divisible per $2$, aleshores no seria
divisible per 4.

M\'{e}s endavant veurem que les implicacions l\`{o}giques seran
extremadament \'{u}tils per construir raonaments v\`{a}lids. En particular,
s'utilitzaran \`{a}mpliament les seg\"{u}ents implicacions l\`{o}giques: Si $%
A,B$ i $C$ s\'{o}n formes proposicionals, aleshores

\begin{enumerate}
\item $(A\longrightarrow B)\wedge A$ $\Longrightarrow B$

\item $(A\longrightarrow B)\wedge\lnot B\Longrightarrow\lnot A$

\item (i) $A\wedge B$ $\Longrightarrow A$; (ii) $A\wedge B$ $\Longrightarrow
B$

\item (i) $A\Longrightarrow A\vee B$; (ii) $B\Longrightarrow A\vee B$

\item (i) $\left( A\vee B\right) \wedge\lnot B\Longrightarrow A$; (ii) $%
\left( A\vee B\right) \wedge\lnot A\Longrightarrow B$

\item (i) $A\longleftrightarrow B\Longrightarrow A\longrightarrow B$; (ii) $%
A\longleftrightarrow B\Longrightarrow B\longrightarrow A$

\item $\left( A\longrightarrow B\right) \wedge\left( B\longrightarrow
A\right) \Longrightarrow A\longleftrightarrow B$

\item $\left( A\longrightarrow B\right) \wedge\left( B\longrightarrow
C\right) \Longrightarrow A\longrightarrow C$
\end{enumerate}

Per veure que $(A\longrightarrow B)\wedge A$ $\Longrightarrow B$ hem de
comprovar que $(\left( A\longrightarrow B)\wedge A\right) $ $\longrightarrow
B$ \'{e}s tautologia. Constru\"{\i}m la taula de veritat corresponent:%
\begin{equation*}
\begin{tabular}{lllll}
$A$ & $B$ & $A\longrightarrow B$ & $\left( A\longrightarrow B\right) \wedge
A $ & $\left( \left( A\longrightarrow B\right) \wedge A\right)
\longrightarrow B$ \\ \hline
\multicolumn{1}{c}{V} & \multicolumn{1}{c|}{V} & \multicolumn{1}{c}{V} &
\multicolumn{1}{c}{V} & \multicolumn{1}{c}{V} \\
\multicolumn{1}{c}{V} & \multicolumn{1}{c|}{F} & \multicolumn{1}{c}{F} &
\multicolumn{1}{c}{F} & \multicolumn{1}{c}{V} \\
\multicolumn{1}{c}{F} & \multicolumn{1}{c|}{V} & \multicolumn{1}{c}{V} &
\multicolumn{1}{c}{F} & \multicolumn{1}{c}{V} \\
\multicolumn{1}{c}{F} & \multicolumn{1}{c|}{F} & \multicolumn{1}{c}{V} &
\multicolumn{1}{c}{F} & \multicolumn{1}{c}{V}%
\end{tabular}
\
\end{equation*}
i observem que efectivament \'{e}s tautologia. An\`{a}logament es fa per les
altres.

\section{Equival\`{e}ncia l\`{o}gica\label{2}}

Si $A$ i $B$ s\'{o}n dues formes proposicionals, llavors es diu que $A$ i $B$
s\'{o}n \textbf{equivalents l\`{o}gicament }quan la forma $%
A\longleftrightarrow B$ \'{e}s tautologia. Com la implicaci\'{o} l\`{o}gica,
l'equival\`{e}ncia \'{e}s una relaci\'{o} entre formes proposicionals que
denotem per $\Longleftrightarrow$. D'aquesta manera, quan escrivim $%
A\Longleftrightarrow B$, que llegim "$A$ \'{e}s equivalent a $B$" o "$A$ i $%
B $ s\'{o}n equivalents", significa que $A\longleftrightarrow B$ \'{e}s
tautologia.

A la pr\`{a}ctica, el fet de tenir l'equival\`{e}ncia $A\Longleftrightarrow
B $ permet fer el seg\"{u}ent argument: Si $A\Longleftrightarrow B$, i $A$
(resp. $B$) \'{e}s tautologia, aleshores necess\`{a}riament $B$ (resp. $A$)
\'{e}s tautologia, i tamb\'{e}, d'aqu\'{\i} surt una interpretaci\'{o} molt
comuna en matem\`{a}tiques: quan $A\Longleftrightarrow B$, aleshores es diu
que $A$ \'{e}s \textbf{condici\'{o} necess\`{a}ria i suficient} per $B$ o a
l'inrev\'{e}s. De fet, aix\`{o} es dedueix de l'equival\`{e}ncia seg\"{u}%
ent: $A\longleftrightarrow B\Longleftrightarrow\left( A\longrightarrow
B\right) \wedge\left( B\longrightarrow A\right) $ perqu\`{e}

\begin{equation*}
\begin{tabular}{lllllll}
$A$ & $B$ & $\alpha=A\longrightarrow B$ & $\beta=B\longrightarrow A$ & $%
\gamma=A\longleftrightarrow B$ & $\alpha\wedge\beta$ & $\gamma
\longleftrightarrow\left( \alpha\wedge\beta\right) $ \\ \hline
\multicolumn{1}{c}{V} & \multicolumn{1}{c|}{V} & \multicolumn{1}{c}{V} &
\multicolumn{1}{c}{V} & \multicolumn{1}{c}{V} & \multicolumn{1}{c}{V} &
\multicolumn{1}{c}{V} \\
\multicolumn{1}{c}{V} & \multicolumn{1}{c|}{F} & \multicolumn{1}{c}{F} &
\multicolumn{1}{c}{V} & \multicolumn{1}{c}{F} & \multicolumn{1}{c}{F} &
\multicolumn{1}{c}{V} \\
\multicolumn{1}{c}{F} & \multicolumn{1}{c|}{V} & \multicolumn{1}{c}{V} &
\multicolumn{1}{c}{F} & \multicolumn{1}{c}{F} & \multicolumn{1}{c}{F} &
\multicolumn{1}{c}{V} \\
\multicolumn{1}{c}{F} & \multicolumn{1}{c|}{F} & \multicolumn{1}{c}{V} &
\multicolumn{1}{c}{V} & \multicolumn{1}{c}{V} & \multicolumn{1}{c}{V} &
\multicolumn{1}{c}{V}%
\end{tabular}
\end{equation*}

A continuaci\'{o} s'enumeren algunes equival\`{e}ncies l\`{o}giques que
seran particularment \'{u}tils i al seu costat el nom pel qual s\'{o}n
conegudes:

\begin{enumerate}
\item $\lnot\left( \lnot A\right) \Longleftrightarrow A$ (llei de la doble
negaci\'{o})

\item $A\wedge B\Longleftrightarrow B\wedge A$ (llei commutativa de $\wedge$)

\item $A\vee B\Longleftrightarrow B\vee A$ (llei commutativa de $\vee$)

\item $\left( A\wedge B\right) \wedge C\Longleftrightarrow A\wedge\left(
B\wedge C\right) $ (llei associativa de $\wedge$)

\item $\left( A\vee B\right) \vee C\Longleftrightarrow A\vee\left( B\vee
C\right) $ (llei associativa de $\vee$)

\item $A\vee\left( B\wedge C\right) \Longleftrightarrow\left( A\vee B\right)
\wedge\left( A\vee C\right) $ (llei distributiva de $\vee$ respecte de $%
\wedge$)

\item $A\wedge\left( B\vee C\right) \Longleftrightarrow\left( A\wedge
B\right) \vee\left( A\wedge C\right) $ (llei distributiva de $\wedge$
respecte de $\vee$)

\item $A\longrightarrow B\Longleftrightarrow\lnot A\vee B$

\item $A\longrightarrow B\Longleftrightarrow\lnot B\longrightarrow\lnot A$
(llei del contrarec\'{\i}proc)

\item $A\longleftrightarrow B\Longleftrightarrow\left( A\longrightarrow
B\right) \wedge\left( B\longrightarrow A\right) $ (llei del bicondicional)

\item $\left( A\vee B\right) \longrightarrow C\Longleftrightarrow\left(
A\longrightarrow C\right) \wedge\left( B\longrightarrow C\right) $

\item $A\longrightarrow\left( B\wedge C\right) \Longleftrightarrow\left(
A\longrightarrow B\right) \wedge\left( A\longrightarrow C\right) $

\item $\lnot(A\wedge B)\Longleftrightarrow\lnot A\vee\lnot B$ (llei de De
Morgan)

\item $\lnot(A\vee B)\Longleftrightarrow\lnot A\wedge\lnot B$ (llei de De
Morgan)
\end{enumerate}

Algunes d'aquestes lleis tenen molt d'inter\`{e}s quan volem fer
demostracions. La llei de la doble negaci\'{o} a la pr\`{a}ctica es usa de
la seg\"{u}ent manera: volem provar que $p$ \'{e}s vertadera. Per a fer-ho,
suposem com hip\`{o}tesi que $\lnot p$ \'{e}s vertadera, i dedu\"{\i}m
contradicci\'{o}, i, per tant, $\lnot\left( \lnot p\right) $ \'{e}s
vertadera. Ara b\'{e}, $\lnot\left( \lnot p\right) $ i $p$ s\'{o}n
equivalents i, per tant, $p$ \'{e}s vertadera. Aquesta manera de demostrar
un enunciat matem\`{a}tic ser\`{a} tractada amb m\'{e}s detall m\'{e}s
endavant.

\bigskip

Es diu \textbf{rec\'{\i}proc} de la proposici\'{o} $p\longrightarrow q$ a la
proposici\'{o} $q\longrightarrow p$, i \textbf{contrarec\'{\i}proc} a la
proposici\'{o} $\lnot q\longrightarrow\lnot p$. La llei del contrarec\'{\i}%
proc tamb\'{e} \'{e}s la base de moltes demostracions. A la pr\`{a}ctica es
usa de la seg\"{u}ent manera: volem provar que $p$ implica $q$. Per a
fer-ho, observem que \'{e}s m\'{e}s senzill provar que $\lnot q$ implica $%
\lnot p$, i, per tant, \'{e}s cert que $p$ implica $q$. Tractarem aquesta
forma de demostraci\'{o} m\'{e}s en detall m\'{e}s endavant.

\bigskip

Finalment, fem alguns comentaris sobre la llei del bicondicional. Quan s'ha
de demostrar una equivalencia $p\longleftrightarrow q$, la llei del
bicondicional permet fer-ho de la seg\"{u}ent manera: primer, provem que $p$
implica $q$, i despr\'{e}s, el rec\'{\i}proc, es a dir, que $q$ implica $p$.
Es habitual en matem\`{a}tiques dir-ho d'alguna d'aquestes maneres: (1) $p$
\'{e}s condici\'{o} necess\`{a}ria i suficient per $q$ o (2) si $p$, llavors
$q$, i rec\'{\i}procament.

\bigskip

Per a completar aquesta secci\'{o} provarem les lleis 10 i 12. Per demostrar
10 hem de comprovar que $\left( A\longleftrightarrow B\right)
\longleftrightarrow\left( A\longrightarrow B\right) \wedge\left(
B\longrightarrow A\right) $ \'{e}s tautologia. En efecte, constriu\"{\i}m la
matriu i s'obt\'{e} tautologia

\begin{equation*}
\begin{tabular}{ccccccc}
$A$ & $B$ & $\alpha=A$ $\longrightarrow$ $B$ & $\beta=B\longrightarrow A$ & $%
\gamma=A\longleftrightarrow B$ & $\alpha\wedge\beta$ & $\gamma
\longleftrightarrow\left( \alpha\wedge\beta\right) $ \\ \hline
V & V & V & V & V & V & V \\
V & F & F & V & F & F & V \\
F & V & V & F & F & F & V \\
F & F & V & V & V & V & V%
\end{tabular}
\text{.}
\end{equation*}

Per demostrar 11 hem de comprovar que $\lnot(A\wedge B)\longleftrightarrow
\left( \lnot A\vee\lnot B\right) $ \'{e}s tautologia. En efecte, constriu%
\"{\i}m la matriu i s'obt\'{e} tautologia

\begin{equation*}
\begin{tabular}{llllllll}
$A$ & $B$ & $A\wedge B$ & $\lnot A$ & $\lnot B$ & $\alpha=\lnot\left(
A\wedge B\right) \,$ & $\beta=\lnot A\vee\lnot B$ & $\alpha%
\longleftrightarrow\beta $ \\ \hline
\multicolumn{1}{c}{V} & \multicolumn{1}{c|}{V} & \multicolumn{1}{c}{V} &
\multicolumn{1}{c}{F} & \multicolumn{1}{c}{F} & \multicolumn{1}{c}{F} &
\multicolumn{1}{c}{F} & \multicolumn{1}{c}{V} \\
\multicolumn{1}{c}{V} & \multicolumn{1}{c|}{F} & \multicolumn{1}{c}{F} &
\multicolumn{1}{c}{F} & \multicolumn{1}{c}{V} & \multicolumn{1}{c}{V} &
\multicolumn{1}{c}{V} & \multicolumn{1}{c}{V} \\
\multicolumn{1}{c}{F} & \multicolumn{1}{c|}{V} & \multicolumn{1}{c}{F} &
\multicolumn{1}{c}{V} & \multicolumn{1}{c}{F} & \multicolumn{1}{c}{V} &
\multicolumn{1}{c}{V} & \multicolumn{1}{c}{V} \\
\multicolumn{1}{c}{F} & \multicolumn{1}{c|}{F} & \multicolumn{1}{c}{F} &
\multicolumn{1}{c}{V} & \multicolumn{1}{c}{V} & \multicolumn{1}{c}{V} &
\multicolumn{1}{c}{V} & \multicolumn{1}{c}{V}%
\end{tabular}
.
\end{equation*}

Les altres \'{e}s demostran de la mateixa manera.

\begin{exemple}
Escriu fent \'{u}s de les connectives les proposicions seg\"{u}ents: (1)
S'ha de revisar. Per exemple, la proposici\'{o} `un nombre natural \'{e}s
parell si i nom\'{e}s si el seu quadrat \'{e}s parell' \'{e}s el
bicondicional: $P(n)$ si i nomes si $P(n^{2})$, on $P$ \'{e}s el predicat
`ser un nombre natural parell'. (2) \'{E}s necess\`{a}ri que $x$ sigui un
nombre real no negativo perqu\`{e} $\sqrt{x}$ sigui real; (3) \'{E}s
suficient que $n=3$ perqu\`{e} $n^{2}-5n+6=0$; (4) $n^{2}-5n+6=0$
precisament si $n=2$ o $n=3$; (5) $\left\vert a\right\vert =\left\vert
b\right\vert $ sii $a=\pm b$.
\end{exemple}

\begin{solucio}
La proposici\'{o} (1) es pot expressar com la implicaci\'{o}: $\sqrt{x}$
\'{e}s real $\Longrightarrow x$ real i $x\geq0$; la proposici\'{o} (2) \'{e}%
s la implicaci\'{o}: $n=3$ $\Longrightarrow$ $n^{2}-5n+6=0$; la proposici%
\'{o} (3) \'{e}s el bicondicional: $n^{2}-5n+6=0$ $\Longleftrightarrow$ $n=2
$ o $n=3$, i, la proposici\'{o} (4) \'{e}s tamb\'{e} un bicondicional: $%
\left\vert a\right\vert =\left\vert b\right\vert $ $\Longleftrightarrow$ $%
a=\pm b$.
\end{solucio}

\section{Infer\`{e}ncia l\`{o}gica}

Molts cops veieu com en resoldre exercicis a matem\`{a}tiques se us demana
que raoneu o justifiqueu la resposta. Ara volem tractar aquest assumpte. Qu%
\`{e} es vol dir que raoneu la vostra resposta? \'{E}s clar, heu de fer un
raonament v\`{a}lid que justifiqui all\`{o} que se us demana demostrar. Aix%
\`{o} t\'{e} dues parts, una \'{e}s conduir correctament el vostre raonament
des del punt vista l\`{o}gic, i l'altra, \'{e}s comprendre correctament els
conceptes que estan presents en el raonament des del punt de vista matem\`{a}%
tic. Aqu\'{\i} tractarem el primer punt perqu\`{e} el segon dep\`{e}n dels
coneixements que cadasc\'{u} t\'{e} de les matem\`{a}tiques.

\bigskip

Si suposem que la proposici\'{o} condicional `$p\rightarrow q$' \'{e}s
vertadera, llavors sabem (recordeu la taula de veritat d'aquesta connectiva)
que si `$p$' \'{e}s vertadera, necess\`{a}riament ho ser\`{a} `$q$'. Ara b%
\'{e}, nom\'{e}s pel fet que `$p\rightarrow q$' sigui vertadera no es t\'{e}
que `$p$' i `$q$' siguin vertaderes, perqu\`{e} podrien ser totes dues
falses, o `$p$' falsa i `$q$' vertadera. Per tant, si `$p\rightarrow q$' i `$%
p$' s\'{o}n vertaderes, llavors s\'{\i} que `$q$' ha de ser vertadera. En
aquest \'{u}ltim cas es diu que `$q$' \'{e}s \textbf{conseq\"{u}\`{e}ncia l%
\`{o}gica} de `$p\rightarrow q$' i `$p$' o tamb\'{e} que `$q$' s'\textbf{%
infereix l\`{o}gicament} de `$p\rightarrow q$' i `$p$'. \'{E}s habitual
representar aquesta infer\`{e}ncia l\`{o}gica seguint el seg\"{u}ent format:%
\begin{equation*}
\begin{tabular}{l}
$p\longrightarrow q$ \\
$p$ \\ \hline
$q$%
\end{tabular}
\ \text{,}
\end{equation*}
on la linea horitzontal separa les proposicions que es diuen \textbf{%
premisses}, de la proposici\'{o} que es diu \textbf{conclusi\'{o}}.

Per exemple: Si $x\in A$, llavors $\left\vert x\right\vert \leq1$; sabem que
$x\in A\cap B$. En conseq\"{u}\`{e}ncia, $\left\vert x\right\vert \leq1$. La
regla d'infer\`{e}ncia anterior ens permet assegurar que aquest argument
\'{e}s v\`{a}lid perqu\`{e} sabem que $x\in A\cap B$ implica $x\in A$.

\bigskip

Observem doncs, que la conseq\"{u}\`{e}ncia l\`{o}gica \'{e}s una relaci\'{o}
entre les premisses i la conclusi\'{o}. De fet, pel raonament que hem fet
abans aquesta infer\`{e}ncia pot tamb\'{e} escriure's com `$\left(
p\rightarrow q\right) \wedge p$' $\Longrightarrow$ `$q$', \'{e}s a dir, com
una implicaci\'{o} l\`{o}gica. \'{E}s clar que tamb\'{e} podr\'{\i}em
reescriure aquesta implicaci\'{o} l\`{o}gica en termes de formes
proposicionals com $(A\longrightarrow B)\wedge A$ $\Longrightarrow B$, on $A$
i $B$ s\'{o}n formes, i tamb\'{e} com a infer\`{e}ncia l\`{o}gica seguint el
format anterior:%
\begin{equation*}
\begin{tabular}{l}
$A\longrightarrow B$ \\
$A$ \\ \hline
$B$%
\end{tabular}
\text{.}
\end{equation*}

En general, des del punt de vista l\`{o}gic un \textbf{raonament} \'{e}s un
condicional que t\'{e} com antecedent la conjunci\'{o} de proposicions $%
P_{1},...,P_{n}$, anomenades premisses, i com a conseq\"{u}ent una proposici%
\'{o} $C$, anomenada conclusi\'{o}:%
\begin{equation*}
\left( P_{1}\wedge\cdots\wedge P_{n}\right) \longrightarrow C\text{.}
\end{equation*}
Es diu que el raonament \'{e}s \textbf{v\`{a}lid} si la conclusi\'{o} necess%
\`{a}riament es deriva de les premisses, o sigui, si $\left(
P_{1}\wedge\cdots\wedge P_{n}\right) \longrightarrow C$ \'{e}s tautologia, o
equivalentment, si $\left( P_{1}\wedge\cdots\wedge P_{n}\right)
\Longrightarrow C$. Pensant en la noci\'{o} d'implicaci\'{o} l\`{o}gica,
podem dir que un raonament \'{e}s v\`{a}lid si no podem assignar valors de
veritat a les proposicions que s'utilitzen en el raonament de manera que les
premisses siguin vertaderes i la conclusi\'{o} sigui falsa.

\bigskip

\'{E}s important no oblidar aqu\'{\i} que la l\`{o}gica s'ocupa nom\'{e}s
d'analitzar la validesa dels raonaments, o sigui de la sintaxi del
llenguatge, no ens pot dir res sobre si la informaci\'{o} continguda en una
hip\`{o}tesi \'{e}s vertadera o falsa, que formaria part de la interpretaci%
\'{o} o sem\`{a}ntica del llenguatge. Per tant, els termes v\`{a}lid i no v%
\`{a}lid es refereixen a l'estructura del raonament, no a la veritat o
falsedat de les proposicions qu\`{e} dep\`{e}n dels nostres coneixements de
matem\`{a}tiques.

\bigskip

Tenint en compte l'\'{u}ltima manera d'escriure el nostre argument, podr%
\'{\i}em intentar demostrar que \'{e}s v\`{a}lid, mostrant que $\left(
P_{1}\wedge\cdots\wedge P_{n}\right) \longrightarrow C$ \'{e}s tautologia,
utilitzant una taula de veritat. Aquest m\`{e}tode realment funcionaria, per%
\`{o} no seria gens efica\c{c}. En primer lloc, at\`{e}s que si hi ha $m$
proposicions implicades, la taula de veritat hauria de tenir $2^{m}$ files,
la qual cosa seria molt feixuga si $m$ \'{e}s gran. En segon lloc, l'\'{u}s
d'una taula de veritat no proporciona cap visi\'{o} intu\"{\i}tiva de per qu%
\`{e} l'argument \'{e}s v\`{a}lid.

Per a aquests motius, en lloc d'utilitzar taules de veritat, intentarem
justificar la validesa dels raonaments fent \'{u}s de les implicacions l\`{o}%
giques que hem donat. Si volem demostrar una implicaci\'{o} l\`{o}gica
complicada, \'{e}s recomanable fer-ho descomponent-la en una
col\textperiodcentered lecci\'{o} d'implicacions m\'{e}s senzilles, preses
d'una en una. Si ja es coneixen les implicacions m\'{e}s simples, podrien
ser blocs per a la implicaci\'{o} m\'{e}s complicada. Algunes de les
implicacions senzilles que fem servir, conegudes com a \textbf{regles d'infer%
\`{e}ncia} \label{5} de la l\`{o}gica proposicional, es detallen a continuaci%
\'{o}.

\bigskip

\begin{tabular}{lllll}
\begin{tabular}{l}
$A\longrightarrow B$ \\
$A$ \\ \hline
$B$%
\end{tabular}
& MP: Modus Ponens &  &
\begin{tabular}{l}
$A\longrightarrow B$ \\
$\lnot B$ \\ \hline
$\lnot A$%
\end{tabular}
& MT: Modus Tollens \\
&  &  &  &  \\
\begin{tabular}{l}
$A$ \\ \hline\hline
$\lnot\lnot A$%
\end{tabular}
& DN: Doble Negaci\'{o} &  &
\begin{tabular}{l}
$A$ \\ \hline\hline
$A$%
\end{tabular}
& R: Repetici\'{o} \\
&  &  &  &  \\
\begin{tabular}{l}
$A\wedge B$ \\ \hline
$A$%
\end{tabular}
& EC: Eliminaci\'{o} Conjunctor &  &
\begin{tabular}{l}
$A\wedge B$ \\ \hline
$B$%
\end{tabular}
& EC: Eliminaci\'{o} Conjunctor \\
&  &  &  &  \\
\begin{tabular}{l}
$A$ \\
$B$ \\ \hline
$A\wedge B$%
\end{tabular}
& IC: Introducci\'{o} conjunctor &  &
\begin{tabular}{l}
$A$ \\ \hline
$A\vee B$%
\end{tabular}
& IC: Introducci\'{o} disjunctor \\
&  &  &  &  \\
\begin{tabular}{l}
$B$ \\ \hline
$A\vee B$%
\end{tabular}
& ID: Introducci\'{o} Disjunctor &  &
\begin{tabular}{l}
$A\vee B$ \\
$\lnot A$ \\ \hline
$B$%
\end{tabular}
& ED: Eliminaci\'{o} Disjunctor \\
&  &  &  &  \\
\begin{tabular}{l}
$A\vee B$ \\
$\lnot B$ \\ \hline
$A$%
\end{tabular}
& ED: Eliminaci\'{o} Disjunctor &  &
\begin{tabular}{l}
$A\longleftrightarrow B$ \\ \hline
$A\longrightarrow B$%
\end{tabular}
& EB: Eliminaci\'{o} Bicondicional \\
&  &  &  &  \\
\begin{tabular}{l}
$A\longleftrightarrow B$ \\ \hline
$B\longrightarrow A$%
\end{tabular}
& EB: Eliminaci\'{o} Bicondicional &  &
\begin{tabular}{l}
$A\longrightarrow B$ \\
$B\longrightarrow A$ \\ \hline
$A\longleftrightarrow B$%
\end{tabular}
& IB: Introducci\'{o} Bicondicional \\
&  &  &  &  \\
\begin{tabular}{l}
$A\longrightarrow B$ \\
$B\longrightarrow C$ \\ \hline
$A\longrightarrow C$%
\end{tabular}
& TC: Transitivitat Condicional &  &
\begin{tabular}{l}
$A\longrightarrow B$ \\
$C\longrightarrow D$ \\
$A\vee C$ \\ \hline
$B\vee D$%
\end{tabular}
& DC: Dilema constructiu \\
&  &  &  &  \\
\begin{tabular}{l}
$\lnot\left( A\vee B\right) $ \\ \hline\hline
$\lnot A\wedge\lnot B$%
\end{tabular}
& DM: De Morgan &  &
\begin{tabular}{l}
$\lnot\left( A\wedge B\right) $ \\ \hline\hline
$\lnot A\vee\lnot B$%
\end{tabular}
& DM: De Morgan%
\end{tabular}

\bigskip

En aquesta taula hem utilitzat el conveni de representar per una l\'{\i}nia
horitzontal doble dues regles d'infer\`{e}ncia com s'indica a continuaci\'{o}%
:%
\begin{equation*}
\begin{tabular}{l}
$A$ \\ \hline\hline
$B$%
\end{tabular}
\ \qquad\text{est\`{a} en lloc de\qquad\ }%
\begin{tabular}{l}
$A$ \\ \hline
$B$%
\end{tabular}
\ \quad\text{ i \quad}%
\begin{tabular}{l}
$B$ \\ \hline
$A$%
\end{tabular}
\
\end{equation*}
on $A$ i $B$ s\'{o}n formes proposicionals. No sols hem tingut present les
implicacions l\`{o}giques de la secci\'{o} \ref{1}, tamb\'{e} les equival%
\`{e}ncies l\`{o}giques de la secci\'{o} \ref{2} com, per exemple, les lleis
de De Morgan.

\begin{exemple}
Volem determinar si el seg\"{u}ent raonament \'{e}s v\`{a}lid:
\textquotedblleft Si en Pau va rebre l'e-mail, llavors va agafar l'avi\'{o}
i ser\`{a} aqu\'{\i} al migdia. Pau no va agafar l'avi\'{o}. Per tant, Pau
no va rebre l'e-mail".
\end{exemple}

\begin{solucio}
Primer procedim a identificar les proposicions simples:

\begin{description}
\item[$p=$] `Pau va rebre l'e-mail'

\item[$q=$] `Pau va agafar l'avi\'{o}'

\item[$r=$] `Pau ser\`{a} aqu\'{\i} al migdia'
\end{description}

Despr\'{e}s, expressem formalment el raonament:%
\begin{equation*}
\begin{tabular}{ll}
$P_{1}:$ & $p\longrightarrow\left( q\wedge r\right) $ \\
$P_{2}:$ & $\lnot q$ \\ \hline
$C:$ & $\lnot p$%
\end{tabular}
\end{equation*}

Per provar la validesa d'aquest argument utilitzarem les regles d'infer\`{e}%
ncia:%
\begin{equation*}
\begin{tabular}{lll}
1. & $p\longrightarrow\left( q\wedge r\right) $ & $P_{1}$ \\
2. & $\lnot q$ & $P_{2}$ \\
3. & $\lnot q\vee\lnot r$ & ID 2 \\
4. & $\lnot\left( q\wedge r\right) $ & DM 3 \\ \hline
5. & $\lnot p$ & MT (1,4)%
\end{tabular}
\
\end{equation*}
Observem que la fila $3$ s'ha dedu\"{\i}t de la fila 2 mitjan\c{c}ant la
regla ID; la fila 4 s'ha dedu\"{\i}t de la fila 3 fent \'{u}s de la llei de
De Morgan i, finalment, la fila 5 que \'{e}s la conclusi\'{o} s'ha obtingut
de les files 1 i 4 per la regla MT.
\end{solucio}

\begin{exemple}
Volem determinar si el seg\"{u}ent raonament \'{e}s v\`{a}lid: "Si robes un
banc, vas a la pres\'{o}. Si anem a la pres\'{o}, no ens divertim. Si tenim
vacances, ens divertim. Robem un banc o tenim vacances. Per tant, anem a la
pres\'{o} o ens divertim"
\end{exemple}

\begin{solucio}
Les proposicions simples d'aquest raonament s\'{o}n:

\begin{description}
\item[$p=$] `Robem un banc'

\item[$q=$] `Anem a la pres\'{o}'

\item[$r=$] `Ens divertim'

\item[$s=$] `Tenim vacances'
\end{description}

Expressem el raonament simb\`{o}licament:%
\begin{equation*}
\begin{tabular}{ll}
$P_{1}:$ & $p\longrightarrow q$ \\
$P_{2}:$ & $q\longrightarrow\lnot r$ \\
$P_{3}:$ & $s\longrightarrow r$ \\
$P_{4}:$ & $p\vee s$ \\ \hline
$C:$ & $q\vee r$%
\end{tabular}
\
\end{equation*}

Intentem provar la validesa d'aquest argument utilitzant les regles d'infer%
\`{e}ncia:%
\begin{equation*}
\begin{tabular}{lll}
1. & $p\longrightarrow q$ & $P_{1}$ \\
2. & $q\longrightarrow\lnot r$ & $P_{2}$ \\
3. & $s\longrightarrow r$ & $P_{3}$ \\
4. & $p\vee s$ & $P_{4}$ \\ \hline
5. & $q\vee r$ & DC (1,3,4)%
\end{tabular}
\
\end{equation*}
Observem, per\`{o} que no hem fet \'{u}s de la hip\`{o}tesi 2. Llavors, la
conclusi\'{o} $q\vee r$ \'{e}s vertadera si $q$ i $r$ ho s\'{o}n, per\`{o}
en canvi la hip\`{o}tesi 2, $q\longrightarrow\lnot r$, \'{e}s falsa i aix%
\`{o} no \'{e}s possible. Per tant, el raonament no \'{e}s v\`{a}lid.
\end{solucio}

\bigskip

En tot raonament se'ns presenten dues q\"{u}estions molt importants: Una
\'{e}s la seva validesa, i l'altra, \'{e}s la seva dedu\"{\i}bilitat. La
primera q\"{u}esti\'{o} fa refer\`{e}ncia a les taules de veritat, i la
segona, a les regles d'infer\`{e}ncia. Quina relaci\'{o} hi ha entre
aquestes dues nocions? Tot i que no \'{e}s absolutament obvi ni f\`{a}cil de
demostrar, resulta molt remarcable que les dues nocions, una de naturalesa
sem\`{a}ntica i l'altre sint\`{a}ctica, sempre donen el mateix resultat,
\'{e}s a dir, un raonament \'{e}s v\`{a}lid si i nom\'{e}s si \'{e}s dedu%
\"{\i}ble. Per tant, si volem demostrar que un argument donat \'{e}s v\`{a}%
lid, n'hi haur\`{a} prou amb demostrar que \'{e}s dedu\"{\i}ble i viceversa.
L'equival\`{e}ncia d'aquests dos enfocaments \'{e}s un resultat important en
la l\`{o}gica. El fet que la validesa implica la dedu\"{\i}bilitat es coneix
com a \textbf{teorema de completesa} de la l\`{o}gica proposicional, i el
fet que la dedu\"{\i}bilitat implica la validesa es coneix com a \textbf{%
teorema de la correcci\'{o}} de la l\`{o}gica proposicional. A m\'{e}s,
aquesta l\`{o}gica \'{e}s \textbf{decidible}, la qual cosa vol dir que
qualsevol raonament expressat en aquest llenguatge podr\`{a} deduir-se en un
nombre finit de passos, mitjan\c{c}ant les taules de veritat, la seva
validesa o no. Aquests resultats i les seves demostracions poden trobar-se
en qualsevol text de l\`{o}gica proposicional.

\bigskip

Per les consideracions de l'apartat anterior observem que per provar que un
raonament \'{e}s v\`{a}lid, simplement hem de trobar una deducci\'{o}, que
sovint \'{e}s una tasca molt m\'{e}s agradable que mostrar directament la
seva validesa. En canvi, per provar que un raonament no \'{e}s v\`{a}lid,
les deduccions no s\'{o}n de gran ajuda, perqu\`{e} haur\'{\i}em de
demostrar que no \'{e}s possible trobar cap deducci\'{o}, i aix\`{o} no
podrem assegurar-ho mai, sempre podria haver-hi una deducci\'{o} que funcion%
\'{e}s. En aquests casos \'{e}s millor fer servir la definici\'{o} de
validesa directament i trobar valors de veritat per als quals les
proposicions que s\'{o}n premisses del raonament siguin vertaderes i la
conclusi\'{o} falsa.

\begin{exemple}
Volem determinar la validesa del seg\"{u}ent raonament: \textquotedblleft Si
el crim va oc\'{o}rrer despr\'{e}s de les 4, llavors en Pep no va poder
haver-lo com\`{e}s. Si el crim va oc\'{o}rrer a les 4 o abans, llavors en
Carles no va poder haver-lo com\`{e}s. El crim involucra a dues persones, si
en Carles no el va cometre. Per tant, el crim involucra a dues
persones\textquotedblright.
\end{exemple}

\begin{solucio}
Primer procedim a identificar les proposicions simples:

\begin{description}
\item[$p=$] `El crim va oc\'{o}rrer despr\'{e}s de les 4'

\item[$q=$] `Pep podia haver com\`{e}s el crim'

\item[$r=$] `Carles podia haver com\`{e}s el crim'

\item[$s=$] `El crim involucra a dues persones'
\end{description}

Despr\'{e}s expressem el raonament simb\`{o}licament:%
\begin{equation*}
\begin{tabular}{ll}
$P_{1}:$ & $p\longrightarrow\lnot q$ \\
$P_{2}:$ & $\lnot p\longrightarrow\lnot r$ \\
$P_{3}:$ & $\lnot r\longrightarrow s$ \\ \hline
$C:$ & $s$%
\end{tabular}
\
\end{equation*}

Intentem provar la validesa d'aquest argument utilitzant les regles d'infer%
\`{e}ncia:%
\begin{equation*}
\begin{tabular}{lll}
1. & $p\longrightarrow\lnot q$ & $P_{1}$ \\
2. & $\lnot p\longrightarrow\lnot r$ & $P_{2}$ \\
3. & $\lnot r\longrightarrow s$ & $P_{3}$ \\
4. & $\lnot p\longrightarrow s$ & TC (2,3) \\
5. & $p\vee\lnot p$ & Tautologia \\ \hline
6. & $\lnot q\vee s$ & DC (1,4,5)%
\end{tabular}
\
\end{equation*}
No hem dedu\"{\i}t $s$ sin\'{o} $\lnot q\vee s$. Aix\`{o} fa pensar que si $%
q $ \'{e}s falsa i $s$ tamb\'{e}, hi ha una interpretaci\'{o} (una fila de
la taula de veritat) que far\`{a} les premisses vertaderes i la conclusi\'{o}%
, falsa; es pot comprovar que aix\`{o} passa si prenem $p$ i $r$ ambdues
vertaderes. Per tant, aquest raonament no \'{e}s v\`{a}lid. A continuaci\'{o}
mostrem la interpretaci\'{o} esmentada:%
\begin{equation*}
\begin{tabular}{l|l|l|l||l|l|l|l|l||l|l|l|l||l}
$p$ & $\longrightarrow$ & $\lnot$ & $q$ & $\lnot$ & $p$ & $\longrightarrow$
& $\lnot$ & $r$ & $\lnot$ & $r$ & $\longrightarrow$ & $s $ & $s$ \\ \hline
V & V & V & F & F & V & V & F & V & F & V & V & F & F%
\end{tabular}
\
\end{equation*}

Volem destacar el fet qu\`{e} construir una taula de veritat per veure que
hi ha una valoraci\'{o} que prova el que hem dit, portaria molta feina perqu%
\`{e} la taula tindria 16 files.
\end{solucio}

\bigskip

Hi ha raonaments que quan intentem provar la seva validesa dedu\"{\i}m una
contradicci\'{o}, \'{e}s a dir, dedu\"{\i}m una proposici\'{o} i la seva
negaci\'{o}. Llavors es diu que el conjunt de premisses d'aquest raonament
\'{e}s \textbf{inconsistent}. Quan les premisses d'un raonament no s\'{o}n
inconsistents es diu que les premisses s\'{o}n \textbf{consistents}. No \'{e}%
s que hi hagi res l\`{o}gicament erroni amb premisses inconsistents,
senzillament que no serveixen per res, perqu\`{e} aleshores podem deduir
qualsevol proposici\'{o}. Volem destacar tamb\'{e} que, des del punt de
vista sem\`{a}ntic, un raonament t\'{e} un conjunt de premisses inconsistent
quan no hi ha cap interpretaci\'{o} (cap fila de la taula de veritat) que
faci a totes les premisses vertaderes.

\begin{exemple}
Volem determinar si les premisses del seg\"{u}ent raonament \'{e}s o no
consistent: "Miquel no toca la piano o Maria toca la guitarra. Si Carla no
toca el viol\'{\i}, Maria no toca la guitarra. Miquel toca el piano i Carla
no toca el viol\'{\i}. Per tant, Jaume toca l'acordi\'{o}".
\end{exemple}

\begin{solucio}
Procedim a formalitzar el raonament, identificant les seves proposicions
simples:

\begin{description}
\item[$p=$] `Miquel toca la piano'

\item[$q=$] `Maria toca la guitarra'

\item[$r=$] `Carla toca el viol\'{\i}'

\item[$s=$] `Jaume toca l'acordi\'{o}'
\end{description}

Llavors el raonament formalitzat \'{e}s%
\begin{equation*}
\begin{tabular}{ll}
$P_{1}:$ & $\lnot p\vee q$ \\
$P_{2}:$ & $\lnot r\longrightarrow\lnot q$ \\
$P_{3}:$ & $p\wedge\lnot r$ \\ \hline
$C:$ & $s$%
\end{tabular}
\
\end{equation*}

Per un costat, utilitzant les regles d'infer\`{e}ncia, s'obt\'{e}:%
\begin{equation*}
\begin{tabular}{lll}
1. & $\lnot p\vee q$ & $P_{1}$ \\
2. & $\lnot r\longrightarrow\lnot q$ & $P_{2}$ \\
3. & $p\wedge\lnot r$ & $P_{3}$ \\
4. & $p$ & EC 3 \\
5. & $p\vee s$ & ID 4 \\
6. & $\lnot r$ & EC 3 \\
7. & $\lnot q$ & MP (2,6) \\
8. & $\lnot p$ & ED (1,7) \\ \hline
9. & $s$ & ED (5,8)%
\end{tabular}
\
\end{equation*}
Per un altre, es t\'{e} tamb\'{e}%
\begin{equation*}
\begin{tabular}{lll}
1. & $\lnot p\vee q$ & Hip\`{o}tesis \\
2. & $\lnot r\longrightarrow\lnot q$ & Hip\`{o}tesis \\
3. & $p\wedge\lnot r$ & Hip\`{o}tesis \\
4. & $p$ & EC 3 \\
5. & $p\vee\lnot s$ & ID 4 \\
6. & $\lnot r$ & EC 3 \\
7. & $\lnot q$ & MP (2,6) \\
8. & $\lnot p$ & ED (1,7) \\ \hline
9. & $\lnot s$ & ED (5,8)%
\end{tabular}
\
\end{equation*}
Observem que hem dedu\"{\i}t $s$ en el primer cas, i $\lnot s$, en el segon.
Com a conseq\"{u}encia, podem dedu\"{\i}r $s\wedge\lnot s$, que \'{e}s una
contradicci\'{o}. Per tant, el conjunt de premisses \'{e}s inconsistent.

Tamb\'{e} podem provar la inconsist\`{e}ncia de les premisses comprovant si
\'{e}s possible que hi hagi una interpretaci\'{o} que faci totes les
premisses vertaderes:
\begin{equation*}
\begin{tabular}{|l|l|l|l|l|l|l|l|l|l|l|l|l|}
$\lnot$ & $p$ & $\vee$ & $q$ & $\lnot$ & $r$ & $\longrightarrow$ & $\lnot$ &
$q$ & $p$ & $\wedge$ & $\lnot$ & $r$ \\ \hline
&  & V &  &  &  & V ! &  &  &  & V &  &  \\ \cline{7-7}
F & V &  & V & V & F & F ! & F & V & V &  & V & F \\ \cline{7-7}
\end{tabular}
\
\end{equation*}
Observem que no \'{e}s possible excepte que acceptem que una proposici\'{o}
sigui vertadera i falsa alhora, cosa que no pot ser.
\end{solucio}

\bigskip

Per acabar aquesta secci\'{o}, volem tractar els raonaments que a vegades es
consideren com a v\`{a}lids i no ho s\'{o}n, sovint coneguts com a \textbf{%
fal\textperiodcentered l\`{a}cies}. En primer lloc tractarem els que tenen a
veure amb un mal \'{u}s de les regles d'infer\`{e}ncia. S\'{o}n arguments
que poden semblar v\`{a}lids a primera vista, per\`{o} que s\'{o}n
fal\textperiodcentered l\`{a}cies. Considerem el seg\"{u}ent raonament: "Si
l'Albert menja un bon dinar, beur\`{a} una cervesa. L'Albert va beure una
cervesa i, per tant, va menjar un bon dinar". Podr\'{\i}em pensar que aquest
raonament \'{e}s v\`{a}lid per MP, per\`{o} no \'{e}s aix\'{\i}. En efecte,
si formalitzem l'argument es t\'{e}:

\begin{description}
\item[$p=$] `Albert menja un bon dinar'

\item[$q=$] `Albert beur\`{a} una cervesa'
\end{description}

i, aleshores,%
\begin{equation*}
\begin{tabular}{ll}
$P_{1}:$ & $p\longrightarrow q$ \\
$P_{2}:$ & $q$ \\ \hline
$C:$ & $p$%
\end{tabular}
\end{equation*}
\'{E}s evident que no poden aplicar MP i, a part, no \'{e}s v\`{a}lid, perqu%
\`{e} pot haver begut una cervesa sense haver tingut un bon sopar. De fet,
si fem la interpretaci\'{o} seg\"{u}ent%
\begin{equation*}
\begin{tabular}{lll||l||l}
$p$ & $\longrightarrow$ & $q$ & $q$ & $p$ \\ \hline
F & V & V & V & F%
\end{tabular}
\end{equation*}
comprovem el que hem dit.

\bigskip

Considerem ara el seg\"{u}ent raonament: "Si en Joan va en bicicleta a
primera hora del mat\'{\i}, aleshores menja un bon esmorzar. En Joan no va
en bicicleta. Per tant, en Joan no menja un bon esmorzar". Podr\'{\i}em
pensar que aquest raonament \'{e}s v\`{a}lid per MT, per\`{o} no \'{e}s aix%
\'{\i}. En efecte, si formalitzem l'argument es t\'{e}:

\begin{description}
\item[$p=$] `Joan va en bicicleta a primera hora del mat\'{\i}'

\item[$q=$] `Joan menja un bon esmorzar'
\end{description}

i, aleshores,%
\begin{equation*}
\begin{tabular}{ll}
$P_{1}:$ & $p\longrightarrow q$ \\
$P_{2}:$ & $\lnot p$ \\ \hline
$C:$ & $\lnot q$%
\end{tabular}
\end{equation*}
\'{E}s evident que no poden aplicar MT i, a part, no \'{e}s v\`{a}lid, perqu%
\`{e} pot haver menjat un bon esmorzar sense haver anat en bicicleta. De
fet, si fem la interpretaci\'{o} seg\"{u}ent%
\begin{equation*}
\begin{tabular}{lll||l|l||ll}
$p$ & $\longrightarrow$ & $q$ & $\lnot$ & $p$ & $\lnot$ & $q$ \\ \hline
F & V & F & V & F & F & V%
\end{tabular}
\end{equation*}
comprovem el que hem dit.

\bigskip

En segon lloc, hi ha raonaments que no s\'{o}n v\`{a}lids perqu\`{e} es fan
suposicions que no estan justificades per les premisses. Per exemple,
considerem el seg\"{u}ent raonament: "Si en D\'{\i}dac t\'{e} febre,
esternuda molt. Per tant, D\'{\i}dac esternuda molt". \'{E}s clar que podr%
\'{\i}em aplicar MP per concloure que en D\'{\i}dac esternuda molt, per\`{o}
no sabem per les hip\`{o}tesis que en D\'{\i}dac t\'{e} febre.

Aquests exemples de fal\textperiodcentered l\`{a}cies poden semblar molt
senzills, per\`{o} quan el raonament \'{e}s m\'{e}s extens i complex, i no
s'escriu sin\'{o} que es parla, a vegades aquests errors l\`{o}gics passen
desapercebuts. Cal doncs tenir cura de no cometre'ls.

\section{Quantificadors}

En la secci\'{o} `Variables i constants' hem escrit
\begin{equation*}
x+y=y+x\text{,}
\end{equation*}
reconeixent en aquesta expressi\'{o} la llei commutativa de l'aritm\`{e}%
tica. Si assumim que les variables designen nombres reals, aquest enunciat
afirma que tots els nombres reals compleixen aquesta propietat i, en
general, tots els enunciats d'aquest tipus que afirmen que objectes
arbitraris d'una classe compleixen una determinada propietat, s'anomenen
\textbf{proposicions universals}.

A vegades tamb\'{e} ens trobem enunciats com%
\begin{equation*}
x^{2}-3x+2=0
\end{equation*}
reconeixent en aquesta expressi\'{o} que hi ha dos nombres que compleixen
aquesta equaci\'{o}. En aquests casos diem que s\'{o}n \textbf{proposicions
existencials} perqu\`{e} afirmen l'exist\`{e}ncia d'objectes (en aquest cas,
els nombres 1 i 2) que compleixen una certa propietat.

\bigskip

Fins ara, mitjan\c{c}ant els s\'{\i}mbols l\`{o}gics $\wedge,\vee
,\lnot,\longrightarrow$ i $\longleftrightarrow$, hem vist com podem
formalitzar molts enunciats de les matem\`{a}tiques i com podem con\`{e}ixer
que dos enunciats diferents tenen el mateix significat si s\'{o}n
equivalents. Tamb\'{e} hem vist com la formalitzaci\'{o} ens pot ajudar a
comprendre l'estructura l\`{o}gica dels raonaments, i com els raonaments s%
\'{o}n v\`{a}lids amb l'ajut d'implicacions l\`{o}giques o regles d'infer%
\`{e}ncia. Per\`{o} amb tot aix\`{o} no n'hi ha prou per captar la totalitat
del significat de molts enunciats de les matem\`{a}tiques. A l'inici
d'aquesta secci\'{o} hem vist com hi ha uns enunciats molt comuns a matem%
\`{a}tiques que cal estudiar amb m\'{e}s detall.

\bigskip

En efecte, imaginem que estem tractant un conjunt infinit $X$ de nombres
enters. Considerem ara l'enunciat `Tots els elements de $X$ s\'{o}n senars'.
Si volem formalitzar aquest enunciat, haur\'{\i}em d'escriure%
\begin{equation*}
P(x_{1})\wedge P(x_{2})\wedge P(x_{3})\wedge\cdots\text{,}
\end{equation*}
on $P(x)$ \'{e}s el predicat `$x$ \'{e}s senar'. I si volem formalitzar
l'enunciat `Hi ha almenys un element de $X$ que \'{e}s senar', haur\'{\i}em
d'escriure%
\begin{equation*}
P(x_{1})\vee P(x_{2})\vee P(x_{3})\vee\cdots\text{.}
\end{equation*}
El problema d'aquestes dues expressions \'{e}s que no s'acaben mai. Per
superar-lo introduirem dos nous s\'{\i}mbols $\exists$ i $\forall$. El s%
\'{\i}mbol $\exists$, anomenat \textbf{quantificador existencial}, est\`{a}
en lloc de frases com `existeix' o `hi ha un'. D'aquesta manera l'enunciat
`Hi ha elements de $X$ que s\'{o}n senars' podem escriure'l aix\'{\i}%
\begin{equation*}
\exists x\in X,P(x)\text{.}
\end{equation*}

El s\'{\i}mbol $\forall$, anomenat \textbf{quantificador universal}, est\`{a}
en lloc de frases com `per a tots' o `per a cadascun'. D'aquesta manera
l'enunciat `Tots els elements de $X$ s\'{o}n senars' podem escriure'l aix%
\'{\i}%
\begin{equation*}
\forall x\in X,P(x)\text{.}
\end{equation*}

\bigskip

Els exemples que hem donat a l'inici d'aquesta secci\'{o} podem ara
expressar-los d'aquesta manera:
\begin{equation*}
\left( \forall x,y\right) \left( x,y\in\mathbb{R}\longrightarrow
x+y=y+x\right)
\end{equation*}
que constitueix una proposici\'{o} universal, i l'altre%
\begin{equation*}
\left( \exists x\right) \left( x\in\mathbb{R\wedge~}x^{2}-3x+2=0\right)
\end{equation*}
que \'{e}s una proposici\'{o} exist\`{e}ncial.

\begin{observacio}
En teoria de conjunts la proposici\'{o} $\exists x\in X,P(x)$ podem
escriure-la d'aquesta manera:%
\begin{equation*}
\left\{ x\in X:P(x)\right\} \neq\emptyset\text{,}
\end{equation*}
indicant que el conjunt dels elements que compleixen la propietat $P$ \'{e}s
no vuit; i la proposici\'{o} $\forall x\in X,P(x)$, com%
\begin{equation*}
\left\{ x\in X:P(x)\right\} =X\text{,}
\end{equation*}
indicant que el conjunt dels elements que compleixen la propietat $P$ \'{e}s
tot el conjunt $X$.
\end{observacio}

\bigskip

L'enunciat `el quadrat d'un nombre real m\'{e}s petit o igual que 4' \'{e}s
un predicat. Si $x$ \'{e}s un nombre real, aleshores `$x^{2}\leq4$' \'{e}s
el predicat abreujat. L'enunciat `$x^{2}\leq4$' no \'{e}s una proposici\'{o}
perqu\`{e} no podem afirmar que sigui vertadera o falsa llevat que assignem
a la varible lliure $x$ un valor real. Si representem per $P(x)$ aquest
predicat, llavors $P(1)$ \'{e}s una proposici\'{o} vertadera, en canvi, $%
P(3) $ \'{e}s falsa.

Hi ha una altra manera de fer que un predicat es converteixi en una proposici%
\'{o}. Es fent que les seves variables lliures quedin lligades per
quantificadors. Considerem que $P(x)$ el predicat anterior, aleshores `$%
\forall x\in\mathbb{R}$, $P(x)$' \'{e}s una proposici\'{o} falsa, i en
canvi, `$\exists x\in\mathbb{R}$, $P(x)$' \'{e}s vertadera.

\bigskip

Podem construir enunciats amb m\'{e}s d'un quantificador. Per exemple,
considerem el predicat $Q(m,n)=$ `$m<n$', on $m,n\in\mathbb{Z}$. Llavors, la
proposici\'{o} `$\forall m\in\mathbb{Z}~\exists n\in\mathbb{Z},m<n$',
escrita en catal\`{a} com `per a tot nombre enter $m$ existeix algun altre
enter $n$ tal que $m<n$', \'{e}s vertadera. En efecte, donat qualsevol enter
$m$, existeix $n=m+1$, tamb\'{e} \'{e}s enter i compleix la condici\'{o} $%
m<n $.

\bigskip

L'ordre dels quantificadors \'{e}s molt important perqu\`{e} per exemple,
canviant l'ordre de l'enunciat anterior, o sigui `$\exists m\in\mathbb{Z}%
~\forall n\in\mathbb{Z},m<n$' es t\'{e} una proposici\'{o} falsa. \'{E}s
important tamb\'{e} observar el paper que fan les variables lliures en una
proposici\'{o} que tamb\'{e} en t\'{e} de lligades. Amb relaci\'{o} al
predicat anterior, considerem la proposici\'{o} `$\exists m\in\mathbb{Z},$ $%
m<n$', qu\`{e} \'{e}s vertadera perqu\`{e} nom\'{e}s cal prendre $m=n-1$. Si
ara posem $x$ en lloc de $m$ en la proposici\'{o}, es t\'{e} `$\exists x\in%
\mathbb{Z},$ $x<n$' que t\'{e} el mateix significat que l'anterior; de fet,
els valors de $m$ o $x$ depenen sempre del valor que assignem a la variable
lliure $n$, i, per tant, si aquesta expressi\'{o} form\'{e}s part d'una
expressi\'{o} m\'{e}s llarga, aquesta \'{u}ltima no canviaria en el seu
significat. En canvi, si substitu\"{\i}m la variable lliure $n$ per una
altra $q$, la proposici\'{o} `$\exists m\in\mathbb{Z},$ $m<q$' pot canviar
el significat d'una expressi\'{o} m\'{e}s llarga de la qual en forma part,
doncs ara $m$ dep\`{e}n de $q$ i no de $n$.

\bigskip

Ara volem veure la negaci\'{o} de proposicions amb quantificadors. Per
exemple, considerem el predicat $Q=$ `tenir els ulls blaus'. Llavors, si
suposem que la variable $x$ t\'{e} per domini el conjunt de totes les
persones, l'enunciat `Totes les persones tenen els ulls blaus' s'escriu aix%
\'{\i}: $\left( \forall x\right) Q(x)$. La negaci\'{o} d'aquest enunciat
\'{e}s `No totes les persones tenen els ulls blaus' que s'escriu com $%
\lnot\left( \forall x\right) Q(x)$. \'{E}s clar que aquest \'{u}ltim
enunciat tamb\'{e} el podem escriure com `Hi ha alguna persona que no te els
ulls blaus', que simb\`{o}licament ho escrivim aix\'{\i}: $\left( \exists
x\right) \lnot Q\left( x\right) $.

Considerem ara el cas d'una proposici\'{o} existencial. Considerem
l'enunciat `Existeix una soluci\'{o} real en l'equaci\'{o} $x^{3}+x=0$'. Si
suposem que el domini de la variable $x$ s\'{o}n els nombres reals,
aleshores escrivim aquesta proposici\'{o} aix\'{\i}: $\left( \exists
x\right) \left( x^{3}+x=0\right) $. La negaci\'{o} d'aquesta proposici\'{o}
\'{e}s `Cap nombre real \'{e}s soluci\'{o} de l'equaci\'{o} $x^{3}+x=0$',
que s'escriu aix\'{\i}: $\left( \forall x\right) \left( x^{3}+x\neq0\right) $%
.

\bigskip

Podem resumir els dos casos anteriors d'aquesta manera: Si $P(x)$ \'{e}s un
predicat i el domini de la variable $x$ \'{e}s el conjunt $U$, aleshores es
compleixen les seg\"{u}ents equival\`{e}ncies:

\begin{itemize}
\item $\lnot\left( \forall x\in U,P(x)\right) \Longleftrightarrow\exists
x\in U,\lnot P\left( x\right) $

\item $\lnot\left( \exists x\in U,P\left( x\right) \right)
\Longleftrightarrow\forall x\in U,\lnot P\left( x\right) $
\end{itemize}

\bigskip

A difer\`{e}ncia de les equival\`{e}ncies discutides a la secci\'{o} \ref{2}%
, no podem utilitzar taules de veritat per verificar aquestes equival\`{e}%
ncies, tot i que s\'{o}n certes, basant-se en els significats dels
quantificadors. Podem utilitzar les equival\`{e}ncies anteriors per negar
proposicions amb m\'{e}s d'un quantificador. Per exemple, suposem que $f$
\'{e}s una funci\'{o} real de variable i considerem l'enunciat `Per a cada
nombre real $x$, existeix un nombre real $y$ tal que $f(x)=y$'; simb\`{o}%
licament s'escriu com $\left( \forall x\right) \left( \exists y\right)
\left( f(x)=y\right) $. La seva negaci\'{o} \'{e}s%
\begin{align*}
\lnot\left( \left( \forall x\right) \left( \exists y\right) \left(
f(x)=y\right) \right) & \Longleftrightarrow\left( \exists x\right) \left(
\lnot\left( \exists y\right) \left( f(x)=y\right) \right) \\
& \Longleftrightarrow\left( \exists x\right) \left( \left( \forall y\right)
\left( f(x)\neq y\right) \right) \text{,}
\end{align*}
i, reformulant aquesta \'{u}ltima expressi\'{o} en catal\`{a} es t\'{e}:
`Existeix un nombre real $x$ tal que per a tot nombre real $y$ es compleix $%
f(x)\neq y$'.

\bigskip

Finalment, passem a les regles d'infer\`{e}ncia amb quantificadors. Hi ha
quatre regles d'infer\`{e}ncia d'aquest tipus i, tot i que el seu \'{u}s
requereix una mica m\'{e}s de cura que les regles d'infer\`{e}ncia de la
secci\'{o} \ref{5}, s'utilitzen amb el mateix prop\`{o}sit, que \'{e}s
mostrar la validesa dels raonaments l\`{o}gics.

\begin{description}
\item[EQU:]
\begin{tabular}{l}
$\left( \forall x\in U\right) P(x)$ \\ \hline
$P(a)$%
\end{tabular}
(Eliminaci\'{o} Quantificador Universal)

on $a$ \'{e}s qualsevol element de $U$.

\item[IQU:]
\begin{tabular}{l}
$P(b)$ \\ \hline
$\left( \forall x\in U\right) P(x)$%
\end{tabular}
(Introducci\'{o} Quantificador Universal)

on $b$ \'{e}s un element arbitrari de $U$.

\item[EQE:]
\begin{tabular}{l}
$\left( \exists x\in U\right) P(x)$ \\ \hline
$P(c)$%
\end{tabular}
(Eliminaci\'{o} Quantificador Existencial)

on $c$ \'{e}s un element de $U$ per\`{o} el s\'{\i}mbol `$c$' no ha d'haver
aparagut en el argument abans.

\item[IQE:]
\begin{tabular}{l}
$P(d)$ \\ \hline
$\left( \exists x\in U\right) P(x)$%
\end{tabular}
(Introducci\'{o} Quantificador Existencial)

on $d$ \'{e}s un element de $U$.
\end{description}

En la regla EQE volem destacar el fet que $c$ no fa refer\`{e}ncia a cap s%
\'{\i}mbol que ja s'ha utilitzat en l'argument. Per tant, hem de triar una
lletra nova, en lloc d'una que ja s'utilitzi per a una altra cosa.

\bigskip

Un exemple de raonament l\`{o}gic senzill que implica quantificadors \'{e}s
el seg\"{u}ent:

\begin{quote}
`A cada gat simp\`{a}tic i intel\textperiodcentered ligent li agrada el
fetge picat. Tots els gats siamesos s\'{o}n agradables. Hi ha un gat siam%
\`{e}s al qual no li agrada el fetge picat. Per tant, hi ha un gat est\'{u}%
pid.'
\end{quote}

Formalitzem els enunciats del raonament. Per facilitar l'escriptura,
considerem que la variable $x$ t\'{e} per domini el conjunt de tots els
gats. Denotem per $S$, $I$, $F$ i $G$ els predicats `\'{e}s simp\`{a}tic o
agradable', $I$, `\'{e}s intel\textperiodcentered ligent', $F$, `agrada el
fetge picat', i $G$, `\'{e}s gat siam\`{e}s', respectivament. Aleshores, el
raonament podem simbolitzar-lo d'aquesta manera:%
\begin{equation*}
\begin{tabular}{ll}
$P_{1}:$ & $\left( \forall x\right) \left( S(x)\wedge I(x)\longrightarrow
F(x)\right) $ \\
$P_{2}:$ & $\left( \forall x\right) \left( G(x)\longrightarrow S(x)\right) $
\\
$P_{3}:$ & $\left( \exists x\right) \left( G(x)\wedge\lnot F(x)\right) $ \\
\hline
$C:$ & $\left( \exists x\right) \lnot I(x)$%
\end{tabular}
\end{equation*}

Fem ara la deducci\'{o} per provar la seva validesa:

\begin{equation*}
\begin{tabular}{lll}
1. & $\left( \forall x\right) \left( S(x)\wedge I(x)\longrightarrow
F(x)\right) $ & $P_{1}$ \\
2. & $\left( \forall x\right) \left( G(x)\longrightarrow S(x)\right) $ & $%
P_{2}$ \\
3. & $\left( \exists x\right) \left( G(x)\wedge\lnot F(x)\right) $ & $P_{3}$
\\
4. & $G(a)\wedge\lnot F(a)$ & EQE 3 \\
5. & $G(a)\longrightarrow S(a)$ & EQU 2 \\
6. & $S(a)\wedge I(a)\longrightarrow F(a)$ & EQU 1 \\
7. & $G(a)$ & EC 4 \\
8. & $\lnot F(a)$ & EC 4 \\
9. & $S(a)$ & MP (5,7) \\
10. & $\lnot\left( S(a)\wedge I(a)\right) $ & MT (6,8) \\
11. & $\lnot S(a)\vee\lnot I(a)$ & DM 10 \\
12. & $\lnot\lnot S(a)$ & DN 9 \\
13. & $\lnot I(a)$ & ED (11,12) \\ \hline
14. & $\left( \exists x\right) \lnot I(x)$ & IQE 13%
\end{tabular}
\end{equation*}

Tinguem en compte que a la l\'{\i}nia (4) hem escollit alguna lletra com `$a$%
'\ que no s'utilitzava abans d'aquesta l\'{\i}nia, perqu\`{e} apliquem la
regla EQE.

\bigskip

\begin{exemple}
Considereu el raonament formalitzat seg\"{u}ent:%
\begin{equation*}
\begin{tabular}{ll}
$P_{1}:$ & $\left( \exists x\in U\right) \left( P(x)\wedge Q(x)\right) $ \\
$P_{2}:$ & $\left( \exists x\in U\right) M(x)$ \\ \hline
$C:$ & $\left( \exists x\in U\right) \left( M(x)\wedge Q(x)\right) $%
\end{tabular}
\
\end{equation*}
Es proposa la prova seg\"{u}ent per a provar la seva validessa, \'{e}s
correcte?%
\begin{equation*}
\begin{tabular}{lll}
1. & $\left( \exists x\in U\right) \left( P(x)\wedge Q(x)\right) $ & $P_{1}$
\\
2. & $\left( \exists x\in U\right) M(x)$ & $P_{2}$ \\
3. & $P(a)\wedge Q(a)$ & EQE 1 \\
4. & $Q(a)$ & EC 3 \\
5. & $M(a)$ & EQE 2 \\
6. & $Q(a)\wedge M(a)$ & IC (4,5) \\
7. & $\left( \exists x\in U\right) \left( M(x)\wedge Q(x)\right) $ & IQE 6%
\end{tabular}
\end{equation*}
$\ $
\end{exemple}

\begin{solucio}
No \'{e}s correcte, perqu\`{e} en el pas 5 el s\'{\i}mbol `$a$' ha aparagut
abans.
\end{solucio}

\section{Definicions i la identitat}

Quan, per exemple, en el conjunt dels nombres enters diem que $x$ \'{e}s
divisible per $y$ si i nom\'{e}s si existeix un nombre enter $k$ tal que $%
x=ky$, simb\`{o}licament%
\begin{equation*}
x\mid y\Longleftrightarrow\left( \exists z\right) \left( z\in \mathbb{Z}%
\wedge x=ky\right) \text{,}
\end{equation*}
volem establir una definici\'{o} del s\'{\i}mbol `$\mid$' donant el seu
significat amb ajuda de termes ja coneguts com `nombre enter', `producte' (o
`quocient'). De la mateixa manera, en el conjunt dels nombres reals diem que
$x\leq y$ si i nom\'{e}s si no \'{e}s el cas que $x>y$, formalment,
\begin{equation*}
\left( \forall x,y\right) \left( x\leq y\Longleftrightarrow\lnot\left(
x>y\right) \right) \text{,}
\end{equation*}
establim la definici\'{o} del s\'{\i}mbol `$\leq$' donant el seu significat
amb ajuda del predicat ja conegut `$>$'. En aquest \'{u}ltim cas, per
exemple, podem substituir en qualsevol proposici\'{o} el predicat `$x\leq y$%
' pel predicat `no \'{e}s el cas que $x>y$' sense que canvi el seu
significat.

\bigskip

No volem donar una definici\'{o} precisa de com cal construir correctament
una definici\'{o}, per\`{o} tota definici\'{o} pot adoptar la forma d'una
equival\`{e}ncia l\`{o}gica; el membre de l'esquerra ha de contenir all\`{o}
que volem definir i, el de la dreta, all\`{o} que est\`{a} ja ben definit o
que el seu significat sigui comprensible immediatament, per\`{o} mai hi ha
d'apar\`{e}ixer el que volem definir.

\bigskip

La noci\'{o} d'identitat present en molts enunciats com "$x$ \'{e}s id\`{e}%
ntic a $y$', `$x$ \'{e}s el mateix que $y$', o senzillament, `$x$ \'{e}s
igual a $y$' i que abreugem simb\`{o}licament per $x=y$ es pot definir des
del punt de vista l\`{o}gic amb l'ajut de la llei de Leibniz. D'acord amb
aquesta llei, $x=y$ si i nom\'{e}s si $x$ i $y$ tenen en com\'{u} totes les
seves propietats. Observeu que aquest enunciat no pertany a la l\`{o}gica
proposicional perqu\`{e} s'hauria de fer \'{u}s d'un quantificador sobre una
variable que designa propietats d'objectes i no objectes com hem fet fins
ara. Formalment, es t\'{e}%
\begin{equation*}
x=y\Longleftrightarrow\left( \forall P\right) \left( Px\longleftrightarrow
Py\right)
\end{equation*}
on $Px$ designa una propietat que compleix $x$. Tot i aix\`{o}, a matem\`{a}%
tiques \'{e}s m\'{e}s com\'{u} interpretar la relaci\'{o} d'identitat com
una relaci\'{o} d'igualtat o de congru\`{e}ncia dins d'un domini determinat
d'objectes. Per exemple, en \`{a}lgebra diem que dos polinomis $A(x)$ i $%
B(x) $ s\'{o}n id\`{e}ntitics si i nom\'{e}s si $A(x)$ i $B(x)$ tenen el
mateix grau i els coeficients dels monomis del mateix grau de cadascun dels
polinomis s\'{o}n iguals. \'{E}s evident que quan $x=y$ es vol significar
que en tota proposici\'{o} on aparegui $x$ pot substituir-se per $y$ si \'{e}%
s necessari, i viceversa.

De la definici\'{o} que hem donat s'obt\'{e} de manera bastant evident que
la relaci\'{o} d'igualtat compleix les tres propietats seg\"{u}ents:

\begin{description}
\item[Reflexiva:] Tot objecte \'{e}s igual a si mateix: $x=x$.

\item[Sim\`{e}trica:] Si $x=y$, llavors $y=x$.

\item[Transitiva:] Si $x=y$ i $y=z$, llavors $x=z$.
\end{description}

Afegint al llenguatge de la l\`{o}gica proposicional el s\'{\i}mbol
d'igualtat, els quantificadors i variables per referir-nos a predicats s'obt%
\'{e} el llenguatge de la \textbf{l\`{o}gica de predicats} o tamb\'{e}
anomenada \textbf{l\`{o}gica de primer ordre}. Si afegim tamb\'{e} les
regles d'infer\`{e}ncia dels quantificadors, podem tractar un punt essencial
de les matem\`{a}tiques que s\'{o}n les demostracions. \'{E}s clar que no
hem tractat gens la l\`{o}gica de predicats, per\`{o} volem observar que
compleix el teorema de completesa, per\`{o} no \'{e}s decidible si algun
dels seus predicats t\'{e} m\'{e}s d'una variable. Tots aquests resultats
poden trobar-se en qualsevol llibre de l\`{o}gica de predicats avan\c{c}at.

\section{Demostracions}

Entrem a la secci\'{o} m\'{e}s important d'aquest document que s\'{o}n les
demostracions a matem\`{a}tiques. \'{E}s important ser conscients de les
raons per les quals hem fet fins ara una petita introducci\'{o} a la l\`{o}%
gica. N'hi ha tres raons de molt significatives:

\begin{enumerate}
\item En primer lloc, les taules de veritat que hem estudiat ens indiquen
els significats exactes de les paraules `i', `o', `no', `si ... llavors ...'
i `... si i nom\'{e}s si ...'. Per exemple, sempre que fem servir o llegim
la construcci\'{o} `Si ..., llavors...' en un context matem\`{a}tic, la l%
\`{o}gica ens indica exactament qu\`{e} es vol dir.

\item En segon lloc, les regles d'infer\`{e}ncia proporcionen un cam\'{\i}
pel qual podem construir nova informaci\'{o} (proposicions) a partir
d'informaci\'{o} coneguda.

\item Finalment, les equival\`{e}ncies l\`{o}giques ens ajuden a canviar
correctament certes proposicions en unes altres proposicions amb el mateix
significat per\`{o} molt m\'{e}s \'{u}tils.
\end{enumerate}

En resum, la l\`{o}gica ens ajuda a entendre els significats dels enunciats
i tamb\'{e} a construir de nous. De fet \'{e}s el llenguatge b\`{a}sic que
ens permet escriure i comprendre b\'{e} els enunciats matem\`{a}tics. Per%
\`{o}, malgrat el seu paper fonamental, el seu lloc queda en un segon pla.
Els s\'{\i}mbols $\wedge,\vee,\lnot,\longrightarrow,\longleftrightarrow,%
\forall$ i $\exists$ poques vegades s'escriuen a la pissarra o en els
llibres de text. Tot i aix\`{o}, hem de ser conscients dels seus significats
constantment; quan llegim o escrivim una frase relacionada amb les matem\`{a}%
tiques, hem d'analitzar-la amb aquests s\'{\i}mbols, sigui mentalment o
sobre paper en brut, per tal d'entendre o comunicar b\'{e} el seu
significat, que ha de ser sempre vertader i inequ\'{\i}voc.

\bigskip

A matem\`{a}tiques un raonament l\`{o}gic \'{e}s senzillament l'enunciat
d'un teorema. La justificaci\'{o} que fa que un teorema sigui v\`{a}lid \'{e}%
s la seva prova o demostraci\'{o}. Quan passem a la construcci\'{o} de
demostracions matem\`{a}tiques, ens centrem en el contingut matem\`{a}tic de
les proposicions implicades en la prova i no ens referim expl\'{\i}citament
a les regles d'infer\`{e}ncia l\`{o}gica discutides en les seccions
anteriors; fer-ho seria una distracci\'{o} de les q\"{u}estions matem\`{a}%
tiques. Tampoc utilitzem la notaci\'{o} l\`{o}gica de les seccions
anteriors. Tot i aix\`{o}, farem servir les regles d'infer\`{e}ncia de
manera impl\'{\i}cita tot el temps, no oblidem que \'{e}s el marc sobre el
qual es basa tot.

Sovint passa que la nostra intu\"{\i}ci\'{o} ens indica el que \'{e}s
important, el qu\`{e} creiem que pot ser cert, el qu\`{e} hem de provar despr%
\'{e}s i m\'{e}s coses. Desafortunadament, els objectes matem\`{a}tics solen
ser tan complicats o abstractes que la nostra intu\"{\i}ci\'{o} falla. Per
aix\`{o} constru\"{\i}m demostracions, per verificar que una afirmaci\'{o}
determinada que ens sembla intu\"{\i}tivament certa \'{e}s realment certa,
nom\'{e}s aix\'{\i} podem avan\c{c}ar a matem\`{a}tiques. Finalment, afegir
tamb\'{e} que ens interessa aprendre a fer demostracions perqu\`{e} ens
ajuda a entendre les nocions o fets relacionats amb el resultat que es vol
demostrar.

\bigskip

Quan es t\'{e} l'enunciat d'un teorema, primer s'ha de comprendre el que
significa, segon s'ha de saber escriure amb rigor (seguint el llenguatge de
la l\`{o}gica) i finalment, amb l'ajut d'altres nocions, proposicions o
teoremes, s'ha d'explorar els possibles camins per portar a terme la seva
demostraci\'{o}. Per construir la prova, el primer que cal fer \'{e}s
especificar en rigor el qu\`{e} se suposa que es compleix, que anomenem les
\textbf{hip\`{o}tesis, }i despr\'{e}s el qu\`{e} s'intenta demostrar, que
se'n diu \textbf{tesi}. A continuaci\'{o}, s'ha d'escollir una estrat\`{e}%
gia per a la prova. La seg\"{u}ent etapa \'{e}s obtenir la prova, fent \'{u}%
s de l'estrat\`{e}gia escollida. Si no es pot idear una prova amb l'estrat%
\`{e}gia escollida, potser s'hauria d'intentar una altra. No hi ha una
manera fixa de trobar la prova; sempre requereix experimentar, jugar i
provar coses diferents. En els apartats seg\"{u}ents tractarem les maneres m%
\'{e}s importants de fer demostracions.

\subsection{Demostracions directes}

Molts teoremes tenen la forma $A\Longrightarrow B$. El cam\'{\i} m\'{e}s
senzill per provar $A\Longrightarrow B$ \'{e}s fer-ho directament: suposem
que $A$ \'{e}s vertadera i hem d'arribar a dedu\"{\i}r $B$ mitjan\c{c}ant
una seq\"{u}\`{e}ncia de passos que comen\c{c}a en $A$ i acaba en $B$.
Aquest tipus de prova es diu \textbf{prova directa} per distinguir-la
d'altres m\`{e}todes de demostraci\'{o}.

\begin{exemple}
Reformuleu simb\`{o}licament cadascun dels teoremes seg\"{u}ents en la forma
$A\longrightarrow B$.

\begin{enumerate}
\item L'\`{a}rea d'un cercle de radi $r$ \'{e}s $\pi r^{2}$.

\item Donada una recta $r$ i un punt $P$ que no \'{e}s de $r$, hi ha
exactament una recta $s$ que passa pel punt $P$ i \'{e}s
paral\textperiodcentered lela a $r$.

\item En tot triangle $ABC$, els costats del qual s\'{o}n $a,b$ i $c$,
llavors es compleix%
\begin{equation*}
\frac{a}{\sin A}=\frac{b}{\sin B}=\frac{c}{\sin C}\text{.}
\end{equation*}

\item $e^{x}\cdot e^{y}=e^{x+y}$.
\end{enumerate}
\end{exemple}

\begin{solucio}
(1) Si $C(r)=$\textquotedblleft$C$ \'{e}s un cercle del pla de radi $r$%
\textquotedblright\ i $A(x)=$\textquotedblleft\ \`{a}rea de la figura $x$%
\textquotedblright, aleshores es t\'{e} $\forall r\left( C(r)\longrightarrow
A(C(r))=\pi r^{2}\right) $. De fet, si $r>0$, aleshores $C(r)=\left\{ \left(
x,y\right) \in\mathbb{R}^{2}:x^{2}+y^{2}=r^{2}\right\} \,$.

(2) Si $R(x)=$\textquotedblleft\'{e}s una recta del pla\textquotedblright, $%
P(x)=$\textquotedblleft\'{e}s un punt del pla\textquotedblright\ i $%
x\parallel y=$\textquotedblleft x \'{e}s paral\textperiodcentered lel a
y\textquotedblright, aleshores%
\begin{equation*}
\forall x,y\left( R(x)\wedge P(y)\longrightarrow\exists z\left( R(z)\wedge
y\in z\wedge z\parallel x\right) \right)
\end{equation*}

(3) Si $T(x,y,z)=$\textquotedblleft$T$ \'{e}s un triangle del pla de costats
$x,y,z$\textquotedblright\ i $A(x)=$\textquotedblleft angle oposat al costat
$x$ del triangle $T$\textquotedblright. Aleshores%
\begin{equation*}
\forall x,y,z\left( T(x,y,z)\longrightarrow\frac{x}{\sin A(x)}=\frac{y}{\sin
A(y)}=\frac{z}{\sin A(z)}\right) \text{.}
\end{equation*}
De fet, $T(x,y,z)=\left\{ (x,y,z)\in\mathbb{R}^{3}:x\leq y+z\right\} $.

(4) $\forall x,y\left( x,y\in\mathbb{R}\longrightarrow e^{x}\cdot
e^{y}=e^{x+y}\right) $.
\end{solucio}

\begin{exemple}
Considerem tres nombres enters qualssevol $a,b$ i $c$. Si $a$ divideix $b$ i
$b$ divideix $c$, prova que $a$ divideix $c$.
\end{exemple}

\begin{solucio}
Primer observem que a l'enunciat hi apareix un predicat "$x$ divideix $y$",
que simbolitzem per $x\mid y$. La seva definici\'{o} \'{e}s: $x\mid y$ sii
(abreujatura de "si i nom\'{e}s si") existeix $k\in\mathbb{Z}$ tal que $kx=y$%
. Ara hem d'escollir una estrategia per la prova. En aquest cas escollim la
prova directa: suposem que $a\mid b$ i $b\mid c$ (hipot\`{e}sis) i tenim que
veure que $a\mid c$ (tesi).

Si $a\mid b$ i $b\mid c$, llavors existeixen $k_{1},k_{2}\in\mathbb{Z}$ tals
que $k_{1}a=b$ i $k_{2}b=c$. D'aqu\'{\i}, per substituci\'{o} es t\'{e}: $%
k_{2}\left( k_{1}a\right) =\left( k_{1}k_{2}\right) a=c$. Per tant, existeix
$k=k_{1}k_{2}\in\mathbb{Z}$ tal que $ka=c$ i, com a conseq\"{u}\`{e}ncia, $%
a\mid c$.
\end{solucio}

\begin{exemple}
Provar que per a qualssevol nombres positius $x$ i $y$ es compleix que la
mitja aritm\`{e}tica \'{e}s m\'{e}s gran que la mitja geom\`{e}trica.
\end{exemple}

\begin{solucio}
Primer observem que a l'enunciat hi apareixen dos conceptes: mitja aritm\`{e}%
tica i geom\`{e}trica de dos nombres positius. La mitja aritm\`{e}tica i geom%
\`{e}trica de $x$ i $y$ s\'{o}n respectivament: $\sqrt{xy}$ i $\dfrac{x+y}{2}
$. Escollim una prova directa: com hip\`{o}tesi suposem $x,y\geq0$, tenim
que veure $\sqrt{xy}\leq\dfrac{x+y}{2}$ (tesi).

Si $x,y\geq0$ aleshores $\sqrt{x}$ i $\sqrt{y}$ existeixen. \'{E}s evident
que $(\sqrt{x}-\sqrt{y})^{2}\geq0$. Desemvolupant aquesta \'{u}ltima expressi%
\'{o} i passant l'arrel quadrada a l'altre costat de la desigualtat, es t%
\'{e}%
\begin{align*}
x-2\sqrt{xy}+y & \geq0 \\
x+y & \geq2\sqrt{xy}
\end{align*}
Finalment, dividind tots dos costats per $2$, es t\'{e} el resultat que
voliem:%
\begin{equation*}
\frac{x+y}{2}\geq\sqrt{xy}\text{.}
\end{equation*}
\end{solucio}

\subsection{Demostracions cap enrere}

Un altre m\`{e}tode de construir una demostraci\'{o} \'{e}s treballant
mentalment des de la tesi cap a les hip\`{o}tesis. Es diu \textbf{prova
pensada cap enrere} per distingir-la de les dem\'{e}s. Tot i aix\`{o}, la
prova finalment es construeix de forma directa. Mirem-ho constru\"{\i}nt la
prova cap enrere de la seg\"{u}ent proposici\'{o}:

\begin{exemple}
Provar que per a qualssevol nombres reals $x,y$ i $x<y$, es t\'{e} que $%
4xy<(x+y)^{2}$.
\end{exemple}

\begin{solucio}
L'hip\`{o}tesi \'{e}s $x,y\in\mathbb{R}$ i $x<y$, i la tesi, $4xy<(x+y)^{2}$%
. La prova cap enrere surt de la tesi, i, per tant, podem fer es seg\"{u}ent:%
\begin{align*}
4xy & <(x+y)^{2} \\
4xy & <x^{2}+2xy+y^{2} \\
0 & <x^{2}-2xy+y^{2} \\
0 & <(x-y)^{2}
\end{align*}
De la \'{u}ltima desig\"{u}altat, s'obt\'{e} $x-y\neq0$ i, per tant, $x\neq y
$. En particular, $x<y$. De fet, aix\`{o} que hem escrit s'havia d'haver
pensat mentalment perqu\`{e} la demostraci\'{o} real \'{e}s una prova
directa: Si $x<y$, llavors $x\neq y$ i, per tant, $x-y\neq0$. D'aqu\'{\i},
es t\'{e} $(x-y)^{2}>0$ i, per tant, $x^{2}-2xy+y^{2}>0$. Sumant a tots dos
costats $4xy$ es t\'{e}: $x^{2}+2xy+y^{2}>4xy$ i, per tant, $(x+y)^{2}>4xy$,
com voliem demostrar.
\end{solucio}

\subsection{Demostracions per contrarec\'{\i}proc}

Recordem que el contrarec\'{\i}proc de $A\longrightarrow B$ \'{e}s $\lnot
B\longrightarrow\lnot A$, i tamb\'{e} l'equival\`{e}ncia l\`{o}gica $\left(
A\longrightarrow B\right) \Longleftrightarrow\left( \lnot B\longrightarrow
\lnot A\right) $, anomenada llei del contrarec\'{\i}proc. Per aix\`{o}, si
volem provar $A\longrightarrow B$, podem fer-ho de manera equivalent, constru%
\"{\i}nt una prova directa de $\lnot B\longrightarrow\lnot A$. L'estrat\`{e}%
gia \'{e}s doncs pendre com hip\`{o}tesi que $B$ \'{e}s falsa i hem
d'arribar a que $A$ \'{e}s tamb\'{e} falsa (tesi). Aquest tipus de demostraci%
\'{o} es diu \textbf{prova per contrarec\'{\i}proc}. Veiem-ho amb un exemple:

\begin{exemple}
\label{3}Provar que per a qualsevol nombre enter $n$, si $n^{2}$ \'{e}s
parell, llavors $n$ \'{e}s parell.
\end{exemple}

\begin{solucio}
Volem constuir una prova per contrarec\'{\i}proc. Aleshores, les hip\`{o}%
tesis s\'{o}n que $n\in\mathbb{Z}$ i $n$ no \'{e}s parell. La tesi \'{e}s
que $n^{2}$ no \'{e}s parell. Si $n$ no \'{e}s parell, $n$ \'{e}s senar i,
per tant, existeix $k\in\mathbb{Z}$ i $n=2k+1$. Hem de veure que $n^{2}$ tamb%
\'{e} \'{e}s senar. En efecte, com $n=2k+1$, aleshores $n^{2}=\left(
2k+1\right) ^{2}=4k^{2}+4k+1=2\left( 2k^{2}+2k\right) +1$ i com $2k^{2}+2k\in%
\mathbb{Z}$, s'obt\'{e} que $n^{2}$ \'{e}s senar, com voliem demostrar.
\end{solucio}

\subsection{Demostracions reducci\'{o} a l'absurd}

El principi del tercer excl\`{o}s permet afirmar que una proposici\'{o} o es
vertadera o b\'{e} falsa. Aquest principi \'{e}s la base del seg\"{u}ent m%
\`{e}tode de demostraci\'{o}. En concret, si volem demostrar que una
proposici\'{o} $p$ \'{e}s certa, aquest m\`{e}tode proposar suposar el
contrari, o sigui que $\lnot p$ \'{e}s certa, i llavors per regles d'infer%
\`{e}ncia hem d'arribar a una contradicci\'{o} com per exemple $q\wedge\lnot
q$, on $q$ \'{e}s una proposici\'{o} qualsevol, per poder concluir que la
suposici\'{o} que $A$ \'{e}s falsa no \'{e}s possible i, en conseq\"{u}\`{e}%
ncia, $A$ \'{e}s certa.

A matem\`{a}tiques els teoremes tenen la forma $A\longrightarrow B$ i, per
tant, hem d'adaptar el raonament anterior a aquest cas. Recordem l'equival%
\`{e}ncia l\`{o}gica $\lnot\left( A\longrightarrow B\right)
\Longleftrightarrow A\wedge\lnot B$. Per tant, si volem construir una
demostraci\'{o} de $A\longrightarrow B$, hem de suposar que $\lnot\left(
A\longrightarrow B\right) $ \'{e}s certa. Aix\`{o} \'{e}s equivalent a
suposar que $A\wedge\lnot B$ \'{e}s certa, o sigui que $A$ \'{e}s certa i $B$
\'{e}s falsa i d'aqu\'{\i} hem de provar que s'arriba a una contradicci\'{o}%
. Una vegada s'ha arribat a la contradicci\'{o}, conclu\"{\i}m que $%
A\wedge\lnot B$ \'{e}s falsa i, per tant, $A\longrightarrow B$ \'{e}s
vertadera.

Aquest m\`{e}tode de demostraci\'{o} es coneix com a \textbf{prova per
reducci\'{o} a l'absurd}. L'hip\`{o}tesi \'{e}s $A\wedge\lnot B$ i la tesi
\'{e}s una contradicci\'{o} o absurd.

\begin{exemple}
Volem provar que $\sqrt{2}$ \'{e}s un nombre irracional.
\end{exemple}

\begin{solucio}
En primer lloc observem que l'enunciat no \'{e}s un condicional. Per\`{o},
recordem la definici\'{o} de un nombre racional diferent de zero: $a\neq0$
\'{e}s racional sii existeixen enters no nuls $m,n$ tals que $a=\dfrac{m}{n}$%
. Llavors, podem reformular l'enunciat com volem. En efecte, l'enunciat "$%
\sqrt{2}$ \'{e}s un nombre irracional" \'{e}s equivalent a la proposici\'{o}
seg\"{u}ent escrita simb\`{o}licament
\begin{equation*}
\left( \forall m,n\in\mathbb{Z}\right) \left( n\neq0\wedge m\neq 0\wedge%
\frac{m}{n}\neq\sqrt{2}\right) .
\end{equation*}

Per demostrar aquesta proposici\'{o} universal, considerem dos nombre enters
no nuls $m,n$. Hem demostrar que no \'{e}s possible que $\dfrac{m}{n}=\sqrt {%
2}$. L'estrat\`{e}gia ser\`{a} fer la prova per reducci\'{o} a l'absurd:
suposem $m$ i $n$ s\'{o}n enters no nuls i que $\dfrac{m}{n}=\sqrt{2}$ (hip%
\`{o}tesi), hem de trobar un contradicci\'{o} (tesi). No \'{e}s restrictiu
suposar que la fracci\'{o} $\dfrac{m}{n}$ \'{e}s irreductible; cas contrari,
escriur\'{\i}em la fracci\'{o} irreductible equivalent. Si $\dfrac{m}{n}=%
\sqrt{2}$, aleshores $m=\sqrt{2}n$ i, per tant, $m^{2}=2n^{2}$. D'aqu\'{\i},
dedu\"{\i}m que $m^{2}$ \'{e}s un enter parell. Llavors $m$ \'{e}s parell
com hem provat en l'exemple \ref{3}. Llavors, $m=2k$, on $k\in\mathbb{Z}$.
D'aqu\'{\i}, $m^{2}=4k^{2}=2n^{2}$ i, per tant, $n^{2}=2k^{2}$. Dedu\"{\i}m
que $n^{2}$ \'{e}s parell i, per tant, $n$ tamb\'{e}. Per\`{o}, aix\`{o}
\'{e}s una contradicci\'{o} perqu\`{e} aleshores la fracci\'{o} $\dfrac{m}{n}
$ \'{e}s reductible. En conseq\"{u}\`{e}ncia, no \'{e}s possible que $\dfrac{%
m}{n}=\sqrt{2}$, i, com $m$ i $n$ son arbitraris, $\sqrt{2}$ no \'{e}s
racional.

Aqu\'{\i} volem fer \`{e}mfasi en el fet seg\"{u}ent: Hem pogut reformular
l'enunciat pensant l\`{o}gicament i aix\`{o} a part d'entendre millor
l'enunciat ens ha perm\`{e}s fer una demostraci\'{o} per reducci\'{o} a
l'absurd.
\end{solucio}

\begin{exemple}
Volem provar que no existeix cap nombre real $x$ tal que $x^{2}+1=0$.
\end{exemple}

\begin{solucio}
Primer, escribim simb\`{o}licament l'enunciat: $\lnot\left( \exists x\in%
\mathbb{R}\right) \left( x^{2}+1=0\right) $. Veiem que \'{e}s la negaci\'{o}
d'una proposici\'{o} existencial. Fent \'{u}s de la seg\"{u}ent equival\`{e}%
ncia l\`{o}gica%
\begin{equation*}
\lnot\left( \exists x\in\mathbb{R}\right) \left( x^{2}+1=0\right)
\Longleftrightarrow\left( \forall x\in\mathbb{R}\right) \left(
x^{2}+1\neq0\right) \text{,}
\end{equation*}
i de la regla d'infer\`{e}ncia IQU:%
\begin{equation*}
\begin{tabular}{l}
$P(b)$ \\
$\left( \forall x\in U\right) P(x)$%
\end{tabular}
\ \text{,}
\end{equation*}
llavors nom\'{e}s cal provar que prenent un nombre real qualsevol $b$, es
compleix que $b^{2}+1\neq0$. Per\`{o}, aix\`{o} \'{e}s immediat perqu\`{e} $%
b^{2}\geq0$ i, per tant, $b^{2}+1\geq1$; en particular, $b^{2}+1\neq0$.
\end{solucio}

\subsection{Demostracions per contraexemple}

A matem\`{a}tiques tamb\'{e} trobem enunciats amb quantificadors que
rebutjant propietats. Per\`{o}, de fet rebutjar una propietat enunciada per
una proposici\'{o} `$p$' \'{e}s el mateix que provar `$\lnot p$'. Per
exemple, per demostrar que no \'{e}s el cas que $\left( \forall x\in
U\right) P(x)$, primer cal pensar en la seg\"{u}ent equival\`{e}ncia l\`{o}%
gica:%
\begin{equation*}
\lnot\left( \forall x\in U,P(x)\right) \Longleftrightarrow\exists x\in
U,\lnot P\left( x\right)
\end{equation*}
i despr\'{e}s recordar la regla d'inferencia IQE:%
\begin{equation*}
\begin{tabular}{l}
$\lnot P(d)$ \\
$\left( \exists x\in U\right) \lnot P(x)$%
\end{tabular}
\text{.}
\end{equation*}
Llavors, nom\'{e}s cal trobar un exemple pel qual $P(x)$ \'{e}s fals.
Aquesta manera de fer la demostraci\'{o} \'{e}s coneix com \textbf{prova per
contraexemple}.

\begin{exemple}
Volem provar que no \'{e}s veritat que la suma de dos nombres irracionals
\'{e}s irracional.
\end{exemple}

\begin{solucio}
Per buscar una estrat\`{e}gia, primer haurem de reformular l'enunciat l\`{o}%
gicament: si $\mathbb{I}=\mathbb{R}\smallsetminus\mathbb{Q}$ \'{e}s el
conjunt dels nombres irracionals, aleshores%
\begin{equation*}
\lnot\left( \forall x,y\in\mathbb{I}\right) \left( x+y\in\mathbb{I}\right)
\end{equation*}
Hem vist que aquest enunciat \'{e}s equivalent al seg\"{u}ent:%
\begin{equation*}
\left( \exists x,y\in\mathbb{I}\right) \left( x+y\notin\mathbb{I}\right)
\end{equation*}
Per tant, per demostrar l'enunciat nom\'{e}s cal trobar dos nombres
irracionals la suma dels quals no \'{e}s irracional.
\end{solucio}

\subsection{Demostracions per casos}

A vegades trobem enunciats que s\'{o}n de la forma $\left( C\vee D\right)
\longrightarrow B$, \'{e}s a dir, que inclouen una disjunci\'{o} en
l'antecedent. En aquests casos, la seg\"{u}ent equival\`{e}ncia
\begin{equation*}
\left( C\vee D\right) \longrightarrow B\Longleftrightarrow\left(
C\longrightarrow B\right) \wedge\left( D\longrightarrow B\right) \text{,}
\end{equation*}
proporciona l'estrat\`{e}gia que hem de seguir. En efecte, per demostrar que
$\left( C\vee D\right) \longrightarrow B$ \'{e}s certa \'{e}s suficient
demostrar que $C\longrightarrow B$ i $D\longrightarrow B$ s\'{o}n ambdues
certes. De fet, la demostraci\'{o} la constru\"{\i}m per casos (en aquest
cas, n'hi ha dos casos) segons la forma que tingui l'antecedent. La
demostraci\'{o} de cadascun dels casos seguir\`{a} l'estrat\`{e}gia que faci
falta. Aquest m\`{e}tode de demostraci\'{o} es coneix com \textbf{prova per
casos}.

\begin{exemple}
Volem provar que si $n$ \'{e}s un nombre enter, llavors $n^{2}+n$ \'{e}s
parell.
\end{exemple}

\begin{solucio}
Sabem que tot nombre enter $n$ \'{e}s o parell o b\'{e} senar. Aquesta idea
proporciona l'estrat\`{e}gia de la demostraci\'{o}: constru\"{\i}r una prova
per casos. Volem provar que (1) si $n$ \'{e}s parell, aleshores $n^{2}+n$
\'{e}s parell, i (2) si $n$ \'{e}s senar, aleshores $n^{2}+n$ \'{e}s senar.

\begin{description}
\item[Cas 1:] Si $n$ \'{e}s parell, llavors per definici\'{o} existeix $k\in%
\mathbb{Z}$ tal que $n=2k$. Aleshores es t\'{e}:
\begin{equation*}
n^{2}+n=4k^{2}+2k=2\left( 2k^{2}+k\right) \text{,}
\end{equation*}
i, per tant, $n^{2}+n$ \'{e}s parell perqu\`{e} $2k^{2}+k\in\mathbb{Z}$.

\item[Cas 2:] Si $n$ \'{e}s senar, llavors per definici\'{o} existeix $m\in%
\mathbb{Z}$ tal que $n=2m+1$. Aleshores es t\'{e}:%
\begin{align*}
n^{2}+n & =\left( 2m+1\right) ^{2}+2m+1 \\
& =4m^{2}+6m+2=2\left( 2m^{2}+3m+1\right) \text{,}
\end{align*}
i, per tant, $n^{2}+n$ \'{e}s parell perqu\`{e} $2m^{2}+3m+1\in\mathbb{Z}$.
\end{description}
\end{solucio}

\bigskip

Ara passem als teoremes de la forma $A\longrightarrow\left( C\vee D\right) $
on hi trobem una disjuntiva en el conseq\"{u}ent. Una estrat\`{e}gia per a
fer la demostraci\'{o} \'{e}s fent \'{u}s primer de la llei del contrarec%
\'{\i}proc i despr\'{e}s una de les lleis de De Morgan:
\begin{align*}
A & \longrightarrow\left( C\vee D\right) \Longleftrightarrow\lnot\left(
C\vee D\right) \longrightarrow\lnot A \\
& \Longleftrightarrow\left( \lnot C\wedge\lnot D\right) \longrightarrow
\lnot A
\end{align*}
A partir d'aqu\'{\i}, constru\"{\i}m la prova directa o per reducci\'{o} al
absurd. Veiem-ho en un exemple.

\begin{exemple}
Considerem dos nombres reals $x$ i $y$. Volem provar que si $xy$ \'{e}s
irracional, llavors $x$ o $y$ \'{e}s irracional.
\end{exemple}

\begin{solucio}
Segons el que hem vist anteriorment, aquest enunciat \'{e}s equivalent al seg%
\"{u}ent: Si $x$ \'{e}s racional i $y$ tamb\'{e} ho \'{e}s, llavors $xy$
\'{e}s racional. Per\`{o} aix\`{o} \'{e}s evident perqu\`{e} sabem que la
multiplicaci\'{o} \'{e}s una operaci\'{o}n interna del cos dels racionals.
\end{solucio}

\bigskip

Despr\'{e}s d'haver discutit sobre l'aparici\'{o} de la connectiva $\vee$,
anem ara a tractar els casos quan aprareix la connectiva $\wedge$ en els
teoremes de la forma $A\longrightarrow B$. Per demostar un enunciat de la
forma $A\longrightarrow\left( C\wedge D\right) $ fem \'{u}s de l'equival\`{e}%
ncia seg\"{u}ent:%
\begin{equation*}
A\longrightarrow\left( C\wedge D\right) \Longleftrightarrow\left(
A\longrightarrow C\right) \wedge\left( A\longrightarrow D\right)
\end{equation*}
A partir d'aqu\'{\i}, la demostraci\'{o} consisteix en provar que \thinspace$%
A\longrightarrow C$ i $A\longrightarrow D$ s\'{o}n certes.

Si l'enunciat \'{e}s de la forma $\left( C\wedge D\right) \longrightarrow B$%
, llavors l'estrat\`{e}gia \'{e}s prendre com hip\`{o}tesis que $C$ i $D$ s%
\'{o}n certes i com a tesi que $B$ \'{e}s certa. La prova pot ser directe o
per reducci\'{o} a l'aburd.

Finalment, volem tractar una altra forma l\`{o}gica que presenten molts
teoremes. S\'{o}n els teoremes de la forma $A\longleftrightarrow B$ i que
ens donen condicions necess\`{a}ries i suficients. La llei del bicondicional
ens dona la resposta per construir proves d'aquest teoremes:%
\begin{equation*}
A\longleftrightarrow B\Longleftrightarrow\left( A\longrightarrow B\right)
\wedge\left( B\longrightarrow A\right) \text{.}
\end{equation*}
Per tant, \'{e}s suficient provar $A\longrightarrow B$ i $B\longrightarrow A$%
, cadascun dels quals es pot demostrar mitjan\c{c}ant qualsevol dels m\`{e}%
todes que hem vist fins ara.

\begin{exemple}
Volem provar que si $x$ i $y$ son dos nombres reals, llavors $xy=0$ sii $x=0$
o $y=0$.
\end{exemple}

\begin{solucio}
L'enunciat expressat formalment \'{e}s:%
\begin{equation*}
\left( \forall x,y\in\mathbb{R}\right) \left( xy=0\longleftrightarrow
x=0\vee y=0\right) \text{.}
\end{equation*}
Per tant, si $x,y$ s\'{o}n nombres reals arbitraris, hem de demostrar $%
xy=0\longleftrightarrow x=0\vee y=0$. Segons hem dit abans, aix\`{o} \'{e}s
equivalent a demostrar cadascun del teoremes seg\"{u}ents: (1) $%
xy=0\longrightarrow x=0\vee y=0$ i (2) $x=0\vee y=0\longrightarrow xy=0$.

(1)\ Demostrem que la condici\'{o} necess\`{a}ria perqu\`{e} \thinspace$xy=0
$ \'{e}s $x=0$ o $y=0$. Aix\`{o} \'{e}s equivalent a provar que si $x\neq0$
i $y\neq0$ aleshores $xy\neq0\,$. No \'{e}s restrictiu suposar que $x\leq y $%
; cas contrari, far\'{\i}em que $x>y$ i la prova es contrueix de forma an%
\`{a}loga. Ara l'estrat\`{e}gia \'{e}s fer la demostraci\'{o} per casos: si $%
x\neq0$, llavors o $x>0$ o b\'{e} $x<0$.

\begin{description}
\item[Cas 1:] Si $0<x\leq y$, llavors $xy>0$ i, per tant, $xy\neq0$.

\item[Cas 2:] Si $x<0$ i $y\geq x$, llavors apareixen dos subcasos: (2.1) $%
x<0$ i $y>0$ (2.2) $x<0$ i $y<0$ i $y\geq x$.

\item[Subcas 2.1:] Si $x<0$ i $y>0$, llavors $xy<0$ i, per tant, $xy\neq0$.

\item[Subcas 2.2:] Si $x<0$ i $y<0$ i $y\geq x$, llavors $xy>0$ i, per tant,
$xy\neq0$.
\end{description}

Com a conseq\"{u}\`{e}ncia, conclu\"{\i}m el que vol\'{\i}em demostrar.

(2) Demostrem que la condici\'{o} suficient perqu\`{e} $xy=0$ \'{e}s que $%
x=0 $ o $y=0$. Aix\`{o} \'{e}s evident perqu\`{e} si $x=0$ aleshores $0\cdot
y=0$ i si $y=0$, $x\cdot0=0$.
\end{solucio}

\subsection{Demostracions d'exist\`{e}ncia i unicitat}

\'{E}s molt freq\"{u}ent trobar demostracions que impliquin l'exist\`{e}ncia
d'un objecte que compleix una determinada propietat i tamb\'{e} la seva
unicitat. Quant a la prova d'exist\`{e}ncia nom\'{e}s cal trobar l'objecte
en concret que compleix la propietat. Quant a la unicitat, l'estrat\`{e}gia m%
\'{e}s comuna \'{e}s suposar que $x$ i $y$ s\'{o}n dos objectes que satisfan
la propietat donada, llavors hem de demostrar que $x=y$.

\begin{exemple}
Volem provar que la proporci\'{o} entre dos nombres reals $a$ i $b$, $a>b>0$%
, que compleixen la propietat%
\begin{equation*}
\frac{a}{b}=\frac{a+b}{a}
\end{equation*}
\'{e}s un nombre irracional i a m\'{e}s \'{e}s \'{u}nic.
\end{exemple}

\begin{solucio}
Primer hem de provar l'exist\`{e}ncia, buscant aquest nombre que representa
la proporci\'{o} donada. Suposem que existeix, aleshores es compleix%
\begin{equation*}
\frac{a}{b}=\frac{a+b}{a}\Longleftrightarrow\frac{a}{b}=\frac{\dfrac{a}{b}+1%
}{\dfrac{a}{b}}
\end{equation*}
Fent que $\dfrac{a}{b}=x$, llavors tenim%
\begin{equation*}
x=\frac{x+1}{x}\Longleftrightarrow x^{2}-x-1=0
\end{equation*}
i, resolent l'equaci\'{o} de segon grau i sabent que $x>0$, es t\'{e} la
soluci\'{o}%
\begin{equation*}
\Phi=\frac{1}{2}\left( 1+\sqrt{5}\right)
\end{equation*}

Ara, hem de veure que $\Phi$ \'{e}s irracional. Considerem el conjunt $%
A=\left\{ a\in\mathbb{N}\text{: }\exists b\in\mathbb{N}\text{ i }a>b>0\text{
i }\Phi=\dfrac{a}{b}\right\} $.Si $A=\emptyset$, llavors $\Phi$ \'{e}s
irracional. Suposem ara que no \'{e}s el vuit. Aleshores, com $A$ \'{e}s un
subconjunt de $\mathbb{N}$, $A$ t\'{e} primer element, que denotem per $m$.
Llavors es t\'{e} $\Phi=\dfrac{m}{b}$ i $m>b>0$, per tant, $m+b>m$ i%
\begin{equation*}
\dfrac{m}{b}<\frac{m+b}{b}
\end{equation*}
per\`{o}, aix\`{o} \'{e}s una contradicci\'{o} perqu\`{e}%
\begin{equation*}
\frac{m}{b}=\frac{m+b}{b}=\Phi\text{.}
\end{equation*}
Dedu\"{\i}m que $\Phi$ \'{e}s irracional.

Finalment, hem de demostrar la unicitat. Suposem que $\Phi_{1}$ i $\Phi_{2} $
compleixen la propietat. Llavors, de (\ref{4}) es t\'{e}%
\begin{align*}
\Phi_{1}^{2}-\Phi_{1}-1 & =\Phi_{2}^{2}-\Phi_{2}-1 \\
\Phi_{1}^{2}-\Phi_{2}^{2}-\Phi_{1}+\Phi_{2} & =0 \\
\left( \Phi_{1}-\Phi_{2}\right) \left( \Phi_{1}+\Phi_{2}-1\right) & =0
\end{align*}
i, per tant, $\Phi_{1}-\Phi_{2}=0$ o $\Phi_{1}+\Phi_{2}-1=0$ per\`{o}, $%
\Phi_{1}+\Phi_{2}>1$ perqu\`{e} $a>b>0$. Per tant, $\Phi_{1}=\Phi_{2}$.
\end{solucio}

\subsection{Demostracions per inducci\'{o}\label{4}}

La inducci\'{o} matem\`{a}tica \'{e}s una t\`{e}cnica \'{u}til per demostrar
enunciats sobre nombres naturals. Considerem, per exemple, l'enunciat seg%
\"{u}ent: "Si $n$ \'{e}s parell, llavors $n^{2}$ \'{e}s divisible per 4".
Volem provar que aquest enunciat \'{e}s correcte. Com ho fem? Aplicant
inducci\'{o} sobre $n$, que vol dir provar els dos passos seg\"{u}ents:

\begin{enumerate}
\item Primer hem de provar que la propietat \'{e}s vertadera pel primer
element. En aquest cas el primer element \'{e}s $2$ perqu\`{e} tractem amb
nombres naturals parells. La propietat es certa perqu\`{e} $2^{2}=4$ que
\'{e}s divisible per 4.

\item Per a tot nombre natural $n$ (no sabem quin \'{e}s), si la propietat
\'{e}s certa per $n$, llavors hem de provar que tamb\'{e} ho \'{e}s per el
seg\"{u}ent element: Suposem \ $n$ \'{e}s parell i que $n^{2}$ \'{e}s
divisible per 4, aleshores hem de provar que el seg\"{u}ent parell $%
(n+2)^{2} $ \'{e}s divisible per 4. En efecte, $%
(n+2)^{2}=n^{2}+4n+4=4k+4n+4=4(k+n+1)$, on $n^{2}=4k$ i $k\in\mathbb{N}$.
\end{enumerate}

Conclu\"{\i}m que la propietat \'{e}s v\`{a}lida per a tot nombre natural
parell.

Intu\"{\i}tivament, aquestes dues condicions permeten assegurar que la
propietat es compleix per tots els naturals. En efecte, $P(2)$ \'{e}s
vertadera, $P(2)\longrightarrow P(4)$ \'{e}s vertadera i, per tant, $P(4)$
tamb\'{e} ho \'{e}s;\ $P(4)\longrightarrow P(6)$ \'{e}s certa i, per tant, $%
P(6)$ tamb\'{e}. I aix\'{\i} succesivament.

\bigskip

La base d'aquest m\`{e}tode de demostraci\'{o} \'{e}s el principi d'inducci%
\'{o} que, de fet \'{e}s un dels axiomes que defineixen els nombres
naturals. Aquest principi diu: Si $P(n)$ \'{e}s un enunciat sobre els
naturals i es compleixen les dues condicions seg\"{u}ents: A la primera
part, anomenada \textbf{cas base}, mostrem que $P(1)$ \'{e}s es compleix (el
primer element no necess\`{a}riament \'{e}s 1, dep\`{e}n de cada cas). A la
segona part, anomenada pas inductiu, suposem que $n$ \'{e}s un natural tal
que $P(n)$ \'{e}s certa, tot i que no sabem qu\`{e} \'{e}s $n$, i hem de dedu%
\"{\i}r que $P(n+1)$ \'{e}s certa. La suposici\'{o} que $P(n)$ \'{e}s
vertadera es diu \textbf{hip\`{o}tesi d'inducci\'{o}}. La conclusi\'{o}
d'aquest raonament \'{e}s que $P(n)$ \'{e}s v\`{a}lida per a tot $n$
natural. Aquest m\`{e}tode de demostraci\'{o} es coneix com \textbf{prova
per inducci\'{o}} sobre $n$.

\bigskip

Veiem en detall aquest m\`{e}tode en un exemple:

\begin{exemple}
Volem demostrar que per a tot $n$ natural es compleix%
\begin{equation*}
1+2+\cdots+n=\frac{n(n+1)}{2}\text{.}
\end{equation*}
\end{exemple}

\begin{solucio}
Constru\"{\i}m una prova d'inducci\'{o} sobre $n$, sent $P(n)=$ `$1+2+\cdots
+n=\dfrac{n(n+1)}{2}$'.

\begin{description}
\item[Cas base:] $P(1)$ \'{e}s $1=\frac{1(1+1)}{2}$ i, evidentment \'{e}s
veritat.

\item[Hip\`{o}tesi d'inducci\'{o}:] Suposem que $P(n)$ \'{e}s certa, o sigui
que $1+2+\cdots+n=\dfrac{n(n+1)}{2}$.

\item[Tesi:] Hem de demostrar que $P(n+1)=$`$1+2+\cdots+n=\dfrac {(n+1)(n+2)%
}{2}$' \'{e}s certa.
\end{description}

En efecte, aplicant l'hip\`{o}tesi d'inducci\'{o}, es t\'{e}%
\begin{align*}
1+2+\cdots+n+n+1 & =\dfrac{n(n+1)}{2}+n+1 \\
& =\frac{\left( n+1\right) (n+2)}{2}
\end{align*}
que \'{e}s el que vol\'{\i}em veure.

\begin{description}
\item[Conclusi\'{o}:] Pel principi d'inducci\'{o} sobre $n$, dedu\"{\i}m que
$P(n)$ \'{e}s certa per a tot $n\in\mathbb{N}$.
\end{description}
\end{solucio}

\begin{exemple}
Volem demostrar que per a tot nombre natural $n$ es compleix que $%
8^{n}-3^{n} $ \'{e}s divisible per 5.
\end{exemple}

\begin{solucio}
El cas base \'{e}s per $n=1$: $8-3=5$ que \'{e}s divisible per $5$. Suposem
ara que per a tot nombre natural $n$ es t\'{e} que $8^{n}-3^{n}$ \'{e}s
divisible per 5 (hip\`{o}tesi d'inducci\'{o}), hem de demostrar que $%
8^{n+1}-3^{n+1}$ tamb\'{e} \'{e}s divisible per 5. En efecte, tenim%
\begin{align*}
8^{n+1}-3^{n+1} & =8\cdot8^{n}-3\cdot3^{n} \\
& =8\cdot8^{n}-\left( 8-5\right) \cdot3^{n} \\
& =8\cdot\left( 8^{n}-3^{n}\right) +5\cdot3^{n}
\end{align*}
per\`{o}, per hip\`{o}tesi d'inducci\'{o}, $8^{n}-3^{n}=5k$, sent $k\in%
\mathbb{N}$. Per tant,%
\begin{equation*}
8^{n+1}-3^{n+1}=5\cdot\left( 8k+3^{n}\right)
\end{equation*}
i, d'aqu\'{\i}, s'obt\'{e} que $8^{n+1}-3^{n+1}$ \'{e}s divisible per 5 perqu%
\`{e} $8k+3^{n}\in\mathbb{N}$. Com a conseq\"{u}\`{e}ncia, $8^{n}-3^{n}$
\'{e}s divisible per 5 per a tot nombre natural $n$.
\end{solucio}

Les definicions \textbf{inductives} estan impl\'{\i}cites en les definicions
de diverses funcions molt comuns que impliquen nombres enters no negatius.
Per exemple, el factorial d'un nombre enter no negatiu $n$, designat per $n!$%
, es defineix inductivament d'aquesta manera:

\begin{itemize}
\item $0!=1;$

\item $\left( n+1\right) !=\left( n+1\right) \cdot n!$.
\end{itemize}

Un altre exemple son les definicions per \textbf{recursi\'{o}}. La successi%
\'{o} de Fibonacci \'{e}s un bon exemple: $1,1,2,3,5,8,13,...$, que podem
definir per recursi\'{o} d'aquesta manera:

\begin{itemize}
\item $u_{1}=1$,

\item $u_{2}=1$,

\item $u_{k+1}=u_{k-1}+u_{k}$ per $k\in\mathbb{N}$ i $k\geq2$.
\end{itemize}

\section{Teories axiom\`{a}tiques}

Resulta que, per demostrar alguna cosa, es necessita el coneixement
d'algunes veritats anteriors. La l\`{o}gica nom\'{e}s proporciona les
maneres que podem deduir una afirmaci\'{o} d'altres, per\`{o} necessitem
algunes afirmacions per comen\c{c}ar. Aquestes afirmacions inicials
s'anomenen \textbf{axiomes} o \textbf{postulats} d'en\c{c}\`{a} que Euclides
va escriure els seus Elements cap al 300 aC. Per exemple, el primer postulat
d'Euclides diu `donats dos punts, existeix una \'{u}nica l\'{\i}nea recta
que passa per ells', enunciat que la majoria de la gent considera evident.
Tot i aix\`{o}, en l'enfocament axiom\`{a}tic modern els axiomes no es veuen
necess\`{a}riament com a veritats evidents per si mateixos, sin\'{o}
simplement com a afirmacions que suposem que s\'{o}n certes. Fem matem\`{a}%
tiques explorant el que es deriva de la veritat d'aquests axiomes mitjan\c{c}%
ant les regles de deducci\'{o} de la l\`{o}gica.

\bigskip

Les matem\`{a}tiques modernes es basen en la teoria de conjunts i la l\`{o}%
gica. La majoria dels objectes matem\`{a}tics, com ara punts, l\'{\i}nies,
nombres, funcions, successions, grups, etc., s\'{o}n realment conjunts. Per
tant, \'{e}s necessari comen\c{c}ar a coneixer els axiomes de la teoria de
conjunts. Aqu\'{\i} no tractarem aquest punt per\`{o} es pot trobar en
qualsevol text de teoria de conjunts de nivell avan\c{c}at.

A m\'{e}s, tenim els axiomes que defineixen cadascuna de les teories. Per
exemple, com hem comentat abans, en geometria euclidiana tenim l'axioma que
afirma que existeix una i nom\'{e}s una recta que passa per dos punts
diferents. En les matem\`{a}tiques modernes no t\'{e} cap sentit discutir
aquest axioma. Si no s'assumeix, simplement no podem anomenar al objectes
amb els noms `l\'{\i}nia recta' i `punt'. Els axiomes serveixen aqu\'{\i}
com a definicions que caracteritzen aquest sistema matem\`{a}tic. De fet,
una teoria matem\`{a}tica \'{e}s com un joc d'escacs i els axiomes
corresponen a les regles del joc. Si no accepteu una regla, el joc deixa de
ser escacs i \'{e}s un altre cosa.

\bigskip

Les matem\`{a}tiques s'ocupen de sistemes abstractes de diversos tipus,
definits cadascun per un conjunt adequat d'axiomes, que serveixen per
caracteritzar la seva estructura. Per\`{o} tot i que, des del punt de vista
de les matem\`{a}tiques pures, cada estructura es considera aut\`{o}noma,
l'esquema matem\`{a}tic sol tenir una o m\'{e}s realitzacions concretes;
\'{e}s a dir, l'estructura sol trobar-se (possiblement nom\'{e}s fins a un
cert grau d'aproximaci\'{o}) en un sistema m\'{e}s concret. La geometria
euclidiana abstracta de tres dimensions, per exemple, t\'{e} una de les
seves realitzacions l'estructura de l'espai ordinari.

\bigskip

Una teoria est\`{a} formada per tots els teoremes que es dedueixen dels seus
axiomes. Com a exemples, presentem dues teories axiom\`{a}tiques cl\`{a}%
ssiques de les matem\`{a}tiques, i completarem aquest document introductori
fent de cadascuna d'elles la demostraci\'{o} d'un teorema.

\subsection{Teoria axiom\`{a}tica de la geometria plana}

La geometria euclidiana plana es pot considerar com el conjunt de resultats
que s'obtenen a partir dels cinc postulats d'Euclides per deducci\'{o} l\`{o}%
gica. Com es pot observar de seguida, els postulats d'Euclides estan pensats
per fer geometria en el pla fent \'{u}s de la regla i el comp\`{a}s. Aquests
postulats s\'{o}n:

\begin{enumerate}
\item Es pot tra\c{c}ar una l\'{\i}nia recta entre dos punts qualsevol i el
resultat \'{e}s un segment de l\'{\i}nia recta.

\item Qualsevol segment de l\'{\i}nia recta es pot ampliar indefinidament.

\item Donat un punt i un segment de l\'{\i}nia de recta que comencen pel
punt, podeu dibuixar un cercle centrat en el punt donat amb el segment de l%
\'{\i}nia donat com a radi.

\item Tots els angles rectes s\'{o}n iguals.

\item Si dues rectes en un pla es troben amb una altra recta i si la suma
dels angles interns d'un costat \'{e}s inferior a dos angles rectes, les
rectes es reuniran si s'estenen prou al costat on es suma la suma dels
angles \'{e}s inferior a dos angles rectes.

\FRAME{dtbpF}{10.4493cm}{7.9693cm}{0pt}{}{}{logi1.jpg}{\special{ language "Scientific Word"; type "GRAPHIC";
maintain-aspect-ratio TRUE; display "USEDEF"; valid_file "F"; width
10.4493cm; height 7.9693cm; depth 0pt; original-width 10.3768cm;
original-height 7.8969cm; cropleft "0"; croptop "1"; cropright "1";
cropbottom "0"; filename 'img/logi1.jpg';file-properties "XNPEU";}}
\end{enumerate}

El cinqu\`{e} axioma \'{e}s conegut com l'\textbf{axioma de les
paral\textperiodcentered leles} quan \'{e}s reformulat d'aquesta manera:
`Donada qualsevol recta i un punt que no estigui sobre aquesta recta,
existeix una \'{u}nica recta que passa pel punt i \'{e}s
paral\textperiodcentered lela a la recta donada'. De fet, la geometria cl%
\`{a}ssica euclidiana es distingeix per aquest axioma. Si no acceptem aquest
resultat, surt per exemple l'\textbf{axioma de Lobachevsky}: Donada una
recta i un punt exterior a ella, existeixen almenys dues rectes que passen
per aquest punt i que no tallen la recta donada. La geometria caracteritzada
pels quatre primers postulats d'Euclides i l'axioma de Lobachevsky
distingeix la geometria anomenada hiperb\`{o}lica.

\begin{teorema}[Teorema de Pit\`{a}gores]
En tots els triangles rectangles es compleix que el quadrat del costat
oposat a l'angle recte \'{e}s igual a la suma dels quadrats dels costats que
comprenen a l'angle recte.
\end{teorema}

\begin{prova}
En efecte, considerem un triangle rectangle qualsevol $ABC$, on hi suposem
l'angle $\angle BAC$ recte. Podem construir un quadrat sobre cada segment
del triangle, i obtenir d'aquesta manera una construcci\'{o} com la de la
figura seg\"{u}ent. \FRAME{dtbpF}{8.4636cm}{9.7969cm}{0pt}{}{}{logi2.jpg}{%
\special{ language "Scientific Word"; type "GRAPHIC"; maintain-aspect-ratio
TRUE; display "USEDEF"; valid_file "F"; width 8.4636cm; height 9.7969cm;
depth 0pt; original-width 8.3933cm; original-height 9.7267cm; cropleft "0";
croptop "1"; cropright "1"; cropbottom "0"; filename
'img/logi2.jpg';file-properties "XNPEU";}}

Observem primer que els triangles $DCB$ i $ABI$ s\'{o}n iguals, perqu\`{e} $%
AB=BD$, $BI=BC$ i $\angle DBC=\angle ABI$. Llavors, l'\`{a}rea del quadrat $%
ABDE$ \'{e}s el doble de l'\`{a}rea del triangle $DCB$, perqu\`{e} les dues
figures tenen la mateixa base i es troben entre les mateixes
paral\textperiodcentered leles. Per la mateixa ra\'{o} l'\`{a}rea del
rectangle $BIJK$ \'{e}s el doble de l'\`{a}rea del triangle $ABI$.
D'aquestes dues conclusions, obtenim que l'\`{a}rea del rectangle $BIJK$
\'{e}s igual que l'\`{a}rea del quadrat $ABDE$. An\`{a}logament, s'obt\'{e}
que l'\`{a}rea del rectangle $CHKJ$ \'{e}s igual que l'\`{a}rea del quadrat $%
ACGF$. Per tant, com l'\`{a}rea del quadrat $BIHC$ \'{e}s igual a la suma de
les \`{a}rees dels rectangles $BIKJ$ i $CHKJ$, per les igualtats demostrades
tenim que l'\`{a}rea del quadrat $BIHC$ de catet oposat a l'angle recte,
\'{e}s igual a la suma de les \`{a}rees dels quadrats $ABDE$ i $ACGF$, que
tenen per catets els segments que comprenen l'angle recte.
\end{prova}

\subsection{Teoria axiom\`{a}tica de l'aritm\`{e}tica}

A partir dels postulats de Peano, \'{e}s possible demostrar totes les
propietats esperades dels nombres naturals i, a partir del nombres naturals,
\'{e}s possible construir primer els enters, despr\'{e}s els nombres
racionals i despr\'{e}s els nombres reals. El conjunt dels nombres naturals
\'{e}s el conjunt $\mathbb{N}$, l'exist\`{e}ncia del qual ve donada per als
postulats de Peano que s\'{o}n:

\begin{enumerate}
\item Existeix un element $1$ que pertany al conjunt dels nombres naturals $%
\mathbb{N}$.

\item Existeix una funci\'{o} $s:\mathbb{N}\longrightarrow\mathbb{N}$ tal
que compleix les propietats seg\"{u}ents:

\begin{enumerate}
\item No hi ha cap nombre natural $n$ tal que $s(n)=1.$

\item Per a tots $m,n\in N$, si $s(m)=s(n)$, llavors $m=n$.

\item Per a cada subconjunt $K\subseteq\mathbb{N}$, si $1\in K$ i per a cada
nombre natural $n\in K$ es t\'{e} que $s(n)\in K$, llavors tot nombre
natural \'{e}s de $K$.
\end{enumerate}
\end{enumerate}

Si pensem intu\"{\i}tivament en la funci\'{o} $s$ com la que assigna a cada
nombre natural el seu \textbf{seg\"{u}ent}, llavors la part (a) dels
postulats diu que $1$ \'{e}s el \textbf{primer element} en $\mathbb{N}$,
perqu\`{e} no \'{e}s el successor de res. Tamb\'{e} es diu que $\mathbb{N}$
\'{e}s un \textbf{conjunt ben ordenat}. La part (c) es coneix com a principi
d'inducci\'{o} i constitueix la base de les demostracions per inducci\'{o}
que hem tractat en l'apartat \ref{4}.

\bigskip

La pregunta que ens fem de seguida \'{e}s com sabem que hi ha un conjunt i
un element del conjunt i una funci\'{o} del conjunt en si mateix, que
satisfan els postulats de Peano? La resposta no la tractarem aqu\'{\i}. Com
hem assenyalat m\'{e}s amunt la teoria de conjunts \'{e}s la base de les \
matem\`{a}tiques modernes, llavors no \'{e}s necessari suposar
addicionalment que els postulats de Peano es compleixen, perqu\`{e} l'exist%
\`{e}ncia d'alguna cosa que compleixi els postulats de Peano es pot deduir
dels axiomes de la teoria de conjunts.

\bigskip

Per a completar aquest apartat proposem demostrar una propietat caracter%
\'{\i}stica del conjunt dels nombres naturals: tot subconjunt $K$ no vuit de
$\mathbb{N}$ est\`{a} ben ordenat, \'{e}s a dir, que $K$ t\'{e} primer
element.

\begin{teorema}
Qualsevol conjunt no buit de nombres naturals t\'{e} un primer element.
\end{teorema}

\begin{prova}
La demostraci\'{o} la farem per reducci\'{o} a l'absurd utilitzant alhora el
principi d'inducci\'{o}: Considerem un conjunt no vuit de nombres naturals $K
$ sense cap primer element. Considerem la propietat que per a qualsevol $n$
de $K$, $n$ no \'{e}s seg\"{u}ent de cap nombre de $K$. Fem inducci\'{o}
sobre $n$. El cas base \'{e}s $1\notin K$, i aix\`{o} \'{e}s evident, perqu%
\`{e} $K$ \'{e}s un subconjunt de nombres naturals i $1$ \'{e}s el primer
element de $\mathbb{N}$. La hip\`{o}tesi d'inducci\'{o} \'{e}s suposar que
per a qualsevol $n$, no hi ha cap element de $K$ del qual $n$ sigui el seu
seg\"{u}ent. Hem de provar que aix\`{o} tamb\'{e} es compleix per $n+1$.
\'{E}s clar que $n+1$ \'{e}s el seg\"{u}ent de $n$. Si $n\in K$, per hip\`{o}%
tesi d'inducci\'{o}, $n$ \'{e}s el primer element, per\`{o} aix\`{o} \'{e}s
contradictori amb el fet que $K$ no t\'{e} cap primer element. Per tant, $%
n+1\notin K$, i en conseq\"{u}\`{e}ncia $K=\mathbb{N}$, \'{e}s a dir, $K$ t%
\'{e} primer element.
\end{prova}

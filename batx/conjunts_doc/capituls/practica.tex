\chapter{Pràctica}
\label{cap:practica}

\section{Conjunts}

\begin{exercici}
Escriu simb\`{o}licament els conjunts seg\"{u}ents i determina els seus
elements en cada cas:

\begin{enumerate}
\item El conjunt dels nombres reals tals que el seu quadrat \'{e}s $12$.

\item El conjunt dels nombres reals tals que el seu quadrat \'{e}s $-9$.

\item El conjunt dels nombres reals tals que el seu quadrat \'{e}s major o
igual que $0$.

\item El conjunt dels nombres naturals compresos entre $3/2$ i $13/3$.

\item El conjunt dels nombres reals que s\'{o}n soluci\'{o} de l'equaci\'{o}
$3x-1=10$.

\item El conjunt dels nombres enters que s\'{o}n soluci\'{o} de l'equaci\'{o}
$3x-1=10$.
\end{enumerate}
\end{exercici}

\begin{solucio}
(1) El conjunt dels nombres reals tals que el seu quadrat \'{e}s $12$
s'escriu com segueix%
\begin{align*}
A& =\left\{ x\in \mathbb{R}:x^{2}=12\right\} \\
& =\left\{ -2\sqrt{3},2\sqrt{3}\right\} \text{.}
\end{align*}

(2) El conjunt dels nombres reals tals que el seu quadrat \'{e}s $-9$ \'{e}s
el conjunt buit, ja que no hi ha cap nombre real el quadrat del qual sigui
negatiu.

(3) El conjunt dels nombres reals tals que el seu quadrat \'{e}s major o
igual que $0$ s'escriu com segueix%
\begin{align*}
C& =\left\{ x\in \mathbb{R}:x\geq 0\right\} \\
& =[0,+\infty )\text{.}
\end{align*}

(4) El conjunt dels nombres naturals compressos entre $3/2$ i $13/3$ \'{e}s%
\begin{align*}
D& =\left\{ x\in \mathbb{N}:3/2\leq x\leq 13/3\right\} \\
& =\left\{ 2,3,4\right\} \text{.}
\end{align*}

(5) El conjunt dels nombres racionals que s\'{o}n soluci\'{o} de l'equaci%
\'{o} $3x-1=10$ \'{e}s%
\begin{align*}
E& =\left\{ x\in \mathbb{Q}:3x-1=10\right\} \\
& =\left\{ \frac{11}{3}\right\} \text{.}
\end{align*}

(6) El conjunt dels nombres enters que s\'{o}n soluci\'{o} de l'equaci\'{o} $%
3x-1=10$ \'{e}s el conjunt buit perqu\`{e} no existeix cap nombre enter que
multiplicat per $3$ doni 11.
\end{solucio}

\begin{exercici}
Defineix els seg\"{u}ents conjunts mitjan\c{c}ant una condici\'{o} que
compleixen tots els seus elements:

\begin{enumerate}
\item $\left\{ 5\right\} $

\item $\left\{ 1,3,5,7,9,11\right\} $

\item $\left\{ -1,0,1\right\} $
\end{enumerate}
\end{exercici}

\begin{solucio}
(1) \'{E}s clar que%
\begin{equation*}
\left\{ 5\right\} =\left\{ x\in \mathbb{N}:4<x<6\right\} \text{.}
\end{equation*}

(2) \'{E}s clar que%
\begin{equation*}
\left\{ 1,3,5,7,9,11\right\} =\left\{ x\in \mathbb{N}:x\text{ \'{e}s senar i
}x\leq 11\right\} \text{.}
\end{equation*}

(3) \'{E}s clar que%
\begin{equation*}
\left\{ -1,0,1\right\} =\left\{ x\in \mathbb{R}:x^{3}-x=0\right\}
\end{equation*}
\end{solucio}

\begin{exercici}
S\'{o}n iguals els conjunts%
\begin{equation*}
A=\left\{ x:x\text{ \'{e}s una lletra de la paraula \textquotedblleft matem%
\`{a}tica\textquotedblright }\right\}
\end{equation*}%
i%
\begin{equation*}
B=\left\{ a,m,i,c,t,e\right\}
\end{equation*}%
Per qu\`{e}?
\end{exercici}

\begin{solucio}
\'{E}s evident que%
\begin{equation*}
A=\left\{ m,a,t,e,i,c\right\}
\end{equation*}%
i, per tant, $A=B$, ja que tenen els mateixos elements.
\end{solucio}

\begin{exercici}
Calcula el conjunt de parts dels conjunts seg\"{u}ents: (1) $A=\left\{
1\right\} $; (2) $B=\left\{ 1,2\right\} $; (3) $C=\left\{ 1,2,3\right\} $.
\end{exercici}

\begin{solucio}
(1) Si $A=\left\{ 1\right\} $, llavors%
\begin{equation*}
\mathcal{P}(A)=\left\{ \emptyset ,A\right\} \text{.}
\end{equation*}

(2) Si $B=\left\{ 1,2\right\} $, llavors%
\begin{equation*}
\mathcal{P}(B)=\left\{ \emptyset ,\left\{ 1\right\} ,\left\{ 2\right\}
,B\right\} \text{.}
\end{equation*}

(3) Si $C=\left\{ 1,2,3\right\} $, llavors%
\begin{equation*}
\mathcal{P}(C)=\left\{ \emptyset ,\left\{ 1\right\} ,\left\{ 2\right\}
,\left\{ 3\right\} ,\left\{ 1,2\right\} ,\left\{ 1,3\right\} ,\left\{
2,3\right\} ,C\right\} \text{.}
\end{equation*}
\end{solucio}

\begin{exercici}
Donat el conjunt $A=\left\{ a,b,c\right\} $, quines s\'{o}n vertaderes de
les seg\"{u}ents expressions?%
\begin{equation*}
\begin{array}{cccccccc}
a) & a\in A &  & b) & \left\{ b\right\} \in A &  & c) & c\subset A \\
d) & \left\{ c\right\} \subset A &  & e) & \left\{ a,b,c\right\} \subset A &
& f) & A\in A%
\end{array}%
\end{equation*}
\end{exercici}

\begin{solucio}
a) \'{E}s clar que $a$ \'{e}s element d'$A$ i, per tant, \'{e}s cert que $%
a\in A$.

b) $\left\{ b\right\} \in A$ \'{e}s falsa, ja que $\left\{ b\right\} $ no
\'{e}s element d'$A$.

c) $c\subset A$ \'{e}s falsa, ja que $c$ \'{e}s element d'$A$ i no
subconjunt.

d) \'{E}s clar que $c$ \'{e}s element d'$A$ i, per tant, \'{e}s cert que $%
\left\{ c\right\} $ \'{e}s subconjunt d'$A$

e) \'{E}s clar que $a,b$ i $c$ s\'{o}n elements d'$A$ i, per tant, \'{e}s
cert que $\left\{ a,b,c\right\} $ \'{e}s subconjunt de $A$.

f) $A\in A$ \'{e}s falsa, ja que $A$ no \'{e}s element de si mateix.
\end{solucio}

\begin{exercici}
Donat el conjunt $A=\left\{ 1,2,3\right\} $, quins de les seg\"{u}ents
relacions s\'{o}n vertaderes?%
\begin{equation*}
\begin{array}{llllllll}
a) & \left\{ 3\right\} \in A &  & b) & \left\{ 1,2\right\} \subset A &  & c)
& 3\in \mathcal{P}(A) \\
d) & \emptyset \in \mathcal{P}(A) &  & e) & \emptyset \subset \mathcal{P}(A)
&  & f) & \left\{ 2,3\right\} \subset \mathcal{P}(A) \\
g) & \left\{ \emptyset \right\} \subset \mathcal{P}(A) &  & h) & \emptyset
\in \left\{ \emptyset \right\} &  & i) & \left\{ \left\{ 1\right\} \right\}
\in \mathcal{P}(A)%
\end{array}%
\end{equation*}
\end{exercici}

\begin{solucio}
a) $\left\{ 3\right\} \in A$ \'{e}s falsa, ja que $\left\{ 3\right\} $ no
\'{e}s element d'$A$.

b) \'{E}s clar que $1$ i $2$ s\'{o}n elements d'$A$ i, per tant, \'{e}s cert
que $\left\{ 1,2\right\} $ \'{e}s subconjunt d'$A$.

c) \'{E}s clar que $3$ no \'{e}s subconjunt d'$A$ i, per tant, $3\in
\mathcal{P}(A)$ \'{e}s falsa.

d) \'{E}s clar que $\emptyset \subset A$ i, per tant, \'{e}s cert que $%
\emptyset \in \mathcal{P}(A)$.

e) \'{E}s clar que $\emptyset $ \'{e}s subconjunt de qualsevol conjunt i,
per tant, \'{e}s cert que $\emptyset \subset \mathcal{P}(A)$.

f) \'{E}s clar que ni $2$ ni $3$ s\'{o}n subconjunts d'$A$ i, per tant, \'{e}%
s fals que $\left\{ 2,3\right\} \subset \mathcal{P}(A)$.

g) \'{E}s clar que $\emptyset \in \mathcal{P}(A)$ i, per tant, \'{e}s cert
que $\left\{ \emptyset \right\} \subset \mathcal{P}(A)$.

h) En ser $\emptyset $ element de $\left\{ \emptyset \right\} $, \'{e}s cert
que $\emptyset \in \left\{ \emptyset \right\} $.

i) \'{E}s clar que $\left\{ \left\{ 1\right\} \right\} $ no \'{e}s
subconjunt d'$A$ i, per tant, \'{e}s fals que $\left\{ \left\{ 1\right\}
\right\} \in \mathcal{P}(A)$.
\end{solucio}

\begin{exercici}
Demostra que es compleixen les seg\"{u}ents propietats de la relaci\'{o}
d'inclusi\'{o}:

\begin{enumerate}
\item Per a tot conjunt $A$, $A\subset A$.

\item Donats dos conjunts $A$ i $B$, si $A\subset B$ i $A\supset B$, llavors
$A=B$.

\item Donats tres conjunts $A$, $B$ i $C$, si $A\subset B$ i $B\subset C$,
llavors $A\subset C$.
\end{enumerate}
\end{exercici}

\begin{solucio}
(1) Donat qualsevol conjunt $A$, per definici\'{o}, tenim%
\begin{equation*}
\begin{array}{ccc}
A\subset A & \Longleftrightarrow & \left( \forall x\right) \left( x\in
A\Longrightarrow x\in A\right)%
\end{array}%
\end{equation*}%
Des del punt de vista l\`{o}gic, la implicaci\'{o}%
\begin{equation*}
x\in A\Longrightarrow x\in A
\end{equation*}%
\'{e}s vertadera qualssevol que sigui $x$. Per tant, $A\subset A$.

(2) Donats dos conjunts qualssevol $A$ i $B$, si $A\subset B$, llavors%
\begin{equation*}
\left( \forall x\right) \left( x\in A\Longrightarrow x\in B\right)
\end{equation*}%
A m\'{e}s, si $B\subset A$, llavors%
\begin{equation*}
\left( \forall x\right) \left( x\in B\Longrightarrow x\in A\right)
\end{equation*}%
Des del punt de vista l\`{o}gic, de les dues implicacions anteriors es
dedueix%
\begin{equation*}
x\in A\Longleftrightarrow x\in B
\end{equation*}%
per a tot $x$. Per tant, per definici\'{o} d'igualtat de conjunts, $A=B$.

(3) Donats tres conjunts qualssevol $A$, $B$ i $C$, si $A\subset B$, llavors%
\begin{equation*}
x\in A\Longrightarrow x\in B
\end{equation*}%
per a tot $x$. A m\'{e}s, si $B\subset C$, llavors%
\begin{equation*}
x\in B\Longrightarrow x\in C
\end{equation*}%
per a tot $x$. Des del punt de vista l\`{o}gic, de les dues implicacions
anteriors es dedueix%
\begin{equation*}
x\in A\Longrightarrow x\in C
\end{equation*}%
per a tot $x$, i per tant, per definici\'{o}, es compleix $A\subset C$.
\end{solucio}

\begin{exercici}
Donades les seg\"{u}ents condicions:%
\begin{equation*}
P(x):\text{ }x\text{ \'{e}s m\'{u}ltiple de }6
\end{equation*}%
i%
\begin{equation*}
Q(x):\text{ }x\text{ \'{e}s m\'{u}ltiple de }3
\end{equation*}%
(a) Demostra que per a tot $x\in \mathbb{Z}$ es compleix la seg\"{u}ents
implicaci\'{o}
\begin{equation*}
P(x)~\Longrightarrow ~Q(x)
\end{equation*}%
i (b) si%
\begin{equation*}
A=\left\{ x\in \mathbb{Z}:P(x)\right\} \text{ \ \ i \ \ }B=\left\{ x\in
\mathbb{Z}:Q(x)\right\}
\end{equation*}%
quins de les seg\"{u}ents relacions \'{e}s correcte $A\subset B$ o $B\subset
A$?
\end{exercici}

\begin{solucio}
(a) \'{E}s evident que tot nombre enter que sigui m\'{u}ltiple de $6$ \'{e}s
tamb\'{e} m\'{u}ltiple de $3$. Per tant, la implicaci\'{o} l\`{o}gica seg%
\"{u}ent%
\begin{equation*}
P(x)~\Longrightarrow ~Q(x)
\end{equation*}%
\'{e}s certa per a tot $x\in \mathbb{Z}$.

(b) Com que, per a tot $x\in \mathbb{Z}$ es compleixen%
\begin{equation*}
\begin{array}{ccc}
x\in A & \Longleftrightarrow & P(x)%
\end{array}%
\text{ \ \ i \ \ }%
\begin{array}{ccc}
x\in B & \Longleftrightarrow & Q(x)%
\end{array}%
\end{equation*}%
i, segons l'apartat anterior,%
\begin{equation*}
P(x)~\Longrightarrow ~Q(x)
\end{equation*}%
llavors,%
\begin{equation*}
x\in A\Longrightarrow x\in B
\end{equation*}%
i, com a conseq\"{u}\`{e}ncia, $A\subset B$.
\end{solucio}

\begin{exercici}
Donats els seg\"{u}ents intervals de la recta real%
\begin{equation*}
A=(-2,5]\text{ \ \ \ i \ \ \ }B=[1,9]
\end{equation*}%
Determina els conjunts $A\cap B$, $A\cup B$, $A-B$ i $B-A$.
\end{exercici}

\begin{solucio}
Es t\'{e} que $A\cap B=[1,5]$, $A\cup B=(-2,9]$, $A-B=(-2,1)$ i $B-A=(5,9]$.
\end{solucio}

\begin{exercici}
Una companyia d'assegurances t\'{e} una cartera de clients $U$ i tracta
d'estudiar algunes caracter\'{\i}stiques d'aquests. Sigui $A$ el conjunt
d'adults, $B$ el de dones i $C$ el dels clients casats. (a) Descriu els seg%
\"{u}ents conjunts: $\complement A$, $\complement B$, $\complement C$, $%
B\cap \complement A$, $A\cap B$, $A\cup B$ i $B\cap \complement C$. (b)
Expressa mitjan\c{c}ant conjunts les seg\"{u}ents enunciats: (1) Adults
casats, (2) Homes menors no casats i (3) Menors o homes casats.
\end{exercici}

\begin{solucio}
(a) Per definici\'{o} de complementari d'un conjunt, si $A$ \'{e}s el
conjunt d'adults, $B$ el de dones i $C$ el dels clients casats, llavors $%
\complement A$ \'{e}s el conjunt de menors, $\complement B$ el d'homes i $%
\complement C$ el dels no casats. Com que%
\begin{equation*}
\begin{array}{ccc}
x\in B\cap \complement A & \Longleftrightarrow & x\in B\text{ i }x\in
\complement A%
\end{array}%
\end{equation*}%
llavors $B\cap \complement A$ \'{e}s el conjunt de dones menors. \'{E}s clar
que $A\cap B$ \'{e}s el conjunt de dones adultes, i, $A\cup B$ \'{e}s el
conjunt de dones o homes adults. Com que%
\begin{equation*}
\begin{array}{ccc}
x\in B\cap \complement C & \Longleftrightarrow & x\in B\text{ i }x\in
\complement C%
\end{array}%
\end{equation*}%
llavors $B\cap \complement C$ \'{e}s el conjunt de dones no casades.

(b) Com que $A$ \'{e}s el conjunt d'adults i $C$ el dels casats, llavors $%
A\cap C$ \'{e}s el conjunt d'adults casats. \'{E}s clar que el conjunt
d'homes menors no casats \'{e}s%
\begin{equation*}
\complement B\cap \complement A\cap \complement C
\end{equation*}%
Finalment, el conjunt de menors o homes casats \'{e}s%
\begin{equation*}
\complement A\cup (\complement B\cap C)
\end{equation*}
\end{solucio}

\begin{exercici}
Donats tres conjunts qualssevol $A$, $B$ i $C$, demostra que es compleixen
les seg\"{u}ents relacions:

\begin{enumerate}
\item $A\cup A=A$ i $A\cap A=A$

\item $A\cup (B\cup C)=(A\cup B)\cup C$ i $A\cap (B\cap C)=(A\cap B)\cap C$

\item $A\cup B=B\cup A$ i $A\cap B=B\cap A$

\item $A\cup (B\cap C)=(A\cup B)\cap (A\cup B)$ i $A\cap (B\cup C)=(A\cap
B)\cup (A\cap C)$

\item $A\cup (B\cap A)=A$ i $A\cap (B\cup A)=A$

\item $A\cup \emptyset =A$ i $A\cap \emptyset =\emptyset $
\end{enumerate}
\end{exercici}

\begin{solucio}
Per a demostrar igualtats de conjunts hi ha dos m\`{e}todes. El primer
consisteix a utilitzar la propietat antisim\`{e}trica de la relaci\'{o}
d'inclusi\'{o}:%
\begin{equation*}
\begin{array}{ccc}
A=B & \Longleftrightarrow & A\subset B\text{ i }B\subset A%
\end{array}%
\end{equation*}%
i, el segon, consisteix a expressar primer les igualtats com a enunciats de l%
\`{o}gica proposicional i despr\'{e}s comprovar que es tracten de
tautologies. Utilitzarem aqu\'{\i} tots dos m\`{e}todes per a provar les
igualtats indicades.

(1) \'{E}s clar que $A\cup A=A$ i $A\cap A=A$ s'expressen com els seg\"{u}%
ents enunciats%
\begin{equation*}
\begin{array}{ccc}
x\in A\text{ o }x\in A & \Longleftrightarrow & x\in A%
\end{array}%
\end{equation*}%
i%
\begin{equation*}
\begin{array}{ccc}
x\in A\text{ i }x\in A & \Longleftrightarrow & x\in A%
\end{array}%
\end{equation*}%
tradu\"{\i}ts com a expressions formals de la l\`{o}gica d'enunciats, tenim%
\begin{equation*}
\begin{array}{ccc}
p\vee p & \longleftrightarrow & p%
\end{array}%
\text{ \ \ i \ \ }%
\begin{array}{ccc}
p\wedge p & \longleftrightarrow & p%
\end{array}%
\end{equation*}%
on $p$ est\`{a} en lloc de l'enunciat $x\in A$. Per a provar que s\'{o}n
tautologies hem de construir les taules de veritat de totes dues
proposicions i comprovar que en l'\'{u}ltima columna s\'{o}n tots $1$ (valor
de veritat). Aix\'{\i}, tenim%
\begin{tabular}{lll}
$p$ & $p\wedge p$ & $\left( p\wedge p\right) \longleftrightarrow p$ \\
$1$ & $1$ & $1$ \\
$0$ & $0$ & $1$%
\end{tabular}%
\begin{equation*}
\begin{tabular}{|l|l|l|}
\hline
$p$ & $p\vee p$ & $\left( p\vee p\right) \longleftrightarrow p$ \\ \hline
$1$ & $1$ & $1$ \\ \hline
$0$ & $0$ & $1$ \\ \hline
\end{tabular}%
\text{ \ \ i \ \ }%
\begin{tabular}{|l|l|l|}
\hline
$p$ & $p\wedge p$ & $\left( p\wedge p\right) \longleftrightarrow p$ \\ \hline
$1$ & $1$ & $1$ \\ \hline
$0$ & $0$ & $1$ \\ \hline
\end{tabular}%
\end{equation*}%
on $0$ \'{e}s el valor de falsedat. Per tant, totes dues igualtats s\'{o}n
vertaderes.

(2) \'{E}s clar que $A\cup (B\cup C)=(A\cup B)\cup C$ i $A\cap (B\cap
C)=(A\cap B)\cap C$ s'expressen com els seg\"{u}ents enunciats%
\begin{equation*}
\begin{array}{ccc}
x\in A\text{ o }\left( x\in B\text{ o }x\in C\right) & \Longleftrightarrow &
\left( x\in A\text{ o }x\in B\right) \text{ o }x\in C%
\end{array}%
\end{equation*}%
i%
\begin{equation*}
\begin{array}{ccc}
x\in A\text{ i }\left( x\in B\text{ i }x\in C\right) & \Longleftrightarrow &
\left( x\in A\text{ i }x\in B\right) \text{ i }x\in C%
\end{array}%
\end{equation*}%
que tradu\"{\i}ts com a expressions formals de la l\`{o}gica d'enunciats,
tenim%
\begin{equation*}
\begin{array}{ccc}
p\vee (q\vee r) & \longleftrightarrow & (p\vee q)\vee r%
\end{array}%
\end{equation*}%
i%
\begin{equation*}
\begin{array}{ccc}
p\wedge (q\wedge r) & \longleftrightarrow & (p\wedge q)\wedge r%
\end{array}%
\end{equation*}%
on $p$ est\`{a} en lloc de $x\in A$, $q$ en lloc de $x\in B$, i $r$ en lloc
de $x\in C$. Per a provar que s\'{o}n tautologies hem de construir les
taules de veritat de totes dues proposicions i comprovar que en l'\'{u}ltima
columna s\'{o}n tots $1$. Aix\'{\i}, tenim%
\begin{equation*}
\begin{tabular}{|l|l|l|l|l|l|l|l|}
\hline
$p$ & $q$ & $r$ & $p\vee q$ & $q\vee r$ & $p\vee (q\vee r)$ & $(p\vee q)\vee
r$ & $\left[ p\vee (q\vee r)\right] \longleftrightarrow \left[ (p\vee q)\vee
r\right] $ \\ \hline
$1$ & $1$ & $1$ & $1$ & $1$ & $1$ & $1$ & $1$ \\ \hline
$1$ & $1$ & $0$ & $1$ & $1$ & $1$ & $1$ & $1$ \\ \hline
$1$ & $0$ & $1$ & $1$ & $1$ & $1$ & $1$ & $1$ \\ \hline
$1$ & $0$ & $0$ & $1$ & $0$ & $1$ & $1$ & $1$ \\ \hline
$0$ & $1$ & $1$ & $1$ & $1$ & $1$ & $1$ & $1$ \\ \hline
$0$ & $1$ & $0$ & $1$ & $1$ & $1$ & $1$ & $1$ \\ \hline
$0$ & $0$ & $1$ & $0$ & $1$ & $1$ & $1$ & $1$ \\ \hline
$0$ & $0$ & $0$ & $0$ & $0$ & $0$ & $0$ & $1$ \\ \hline
\end{tabular}%
\end{equation*}%
i%
\begin{equation*}
\begin{tabular}{|l|l|l|l|l|l|l|l|}
\hline
$p$ & $q$ & $r$ & $p\wedge q$ & $q\wedge r$ & $p\wedge (q\wedge r)$ & $%
(p\wedge q)\wedge r$ & $\left[ p\wedge (q\wedge r)\right]
\longleftrightarrow \left[ (p\wedge q)\wedge r\right] $ \\ \hline
$1$ & $1$ & $1$ & $1$ & $1$ & $1$ & $1$ & $1$ \\ \hline
$1$ & $1$ & $0$ & $1$ & $0$ & $0$ & $0$ & $1$ \\ \hline
$1$ & $0$ & $1$ & $0$ & $0$ & $0$ & $0$ & $1$ \\ \hline
$1$ & $0$ & $0$ & $0$ & $0$ & $0$ & $0$ & $1$ \\ \hline
$0$ & $1$ & $1$ & $0$ & $1$ & $0$ & $0$ & $1$ \\ \hline
$0$ & $1$ & $0$ & $0$ & $0$ & $0$ & $0$ & $1$ \\ \hline
$0$ & $0$ & $1$ & $0$ & $0$ & $0$ & $0$ & $1$ \\ \hline
$0$ & $0$ & $0$ & $0$ & $0$ & $0$ & $0$ & $1$ \\ \hline
\end{tabular}%
\end{equation*}%
Per tant, totes dues igualtats s\'{o}n vertaderes.

(3) \'{E}s clar que $A\cup B=B\cup A$ i $A\cap B=B\cap A$ s'expressen com
els seg\"{u}ents enunciats%
\begin{equation*}
\begin{array}{ccc}
x\in A\text{ o }x\in B & \Longleftrightarrow & x\in B\text{ o }x\in A%
\end{array}%
\end{equation*}%
i%
\begin{equation*}
\begin{array}{ccc}
x\in A\text{ i }x\in B & \Longleftrightarrow & x\in B\text{ i }x\in A%
\end{array}%
\end{equation*}%
tradu\"{\i}ts com a expressions formals de la l\`{o}gica d'enunciats, tenim%
\begin{equation*}
\begin{array}{ccc}
p\vee q & \longleftrightarrow & q\vee p%
\end{array}%
\text{ \ \ i \ \ }%
\begin{array}{ccc}
p\wedge q & \longleftrightarrow & q\wedge p%
\end{array}%
\end{equation*}%
on $p$ est\`{a} en lloc de l'enunciat $x\in A$ i $q$ en lloc de $x\in B$ .
Per a provar que s\'{o}n tautologies hem de construir les taules de veritat
de totes dues proposicions i comprovar que en l'\'{u}ltima columna s\'{o}n
tots $1$.%
\begin{equation*}
\begin{tabular}{|l|l|l|l|l|}
\hline
$p$ & $q$ & $p\vee q$ & $q\vee p$ & $(p\vee q)\longleftrightarrow (q\vee p)$
\\ \hline
$1$ & $1$ & $1$ & $1$ & $1$ \\ \hline
$1$ & $0$ & $1$ & $1$ & $1$ \\ \hline
$0$ & $1$ & $1$ & $1$ & $1$ \\ \hline
$0$ & $0$ & $0$ & $0$ & $1$ \\ \hline
\end{tabular}%
\end{equation*}%
i%
\begin{equation*}
\begin{tabular}{|l|l|l|l|l|}
\hline
$p$ & $q$ & $p\wedge q$ & $q\wedge p$ & $(p\wedge q)\longleftrightarrow
(q\wedge p)$ \\ \hline
$1$ & $1$ & $1$ & $1$ & $1$ \\ \hline
$1$ & $0$ & $0$ & $0$ & $1$ \\ \hline
$0$ & $1$ & $0$ & $0$ & $1$ \\ \hline
$0$ & $0$ & $0$ & $0$ & $1$ \\ \hline
\end{tabular}%
\end{equation*}%
Per tant, totes dues igualtats s\'{o}n vertaderes.

(4) Aqu\'{\i}, utilitzarem el primer m\`{e}tode per a provar $A\cup (B\cap
C)=(A\cup B)\cap (A\cup B)$ i $A\cap (B\cup C)=(A\cap B)\cup (A\cap C)$. Aix%
\'{\i}, tenim%
\begin{equation*}
\begin{array}{ccc}
A\cup (B\cap C)=(A\cup B)\cap (A\cup B) & \Longleftrightarrow & \left\{
\begin{array}{c}
A\cup (B\cap C)\subset (A\cup B)\cap (A\cup B) \\
(A\cup B)\cap (A\cup B)\subset A\cup (B\cap C)%
\end{array}%
\right.%
\end{array}%
\end{equation*}%
Com que%
\begin{equation*}
\begin{array}{lll}
x\in A\cup (B\cap C) & \Longleftrightarrow & x\in A\text{ o }x\in B\cap C \\
& \Longleftrightarrow & x\in A\text{ o }\left( x\in B\text{ i }x\in C\right)
\\
& \Longleftrightarrow & \left( x\in A\text{ o }x\in B\right) \text{ i }%
\left( x\in A\text{ o }x\in C\right) \\
& \Longleftrightarrow & x\in A\cup B\text{ i }x\in A\cup C \\
& \Longleftrightarrow & x\in (A\cup B)\cap (A\cup C)%
\end{array}%
\end{equation*}%
Per tant, $A\cup (B\cap C)=(A\cup B)\cap (A\cup B)$.

D'altra banda, tenim%
\begin{equation*}
\begin{array}{ccc}
A\cap (B\cup C)=(A\cap B)\cup (A\cap C) & \Longleftrightarrow & \left\{
\begin{array}{c}
A\cap (B\cup C)\subset (A\cap B)\cup (A\cap C) \\
(A\cap B)\cup (A\cap C)\subset A\cap (B\cup C)%
\end{array}%
\right.%
\end{array}%
\end{equation*}%
Com que%
\begin{equation*}
\begin{array}{lll}
x\in A\cap (B\cup C) & \Longleftrightarrow & x\in A\text{ i }x\in B\cup C \\
& \Longleftrightarrow & x\in A\text{ i }\left( x\in B\text{ o }x\in C\right)
\\
& \Longleftrightarrow & \left( x\in A\text{ i }x\in B\right) \text{ o }%
\left( x\in A\text{ i }x\in C\right) \\
& \Longleftrightarrow & x\in A\cap B\text{ o }x\in A\cap C \\
& \Longleftrightarrow & x\in (A\cap B)\cup (A\cap C)%
\end{array}%
\end{equation*}%
Per tant, $A\cap (B\cup C)=(A\cap B)\cup (A\cap C)$.

(5) Com que
\begin{equation*}
\begin{array}{lll}
x\in A\cup (B\cap A) & \Longleftrightarrow & x\in A\text{ o }x\in B\cap A \\
& \Longleftrightarrow & x\in A\text{ o }\left( x\in B\text{ i }x\in A\right)
\\
& \Longleftrightarrow & \left( x\in A\text{ o }x\in B\right) \text{ i }%
\left( x\in A\text{ o }x\in A\right) \\
& \Longleftrightarrow & \left( x\in A\text{ o }x\in B\right) \text{ i }x\in A
\\
& \Longleftrightarrow & x\in A%
\end{array}%
\end{equation*}%
Aleshores, $A\cup (B\cap A)=A$. D'altra banda,%
\begin{equation*}
\begin{array}{lll}
x\in A\cap (B\cup A) & \Longleftrightarrow & x\in A\text{ i }x\in B\cup A \\
& \Longleftrightarrow & x\in A\text{ i }\left( x\in B\text{ o }x\in A\right)
\\
& \Longleftrightarrow & \left( x\in A\text{ i }x\in B\right) \text{ o }%
\left( x\in A\text{ i }x\in A\right) \\
& \Longleftrightarrow & \left( x\in A\text{ i }x\in B\right) \text{ o }x\in A
\\
& \Longleftrightarrow & x\in A%
\end{array}%
\end{equation*}%
Per tant, $A\cap (B\cup A)=A$.

(6) Com que%
\begin{equation*}
\begin{array}{lll}
x\in A\cup \emptyset & \Longleftrightarrow & x\in A\text{ o }x\in \emptyset
\\
& \Longleftrightarrow & x\in A%
\end{array}%
\end{equation*}%
Per tant, $A\cup \emptyset =A$. D'altra banda, si fos $A\cap \emptyset $ no
buit, existiria un element $x$ tal que $x\in A$ i $x\in \emptyset $, per\`{o}
aix\`{o} no \'{e}s possible, ja que $x\in \emptyset $ \'{e}s una relaci\'{o}
que sempre \'{e}s falsa. Per tant, no pot haver-hi cap element en $A\cap
\emptyset $ i, per tant, $A\cap \emptyset =\emptyset $.
\end{solucio}

\begin{exercici}
Prenent com univers $\mathbb{R}$, determina els complementaris dels seg\"{u}%
ents conjunts: $(2,+\infty )$, $(-\infty ,0]$, $(-3,1]$ i $[0.5,0.7]\cup
\lbrack 2,3)$.
\end{exercici}

\begin{solucio}
Per definici\'{o} de complementari d'un conjunt tenim%
\begin{align*}
\complement (2,+\infty )& =(-\infty ,2] \\
\complement (-\infty ,0]& =(0,+\infty ) \\
\complement (-3,1]& =(-\infty ,-3]\cup (1,+\infty ) \\
\complement \left( \lbrack 0.5,0.7]\cup \lbrack 2,3)\right) & =(-\infty
,0.5)\cup (0.7,2)\cup \lbrack 3,+\infty )
\end{align*}
\end{solucio}

\begin{exercici}
Si $E$ \'{e}s el conjunt referencial, demostra que es compleixen les seg\"{u}%
ents propietats:

\begin{enumerate}
\item $\complement E=\emptyset $ i $\complement \emptyset =E$

\item $\complement \left( \complement A\right) =A$

\item $\complement \left( A\cup B\right) =\complement A\cap \complement B$

\item $\complement \left( A\cap B\right) =\complement A\cup \complement B$
\end{enumerate}
\end{exercici}

\begin{solucio}
(1) Si $\complement E$ no fos buit, existiria un element $x$ tal que $%
x\notin E$, per\`{o} aix\`{o} no \'{e}s possible perqu\`{e} $E$ \'{e}s
l'univers. Per tant, no pot haver-hi cap element en $\complement E$ i, com a
consequ\`{e}ncia, $\complement E=\emptyset $.

Com que%
\begin{equation*}
\begin{array}{lll}
x\in \complement \emptyset & \Longleftrightarrow & x\in E\text{ i }x\notin
\emptyset \\
& \Longleftrightarrow & x\in E%
\end{array}%
\end{equation*}%
Per tant, $\complement \emptyset =E$.

(2) \'{E}s clar que $\complement \left( \complement A\right) =A$ es tradueix
com la seg\"{u}ent expressi\'{o} formal de la l\`{o}gica d'enunciats,%
\begin{equation*}
\lnot \lnot p\longleftrightarrow p
\end{equation*}%
on $p$ est\`{a} en lloc de l'enunciat $x\in A$. Per a provar que \'{e}s una
tautologia hem de construir la taula de veritat i comprovar que en l'\'{u}%
ltima columna s\'{o}n tots $1$. Aix\'{\i}, tenim%
\begin{equation*}
\begin{tabular}{|c|c|c|c|}
\hline
$p$ & $\lnot p$ & $\lnot \lnot p$ & $\lnot \lnot p\longleftrightarrow p$ \\
\hline
$1$ & $0$ & $1$ & $1$ \\ \hline
$0$ & $1$ & $0$ & $1$ \\ \hline
\end{tabular}%
\end{equation*}%
Per tant, la igualtat $\complement \left( \complement A\right) =A$ \'{e}s
certa.

(3) Com que%
\begin{equation*}
\begin{array}{lll}
x\in \complement \left( A\cup B\right) & \Longleftrightarrow & x\in E\text{
i }x\notin A\cup B \\
& \Longleftrightarrow & x\in E\text{ i }\left( x\notin A\text{ i }x\notin
B\right) \\
& \Longleftrightarrow & \left( x\in E\text{ i }x\notin A\right) \text{ i }%
\left( x\in E\text{ i }x\notin B\right) \\
& \Longleftrightarrow & x\in \complement A\text{ i }x\in \complement B \\
& \Longleftrightarrow & x\in \complement A\cap \complement B%
\end{array}%
\end{equation*}%
Per tant, $\complement \left( A\cup B\right) =\complement A\cap \complement
B $.

(4) \'{E}s clar que $\complement \left( A\cap B\right) =\complement A\cup
\complement B$ es tradueix com la seg\"{u}ent expressi\'{o} formal de la l%
\`{o}gica d'enunciats,%
\begin{equation*}
\lnot (p\wedge q)\longleftrightarrow \lnot p\vee \lnot q
\end{equation*}%
on $p$ est\`{a} en lloc de l'enunciat $x\in A$ i $q$ de $x\in B$. Per a
provar que \'{e}s una tautologia hem de construir la taula de veritat i
comprovar que en l'\'{u}ltima columna s\'{o}n tots $1$. Aix\'{\i}, tenim%
\begin{equation*}
\begin{tabular}{|c|c|c|c|c|c|c|c|}
\hline
$p$ & $q$ & $\lnot p$ & $\lnot q$ & $p\wedge q$ & $\lnot (p\wedge q)$ & $%
\lnot p\vee \lnot q$ & $\lnot (p\wedge q)\longleftrightarrow \lnot p\vee
\lnot q$ \\ \hline
$1$ & $1$ & $0$ & $0$ & $1$ & $0$ & $0$ & $1$ \\ \hline
$1$ & $0$ & $0$ & $1$ & $0$ & $1$ & $1$ & $1$ \\ \hline
$0$ & $1$ & $1$ & $0$ & $0$ & $1$ & $1$ & $1$ \\ \hline
$0$ & $0$ & $1$ & $1$ & $0$ & $1$ & $1$ & $1$ \\ \hline
\end{tabular}%
\end{equation*}%
Per tant, la igualtat $\complement \left( A\cap B\right) =\complement A\cup
\complement B$ \'{e}s vertadera.
\end{solucio}

\begin{exercici}
Suposant que $E$ \'{e}s el conjunt referencial, simplifica les seg\"{u}ents
expressions:

\begin{enumerate}
\item $(A\cap \complement B)\cap (\complement A\cap \complement B)$

\item $(A\cap B\cap C)\cup (\complement A\cup \complement B\cup \complement
C)$

\item $\left[ A\cap (\complement A\cup B)\right] \cup \left[ B\cap (B\cup C)%
\right] \cup B$
\end{enumerate}
\end{exercici}

\begin{solucio}
En tots aquests apartats aplicarem les propietats de la uni\'{o} i intersecci%
\'{o} entre conjunts i del complementari d'un conjunt.

(1) Aix\'{\i} tenim%
\begin{align*}
(A\cap \complement B)\cap (\complement A\cap \complement B)& =(A\cap
\complement B)\cap (\complement B\cap \complement A) \\
& =A\cap (\complement B\cap \complement B)\cap \complement A \\
& =A\cap \complement B\cap \complement A \\
& =A\cap (\complement B\cap \complement A) \\
& =A\cap (\complement A\cap \complement B) \\
& =(A\cap \complement A)\cap \complement B \\
& =\emptyset \cap \complement B \\
& =\emptyset
\end{align*}

(2) Aix\'{\i}, tenim%
\begin{align*}
(A\cap B\cap C)\cup (\complement A\cup \complement B\cup \complement C)&
=(A\cap B\cap C)\cup (\complement \left( A\cap B\right) \cup \complement C)
\\
& =(A\cap B\cap C)\cup \complement \left( (A\cap B)\cap C\right) \\
& =(A\cap B\cap C)\cup \complement \left( A\cap B\cap C\right) \\
& =E
\end{align*}

(3) Aix\'{\i}, tenim%
\begin{align*}
\left[ A\cap (\complement A\cup B)\right] \cup \left[ B\cap (B\cup C)\right]
\cup B& =\left[ (A\cap \complement A)\cup (A\cap B)\right] \cup \left[
(B\cap B)\cup (B\cap C)\right] \cup B \\
& =\emptyset \cup (A\cap B)\cup B\cup (B\cap C)\cup B \\
& =(A\cap B)\cup (B\cap C)\cup B \\
& =(A\cap B)\cup \left[ (B\cap C)\cup B\right] \\
& =(A\cap B)\cup B \\
& =B
\end{align*}
\end{solucio}

\begin{exercici}
Si $A=\left\{ 1,2\right\} $, $B=\left\{ 2,4\right\} $ i $C=\left\{ a\right\}
$, determina els conjunts seg\"{u}ents: (a) $A\times (B\cup C)$; (b) $%
(C\times A)\cap (C\times B)$.
\end{exercici}

\begin{solucio}
(a) \'{E}s clar que%
\begin{equation*}
B\cup C=\left\{ 2,4,a\right\}
\end{equation*}%
Llavors, segons la definici\'{o} de producte cartesi\`{a} de dos conjunts,
obtenim%
\begin{equation*}
A\times (B\cup C)=\left\{ (1,2),(1,4),(1,a),(2,2),(2,4),(2,a)\right\}
\end{equation*}

(b) De la mateixa manera obtenim%
\begin{equation*}
C\times A=\left\{ (a,1),(a,2)\right\} \text{ \ \ \ y \ \ \ }C\times
B=\left\{ (a,2),(a,4)\right\}
\end{equation*}%
Despr\'{e}s,%
\begin{equation*}
(C\times A)\cap (C\times B)=\left\{ (a,2)\right\}
\end{equation*}
\end{solucio}

\begin{exercici}
Representa gr\`{a}ficament el conjunt $A\times B$, sabent que $A=\left\{
x\in \mathbb{R}:-3\leq x\leq 1\right\} $ i $B=\left\{ x\in \mathbb{R}:0\leq
x\leq 2\right\} $.
\end{exercici}

\begin{solucio}
Segons la definici\'{o} de producte cartesi\`{a} de dos conjunts, obtenim%
\begin{equation*}
A\times B=\left\{ (x,y)\in \mathbb{R}\times \mathbb{R}:-3\leq x\leq 1\text{
\ i \ }0\leq y\leq 2\right\}
\end{equation*}%
Gr\`{a}ficament, aquest conjunt representa la regi\'{o} del pla assenyalada
en la figura seg\"{u}ent\FRAME{dtbpF}{1.433in}{1.0378in}{0pt}{}{}{set8.jpg}{%
\special{language "Scientific Word";type "GRAPHIC";maintain-aspect-ratio
TRUE;display "USEDEF";valid_file "F";width 1.433in;height 1.0378in;depth
0pt;original-width 1.4062in;original-height 1.0101in;cropleft "0";croptop
"1";cropright "1";cropbottom "0";filename
'../../conjunts/set8.jpg';file-properties "XNPEU";}}
\end{solucio}

\section{Relacions}

\begin{exercici}
Estudiar les propietats de les seg\"{u}ents relacions:

\begin{enumerate}
\item \textquotedblleft Ser divisor de\textquotedblright\ en el conjunt dels
nombres naturals.

\item \textquotedblleft Ser quadrat de\textquotedblright\ en el conjunt dels
nombres naturals.

\item \textquotedblleft Tenir igual \`{a}rea que\textquotedblright\ en el
conjunt dels triangles del pla.

\item \textquotedblleft Ser perpendicular\textquotedblright\ en el conjunt
de les rectes de l'espai.

\item \textquotedblleft Tenir el mateix color d'ulls que\textquotedblright\
en el conjunt dels habitants de la terra.
\end{enumerate}
\end{exercici}

\begin{solucio}
(1) En $\mathbb{N}$ considerem la relaci\'{o} \textquotedblleft Ser divisor
de\textquotedblright . \'{E}s evident que la relaci\'{o} \'{e}s reflexiva
(Tot nombre natural \'{e}s divisor de si mateix) i, per tant, no \'{e}s
irreflexiva ni asim\`{e}trica. Tampoc \'{e}s sim\`{e}trica (Per exemple, $3$
\'{e}s divisor d'i, $6$ en canvi, $6$ no \'{e}s divisor de $3$). \'{E}s
evident que la relaci\'{o} \'{e}s antisim\`{e}trica i transitiva.

(2) En $\mathbb{N}$ considerem la relaci\'{o} \textquotedblleft Ser quadrat
de\textquotedblright . \'{E}s evident que la relaci\'{o} no \'{e}s reflexiva
(Per exemple, $3$ no \'{e}s quadrat de $3$). Tampoc \'{e}s irreflexiva ja
que $1$ \'{e}s quadrat de si mateix. No \'{e}s asim\`{e}trica ni sim\`{e}%
trica per\`{o}, en canvi, s\'{\i} que \'{e}s antisim\`{e}trica. No \'{e}s
transitiva (Si $a=b^{2}$ i $b=c^{2}$, llavors $a=(c^{2})^{2}=c^{4}\neq c^{2}$%
).

(3) En el conjunt dels triangles del pla considerem la relaci\'{o}
\textquotedblleft Tenir igual \`{a}rea que\textquotedblright . \'{E}s
evident que aquesta relaci\'{o} \'{e}s reflexiva, sim\`{e}trica i transitiva.

(4) En el conjunt de les rectes de l'espai considerem la relaci\'{o}
\textquotedblleft Ser perpendicular\textquotedblright . \'{E}s evident que
la relaci\'{o} no \'{e}s reflexiva (Cap recta \'{e}s perpendicular a si
mateixa). \'{E}s irreflexiva, sim\`{e}trica i no transitiva, com pot
comprovar-se de seguida.

(5) En conjunt dels habitants de la terra considerem la relaci\'{o}
\textquotedblleft Tenir el mateix color d'ulls que\textquotedblright . \'{E}%
s evident que aquesta relaci\'{o} \'{e}s reflexiva, sim\`{e}trica i
transitiva.
\end{solucio}

\begin{exercici}
De les seg\"{u}ents relacions bin\`{a}ries $R$, esbrina quins s\'{o}n
d'equival\`{e}ncia i descriu les seves classes: (a)En $\mathbb{R}$, $\left(
x,y\right) \in R$ si i nom\'{e}s si $\left\vert x-y\right\vert <1$; (b) En $%
\mathbb{R}-\left\{ 0\right\} $, $\left( x,y\right) \in R$ si i nom\'{e}s si $%
\frac{x}{y}\in \mathbb{Q}$.
\end{exercici}

\begin{solucio}
(a) La relaci\'{o} \'{e}s reflexiva, doncs, per a tot $x\in \mathbb{R}$ es
compleix%
\begin{equation*}
\left\vert x-x\right\vert =0<1
\end{equation*}%
Tamb\'{e} \'{e}s sim\`{e}trica, doncs, si $\left\vert x-y\right\vert <1$,
llavors%
\begin{equation*}
\left\vert x-y\right\vert =\left\vert -\left( x-y\right) \right\vert
=\left\vert y-x\right\vert <1
\end{equation*}%
En canvi, la relaci\'{o} no \'{e}s transitiva, ja que $\left( -1,-0.5\right)
\in R$ i $\left( -0.5,0.25\right) \in R$ i, en canvi, $\left( -1,0.25\right)
\notin R$ perqu\`{e}%
\begin{equation*}
\left\vert -1-0.25\right\vert =1.25>1
\end{equation*}%
Per conseg\"{u}ent, aquesta relaci\'{o} no \'{e}s d'equival\`{e}ncia.

(b) La relaci\'{o} \'{e}s evidentment reflexiva, sim\`{e}trica i transitiva.
Per tant, la relaci\'{o} \'{e}s d'equival\`{e}ncia. La classe d'un element
arbitrari $a\in \mathbb{R}-\left\{ 0\right\} $ \'{e}s%
\begin{align*}
\left[ a\right] & =\left\{ x\in \mathbb{R}-\left\{ 0\right\} :\frac{x}{a}\in
\mathbb{Q}\right\} \\
& =\left\{ x\in \mathbb{R}-\left\{ 0\right\} :\frac{x}{a}=q\text{ \ i \ }%
q\in \mathbb{Q}-\left\{ 0\right\} \right\} \\
& =\left\{ a\cdot q:q\in \mathbb{Q}-\left\{ 0\right\} \right\}
\end{align*}%
Per tant, si $b\in \mathbb{Q}-\left\{ 0\right\} $, llavors%
\begin{equation*}
\left[ b\right] =\left\{ b\cdot q:q\in \mathbb{Q}-\left\{ 0\right\} \right\}
=\mathbb{Q}-\left\{ 0\right\}
\end{equation*}%
En resum, hi ha dos tipus de classes d'equival\`{e}ncia: les classes de la
forma $\left[ a\right] $ amb $a\in \mathbb{R}-\mathbb{Q}$ i la classe
formada per tots els nombres racionals no nuls.
\end{solucio}

\begin{exercici}
En $\mathbb{Z}$ es defineix la seg\"{u}ent relaci\'{o}%
\begin{equation*}
\begin{array}{ccc}
x\equiv y & \Longleftrightarrow & x-y\text{ \ \'{e}s m\'{u}ltiple de }5\text{%
.}%
\end{array}%
\end{equation*}%
(a) Demostra que $\equiv $ \'{e}s una relaci\'{o} d'equival\`{e}ncia, (b)
troba el conjunt quocient $\mathbb{Z}/\equiv $; (c) calcula un representant $%
x$ de la classe a la que pertany $127$ que compleixi $8<x<15$ i un
representant $y$ a la que pertany $-34$ que compleixi $5<y<10$.
\end{exercici}

\begin{solucio}
(a) La relaci\'{o} $\equiv $ \'{e}s d'equival\`{e}ncia, perqu\`{e} es
compleixen les propietats

\begin{enumerate}
\item Reflexiva: $x\equiv x$, per a tot $x\in \mathbb{Z}$, ja que $x-x=0$
\'{e}s m\'{u}ltiple de $5$.

\item Sim\`{e}trica: $x\equiv y$ implica $y\equiv x$ per a tot $x,y\in
\mathbb{Z}$, ja que si $x-y$ \'{e}s m\'{u}ltiple de $5$, tamb\'{e} ho \'{e}s
$-(x-y)=y-x$.

\item Transitiva: $x\equiv y$ i $y\equiv z$ implica $x\equiv z$ per a tot $%
x,y,z\in \mathbb{Z}$, ja que si $x-y$ i $y-z$ s\'{o}n m\'{u}ltiples de $5$,
tamb\'{e} ho \'{e}s la seva suma $x-z$.
\end{enumerate}

(b) Considerem un nombre enter arbitrari $a$ i determinem la seva classe
d'equival\`{e}ncia. Tenim,%
\begin{align*}
\left[ a\right] & =\left\{ x\in \mathbb{Z}:x\equiv a\right\} \\
& =\left\{ x\in \mathbb{Z}:x-a\text{ \ \'{e}s m\'{u}ltiple de }5\right\} \\
& =\left\{ x\in \mathbb{Z}:x-a=5k\text{ \ i \ }k\in \mathbb{Z}\right\} \\
& =\left\{ a+5k:k\in \mathbb{Z}\right\} \text{.}
\end{align*}%
Per tant, distingim 5 classes d'equival\`{e}ncia:%
\begin{equation*}
\begin{array}{c}
\left[ 0\right] =\left\{ 0+5k:k\in \mathbb{Z}\right\} =\left\{
...,-10,-5,0,5,10,...\right\} \\
\left[ 1\right] =\left\{ 1+5k:k\in \mathbb{Z}\right\} =\left\{
...,-9,-4,1,6,11,...\right\} \\
\left[ 2\right] =\left\{ 2+5k:k\in \mathbb{Z}\right\} =\left\{
...,-8,-3,2,7,12,...\right\} \\
\left[ 3\right] =\left\{ 3+5k:k\in \mathbb{Z}\right\} =\left\{
...,-7,-2,3,8,13,...\right\} \\
\left[ 4\right] =\left\{ 4+5k:k\in \mathbb{Z}\right\} =\left\{
...,-6,-1,4,9,14,...\right\}%
\end{array}%
\end{equation*}%
Per tant, el conjunt quocient \'{e}s%
\begin{equation*}
\mathbb{Z}/\equiv ~=\left\{ \left[ 0\right] ,\left[ 1\right] ,\left[ 2\right]
,\left[ 3\right] ,\left[ 4\right] \right\} \text{.}
\end{equation*}

(c) \'{E}s clar que%
\begin{equation*}
127=2+5\cdot 25
\end{equation*}%
i, per tant, $127\in \left[ 2\right] $. Aleshores, un representant $x$ de la
classe a la que pertany $127$ que compleixi $8<x<15$ \'{e}s $12$. De la
mateixa manera, observa primer que%
\begin{equation*}
-34=-4+5\cdot (-6)
\end{equation*}%
i, per tant, $-34\in \left[ -4\right] =\left[ 1\right] $. Per tant, un
representant $y$ a la que pertany $-34$ que compleixi $5<y<10$ \'{e}s $6$.
\end{solucio}

\begin{exercici}
Sigui $A=\left\{ 0,1,2,3,...\right\} $ i considerem en el conjunt $A\times A$
la seg\"{u}ent relaci\'{o}%
\begin{equation*}
\begin{array}{ccc}
(a,b)\sim (c,d) & \Longleftrightarrow & a+d=b+c%
\end{array}%
\end{equation*}%
(a) Demostra que $\sim $ \'{e}s una relaci\'{o} d'equival\`{e}ncia i (b)
troba el conjunt quocient.
\end{exercici}

\begin{solucio}
(a) La relaci\'{o} $\sim $ \'{e}s d'equival\`{e}ncia, doncs es compleixen
les propietats:

\begin{itemize}
\item Reflexiva: En efecte, per a tot $(a,b)\in A\times A$, es compleix $%
a+b=b+a$ i, per tant, $(a,b)\sim (a,b)$.

\item Sim\`{e}trica: En efecte, per a tot $(a,b),(c,d)\in A\times A$, si $%
(a,b)\sim (c,d)$, \'{e}s a dir si $a+d=b+c$, llavors $c+b=d+a$ i, per tant, $%
(c,d)\sim (a,b)$.

\item Transitiva: En efecte, per a tot $(a,b),(c,d),(e,f)\in A\times A$, si $%
(a,b)\sim (c,d)$ i $(c,d)\sim (e,f)$, \'{e}s a dir si%
\begin{equation*}
\begin{array}{c}
a+d=b+c \\
c+f=d+e%
\end{array}%
\end{equation*}%
llavors, sumant membre a membre totes dues igualtats, obtenim%
\begin{equation*}
a+d+c+f=b+c+d+e
\end{equation*}%
i d'aqu\'{\i}, simplificant, obtenim%
\begin{equation*}
a+f=b+e
\end{equation*}%
i, per tant, $(a,b)\sim (e,f)$.
\end{itemize}

(b) Considerem un element arbitrari $(a,b)$ de $A\times A$ i determinem la
seva classe d'equival\`{e}ncia. Observa primer que%
\begin{equation*}
(a,b)\sim (a+k,b+k)
\end{equation*}%
per a tot $k\in A$ i, per tant,%
\begin{equation*}
\left[ (a,b)\right] =\left[ (a+k,b+k)\right]
\end{equation*}%
per a tot $k\in A$. D'aqu\'{\i}, obtenim que%
\begin{equation*}
\begin{array}{l}
\left[ (0,0)\right] =\left\{ (0,0),(1,1),(2,2),...\right\} \\
\left[ (1,0)\right] =\left\{ (1,0),(2,1),(3,2),...\right\} \\
\left[ (0,1)\right] =\left\{ (0,1),(1,2),(2,3),...\right\} \\
\left[ (2,0)\right] =\left\{ (2,0),(3,1),(4,2),...\right\} \\
\left[ (0,2)\right] =\left\{ (0,2),(1,3),(2,4),...\right\} \\
\multicolumn{1}{c}{\vdots}%
\end{array}%
\end{equation*}%
s\'{o}n classes d'equival\`{e}ncia d'aquesta relaci\'{o}.

En general,

\begin{itemize}
\item si $a>b$, llavors es compleix%
\begin{equation*}
\left[ (a,b)\right] =\left[ (a-b,0)\right]
\end{equation*}

\item si $a=b$, llavors%
\begin{equation*}
\left[ (a,b)\right] =\left[ (0,0)\right]
\end{equation*}

\item si $a<b$, llavors%
\begin{equation*}
\left[ (a,b)\right] =\left[ (0,b-a)\right]
\end{equation*}
\end{itemize}

Per conseg\"{u}ent, hi ha tantes classes d'equival\`{e}ncia en el conjunt
quocient com a nombres enters.%
\begin{equation*}
\begin{array}{ccc}
\left[ (0,0)\right] & \text{es correspon amb} & 0 \\
\left[ (1,0)\right] &  & 1 \\
\left[ (0,1)\right] &  & -1 \\
\left[ (2,0)\right] &  & 2 \\
\left[ (0,2)\right] &  & -2 \\
\vdots &  & \vdots%
\end{array}%
\end{equation*}
\end{solucio}

\begin{exercici}
En el conjunt $\mathbb{Z}\times \left( \mathbb{Z}-\left\{ 0\right\} \right) $
es defineix la seg\"{u}ent relaci\'{o}%
\begin{equation*}
\begin{array}{ccc}
(a,b)\sim (c,d) & \Longleftrightarrow & ad=bc%
\end{array}%
\end{equation*}%
(a) Demostra que $\sim $ \'{e}s una relaci\'{o} d'equival\`{e}ncia i (b)
Troba el conjunt quocient.
\end{exercici}

\begin{solucio}
(a) La relaci\'{o} $\sim $ \'{e}s d'equival\`{e}ncia, perqu\`{e} es
compleixen les propietats:

\begin{itemize}
\item Reflexiva: En efecte, per a tot $(a,b)\in \mathbb{Z}\times \left(
\mathbb{Z}-\left\{ 0\right\} \right) $ es compleix $ab=ba$ i, per tant, $%
(a,b)\sim (a,b)$.

\item Sim\`{e}trica: En efecte, per a tot $(a,b),(c,d)\in \mathbb{Z}\times
\left( \mathbb{Z}-\left\{ 0\right\} \right) $, si $(a,b)\sim (c,d)$, \'{e}s
a dir si $ad=bc$, llavors $cb=da$ i, per tant, $(c,d)\sim (a,b)$.

\item Transitiva: En efecte, per a tot $(a,b),(c,d),(e,f)\in \mathbb{Z}%
\times \left( \mathbb{Z}-\left\{ 0\right\} \right) $, si $(a,b)\sim (c,d)$ y
$(c,d)\sim (e,f)$, o sigui si%
\begin{equation*}
\begin{array}{c}
ad=bc \\
cf=de%
\end{array}%
\end{equation*}%
Llavors, multiplicat la primera igualtat per $f\neq 0$, s'obt\'{e}%
\begin{equation*}
adf=bcf
\end{equation*}%
D'aqu\'{\i}, mitjan\c{c}ant la segona igualtat, obtenim%
\begin{equation*}
adf=bde
\end{equation*}%
Ara, dividint per $d\neq 0$, es dedueix%
\begin{equation*}
af=be
\end{equation*}%
i, per tant, $(a,b)\sim (e,f)$.
\end{itemize}

(b) Considerem un element arbitrari $(a,b)$ de $\mathbb{Z}\times \left(
\mathbb{Z}-\left\{ 0\right\} \right) $ i determinem la seva classe d'equival%
\`{e}ncia. Observa primer que%
\begin{equation*}
(a,b)\sim (ak,bk)
\end{equation*}%
per tot $k\in \mathbb{Z}-\left\{ 0\right\} $ i, per tant,%
\begin{equation*}
\left[ (a,b)\right] =\left[ (ak,bk)\right]
\end{equation*}%
per tot $k\in \mathbb{Z}-\left\{ 0\right\} $. D'aqu\'{\i}, s'obt\'{e}%
\begin{equation*}
\begin{array}{l}
\left[ (0,1)\right] =\left\{ ...,(0,-2),(0,-1),(0,1),(0,2),...\right\} \\
\left[ (1,1)\right] =\left\{ ...,(-2,-2),(-1,-1),(1,1),(2,2),...\right\} \\
\left[ (1,2)\right] =\left\{ ...,(-2,-4),(-1,-2),(1,2),(2,4),...\right\} \\
\left[ (2,1)\right] =\left\{ ...,(-4,-2),(-2,-1),(2,1),(4,2),...\right\} \\
\multicolumn{1}{c}{\vdots}%
\end{array}%
\end{equation*}%
s\'{o}n classes d'equival\`{e}ncia d'aquesta relaci\'{o}.

En general, hi ha tantes classes d'equival\`{e}ncia en el conjunt quocient
com a n\'{u}meros de la forma $a/b$, amb $a\in \mathbb{Z}$ i $b\in \mathbb{Z}%
-\left\{ 0\right\} $, \'{e}s a dir, com a nombres racionals.%
\begin{equation*}
\begin{array}{ccc}
\left[ (0,1)\right] & \text{es correspon amb} & ...=\frac{0}{-1}=\frac{0}{1}%
=... \\
\left[ (1,1)\right] &  & ...=\frac{-1}{-1}=\frac{1}{1}=... \\
\left[ (1,2)\right] &  & ...=\frac{-1}{-2}=\frac{1}{2}=... \\
\left[ (2,1)\right] &  & ...=\frac{-2}{-1}=\frac{2}{1}=... \\
\vdots &  & \vdots%
\end{array}%
\end{equation*}
\end{solucio}

\begin{exercici}
(a) Demostra que la seg\"{u}ent relaci\'{o}%
\begin{equation*}
\begin{array}{ccc}
x\sim y & \Longleftrightarrow & \text{existeix }n\in \mathbb{Z}\text{ \ tal
que }x,y\in (n-1,n]%
\end{array}%
\end{equation*}%
\'{e}s d'equival\`{e}ncia en $\mathbb{R}$. Quines s\'{o}n les seves classes
d'equival\`{e}ncia? (b) Demostra que els intervals de la forma $(n,n+1]$,
amb $n\in \mathbb{Z}$, constitueixen una partici\'{o} de la recta real.
\end{exercici}

\begin{solucio}
(a) La relaci\'{o} $\sim $ \'{e}s d'equival\`{e}ncia, perqu\`{e} es
compleixen les propietats:

\begin{itemize}
\item Reflexiva: Donat qualsevol $x\in \mathbb{R}$, si $n$ \'{e}s el menor
nombre enter que \'{e}s major o igual que $x$, llavors $x\in (n-1,n]$ i, per
tant, $x\sim x$.

\item Sim\`{e}trica: \'{E}s evident que $x\sim y$ implica $y\sim x$ per a
tot $x,y\in \mathbb{R}$.

\item Transitiva: En efecte, si $x\sim y$, llavors existeix $n\in \mathbb{Z}$
tal que $x,y\in (n-1,n]$. A m\'{e}s, si $y\sim z$, llavors tamb\'{e} es
compleix que $y,z\in (n-1,n]$. Aleshores, $x,z\in (n-1,n]$ i, per tant, $%
x\sim z$.
\end{itemize}

\'{E}s clar que la classe d'equival\`{e}ncia de qualsevol $a\in \mathbb{R}$
\'{e}s l'interval $(n-1,n]$ tal que $a\in (n-1,n]$. A m\'{e}s, qualsevol
altre nombre real d'aquest interval est\`{a} relacionat amb $a$ i, per tant,
la seva classe coincideix amb la de $a$. En definitiva, les classes del
conjunt quocient s\'{o}n els intervals de la forma $(n-1,n]$ amb $n\in
\mathbb{Z}$.

(b) En tractar-se d'una relaci\'{o} d'equival\`{e}ncia, el conjunt quocient
format pels intervals de la forma $(n-1,n]$, amb $n\in \mathbb{Z}$,
constitueixen una partici\'{o} de $\mathbb{R}$.
\end{solucio}

\begin{exercici}
Es considera en $\mathbb{R}$ la relaci\'{o} "menor o igual que" designada
per $\leq $. Comprova que $\leq $ \'{e}s una relaci\'{o} d'ordre. \'{E}s
total o parcial? Hi ha algun element maximal? Hi ha algun element minimal?
\end{exercici}

\begin{solucio}
La relaci\'{o} \'{e}s d'ordre parcial ja que es compleixen les propietats:

\begin{itemize}
\item Reflexiva: Per a tot $x\in \mathbb{R}$, \'{e}s evident que $x\leq x$.

\item Antisim\`{e}trica: Per a tot $x,y\in \mathbb{R}$, si $x\leq y$ i $%
y\leq x$, \'{e}s clar que $x=y$.

\item Transitiva: Per a tot $x,y,z\in \mathbb{R}$, si $x\leq y$ i $y\leq z$,
llavors \'{e}s evident que $x\leq z$.
\end{itemize}

La relaci\'{o} \'{e}s d'ordre total ja que per a qualsevol parell d'elements
$x,y\in \mathbb{R}$ es compleix $x\leq y$ o b\'{e} $y\leq x$. \'{E}s clar
que no hi ha elements maximals ni minimals en aquesta relaci\'{o}.
\end{solucio}

\begin{exercici}
En el conjunt $\mathcal{P}(A)$ de les parts d'un conjunt donat $A$ es
considera la relaci\'{o} d'inclusi\'{o} $\subset $. Comprova que $\subset $
\'{e}s una relaci\'{o} d'ordre. \'{E}s total o parcial? Hi ha algun element
maximal? Hi ha algun element minimal?
\end{exercici}

\begin{solucio}
La relaci\'{o} \'{e}s d'ordre parcial ja que es compleixen les propietats:

\begin{itemize}
\item Reflexiva: Per a tot $X\in \mathcal{P}(A)$, \'{e}s evident que $%
X\subset X$.

\item Antisim\`{e}trica: Per a tot $X,Y\in \mathcal{P}(A)$, si $X\subset Y$
i $Y\subset X$, \'{e}s clar que $X=Y$.

\item Transitiva: Per a tot $X,Y,Z\in \mathcal{P}(A)$, si $X\subset Y$ i $%
Y\subset Z$, llavors \'{e}s evident que $X\subset Z$.
\end{itemize}

La relaci\'{o} no \'{e}s d'ordre total ja que, per exemple, si $X\in
\mathcal{P}(A)$, llavors $A-X\in \mathcal{P}(A)$ i $X$ no \'{e}s subconjunt
de $A-X$ ni $A-X$ \'{e}s subconjunt de $X$. \'{E}s evident que $A$ \'{e}s un
element maximal i $\emptyset $ \'{e}s un element minimal en $\mathcal{P}(A)$
segons aquesta relaci\'{o}.
\end{solucio}

\begin{exercici}
Es considera $\mathbb{R}$ amb l'ordre usual $\leq $ i els subconjunts seg%
\"{u}ents: (1) $\mathbb{Z}$; (2) $(0,2]\cup (3,5]$\ ; (3) $(-\infty ,-2)\cup
\lbrack 13,19)$. Calcula (a) els extrems superiors i inferiors i (b) els m%
\`{a}xims i m\'{\i}nims, si existeixen.
\end{exercici}

\begin{solucio}
\'{E}s clar que%
\begin{equation*}
\begin{tabular}{|l|l|l|l|}
\hline
& $\mathbb{Z}$ & $(0,2]\cup (3,5]$ & $(-\infty ,-2)\cup \lbrack 13,19)$ \\
\hline
Suprem & No existeix & $5$ & $19$ \\ \hline
\'{I}nfim & No existeix & $0$ & No existeix \\ \hline
M\`{a}xim & No existeix & $5$ & No existeix \\ \hline
M\'{\i}nim & No existeix & No existeix & No existeix \\ \hline
\end{tabular}%
\end{equation*}
\end{solucio}

\begin{exercici}
En el conjunt $A=\left\{ 1,2,3,4,5,6,15,60\right\} $ es defineix la relaci%
\'{o}%
\begin{equation*}
\begin{array}{ccc}
a\mid b & \Longleftrightarrow & a\text{ \'{e}s divisor de }b%
\end{array}%
\end{equation*}%
(a) Demostra que $\mid $ \'{e}s una relaci\'{o} d'ordre en $A$. \'{E}s total
o parcial? (b) Troba el m\`{a}xim, m\'{\i}nim, suprem i \'{\i}nfim del
conjunt $B=\left\{ 2,3,6,15\right\} $. (c) Calcula el m\`{a}xim, m\'{\i}nim,
suprem i \'{\i}nfim de $A$ . (d) Hi ha elements maximals i minimals en $A?$
\end{exercici}

\begin{solucio}
(a) La relaci\'{o} $\mid $ \'{e}s d'ordre, ja que es compleixen les
propietats:

\begin{itemize}
\item Reflexiva: Per a tot $x\in A$ \'{e}s evident que es compleix $x\mid x$.

\item Antisim\`{e}trica: Per a tot $x,y\in A$, si $x\mid y$ i $y\mid x$,
\'{e}s clar que $x=y$.

\item Transitiva: Per a tot $x,y,z\in A$, si $x\mid y$ i $y\mid z$, \'{e}s
tamb\'{e} clar que $x\mid z$.
\end{itemize}

La relaci\'{o} no \'{e}s d'ordre total ja que $4,5\in A$ i $4\nmid 5$ ni $%
5\nmid 4$.

(b) Una representaci\'{o} gr\`{a}fica d'aquesta relaci\'{o} \'{e}s%
\begin{equation*}
\begin{array}{ccccc}
&  & 60 &  &  \\
& \diagup & | & \diagdown &  \\
4 &  & 6 &  & 15 \\
| & \diagup & | & \diagup & | \\
2 &  & 3 &  & 5 \\
& \diagdown & | & \diagup &  \\
&  & 1 &  &
\end{array}%
\end{equation*}%
A partir d'ella, \'{e}s evident que $\sup B=60$, $\inf B=1$, i no existeixen
m\`{a}xim ni m\'{\i}nim de $B$.

(c) A partir del mateix gr\`{a}fic de l'apartat anterior, \'{e}s clar que $%
\sup A=\max A=60$ i $\inf A=\min A=1$.

(d) Els elements maximal i minimal de $A$ s\'{o}n, respectivament, $60$ i $1$%
.
\end{solucio}

\begin{exercici}
Es considera en el conjunt $A=\left\{ 1,2,3,4,5,6\right\} $ la seg\"{u}ent
relaci\'{o}
\begin{equation*}
R=\left\{
(6,5),(5,1),(1,2),(6,4),(4,1),(4,2),(3,2),(5,2),(6,1),(6,2)\right\} \cup
\Delta _{A}
\end{equation*}%
on $\Delta _{A}$ \'{e}s la relaci\'{o} d'identitat en $A$. (a) Representa gr%
\`{a}ficament aquesta relaci\'{o}. (b) Calcula cotes inferiors i superiors
de $B=\left\{ 1,4,5\right\} $ i determina $\sup B$ i $\inf B$. (c) Calcula
els elements maximals i minimals d'$A$. Hi ha m\`{a}xim i m\'{\i}nim d'$A$?
\end{exercici}

\begin{solucio}
(a) Una representaci\'{o} gr\`{a}fica de la relaci\'{o} \'{e}s%
\begin{equation*}
\begin{array}{ccccc}
&  & 2 &  &  \\
& \diagup & | & \diagdown &  \\
1 &  & | &  & 3 \\
| & \diagdown & | &  &  \\
5 &  & 4 &  &  \\
& \diagdown & | &  &  \\
&  & 6 &  &
\end{array}%
\end{equation*}

(b) Per al conjunt $B$ nom\'{e}s hi ha una cota inferior $6$ i t\'{e} $1$ i $%
2$ com a cotes superiors. Llavors, \'{e}s clar que $\sup B=1$ i $\inf B=6$.

(c) Per al conjunt $A$, observant el gr\`{a}fic de la relaci\'{o}, es t\'{e}
que $3$ i $6$ s\'{o}n elements minimals i nom\'{e}s hi ha un element maximal
$2$. Per tant, no hi ha m\'{\i}nim d'$A$ i $\max A=2$.
\end{solucio}

\begin{exercici}
Es considera el conjunt ordenat $\mathbb{Q}$ per la relaci\'{o} d'ordre
usual $\leq $. Quin subconjunt de $\mathbb{Q}$ est\`{a} ben ordenat? (a) $%
\mathbb{Q}$; (b) Els nombres enters majors que $9$; (c) Els nombres enters
parells menors que $0$; i (d) Els nombres enters positius m\'{u}ltiples de $%
5 $.
\end{exercici}

\begin{solucio}
(a) $\mathbb{Q}$ no est\`{a} ben ordenat ja que no t\'{e} element m\'{\i}nim.

(b) El conjunt de nombres enters majors que $9$ est\`{a} ben ordenat perqu%
\`{e} \'{e}s un subconjunt de $\mathbb{N}$, que est\`{a} ben ordenat.

(c) El conjunt de nombres enters parells menors que $0$ no est\`{a} ben
ordenat ja que no t\'{e} element m\'{\i}nim.

(d) El conjunt de nombres enters positius m\'{u}ltiples d'est\`{a} $5$ ben
ordenat perqu\`{e} \'{e}s un subconjunt de $\mathbb{N}$, que est\`{a} ben
ordenat.
\end{solucio}

\section{Aplicacions}

\begin{exercici}
Donats $A=\left\{ 1,2,3\right\} $ i $B=\left\{ 2,3,4,5\right\} $, \'{e}s
aplicaci\'{o} de $A$ en $B$ la relaci\'{o} entre $A$ i $B$ definida per%
\begin{equation*}
\left\{ (1,3),(2,2),(1,5),(3,5)\right\}
\end{equation*}%
Raona la resposta.
\end{exercici}

\begin{solucio}
No \'{e}s aplicaci\'{o} ja que $1\in A$ est\`{a} relacionat amb dos elements
de $B$ i aix\`{o} no pot passar.
\end{solucio}

\begin{exercici}
Estudia si les relacions bin\`{a}ries seg\"{u}ents en $\mathbb{R}$ s\'{o}n o
no aplicacions. Quan ho siguin, calcula el seu domini i imatge. (a) $%
R_{1}=\left\{ (x,y)\in \mathbb{R}\times \mathbb{R}:y^{2}-x=0\right\} $; (b) $%
R_{2}=\left\{ (x,y)\in \mathbb{R}\times \mathbb{R}:x+y=2\right\} $; (c) $%
R_{3}=\left\{ (x,y)\in \mathbb{R}\times \mathbb{R}:x^{2}+y^{2}=1\right\} $;
i (d) $R_{4}=\left\{ (x,y)\in \mathbb{R}\times \mathbb{R}:y=\sqrt{4-x^{2}}%
\right\} $.
\end{exercici}

\begin{solucio}
(a) La relaci\'{o} $R_{1}$ no \'{e}s aplicaci\'{o} ja que $(1,1),(1,-1)\in
R_{1}$.

(b) La relaci\'{o} $R_{2}$ \'{e}s aplicaci\'{o}. \'{E}s clar que defineix
l'aplicaci\'{o} $f:\mathbb{R}~\longrightarrow ~\mathbb{R}$ mitjan\c{c}ant $%
f(x)=2-x$. El domini de $R_{2}$ \'{e}s $\mathbb{R}$ i la imatge \'{e}s tamb%
\'{e} $\mathbb{R}$.

(c) La relaci\'{o} $R_{3}$ no \'{e}s aplicaci\'{o} ja que $(0,1),(0,-1)\in
R_{3}$.

(d) La relaci\'{o} $R_{4}$ \'{e}s aplicaci\'{o}. \'{E}s clar que defineix
l'aplicaci\'{o} $f:\mathbb{R}~\longrightarrow ~\mathbb{R}$ mitjan\c{c}ant $%
f(x)=\sqrt{4-x^{2}}$. El domini de $R_{4}$ \'{e}s $[-2,2]$ i la imatge \'{e}%
s $[0,2]$.
\end{solucio}

\begin{exercici}
Es considera l'aplicaci\'{o} $f:\mathbb{Z}~\longrightarrow ~\mathbb{R}$
definida per $f(x)=3x+1$. (a) Calcula les imatges de $-2,0,3$, i les
antiimatges, si existeixen, de $-5,4/5$ i $9$. Quins s\'{o}n els elements
que tenen antiimatge? (b) Contesta a les mateixa q\"{u}estions prenent com a
conjunt de sortida $\mathbb{R}$ en lloc de $\mathbb{Z}$
\end{exercici}

\begin{solucio}
(a) Les imatges de $-2,0$ i $3$ s\'{o}n:%
\begin{equation*}
\begin{array}{l}
f(-2)=3\cdot \left( -2\right) +1=-5 \\
f(0)=3\cdot 0+1=1 \\
f(3)=3\cdot 3+1=10%
\end{array}%
\end{equation*}%
Calculem les antiimatges de $-5,4/5$ i $9$. Com que%
\begin{equation*}
\begin{array}{lll}
f(x)=-5 & \Longrightarrow & 3x+1=-5 \\
& \Longrightarrow & x=-2%
\end{array}%
\end{equation*}%
Llavors $-2$ \'{e}s antiimatge de $-5$. De la mateixa manera,
\begin{equation*}
\begin{array}{lll}
f(x)=\frac{4}{5} & \Longrightarrow & 3x+1=\frac{4}{5} \\
& \Longrightarrow & x=-\frac{1}{15}%
\end{array}%
\end{equation*}%
Per tant, no existeix antiimatge de $4/5$ ja que $-1/15\notin \mathbb{Z}$.
Finalment,%
\begin{equation*}
\begin{array}{lll}
f(x)=9 & \Longrightarrow & 3x+1=9 \\
& \Longrightarrow & x=\frac{8}{3}%
\end{array}%
\end{equation*}%
Per tant, tampoc existeix antiimatge de $9$ ja que $8/3\notin \mathbb{Z}$.

Observa que%
\begin{equation*}
\begin{array}{lll}
f(x)=y & \Longrightarrow & 3x+1=y \\
& \Longrightarrow & x=\dfrac{y-1}{3}%
\end{array}%
\end{equation*}%
Per tant,%
\begin{equation*}
\begin{array}{lll}
\dfrac{y-1}{3}\in \mathbb{Z} & \Longrightarrow & y-1\text{ \'{e}s m\'{u}%
ltiple de }3 \\
& \Longrightarrow & y=1+3k\text{, amb }k\in \mathbb{Z}%
\end{array}%
\end{equation*}%
Per conseg\"{u}ent, els elements que tenen antiimatge s\'{o}n%
\begin{equation*}
\left\{ ...,-5,-2,1,4,7,...\right\}
\end{equation*}

(b) Les respostes s\'{o}n les mateixes que abans per\`{o} amb la difer\`{e}%
ncia que ara $-1/15\in \mathbb{R}$ \'{e}s antiimatge de $4/5$ i $8/3\in
\mathbb{R}$ ho \'{e}s de $9$. A m\'{e}s, els elements que tenen antiimatge
\'{e}s ara tot $\mathbb{R}$.
\end{solucio}

\begin{exercici}
Donades les aplicacions $f,g,h:\mathbb{Z}~\longrightarrow ~\mathbb{Z}$
definides per $f(x)=x+2$ $g(x)=4x$ , i $h(x)=x^{2}-x$, esbrina si s\'{o}n
injectives, exhaustives o bijectivas.
\end{exercici}

\begin{solucio}
L'aplicaci\'{o} $f$ \'{e}s bijectiva. En efecte, \'{e}s injectiva doncs%
\begin{equation*}
\begin{array}{lll}
f(x)=f(y) & \Longrightarrow & x+2=y+2 \\
& \Longrightarrow & x=y%
\end{array}%
\end{equation*}%
i tamb\'{e} \'{e}s exhaustiva ja que donat qualsevol $y\in \mathbb{Z}$ tenim%
\begin{equation*}
\begin{array}{lll}
f(x)=y & \Longrightarrow & x+2=y \\
& \Longrightarrow & x=y-2\in \mathbb{Z}%
\end{array}%
\end{equation*}%
i, per tant, cada element $y\in \mathbb{Z}$ t\'{e} antiimatge $y-2\in
\mathbb{Z}$.

L'aplicaci\'{o} $g$ \'{e}s injectiva per\`{o} no exhaustiva. En efecte, \'{e}%
s injectiva doncs%
\begin{equation*}
\begin{array}{lll}
g(x)=g(y) & \Longrightarrow & 4x=4y \\
& \Longrightarrow & x=y%
\end{array}%
\end{equation*}%
En canvi, no \'{e}s exhaustiva perqu\`{e} qualsevol nombre enter que no
sigui m\'{u}ltiple de $4$ no t\'{e} antiimatge en $\mathbb{Z}$.

Finalment, l'aplicaci\'{o} $h$ no \'{e}s injectiva ni exhaustiva. En efecte,
ja que $h(0)=h(1)=0$ i $0\neq 1$, l'aplicaci\'{o} no \'{e}s injectiva.
Tampoc \'{e}s exhaustiva ja que, per exemple, $-1$ no t\'{e} antiimatge per
al $h$ no tenir solucions senceres la seg\"{u}ent equaci\'{o} de segon grau%
\begin{align*}
x^{2}-x& =-1 \\
x^{2}-x+1& =0
\end{align*}
\end{solucio}

\begin{exercici}
Donades les aplicacions $f,g,h:\mathbb{R}~\longrightarrow ~\mathbb{R}$
definides per $f(x)=e^{x}$, $g(x)=\frac{1}{1+x^{2}}$ , i $h(x)=\cos x$,
esbrina si s\'{o}n injectives, exhaustives o bijectivas.
\end{exercici}

\begin{solucio}
L'aplicaci\'{o} $f$ \'{e}s injectiva per\`{o} no exhaustiva. \'{E}s
injectiva ja que si $x\neq y$, llavors \'{e}s evident que $e^{x}\neq e^{y}$.
En canvi, no \'{e}s exhaustiva ja que $e^{x}>0$ per tot $x\in \mathbb{R}$ i,
per tant, $\func{Im}f=(0,+\infty )\neq \mathbb{R}$.

L'aplicaci\'{o} $g$ no \'{e}s injectiva ni exhaustiva. No \'{e}s injectiva
ja que, per exemple , $g(1)=g(-1)=1/2$ i $1\neq -1$. Tampoc \'{e}s
exhaustiva ja que evidentment
\begin{equation*}
0<\frac{1}{1+x^{2}}<1
\end{equation*}%
per a tot $x\in \mathbb{R}$ i, per tant, $\func{Im}g=(0,1]\neq \mathbb{R}$.

L'aplicaci\'{o} $h$ no \'{e}s injectiva ni exhaustiva. No \'{e}s injectiva
ja que, per exemple , $h(0)=h(2\pi )=1$ i $0\neq 2\pi $. Tampoc \'{e}s
exhaustiva ja que $-1\leq \cos x\leq 1$ per a tot $x\in \mathbb{R}$ i, per
tant, $\func{Im}h=[-1,1]$.
\end{solucio}

\begin{exercici}
Considerem l'aplicaci\'{o} $f:\mathbb{R}-\left\{ -1,1\right\}
~\longrightarrow ~\mathbb{R}$ definida per%
\begin{equation*}
f(x)=\frac{x^{2}}{x^{2}-1}
\end{equation*}%
(a) Si $A=\left\{ -1/2,0,1/2\right\} $, calcula $f(A)$ i $f^{-1}(A)$. (b)
Esbrina si $f$ \'{e}s injectiva o exhaustiva.
\end{exercici}

\begin{solucio}
Com que les imatges de $-1/2,0$ i $1/2$ s\'{o}n%
\begin{equation*}
\begin{array}{l}
f(-\frac{1}{2})=-\frac{1}{3} \\
f(0)=0 \\
f(\frac{1}{2})=-\frac{1}{3}%
\end{array}%
\end{equation*}%
llavors $f(A)=\left\{ -1/3,0\right\} $.

Calculem les antiimatges de $-1/2,0$ i $1/2$. Com que%
\begin{equation*}
\begin{array}{lll}
f(x)=-\frac{1}{2} & \Longrightarrow & \dfrac{x^{2}}{x^{2}-1}=-\frac{1}{2} \\
& \Longrightarrow & 3x^{2}=1 \\
& \Longrightarrow & x=\pm \frac{1}{\sqrt{3}}%
\end{array}%
\end{equation*}%
dedu\"{\i}m que $-1/\sqrt{3}$ i $1/\sqrt{3}$ s\'{o}n antiimatges de $-1/2$.
De la mateixa manera ,%
\begin{equation*}
\begin{array}{lll}
f(x)=0 & \Longrightarrow & \frac{x^{2}}{x^{2}-1}=0 \\
& \Longrightarrow & x^{2}=0 \\
& \Longrightarrow & x=0%
\end{array}%
\end{equation*}%
Per tant, $0$ \'{e}s antiimatge de $0$. Finalment,%
\begin{equation*}
\begin{array}{lll}
f(x)=\frac{1}{2} & \Longrightarrow & \frac{x^{2}}{x^{2}-1}=\frac{1}{2} \\
& \Longrightarrow & x^{2}=-1%
\end{array}%
\end{equation*}%
al no tenir solucions reals aquesta \'{u}ltima equaci\'{o} de segon grau,
dedu\"{\i}m que $1/2$ no t\'{e} antiimatges.

Dels resultats obtinguts, dedu\"{\i}m tamb\'{e} que $f$ no \'{e}s injectiva
(Hem vist que $f(-1/2)=f(-1/2)$) ni exhaustiva (Hem vist que $0$ no t\'{e}
antiimatge).
\end{solucio}

\begin{exercici}
Donada una aplicaci\'{o} $f$ de $A$ en $B$, considerem $X,Y\subset A$ i $%
Z,T\subset B$. Demostra que es compleixen les seg\"{u}ents propietats: (a) $%
X\subset Y$ implica $f(X)\subset f(Y)$; (b) $Z\subset T$ implica $%
f^{-1}(Z)\subset f^{-1}(T)$; (c) $f(X\cup Y)=f(X)\cup f(Y)$; (d) $f(X\cap
Y)\subset f(X)\cap f(Y)$; (e) $f^{-1}(Z\cup T)=f^{-1}(Z)\cup f^{-1}(T)$; (f)
$f^{-1}(Z\cap T)=f^{-1}(Z)\cap f^{-1}(T)$; (g) $X\subset f^{-1}\left(
f(X)\right) $; (h) $f\left( f^{-1}(Z)\right) \subset Z$; (i) $f^{-1}\left(
\complement _{B}Z\right) =\complement _{A}\left( f^{-1}(Z)\right) $.
\end{exercici}

\begin{solucio}
(a) Considerem qualsevol element $b\in f(X)$. Llavors, existeix $a\in X$ tal
que $f(a)=b$. Ara b\'{e}, per hip\`{o}tesi, $X\subset Y$, despr\'{e}s $a\in
Y $ i $f(a)=b\in f(Y)$. D'aquesta manera hem demostrat que $f(X)\subset f(Y)$%
.

(b) Considerem qualsevol element $a\in f^{-1}(Z)$. Llavors, $f(a)\in Z$ i
com, per hip\`{o}tesi, $Z\subset T$, dedu\"{\i}m que $f(a)\in T$. Despr\'{e}%
s, $a\in f^{-1}(T)$. Per tant, $f^{-1}(Z)\subset f^{-1}(T)$.

(c) Provarem (1) $f(X\cup Y)\subset f(X)\cup f(Y)$ i (2) $f(X)\cup
f(Y)\subset f(X\cup Y)$. Llavors, de (1) i (2), deduirem la igualtat. (1)
Considerem qualsevol element $b\in f(X\cup Y)$. Llavors, existeix $a\in
X\cup Y$ tal que $f(a)=b$. Ara b\'{e}, si $a\in X\cup Y$, llavors $a\in X$ o
$a\in Y$. Si $a\in X$, llavors $f(a)=b\in f(X)$ i, per tant, $b\in f(X)\cup
f(Y)$. De la mateixa manera, si $a\in Y$, llavors $f(a)=b\in f(Y)$ i, per
tant, $b\in b\in f(X)\cup f(Y)$. En qualsevol cas $b\in f(X)\cup f(Y)$, amb
el que dedu\"{\i}m que $f(X\cup Y)\subset f(X)\cup f(Y)$. (2) Considerem
qualsevol element $b\in f(X)\cup f(Y)$. Llavors, $b\in f(X)$ o $b\in f(Y)$.
Si $b\in f(X)$, llavors existeix $a\in X$ tal que $f(a)=b$. Ara b\'{e}, si $%
a\in X$, llavors $a\in X\cup Y$ i, per tant, $f(a)=b\in f(X\cup Y)$. Si $%
b\in f(Y)$, llavors existeix $c\in Y$ tal que $f(c)=b$. De la mateixa manera
que abans, si $c\in Y$, llavors $c\in X\cup Y$ i, per tant, $f(c)=b\in
f(X\cup Y)$. En qualsevol cas $b\in f(X\cup Y)$, amb el que dedu\"{\i}m que $%
f(X)\cup f(Y)\subset f(X\cup Y)$.

(d) Considerem qualsevol element $b\in f(X\cap Y)$. Llavors, existeix $a\in
X\cap Y$ tal que $f(a)=b$. Ara b\'{e}, si $a\in X\cap Y$, llavors $a\in X$ i
$a\in Y$. Per tant, $f(a)=b\in f(X)\cap f(Y)$, amb el que dedu\"{\i}m que $%
f(X\cap Y)\subset f(X)\cap f(Y)$.

(e) Considerem qualsevol element $a\in f^{-1}(Z\cup T)$. Llavors,
\begin{equation*}
\begin{array}{lll}
a\in f^{-1}(Z\cup T) & \Longleftrightarrow & f(a)\in Z\cup T \\
& \Longleftrightarrow & f(a)\in Z\text{ \ o \ }f(a)\in T \\
& \Longleftrightarrow & a\in f^{-1}(Z)\text{ \ o \ }a\in f^{-1}(T) \\
& \Longleftrightarrow & a\in f^{-1}(Z)\cup f^{-1}(T)%
\end{array}%
\end{equation*}%
D'aquestes equival\`{e}ncies s'obt\'{e} directament $f^{-1}(Z\cup
T)=f^{-1}(Z)\cup f^{-1}(T)$.

(f) Considerem qualsevol element $a\in f^{-1}(Z\cap T)$. Llavors,%
\begin{equation*}
\begin{array}{lll}
a\in f^{-1}(Z\cap T) & \Longleftrightarrow & f(a)\in Z\cap T \\
& \Longleftrightarrow & f(a)\in Z\text{ \ i \ }f(a)\in T \\
& \Longleftrightarrow & a\in f^{-1}(Z)\text{ \ i \ }a\in f^{-1}(T) \\
& \Longleftrightarrow & a\in f^{-1}(Z)\cap f^{-1}(T)%
\end{array}%
\end{equation*}%
D'aquestes equival\`{e}ncies s'obt\'{e} directament $f^{-1}(Z\cap
T)=f^{-1}(Z)\cap f^{-1}(T)$.

(g) Considerem qualsevol element $a\in X$. Llavors $f(a)\in f(X)$ i, per
tant, $a\in f^{-1}\left( f(X)\right) $. Com a conseq\"{u}\`{e}ncia, $%
X\subset f^{-1}\left( f(X)\right) $.

(h) Considerem qualsevol element $b\in f\left( f^{-1}(Z)\right) $. Llavors,
existeix $a\in f^{-1}(Z)$ tal que $f(a)=b$. Ara b\'{e}, si $a\in f^{-1}(Z)$,
llavors $f(a)=b\in Z$. Com a conseq\"{u}\`{e}ncia, $f\left( f^{-1}(Z)\right)
\subset Z$.

(i) Considerem qualsevol element $a\in f^{-1}\left( \complement _{B}Z\right)
$. Llavors,%
\begin{equation*}
\begin{array}{lll}
a\in f^{-1}\left( \complement _{B}Z\right) & \Longleftrightarrow & f(a)\in
\complement _{B}Z \\
& \Longleftrightarrow & f(a)\notin Z \\
& \Longleftrightarrow & a\notin f^{-1}(Z) \\
& \Longleftrightarrow & a\in \complement _{A}\left( f^{-1}(Z)\right)%
\end{array}%
\end{equation*}%
D'aquestes equival\`{e}ncies s'obt\'{e} directament $f^{-1}\left(
\complement _{B}Z\right) =\complement _{A}\left( f^{-1}(Z)\right) $.
\end{solucio}

\begin{exercici}
Si $f:A~\longrightarrow ~B$ \'{e}s injectiva i $X,Y\subset A$, demostra que
(a) $X=f^{-1}\left( f(X)\right) $ i (b) $f(X\cap Y)\subset f(X)\cap f(Y)$.
\end{exercici}

\begin{solucio}
(a) Per l'exercici anterior, nom\'{e}s cal provar que $f^{-1}\left(
f(X)\right) \subset X$. Considerem qualsevol element $a\in f^{-1}\left(
f(X)\right) $. Llavors, $f(a)\in f(X)$ i, per tant, existeix $c\in X$ tal
que $f(c)=f(a)$. Ara b\'{e}, per hip\`{o}tesi, $f$ \'{e}s injectiva i, per
tant, dedu\"{\i}m $c=a$. Despr\'{e}s, $a\in X$ i, com a conseq\"{u}\`{e}%
ncia, $f^{-1}\left( f(X)\right) \subset X$.

(b) Per l'exercici anterior, nom\'{e}s cal provar que $f(X)\cap f(Y)\subset
f(X\cap Y)$. Considerem qualsevol element $b\in f(X)\cap f(Y)$. Llavors, $%
b\in f(X)$ i $b\in f(Y)$. Per tant, existeixen $a\in X$ i $c\in Y$ tals que $%
f(a)=f(c)=b$. Ara b\'{e}, per hip\`{o}tesi, $f$ \'{e}s injectiva i, per
tant, dedu\"{\i}m $a=c$. Com conseq\"{u}\`{e}ncia, $a\in X\cap Y$ i, per
tant, $f(a)=b\in f(X\cap Y)$. Aix\'{\i}, hem demostrat que $f(X)\cap
f(Y)\subset f(X\cap Y)$.
\end{solucio}

\begin{exercici}
Si $f:A~\longrightarrow ~B$ \'{e}s exhaustiva i $Z\subset B$, demostra que $%
f\left( f^{-1}(Z)\right) =Z$.
\end{exercici}

\begin{solucio}
Per l'exercici anterior, nom\'{e}s cal provar que $Z\subset f\left(
f^{-1}(Z)\right) $. Considerem qualsevol element $b\in Z$. Per hip\`{o}tesi,
$f$ \'{e}s exhaustiva i, per tant, existeix $a\in A$ tal que $f(a)=b$. Despr%
\'{e}s, $f(a)\in Z$ i, per tant, $a\in f^{-1}(Z)$. D'aqu\'{\i}, dedu\"{\i}m
que $f(a)=b\in f\left( f^{-1}(Z)\right) $. Com a conseq\"{u}\`{e}ncia, $%
Z\subset f\left( f^{-1}(Z)\right) $.
\end{solucio}

\begin{exercici}
Donades les aplicacions $f,g:\mathbb{R}~\longrightarrow ~\mathbb{R}$
definides per $f(x)=x^{2}$ $g(x)=2x+1$i . Calcula (a) $g\circ f$, (b) $%
f\circ g$, (c) $f\circ (g\circ f)$ i (d) $(f\circ f)\circ g$.
\end{exercici}

\begin{solucio}
(a)
\begin{align*}
(g\circ f)(x)& =g\left( f(x)\right) \\
& =g(x^{2}) \\
& =2x^{2}+1
\end{align*}

(b)%
\begin{align*}
(f\circ g)(x)& =f\left( g(x)\right) \\
& =f(2x+1) \\
& =(2x+1)^{2}
\end{align*}

(c)%
\begin{align*}
\left( f\circ (g\circ f)\right) (x)& =f\left( (g\circ f)(x)\right) \\
& =f\left( 2x^{2}+1\right) \\
& =(2x^{2}+1)^{2}
\end{align*}

(d)%
\begin{align*}
\left( (f\circ f)\circ g\right) (x)& =(f\circ f)\left( g(x)\right) \\
& =f\left( f\left( g(x)\right) \right) \\
& =f\left( (2x+1)^{2}\right) \\
& =\left[ (2x+1)^{2}\right] ^{2} \\
& =(2x+1)^{4}
\end{align*}
\end{solucio}

\begin{exercici}
Sean $f:A~\longrightarrow ~B$ i $g:B~\longrightarrow ~C$ dues aplicacions.
Demostra que es compleixen les seg\"{u}ents propietats: (a) Si $f$ i $g$ s%
\'{o}n injectives, llavors $g\circ f$ \'{e}s injectiva; (b) Si $f$ i $g$ s%
\'{o}n exhaustives, llavors $g\circ f$ \'{e}s exhaustiva; (c) Si $f$ i $g$ s%
\'{o}n bijectivas, llavors $g\circ f$ \'{e}s bijectiva i, a m\'{e}s, $%
(g\circ f)^{-1}=f^{-1}\circ g^{-1}$; (d) Si $g\circ f$ \'{e}s injectiva,
llavors $f$ \'{e}s injectiva; (e) Si $g\circ f$ \'{e}s exhaustiva, llavors $%
g $ \'{e}s exhaustiva; (f) Si $g\circ f$ \'{e}s injectiva i $f$ \'{e}s
exhaustiva, llavors $g$ \'{e}s injectiva; (g) Si $g\circ f$ \'{e}s
exhaustiva i $g$ \'{e}s injectiva, llavors $f$ \'{e}s exhaustiva.
\end{exercici}

\begin{solucio}
(a) Suposem que $x,y\in A$ tals que $(g\circ f)(x)=(g\circ f)(y)$. Llavors,%
\begin{equation*}
\begin{array}{lll}
g\left( f(x)\right) =g\left( f(y)\right) & \Longrightarrow & f(x)=f(y) \\
& \Longrightarrow & x=y%
\end{array}%
\end{equation*}%
i, per tant, $g\circ f$ \'{e}s injectiva.

(b) Donat qualsevol $z\in C$ hem de provar que existeix $x\in A$ tal que $%
(g\circ f)(x)=z$. Per ser $g$ exhaustiva, existeix $u\in B$ tal que $g(u)=z$%
. Ara, per ser $f$ exhaustiva, existeix $x\in A$ tal que $f(x)=u$. Per tant,%
\begin{align*}
(g\circ f)(x)& =g\left( f(x)\right) \\
& =g(u) \\
& =z
\end{align*}%
i, com a conseq\"{u}\`{e}ncia, $g\circ f$ \'{e}s exhaustiva.

(c) Pels dos apartats anteriors, \'{e}s clar que si $f$ i $g$ s\'{o}n
bijectivas, llavors $g\circ f$ \'{e}s bijectiva. En ser $f,g$ bijectivas,
existeixen les aplicacions inverses $f^{-1}$ i $g^{-1}$ de $f$ i $g$,
respectivament. Llavors,%
\begin{equation*}
\begin{array}{lll}
(g\circ f)(x)=g\left( f(x)\right) =z & \Longleftrightarrow & f(x)=g^{-1}(z)
\\
& \Longleftrightarrow & x=f^{-1}\left( g^{-1}(z)\right) =(f^{-1}\circ
g^{-1})(z)%
\end{array}%
\end{equation*}%
Per tant,
\begin{equation*}
(g\circ f)^{-1}=f^{-1}\circ g^{-1}
\end{equation*}

(d) Suposem que $x,y\in A$. Llavors,%
\begin{equation*}
\begin{array}{lll}
f(x)=f(y) & \Longrightarrow & g\left( f(x)\right) =g\left( f(y)\right) \\
&  & (g\circ f)(x)=(g\circ f)(y) \\
& \Longrightarrow & x=y%
\end{array}%
\end{equation*}%
i, per tant, $f$ \'{e}s injectiva.

(e) Donat qualsevol $z\in C$ hem de provar que existeix $u\in B$ tal que $%
g(u)=z$. Per ser $g\circ f$ exhaustiva, existeix $x\in A$ tal que $(g\circ
f)(x)=g\left( f(x)\right) =z$. Prenent $u=f(x)\in B$, llavors tenim que $%
g(u)=g\left( f(x)\right) =z$ i, per tant, $g$ \'{e}s exhaustiva.

(f) Suposem que $u,v\in B$ tals que $g(u)=g(v)$. Per ser $f$ exhaustiva,
existeixen $x,y\in A$ tals que $f(x)=u$ i $f(y)=v$. Llavors, $g\left(
f(x)\right) =g(u)$ i $g\left( f(y)\right) =g(v)$ i, per tant, $(g\circ
f)(x)=(g\circ f)(y)$. Ara b\'{e}, per hip\`{o}tesi, $g\circ f$ \'{e}s
injectiva, amb el que dedu\"{\i}m que $x=y$. Despr\'{e}s, $f(x)=f(y)$, \'{e}%
s a dir, $u=v$. En conseq\"{u}\`{e}ncia, $g$ \'{e}s injectiva.

(g) Donat qualsevol $u\in B$ hem de provar que existeix $x\in A$ tal que $%
f(x)=u$. \'{E}s clar que $g(u)\in C$. Per ser $g\circ f$ exhaustiva,
existeix $x\in A$ tal que $(g\circ f)(x)=g(u)$, \'{e}s a dir, $g\left(
f(x)\right) =g(u)$. Ara b\'{e}, per hip\`{o}tesi, $g$ \'{e}s injectiva, amb
el que dedu\"{\i}m que $f(x)=u$. En conseq\"{u}\`{e}ncia, $f$ \'{e}s
exhaustiva.
\end{solucio}

\begin{exercici}
Demostra que l'aplicaci\'{o} $f:\mathbb{R}-\left\{ -1/2\right\}
\longrightarrow \mathbb{R}-\left\{ 1/2\right\} $ definida per%
\begin{equation*}
f(x)=\frac{x+3}{1+2x}
\end{equation*}%
\'{e}s bijectiva. Calcula l'aplicaci\'{o} inversa $f^{-1}$.
\end{exercici}

\begin{solucio}
Vegem que $f$ \'{e}s injectiva. Per a aix\`{o}, suposem que $x,y\in \mathbb{R%
}-\left\{ -1/2\right\} $ i $f(x)=f(y)$. Llavors,%
\begin{equation*}
\begin{array}{lll}
\dfrac{x+3}{1+2x}=\dfrac{y+3}{1+2y} & \Longrightarrow &
(x+3)(1+2y)=(1+2x)(y+3) \\
& \Longrightarrow & x+2xy+3+6y=y+3+2xy+6x \\
& \Longrightarrow & 5y=5x \\
& \Longrightarrow & x=y%
\end{array}%
\end{equation*}%
i, per tant, $f$ \'{e}s injectiva.

Vegem que $f$ \'{e}s exhaustiva. Per a aix\`{o}, donat qualsevol $y\in
\mathbb{R}-\left\{ 1/2\right\} $ hem de provar que existeix $x\in \mathbb{R}%
-\left\{ -1/2\right\} $ tal que $f(x)=y$. En efecte, suposem que $x$ exist%
\'{\i}s i vegem quin \'{e}s. Llavors,%
\begin{equation*}
\begin{array}{lll}
f(x)=y & \Longrightarrow & \frac{x+3}{1+2x}=y \\
& \Longrightarrow & x+3=(1+2x)y \\
& \Longrightarrow & x-2xy=y-3 \\
& \Longrightarrow & x(1-2y)=y-3 \\
& \Longrightarrow & x=\frac{y-3}{1-2y}%
\end{array}%
\end{equation*}%
\'{e}s a dir, hauria de ser
\begin{equation*}
x=\frac{y-3}{1-2y}
\end{equation*}%
Ara b\'{e}, com $y\neq 1/2$ tenim que $1-2y\neq 0$ i, a m\'{e}s,
\begin{equation*}
\frac{y-3}{1-2y}\neq -\frac{1}{2}
\end{equation*}%
per a tot $y\neq 1/2$. Per tant,%
\begin{equation*}
x=\frac{y-3}{1-2y}\in \mathbb{R}-\left\{ -1/2\right\}
\end{equation*}%
i $f$ \'{e}s exhaustiva. En ser $f$ injectiva i exhaustiva, tamb\'{e} \'{e}s
bijectiva. Per tant, $f$ t\'{e} aplicaci\'{o} inversa $f^{-1}$. Com que es
compleix%
\begin{equation*}
\begin{array}{ccc}
\dfrac{x+3}{1+2x}=y & \Longleftrightarrow & x=\dfrac{y-3}{1-2y}%
\end{array}%
\end{equation*}%
obtenim que%
\begin{equation*}
f^{-1}(x)=\frac{x-3}{1-2x}
\end{equation*}
\end{solucio}

\begin{exercici}
Donades les aplicacions $f,g:\mathbb{R}~\longrightarrow ~\mathbb{R}$
definides per%
\begin{equation*}
f(x)=x^{3}+1\text{ \ \ \ \ y \ \ \ \ }g(x)=\sqrt[3]{x-1}
\end{equation*}%
calcula $g\circ f$ i $g^{-1}\circ f^{-1}$, si existeixen.
\end{exercici}

\begin{solucio}
Les aplicacions $f$ i $g$ s\'{o}n bijectivas com pot comprovar-se de
seguida. Per tant, existeixen les aplicacions inverses $f^{-1}$ i $g^{-1}$
d'i $f$ $g$, respectivament.

Calcularem ara $g\circ f$ i $f\circ g$. Aix\'{\i}, tenim%
\begin{align*}
(g\circ f)(x)& =g\left( f(x)\right) \\
& =g(x^{3}+1) \\
& =\sqrt[3]{x^{3}+1-1} \\
& =\sqrt[3]{x^{3}} \\
& =x
\end{align*}%
i%
\begin{align*}
(f\circ g)(x)& =f\left( g(x)\right) \\
& =f\left( \sqrt[3]{x-1}\right) \\
& =\left( \sqrt[3]{x-1}\right) ^{3}+1 \\
& =x-1+1 \\
& =x
\end{align*}%
D'aquests resultats, dedu\"{\i}m que $f^{-1}=g$ ja que $f\circ g=g\circ f=I_{%
\mathbb{R}}$. Per conseg\"{u}ent,%
\begin{equation*}
g^{-1}\circ f^{-1}=g^{-1}\circ g=I_{\mathbb{R}}
\end{equation*}%
\'{e}s a dir, $g^{-1}\circ f^{-1}$ \'{e}s l'aplicaci\'{o} identitat en $%
\mathbb{R}$.
\end{solucio}

\section{Cardinal d'un conjunt}

\begin{exercici}
Sigui $E$ un conjunt referencial i considerem dos conjunts finits $A$ i $B$.
Demostra les seg\"{u}ents propietats: (a) Si $A\subset B$, llavors $\#A\leq $
$\#B$; (b) $\#\left( A\cup B\right) =$ $\#A+$ $\#B-$ $\#\left( A\cap
B\right) $; (c) $\#\left( A\cup B\cup C\right) =$ $\#A+$ $\#B+$ $\#C-$ $%
\#\left( A\cap B\right) -$ $\#\left( A\cap C\right) -$ $\#\left( B\cap
C\right) +$ $\#\left( A\cap B\cap C\right) $; (d) $\#\complement A=$ $\#E-$ $%
\#A$; (e) $\#\left( A\times B\right) =$ $\#A\cdot $ $\#B$; (f) $\#\mathcal{P}%
(A)=2^{\#A}$.
\end{exercici}

\begin{solucio}
\'{E}s clar que si $A$ i $B$ s\'{o}n disjunts, llavors%
\begin{equation*}
\#\left( A\cup B\right) =\#A+\#B
\end{equation*}

(a) Com que $\left\{ A\cap B,\complement A\cap B\right\} $ \'{e}s una partici%
\'{o} de $B$, llavors%
\begin{equation*}
\#B=\text{ }\#\left( A\cap B\right) +\#\left( \complement A\cap B\right)
\end{equation*}%
Ara b\'{e}, com $A\subset B$, llavors $A\cap B=A$ i, per tant, obtenim%
\begin{equation*}
\#B=\text{ }\#A+\#\left( \complement A\cap B\right)
\end{equation*}%
En ser $\#\left( \complement A\cap B\right) \geq 0$, dedu\"{\i}m%
\begin{equation*}
\#A\leq \text{ }\#B
\end{equation*}

(b) Com que $\left\{ A,\complement A\cap B\right\} $, $\left\{ B,A\cap
\complement B\right\} $ i $\left\{ A\cap \complement B,\complement A\cap
B,A\cap B\right\} $ constitueixen particions del conjunt $A\cup B$, llavors%
\begin{equation*}
\#\left( A\cup B\right) =\#A+\#\left( B\cap \complement A\right)
\end{equation*}%
i%
\begin{equation*}
\#\left( A\cup B\right) =\#A+\#\left( A\cap \complement B\right)
\end{equation*}%
i%
\begin{equation*}
\#\left( A\cup B\right) =\#\left( A\cap \complement B\right) +\#\left(
\complement A\cap B\right) +\#\left( A\cap B\right)
\end{equation*}%
Sumant ara les dues primeres igualtats i restant la tercera, obtenim%
\begin{equation*}
\#\left( A\cup B\right) =\#A+\#B-\#\left( A\cap B\right)
\end{equation*}

(c) Segons l'apartat anterior, tenim%
\begin{align*}
\#\left( A\cup B\cup C\right) & =\#\left( (A\cup B)\cup C\right) \\
& =\#(A\cup B)+\#C-\#\left( (A\cup B)\cap C\right) \\
& =\#A+\#B-\#\left( A\cap B\right) +\#C-\#\left( (A\cup B)\cap C\right)
\end{align*}%
Ara b\'{e}, $(A\cup B)\cap C=(A\cap C)\cup (B\cap C)$ i, per tant,%
\begin{align*}
\#\left( (A\cup B)\cap C\right) & =\#\left( (A\cap C)\cup (B\cap C)\right) \\
& =\#\left( A\cap C\right) +\#\left( B\cap C\right) -\#\left( (A\cap C)\cap
(B\cap C)\right) \\
& =\#\left( A\cap C\right) +\#\left( B\cap C\right) -\#\left( A\cap B\cap
C\right)
\end{align*}%
Per conseg\"{u}ent, obtenim%
\begin{align*}
\#\left( A\cup B\cup B\right) & =\#A+\#B+\#C-\#\left( A\cap B\right) \\
& -\#\left( A\cap C\right) -\#\left( B\cap C\right) +\#\left( A\cap B\cap
C\right)
\end{align*}

(d) Com que $\left\{ A,\complement _{E}A\right\} $ \'{e}s una partici\'{o}
de $E$ , llavors%
\begin{equation*}
\#E=\#A+\#\left( \complement _{E}A\right)
\end{equation*}%
i, per tant,
\begin{equation*}
\#\left( \complement _{E}A\right) =\#E-\#A
\end{equation*}

(e) Suposem que $\#A=n$i $\#B=m$. Sigui $\varphi :\left\{ 1,2,...,n\right\}
\longrightarrow A$una aplicaci\'{o} tal que $\varphi (i)=a_{i}\in A$i $%
a_{i}\neq a_{j}$si $i\neq j$. Aix\'{\i}, podem escriure%
\begin{equation*}
A=\left\{ a_{1},a_{2},...,a_{n}\right\}
\end{equation*}%
De la mateixa manera, obtenim%
\begin{equation*}
B=\left\{ b_{1},b_{2},...,b_{m}\right\}
\end{equation*}%
Llavors, els conjunts%
\begin{equation*}
\begin{array}{c}
F_{1}=\left\{ (a_{1},b_{1}),(a_{1},b_{2}),...(a_{1},b_{m})\right\} \\
F_{2}=\left\{ (a_{2},b_{1}),(a_{2},b_{2}),...(a_{2},b_{m})\right\} \\
\vdots \\
F_{n}=\left\{ (a_{n},b_{1}),(a_{n},b_{2}),...(a_{n},b_{m})\right\}%
\end{array}%
\end{equation*}%
constitueixen una partici\'{o} de $A\times B$ i, per tant,%
\begin{equation*}
\#\left( A\times B\right) =\#F_{1}+\#F_{2}+\cdots +\#F_{n}=n\cdot m
\end{equation*}%
doncs%
\begin{equation*}
\#F_{1}=\#F_{2}=\cdots =\#F_{n}=m
\end{equation*}%
Per tant,%
\begin{equation*}
\#\left( A\times B\right) =\#A\times \#B
\end{equation*}

(f) Suposem que $\#A=n$. Sabem que $\mathcal{P}(A)$ \'{e}s el conjunt els
elements del qual s\'{o}n subconjunts d'$A$. Sigui $0\leq m\leq n$, quants
subconjunts de $m$ elements t\'{e} $A$? Aquest n\'{u}mero \'{e}s per definici%
\'{o} el \textbf{n\'{u}mero combinatori}
\begin{equation*}
\binom{n}{m}
\end{equation*}%
que es calcula mitjan\c{c}ant la f\'{o}rmula seg\"{u}ent%
\begin{equation*}
\binom{n}{m}=\frac{n!}{m!\left( n-m\right) !}
\end{equation*}%
sent el \textbf{factorial d'un n\'{u}mero} $n$%
\begin{equation*}
n!=n\cdot (n-1)\cdot (n-2)\cdot \cdots \cdot 2\cdot 1
\end{equation*}%
i, per definici\'{o}, $0!=1$. Aix\'{\i}, tenim%
\begin{equation*}
\begin{array}{cc}
\text{N\'{u}mero d'elements del subconjunt} & \text{N\'{u}mero de subconjunts%
} \\
0 & \binom{n}{0} \\
1 & \binom{n}{1} \\
2 & \binom{n}{2} \\
\vdots & \vdots \\
n-1 & \binom{n}{n-1} \\
n & \binom{n}{n}%
\end{array}%
\end{equation*}%
Llavors,%
\begin{equation*}
\#\mathcal{P}(A)=\binom{n}{0}+\binom{n}{1}+\binom{n}{2}+\cdots +\binom{n}{n-1%
}+\binom{n}{n}
\end{equation*}%
Aquesta suma pot calcular-se mitjan\c{c}ant la f\'{o}rmula de la pot\`{e}%
ncia del binomi de Newton
\begin{equation*}
(A+B)^{n}=\binom{n}{0}A^{n}+\binom{n}{1}A^{n-1}B+\binom{n}{2}A^{n-2}B+\cdots
+\binom{n}{n-1}AB^{n-1}+\binom{n}{n}B^{n}
\end{equation*}%
Prenent $A=B=1$, resulta%
\begin{equation*}
(1+1)^{n}=\binom{n}{0}+\binom{n}{1}+\binom{n}{2}+\cdots +\binom{n}{n-1}+%
\binom{n}{n}
\end{equation*}%
i, per conseg\"{u}ent, obtenim%
\begin{equation*}
\#\mathcal{P}(A)=2^{n}=2^{\#A}
\end{equation*}
\end{solucio}

\begin{exercici}
Suposem que en Joan menja cada mat\'{\i} ous o cereals per esmorzar durant
el mes de gener. Si en 25 matins ha menjat cereals, i en 18, ous, en quants
matins ha menjat ous i cereals?
\end{exercici}

\begin{solucio}
Sigui $A$ el conjunt de dies del mes de gener que en Joan menja ous per
esmorzar i $B$ el conjunt de dies que menja cereals. Segons la informaci\'{o}
de l'enunciat, tenim $\#A=25$ i $\#B=18$ i, a m\'{e}s, \'{e}s clar que $%
\#\left( A\cup B\right) =31$. Llavors,
\begin{align*}
\#\left( A\cup B\right) & =\#A+\#B-\#\left( A\cap B\right) \\
31& =25+18-\#\left( A\cap B\right) \\
\#\left( A\cap B\right) & =12
\end{align*}%
Ara b\'{e}, $\#\left( A\cap B\right) $ representa els dies que en Joan menja
ous i cereals per esmorzar durant el mes de gener. Per tant, en Joan menja
ous i cereals en 12 matins.
\end{solucio}

\begin{exercici}
Se sap que dels 30 alumnes d'una classe 15 juguen al ping-pong i 20 al
tennis. A m\'{e}s, no hi ha cap alumne que no practiqui algun d'aquests dos
esports. Quants alumnes practiquen els dos esports alhora?
\end{exercici}

\begin{solucio}
Sigui $A$ el conjunt d'alumnes de la classe que practiquen ping-pong i $B$
el conjunt d'alumnes que practiquen tennis. Segons l'enunciat, $\#A=15$ i $%
\#B=20$. Com sabem tamb\'{e} que no hi ha cap alumne que no practiqui algun
d'aquests dos esports, tenim que $\#\left( A\cup B\right) =30$. Llavors,%
\begin{align*}
\#\left( A\cup B\right) & =\#A+\#B-\#\left( A\cap B\right) \\
30& =15+20-\#\left( A\cap B\right) \\
\#\left( A\cap B\right) & =5
\end{align*}%
i, per tant, hi ha 5 alumnes de la classe que practiquen tots dos esports.
\end{solucio}

\begin{exercici}
En una classe, 30 alumnes llegeixen el diari $A$, 20 llegeixen el $B$, 13
llegeixen l'i $A$ el $C$, 10 llegeixen el $B$ per\`{o} no el $C$, 24 no
llegeixen $C$, 7 llegeixen l'i $A$ el $C$ per\`{o} no el $B$, 9 llegeixen el
$C$ per\`{o} no el $A$ ni el $B$, i 11 llegeixen el $A$ per\`{o} no el $B$
ni el $C$. (a) Quants alumnes llegeixen almenys un dels tres diaris? (b)
Quants alumnes hi ha en la classe?
\end{exercici}

\begin{solucio}
Reunint la informaci\'{o} en un diagrama de Venn, obtenim\FRAME{dtbpF}{%
2.1638in}{1.4754in}{0pt}{}{}{set9.jpg}{\special{language "Scientific
Word";type "GRAPHIC";maintain-aspect-ratio TRUE;display "USEDEF";valid_file
"F";width 2.1638in;height 1.4754in;depth 0pt;original-width
2.1352in;original-height 1.4477in;cropleft "0";croptop "1";cropright
"1";cropbottom "0";filename '../../conjunts/set9.jpg';file-properties
"XNPEU";}}A partir d'aquest diagrama podem respondre directament les q\"{u}%
estions plantejades. Tot i aix\`{o}, aqu\'{\i} el farem mitjan\c{c}ant les f%
\'{o}rmules estudiades sobre cardinals de conjunts finits.

Sigui $E$ el conjunt d'alumnes de la classe, $A$ el conjunt d'alumnes
d'aquesta classe que llegeixen el diari $A$, $B$ el conjunt d'alumnes que
llegeixen el diari $B$, i $C$ el conjunt d'alumnes que llegeixen el diari $C$%
.

Per l'enunciat, sabem que $\#A=30$, $\#B=20$, $\#\left( A\cap C\right) =13$,
$\#\left( B\cap \complement C\right) =10$, $\#\complement C=24$, $\#\left(
A\cap C\cap \complement B\right) =7$, $\#\left( C\cap \complement A\cap
\complement B\right) =9$ i $\#\left( A\cap \complement B\cap \complement
C\right) =11$.

(a) Ens demanen calcular $\#\left( A\cup B\cup C\right) $. Segons la
informaci\'{o} que tenim, els conjunts $A\cap C\cap \complement B$, $C\cap
\complement A\cap \complement B$, $A\cap \complement B\cap \complement C$ i $%
B$ s\'{o}n disjunts i, a m\'{e}s, com%
\begin{equation*}
A\cup B\cup C=\left( A\cap C\cap \complement B\right) \cup \left( C\cap
\complement A\cap \complement B\right) \cup \left( A\cap \complement B\cap
\complement C\right) \cup B
\end{equation*}%
s'obt\'{e}%
\begin{align*}
\#\left( A\cup B\cup B\right) & =\#\left( A\cap C\cap \complement B\right)
+\#\left( C\cap \complement A\cap \complement B\right) +\#\left( A\cap
\complement B\cap \complement C\right) +\#B \\
& =7+9+11+20 \\
& =47
\end{align*}%
Per tant, hi ha 47 alumnes que llegeixen almenys un dels tres diaris.

(b) Ens demanen en aquest cas $\#E$. Els conjunts $A\cap B\cap C$, $A\cap
\complement B\cap C$ s\'{o}n disjunts i, a m\'{e}s,%
\begin{equation*}
A\cap C=\left( A\cap B\cap C\right) \cup \left( A\cap \complement B\cap
C\right)
\end{equation*}%
Per tant,%
\begin{align*}
\#\left( A\cap C\right) & =\#\left( A\cap B\cap C\right) +\#\left( A\cap
\complement B\cap C\right) \\
13& =\#\left( A\cap B\cap C\right) +7
\end{align*}%
Llavors, $\#\left( A\cap B\cap C\right) =6$. D'altra banda, els conjunts $%
B\cap \complement C$, $A\cap B\cap C$ i $\complement A\cap B\cap C$ s\'{o}n
disjunts i, a m\'{e}s,%
\begin{equation*}
B=\left( B\cap \complement C\right) \cup \left( A\cap B\cap C\right) \cup
\left( \complement A\cap B\cap C\right)
\end{equation*}%
Aleshores,%
\begin{align*}
\#B& =\#\left( B\cap \complement C\right) +\#\left( A\cap B\cap C\right)
+\#\left( \complement A\cap B\cap C\right) \\
20& =10+6+\#\left( \complement A\cap B\cap C\right)
\end{align*}%
Per tant, $\#\left( \complement A\cap B\cap C\right) =4$. Finalment, els
conjunts $A\cap C$, $\complement A\cap B\cap C$ i $A\cap \complement B\cap
\complement C$ s\'{o}n disjunts i, a m\'{e}s,%
\begin{equation*}
C=\left( A\cap C\right) \cup \left( \complement A\cap B\cap C\right) \cup
\left( A\cap \complement B\cap \complement C\right)
\end{equation*}%
Per tant,%
\begin{align*}
\#C& =\#\left( A\cap C\right) +\#\left( \complement A\cap B\cap C\right)
+\#\left( A\cap \complement B\cap \complement C\right) \\
& =13+4+9 \\
& =26
\end{align*}%
i, com%
\begin{equation*}
\#\complement C=\#E-\#C
\end{equation*}%
dedu\"{\i}m que%
\begin{align*}
\#E& =\#\complement C+\#C \\
& =24+26 \\
& =50
\end{align*}%
\'{e}s a dir, hi ha 50 alumnes en classe.
\end{solucio}

\begin{exercici}
En una acarnissada batalla almenys el 70 \% dels combatents perden un ull,
almenys un 75 \% perden una orella, com a m\'{\i}nim un 80 \% perden un bra%
\c{c} i almenys el 85 \% una cama. Quants combatents han perdut almenys les
quatre coses?
\end{exercici}

\begin{solucio}
Sigui $A$ el conjunt de combatents que perden un ull, $B$ el conjunt dels
quals perden una orella, $C$ el conjunt dels quals perden un bra\c{c} i $D$
el conjunt dels quals perden una cama. Per a simplificar els c\`{a}lculs
suposarem que hi ha 100 combatents en la batalla (En treballar amb tants per
cent \'{e}s igual el nombre inicial de combatents). Llavors, per l'enunciat,
tenim que $\#A\geq 70$, $\#B\geq 75$, $\#C\geq 80$ i $\#D\geq 85$. Volem
calcular el valor m\'{\i}nim de $\#\left( A\cap B\cap C\cap D\right) $.

Sabem que%
\begin{equation*}
\#\left( \left( A\cap B\right) \cup \left( C\cap D\right) \right) =\#\left(
A\cap B\right) +\#\left( C\cap D\right) -\#\left( A\cap B\cap C\cap D\right)
\end{equation*}%
Aleshores es t\'{e} tamb\'{e}%
\begin{equation*}
\#\left( A\cap B\cap C\cap D\right) =\#\left( A\cap B\right) +\#\left( C\cap
D\right) -\#\left( \left( A\cap B\right) \cup \left( C\cap D\right) \right)
\end{equation*}%
Ara b\'{e}, d'altra banda, sabem que%
\begin{align*}
\#\left( A\cap B\right) & =\#A+\#B-\#\left( A\cup B\right) \\
& \geq 70+75-100 \\
& =45
\end{align*}%
i%
\begin{align*}
\#\left( C\cap D\right) & =\#C+\#D-\#\left( C\cup D\right) \\
& \geq 80+85-100 \\
& =65
\end{align*}%
Llavors, d'aquests dos resultats dedu\"{\i}m que%
\begin{equation*}
\#\left( A\cap B\cap C\cap D\right) \geq 45+65-100=10
\end{equation*}%
i com que $0\leq $ $\#\left( \left( A\cap B\right) \cup \left( C\cap
C\right) \right) \leq 100$, s'obt\'{e} que almenys el 10 \% perd les quatre
coses.
\end{solucio}

\begin{exercici}
En una reuni\'{o} hi ha m\'{e}s homes que dones, m\'{e}s dones que beuen que
homes que fumen i m\'{e}s dones que fumen i no beuen que homes que no beuen
ni fumen. Demostrar que hi ha menys dones que no beuen ni fumen que homes
que beuen i no fumen.
\end{exercici}

\begin{solucio}
Fem el seg\"{u}ent diagrama per a descriure la informaci\'{o} de l'enunciat.%
\FRAME{dtbpF}{2.2883in}{1.8299in}{0pt}{}{}{set10.jpg}{\special{language
"Scientific Word";type "GRAPHIC";maintain-aspect-ratio TRUE;display
"USEDEF";valid_file "F";width 2.2883in;height 1.8299in;depth
0pt;original-width 2.2606in;original-height 1.8023in;cropleft "0";croptop
"1";cropright "1";cropbottom "0";filename
'../../conjunts/set10.jpg';file-properties "XNPEU";}}Si $H$ \'{e}s el
conjunt d'homes i $M$ el de dones, llavors per l'enunciat es compleix que%
\begin{equation*}
\#H>\#M
\end{equation*}%
Si $F$ \'{e}s el conjunt de persones fumadores i $B$ \'{e}s el conjunt de
persones que beuen, llavors per l'enunciat tamb\'{e} es compleix%
\begin{equation*}
\#\left( M\cap B\right) >\#\left( H\cap F\right)
\end{equation*}%
i%
\begin{equation*}
\#\left( M\cap F\cap \complement B\right) >\#\left( H\cap \complement B\cap
\complement F\right)
\end{equation*}%
Cal provar que%
\begin{equation*}
\#\left( M\cap \complement B\cap \complement F\right) <\#\left( H\cap B\cap
\complement F\right)
\end{equation*}

Amb l'ajuda del diagrama anterior, podem escriure%
\begin{equation*}
\#H=a+c+e+g>b+d+f+h=\#M
\end{equation*}%
\begin{equation*}
\#\left( M\cap B\right) =d+f>a+c=\#\left( H\cap F\right)
\end{equation*}%
\begin{equation*}
\#\left( M\cap F\right) =b>g=\#\left( H\cap \complement B\cap \complement
F\right)
\end{equation*}%
Sumant membre a membre les tres desigualtats anteriors, obtenim%
\begin{equation*}
\begin{array}{ccc}
a+c+e+g+d+f+b>b+d+f+h+a+c+g & \Longrightarrow & e>h%
\end{array}%
\end{equation*}%
\'{e}s a dir,%
\begin{equation*}
\#\left( M\cap \complement B\cap \complement F\right) =b<e=\#\left( H\cap
B\cap \complement F\right)
\end{equation*}
\end{solucio}


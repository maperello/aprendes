
\section{Aplicaciones i Cardinals}

\begin{enumerate}
\item Quin de les seg\"{u}ents relacions $R$ entre $A$ i $B$ \'{e}s una
aplicaci\'{o} d'$A$ en $B$?

\begin{enumerate}
\item $A=\left\{ 1,2,3,4,5\right\} $, $B=\left\{ 1,2\right\} $ i $R=\left\{
(1,1),(3,2),(5,1),(4,1)\right\} $

\item $A=B=\mathbb{R}$ i $R=\left\{ (x,y)\in \mathbb{R}\times \mathbb{R}%
:x+y^{3}=0\right\} $ *

\item $A=B=\mathbb{R}$ i $R=\left\{ (x,y)\in\mathbb{R}\times\mathbb{R}%
:xy=1\right\} $

\item $A=B=\mathbb{R}$ i $R=\left\{ (x,y)\in\mathbb{R}\times\mathbb{R}:y=%
\sqrt{x-1}\right\} $
\end{enumerate}

\item Es defineix l'aplicaci\'{o} $f:\mathbb{R}~\longrightarrow~\mathbb{R}$
mitjan\c{c}ant $f(x)=x^{2}+4$. Llavors,

\begin{enumerate}
\item $f^{-1}(\left\{ 0\right\} )=\{-2,2\}$

\item $f\left( [0,1]\right) =[0,1]$

\item $f^{-1}\left( (0,4)\right) =(0,2)$

\item Cap de les anteriors \'{e}s certa *
\end{enumerate}

\item La gr\`{a}fica d'una aplicaci\'{o} $f:\mathbb{R}~\longrightarrow~%
\mathbb{R}$ \'{e}s\FRAME{dtbpF}{2.9447in}{2.0072in}{0pt}{}{}{set11.jpg}{%
\special{ language "Scientific Word"; type "GRAPHIC"; maintain-aspect-ratio
TRUE; display "USEDEF"; valid_file "F"; width 2.9447in; height 2.0072in;
depth 0pt; original-width 2.917in; original-height 1.9796in; cropleft "0";
croptop "1"; cropright "1"; cropbottom "0"; filename
'../../conjunts/set11.jpg';file-properties "XNPEU";}}llavors:

\begin{enumerate}
\item $f$ \'{e}s injectiva

\item $f$ \'{e}s exhaustiva

\item $f:[0,+\infty )~\longrightarrow ~[0,1]$ \'{e}s bijectiva *

\item $f:[0,+\infty)~\longrightarrow~\mathbb{R}$ \'{e}s exhaustiva
\end{enumerate}

\item Considerem les aplicacions: $f:\mathbb{R}-\{-2\}~\longrightarrow ~%
\mathbb{R}-\{3\}$ definida per%
\begin{equation*}
f(x)=\frac{3+3x}{x+2}
\end{equation*}
i $g:\mathbb{R}-\{3\}~\longrightarrow~\mathbb{R}$ definida per $g(x)=x^{3}$.
Llavors, quin de les seg\"{u}ents afirmacions \'{e}s falsa?

\begin{enumerate}
\item $f$ \'{e}s bijectiva i $f^{-1}(x)=\frac{3-2x}{x-3}$

\item $g$ \'{e}s bijectiva i $g^{-1}(x)=\sqrt[3]{x}$ *

\item $f\circ g$ no \'{e}s aplicaci\'{o}

\item $g\circ f$ \'{e}s injectiva i $(g\circ f)(x)=\left( \frac{3+3x}{x+2}%
\right) ^{3}$
\end{enumerate}

\item Sigui $f:A~\longrightarrow~B$ una aplicaci\'{o} i suposem que $\#A=n$
i $\#B=m$. Llavors:

\begin{enumerate}
\item Si $f$ \'{e}s injectiva, llavors $n\leq m$ *

\item Si $f$ \'{e}s exhaustiva, llavors $n\leq m$

\item Si $n=m$, llavors $f$ \'{e}s bijectiva

\item No pot oc\'{o}rrer que $n<m$
\end{enumerate}

\item Si $f,g$ s\'{o}n aplicacions d'en $\mathbb{R}$ $\mathbb{R}$ tals que $%
g(x)=x^{3}$ i $(g\circ f)(x)=x^{3}-3x^{2}+3x-1$. Llavors:

\begin{enumerate}
\item $f(x)=x+1$

\item $f(x)=1-x$ *

\item $f(x)=x-1$

\item $f(x)=-1-x$
\end{enumerate}

\item Sigui $f:A~\longrightarrow~B$ una aplicaci\'{o} i considerem $%
X,Y\subset A$ i $Z,T\subset B$, llavors

\begin{enumerate}
\item $f(X\cap Y)=f(X)\cap f(Y)$

\item $f^{-1}\left( f(X)\right) =X$

\item $f^{-1}(Z\cap T)=f^{-1}(Z)\cap f^{-1}(T)$ *

\item $f\left( f^{-1}(Z)\right) =Z$
\end{enumerate}

\item Efectuant una mostra de 1000 individus s'observa que mengen peix i
carn per\`{o} no ous 60, peix i ous per\`{o} no carn 40, carn i ous per\`{o}
no pescat 30, nom\'{e}s pescat 50, nom\'{e}s carn 40 i nom\'{e}s ous 30.
Tots mengen carn, ous o peix. Quants mengen peix?

\begin{enumerate}
\item 900 *

\item 750

\item 800

\item Cap de les anteriors
\end{enumerate}

\item En una classe de 100 alumnes que s'han examinat de matem\`{a}tiques i F%
\'{\i}sica es coneixen els seg\"{u}ents resultats: No han aprovat cap
assignatura 20 alumnes. Han aprovat les dues assignatures 25 alumnes. Han
aprovat el doble d'alumnes Matem\`{a}tiques que F\'{\i}sica.
\textquestiondown Quants alumnes han aprovat Matem\`{a}tiques?

\begin{enumerate}
\item 10

\item 20

\item 35

\item 45 *
\end{enumerate}

\item En el conjunt dels nombres naturals menors que 500, \textquestiondown %
quants n\'{u}meros cal no siguin m\'{u}ltiples de 2, ni de 3, ni de 5?

\begin{enumerate}
\item 120

\item 134 *

\item 100

\item Cap de les anteriors
\end{enumerate}
\end{enumerate}



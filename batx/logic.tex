% Plantilla LaTeX unificada compatible con LaTeXML
% Combina elementos de aprendes.tex
% ——— Llengua i tipografia ———
% ——— Matemàtiques i entorns ———
% ——— Estructures i enllaços ———
% ——— Colors per a LaTeXML (evitar tcolorbox) ———
% ——— Definició d'estils d'enunciats ———
% Numeració per capítol pels teoremes
% Numeració independent per a definicions (sense número de capítulo)
% ——— Entorns personalitzats sense tcolorbox ———
% --- Llistes ---
% ——— Estils per a seccions ———
% ——— Configuració general ———
% --- Macros útils (afegeix aquí les teves comandes) ---
% ——— Inici del document ———


\documentclass[12pt,a4paper,oneside]{book}
\usepackage{eurosym}
%%%%%%%%%%%%%%%%%%%%%%%%%%%%%%%%%%%%%%%%%%%%%%%%%%%%%%%%%%%%%%%%%%%%%%%%%%%%%%%%%%%%%%%%%%%%%%%%%%%%%%%%%%%%%%%%%%%%%%%%%%%%%%%%%%%%%%%%%%%%%%%%%%%%%%%%%%%%%%%%%%%%%%%%%%%%%%%%%%%%%%%%%%%%%%%%%%%%%%%%%%%%%%%%%%%%%%%%%%%%%%%%%%%%%%
\usepackage[T1]{fontenc}
\usepackage[utf8]{inputenc}
\usepackage{lmodern}
\usepackage{microtype}
\usepackage[catalan]{babel}
\usepackage{csquotes}
\usepackage{amsmath,amssymb,amsthm,mathtools}
\usepackage{geometry}
\usepackage{import}
\usepackage{hyperref}
\usepackage{xcolor}
\usepackage{enumitem}
\usepackage{sectsty}

\setcounter{MaxMatrixCols}{10}
%TCIDATA{OutputFilter=Latex.dll}
%TCIDATA{Version=5.50.0.2960}
%TCIDATA{<META NAME="SaveForMode" CONTENT="1">}
%TCIDATA{BibliographyScheme=Manual}
%TCIDATA{LastRevised=Sunday, September 07, 2025 19:37:50}
%TCIDATA{<META NAME="GraphicsSave" CONTENT="32">}

\geometry{hmargin={3cm,3cm},vmargin={2.5cm,2.5cm},offset=0pt,footskip=1cm,headsep=1cm}
\hypersetup{
  pdftitle={Títol del llibre},
  pdfauthor={Autor/a},
  colorlinks=true,
  linktoc=page,
  linkcolor=blue,
  urlcolor=blue,
  citecolor=blue
}
\definecolor{fons}{RGB}{243,231,220}
\definecolor{apren}{RGB}{255,64,87}
\definecolor{cdefn}{RGB}{101,255,205}
\definecolor{cexem}{RGB}{23,205,255}
\definecolor{cexer}{RGB}{137,255,133}
\definecolor{cobs}{RGB}{250,163,92}
\definecolor{cdemo}{RGB}{255,115,100}
\definecolor{csolu}{RGB}{110,132,169}
\theoremstyle{plain}
\newtheorem{teorema}{\color{apren}\textbf{Teorema}}[chapter]
\newtheorem{lema}[teorema]{\color{apren}Lema}
\newtheorem{corollari}[teorema]{\color{apren}Coro\l.lari}
\newtheorem{proposicio}[teorema]{\color{apren}Proposici\'{o}}
\theoremstyle{definition}
\newtheorem{definicio}{\color{apren}Definici\'{o}}[chapter]
\newtheorem{exemple}{\color{apren}Exemple}[chapter]
\newtheorem{exercici}{\color{apren}Exercici}[chapter]
\theoremstyle{remark}
\newtheorem{observacio}{\color{apren}Observaci\'{o}}[chapter]
\newtheorem*{nota}{\color{apren}Nota}
\newenvironment{prova}
  {\par\noindent{\color{apren}\textbf{Demostraci\'{o}:}\ }}{\quad\hfill $\blacksquare$\par}
\newenvironment{solucio}
  {\par\noindent{\color{apren}\textbf{Soluci\'{o}:}\ }}{\quad\hfill $\square$\par}
\setlist{itemsep=0.25em, topsep=0.25em}
\sectionfont{\color{apren}}
\subsectionfont{\color{apren}}
\subsubsectionfont{\color{apren}}
\setlength{\parindent}{0pt}
\numberwithin{equation}{section}
\setcounter{secnumdepth}{3}
\setcounter{tocdepth}{2}
\input{tcilatex}
\renewcommand{\FRAME}[8]{
\begin{center}
\includegraphics[width=#2,height=#3]{img/#7}
\end{center}}
\newcommand{\R}{\mathbb{R}}
\newcommand{\N}{\mathbb{N}}
\newcommand{\Z}{\mathbb{Z}}
\begin{document}

\frontmatter
\title{Lògica, Raonament i Demostració}
\author{Miquel Àngel Perelló}
\date{\today}
\maketitle
\tableofcontents
\frontmatter

\pagecolor{fons!20}

\let\cleardoublepage\clearpage

\chapter*{Prefaci}

Aquests apunts neixen de la voluntat de proporcionar a l'alumnat de
batxillerat una base s\`{o}lida i rigorosa en el llenguatge de les matem\`{a}%
tiques: la l\`{o}gica i les demostracions.

El cam\'{\i} cap a la comprensi\'{o} de les matem\`{a}tiques superiors est%
\`{a} pavimentat amb la capacitat de comprendre i construir arguments l\`{o}%
gics. Aquest document vol ser la primera pedra d'aquest cam\'{\i}. S'ha
elaborat amb la intenci\'{o} de ser una eina clara, accessible i, alhora,
precisa, que gui\"{\i} l'estudiant a trav\'{e}s dels conceptes fonamentals
de la l\`{o}gica proposicional, els quantificadors i les t\`{e}cniques de
demostraci\'{o}.

El material que es presenta aqu\'{\i} \'{e}s el fruit de l'experi\`{e}ncia
docent i la convicci\'{o} que la bellesa de les matem\`{a}tiques rau no nom%
\'{e}s en els seus resultats, sin\'{o} sobretot en el proc\'{e}s rigor\'{o}s
i elegant que condueix a ells. Especial \`{e}mfasi s'ha posat en la relaci%
\'{o} entre el llenguatge natural i el simb\`{o}lic, i en la traducci\'{o}
d'uns en altres, ja que \'{e}s en aquest pas on sovint es troben les majors
dificultats.

Agra\"{\i}ments a tots els estudiants les seves preguntes i curiositat han
anat donant forma a aquest text, i als companys docents pels seus valuosos
comentaris i suggeriments.

Esperem que aquests apunts us acompanyin en el vostre descobriment de com es
construeix i es valida el coneixement matem\`{a}tic.

\let\cleardoublepage\clearpage

\chapter{Introducci\'{o}}

Aquest document \'{e}s una introducci\'{o} als m\`{e}todes de raonament i
demostraci\'{o} propis de les matem\`{a}tiques. El seu objectiu \'{e}s
proporcionar les eines l\`{o}giques necess\`{a}ries per comprendre i
construir arguments matem\`{a}tics v\`{a}lids i rigorosos.

La finalitat \'{u}ltima \'{e}s que l'estudiant sigui capa\c{c} de:

\begin{itemize}
\item Entendre el significat prec\'{\i}s dels enunciats matem\`{a}tics.

\item Analitzar la seva estructura l\`{o}gica.

\item Construir demostracions correctes de resultats simples.
\end{itemize}

Per assolir-ho, el llibre est\`{a} estructurat en tres grans blocs:

\begin{enumerate}
\item Coneixements generals de L\`{o}gica (Cap\'{\i}tol 1: seccions 1 a 11):
S'introdueixen els conceptes b\`{a}sics de la l\`{o}gica proposicional i de
predicats. S'apren a treballar amb connectives l\`{o}giques (i, o, no,
si...llavors..., si i nom\'{e}s si), taules de veritat, tautologies, equival%
\`{e}ncies i regles d'infer\`{e}ncia. Aquests elements s\'{o}n l'alfabet amb
qu\`{e} s'escriuen les matem\`{a}tiques.

\item T\`{e}cniques de Demostraci\'{o} (Cap\'{\i}tol 1: secci\'{o} 12): Es
presenten i es practiquen les estrat\`{e}gies de demostraci\'{o} m\'{e}s
comunes i essencials: demostracions directes, per contrarec\'{\i}proc, per
reducci\'{o} a l'absurd, per inducci\'{o}, per casos i per contraexemple.
Aquestes s\'{o}n les "jugades" b\`{a}siques del joc de demostrar.

\item Pr\`{a}ctica i Aplicaci\'{o} (Cap\'{\i}tols 2, 3 i 4): La part te\`{o}%
rica es consolida amb una \`{a}mplia col\textperiodcentered lecci\'{o}
d'exercicis resolts, exercicis proposats i tests d'autoevaluaci\'{o} que
permeten posar a prova la comprensi\'{o} i aplicar els coneixements
adquirits.
\end{enumerate}

El document es tanca amb una molt breu mirada a com s'apliquen aquests
principis en el context de teories axiom\`{a}tiques concretes, com la
geometria euclidiana i l'aritm\`{e}tica, il\textperiodcentered lustrant que
tot l'edifici matem\`{a}tic es construeix sobre els fonaments l\`{o}gics que
aqu\'{\i} es presenten (Cap\'{\i}tul 1: secci\'{o} 13).

Aquests apunts no nom\'{e}s us prepararan per a assignatures futures sin\'{o}
que, \'{e}s el nostre desig, us obriran la porta a una visi\'{o} m\'{e}s
profunda i satisfact\`{o}ria del fascinant m\'{o}n de les matem\`{a}tiques.

Com utilitzar aquest document? Es recomana llegir la teoria amb atenci\'{o},
assegurant-se de comprendre cada concepte abans de passar al seg\"{u}ent.
Els exercicis de pr\`{a}ctica s\'{o}n una part indispensable de
l'aprenentatge; no si val a saltar-se'n cap! Us animem a intentar
resoldre'ls abans de mirar la soluci\'{o} proposada.

\let\cleardoublepage\clearpage

\mainmatter

\import{./capituls/}{teoria.tex} \import{./capituls/}{practica.tex} %
\import{./capituls/}{exercicis.tex}

\chapter{Tests}

\label{cap:tests}

\import{./tests/}{test1.tex} \import{./tests/}{test2.tex} %
\import{./tests/}{test3.tex}

\backmatter

\end{document}

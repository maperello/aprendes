\chapter{Exercicis proposats}
\label{cap:exercicis}

\begin{enumerate}
\item Reformuleu els enunciats seg\"{u}ents fent \'{u}s de variables i constants:

\begin{enumerate}
\item El cub d'un nombre senar \'{e}s senar.

\item Si el quadrat de un nombre enter no \'{e}s divisible per 4, aleshores
aquest nombre \'{e}s senar.

\item El producte de dos nombres consecutius \'{e}s parell.
\end{enumerate}

\item Quines de les expressions seg\"{u}ents s\'{o}n proposicions i quines predicats?

\begin{enumerate}
\item Si $x\geq2$, llavors $x^{3}\geq1.$

\item $(x+y)^{2}=x^{2}+2xy+y^{2}$.

\item Si $w=3$, llavors $z^{w}\neq0$.
\end{enumerate}

\item Quins dels seg\"{u}ents enunciats s\'{o}n predicats i quins no ho
s\'{o}n? Justifica la teva resposta.\qquad

\begin{enumerate}
\item $x$ \'{e}s divisible per 3.

\item La suma dels nombres \thinspace x\thinspace\ i \thinspace2.

\item $x^{2}-y^{2}$.

\item $x+2<y-3$.
\end{enumerate}

\item Considereu els enunciats seg\"{u}ents: $p=$ \textquotedblleft Estic
content\textquotedblright, $q=$ \textquotedblleft Estic veient una
pel\textperiodcentered l\'{\i}cula\textquotedblright\ i $r=$ \textquotedblleft
Estic menjant espaguetis\textquotedblright. Expresseu en paraules els
enunciats seg\"{u}ents: (a) $r\longrightarrow p$; (b) $p\longleftrightarrow
q$; (3) $q\vee\left(  r\longrightarrow p\right)  $; (4) $\left(
q\longrightarrow\lnot p\right)  \wedge\left(  r\longrightarrow\lnot p\right)
$.

\item Considereu els enunciats seg\"{u}ents: $p=$ \textquotedblleft L'Eduard
t\'{e} els cabells vermells\textquotedblright, $q=$ \textquotedblleft L'Eduard
t\'{e} un nas gran\textquotedblright\ i $r=$ \textquotedblleft A l'Eduard li
agrada menjar crispetes\textquotedblright. Tradueix els enunciats seg\"{u}ents
a s\'{\i}mbols: (a) \textquotedblleft A l'Eduard no li agrada menjar
crispetes\textquotedblright; (b) \textquotedblleft L'Eduard t\'{e} un nas gran
i els cabells vermells, o b\'{e} t\'{e} un nas gran i li agrada menjar
crispetes\textquotedblright.

\item Construeix les taules de veritat de les seg\"{u}ents expressions
l\`{o}giques: (a) $p\longrightarrow(q\wedge p)$; (b) $(p\wedge
q)\longleftrightarrow(p\vee\lnot r)$.

\item Considerem els predicats $P(x,y)=$ \textquotedblleft$x+y=4$%
\textquotedblright\ i $Q(x,y)=$ \textquotedblleft$x<y$\textquotedblright.
Troba els valors de veritat de les seg\"{u}ents expressions l\`{o}giques: (a)
$P(x,y)\wedge Q(x,y)$ i (b) $P(x,y)\longrightarrow\lnot Q(x,y)$ quan usem els
valos seg\"{u}ents (1) $x=3,y=1$; (2) $x=1,y=2$.

\item Proveu les implicacions seg\"{u}ents:

\begin{enumerate}
\item $(A\longrightarrow B)\wedge A$ $\Longrightarrow B$

\item (i) $A\wedge B$ $\Longrightarrow A$; (ii) $A\wedge B$ $\Longrightarrow
B$

\item (i) $A\Longrightarrow A\vee B$; (ii) $B\Longrightarrow A\vee B$

\item (i) $\left(  A\vee B\right)  \wedge\lnot B\Longrightarrow A$; (ii)
$\left(  A\vee B\right)  \wedge\lnot A\Longrightarrow B$

\item (i) $A\longleftrightarrow B\Longrightarrow A\longrightarrow B$; (ii)
$A\longleftrightarrow B\Longrightarrow B\longrightarrow A$

\item $\left(  A\longrightarrow B\right)  \wedge\left(  B\longrightarrow
A\right)  \Longrightarrow A\longleftrightarrow B$
\end{enumerate}

\item Proveu les equival\`{e}ncies l\`{o}giques seg\"{u}ents:

\begin{enumerate}
\item $\lnot\left(  \lnot A\right)  \Longleftrightarrow A$ (llei de la doble negaci\'{o})

\item $A\wedge B\Longleftrightarrow B\wedge A$ (llei commutativa de $\wedge$)

\item $A\vee B\Longleftrightarrow B\vee A$ (llei commutativa de $\vee$)

\item $\left(  A\wedge B\right)  \wedge C\Longleftrightarrow A\wedge\left(
B\wedge C\right)  $ (llei associativa de $\wedge$)

\item $\left(  A\vee B\right)  \vee C\Longleftrightarrow A\vee\left(  B\vee
C\right)  $ (llei associativa de $\vee$)

\item $A\vee\left(  B\wedge C\right)  \Longleftrightarrow\left(  A\vee
B\right)  \wedge\left(  A\vee C\right)  $ (llei distributiva de $\vee$
respecte de $\wedge$)

\item $A\wedge\left(  B\vee C\right)  \Longleftrightarrow\left(  A\wedge
B\right)  \vee\left(  A\wedge C\right)  $ (llei distributiva de $\wedge$
respecte de $\vee$)

\item $A\longrightarrow B\Longleftrightarrow\lnot B\longrightarrow\lnot A$
(llei del contrarec\'{\i}proc)

\item $A\longleftrightarrow B\Longleftrightarrow\left(  A\longrightarrow
B\right)  \wedge\left(  B\longrightarrow A\right)  $ (llei del bicondicional)

\item $\left(  A\vee B\right)  \longrightarrow C\Longleftrightarrow\left(
A\longrightarrow C\right)  \wedge\left(  B\longrightarrow C\right)  $

\item $\lnot(A\wedge B)\Longleftrightarrow\lnot A\vee\lnot B$ (llei de De Morgan)

\item $\lnot(A\vee B)\Longleftrightarrow\lnot A\wedge\lnot B$ (llei de De Morgan)
\end{enumerate}

\item Simplifiqueu les afirmacions seg\"{u}ents. Podeu fer \'{u}s de les
equival\`{e}ncies de l'exercici 9 a m\'{e}s de les equival\`{e}ncies
comentades a teoria i pr\`{a}ctica: (a) $\lnot(X\longrightarrow\lnot Y)$; (b)
$\left(  Y\wedge Z\right)  \longrightarrow Y$; (c) $\lnot(X\longrightarrow
Y)\vee Y$; (d) $\left(  X\longrightarrow Y\right)  \vee Y$.

\item Escriu la negaci\'{o} dels enunciats seg\"{u}ents: (a) $e^{3}>0$; (b)
$\sin\left(  \frac{\pi}{2}\right)  >0$ i $\tan0\geq0$; (c) $y-3>0$ implica
$y^{2}+9>6y$.

\item Per a cadascun dels arguments seg\"{u}ents, si \'{e}s v\`{a}lid, doneu
una deducci\'{o}, i si no \'{e}s v\`{a}lid, mostreu el perqu\`{e}.

\begin{enumerate}
\item $%
\begin{tabular}
[c]{ll}%
$P_{1}:$ & $p\longrightarrow q$\\
$P_{2}:$ & $\lnot r\longrightarrow\lnot q$\\
$P_{3}:$ & $s\longrightarrow t$\\
$P_{4}:$ & $p\vee s$\\\hline
$C:$ & $r\vee t$%
\end{tabular}
\ \ \ \ \ $

\item $%
\begin{tabular}
[c]{ll}%
$P_{1}:$ & $p\longrightarrow q$\\
$P_{2}:$ & $\lnot r\longrightarrow\left(  s\longrightarrow t\right)  $\\
$P_{3}:$ & $r\vee\left(  p\vee t\right)  $\\
$P_{4}:$ & $\lnot r$\\\hline
$C:$ & $q\vee s$%
\end{tabular}
\ \ \ \ \ \ $

\item $%
\begin{tabular}
[c]{ll}%
$P_{1}:$ & $\lnot p\longrightarrow\left(  q\longrightarrow\lnot r\right)  $\\
$P_{2}:$ & $r\longrightarrow\lnot p$\\
$P_{3}:$ & $\left(  \lnot s\vee p\right)  \longrightarrow\lnot\lnot r$\\
$P_{4}:$ & $\lnot s$\\\hline
$C:$ & $\lnot q$%
\end{tabular}
\ \ \ \ \ $
\end{enumerate}

\item Per a cadascun dels arguments seg\"{u}ents, si \'{e}s v\`{a}lid, doneu
una deducci\'{o}, i si no \'{e}s v\`{a}lid, mostreu el perqu\`{e}.

\begin{enumerate}
\item Si el rellotge est\`{a} avan\c{c}at, llavors en Joan va arribar abans de
les deu i va veure sortir el cotxe de l'Andreu. Si l'Andreu no diu la veritat,
llavors en Joan no va veure sortir el cotxe de l'Andreu. L'Andreu diu la
veritat o estava en l'edifici en el moment del crim. El rellotge est\`{a}
avan\c{c}at. Per tant, l'Andreu estava en l'edifici en el moment del crim.

\item Si $\alpha$ i $\beta$ s\'{o}n dos angles iguals, llavors $\alpha=45%
%TCIMACRO{\U{ba}}%
%BeginExpansion
{{}^o}%
%EndExpansion
$. Si $\beta=45%
%TCIMACRO{\U{ba}}%
%BeginExpansion
{{}^o}%
%EndExpansion
$, llavors $\alpha=90%
%TCIMACRO{\U{ba}}%
%BeginExpansion
{{}^o}%
%EndExpansion
$. O $\beta$ \'{e}s recte o b\'{e} $\beta=45%
%TCIMACRO{\U{ba}}%
%BeginExpansion
{{}^o}%
%EndExpansion
$. $\beta$ no \'{e}s recte. Per tant, $\alpha$ i $\beta$ no s\'{o}n iguals i
cap d'ells \'{e}s recte.
\end{enumerate}

\item Escriviu una deducci\'{o} per a cadascun dels arguments seg\"{u}ents,
tots ells s\'{o}n v\`{a}lids. Indiqueu si les premisses s\'{o}n consistents o inconsistents:

\begin{enumerate}
\item No \'{e}s el cas que la roba sigui molesta o no sigui barata. La roba no
\'{e}s barata o no est\`{a} de moda. Si la roba no est\`{a} de moda, \'{e}s
una ximpleria. Per tant, la roba \'{e}s una ximpleria.

\item Si al Marc li agrada la pizza, li agrada la cervesa. Si al Marc li
agrada la cervesa, no li agrada l'arengada. Si al Marc li agrada la pizza, li
agraden les arengades. Al Marc li agrada la pizza. Per tant, li agrada la
pizza d'arengada.
\end{enumerate}

\item Trobeu la fal\textperiodcentered l\`{a}cia o fal\textperiodcentered
l\`{a}cies en cadascun dels arguments seg\"{u}ents:

\begin{enumerate}
\item Si la meva tortuga menja una hamburguesa, es posa malalta. Si la meva
tortuga es posa malalta, llavors no \'{e}s feli\c{c}. Per tant, la meva
tortuga es posa malalta.

\item Si Frida es menja una granota, Susana menjar\`{a} una serp. Frida no
menja una granota. Per tant, Susana no menja una serp.
\end{enumerate}

\item Suposem que els valors possibles de $x$ s\'{o}n totes les persones.
Considerem els predicats seg\"{u}ents: $Y(x)=$ `$x$ t\'{e} el cabell verd',
$Z(x)=$ `$x$ li agrada les croquetes' i $W(x)=$ `$x$ t\'{e} una granota de
mascota', $L(x,y)=$`$x$ \'{e}s tan r\`{a}pid com $y$', $M(x,y)=$`$x$ \'{e}s
tan car com $y$' i $N(x,y)=$`$x$ \'{e}s tan vell com $y$'. Tradueix les
expressions seg\"{u}ents en paraules:%
\[%
\begin{tabular}
[c]{llllll}%
(1) & $\left(  \forall x\right)  Y(x)$ &  &  & (3) & $\left(  \exists
x\right)  \left(  Y(x)\longrightarrow Z(x)\right)  $\\
(2) & $\left(  \exists x\right)  Z(x)$ &  &  & (4) & $\left(  \forall
x\right)  \left(  W(x)\longleftrightarrow\lnot Z(x)\right)  $%
\end{tabular}
\ \ \ \
\]


\item Suposem que els valors possibles de $x$ i $y$ s\'{o}n tots els vehicles.
Considerem els predicats seg\"{u}ents: $L(x,y)=$`$x$ \'{e}s tan r\`{a}pid com
$y$', $M(x,y)=$`$x$ \'{e}s tan car com $y$' i $N(x,y)=$`$x$ \'{e}s tan vell
com $y$'. Tradueix les expressions seg\"{u}ents en paraules:%
\[%
\begin{tabular}
[c]{llllll}%
(1) & $\left(  \exists x\right)  \left(  \forall y\right)  L(x,y)$ &  &  &
(3) & $\left(  \exists y\right)  \left(  \forall x\right)  \left(  L(x,y)\vee
N(x,y)\right)  $\\
(2) & $\left(  \forall x\right)  \left(  \exists y\right)  M(x,y)$ &  &  &
(4) & $\left(  \forall y\right)  \left(  \exists x\right)  \left(  \lnot
M(x,y)\longrightarrow L(x,y)\right)  $%
\end{tabular}
\ \ \
\]


\item Expresseu formalment els enunciats seg\"{u}ents:%
\[%
\begin{tabular}
[c]{ll}%
(1) & La gent \'{e}s simp\`{a}tica.\\
(2) & A ning\'{u} li agraden els gelats i els embutits.\\
(3) & Em va agradar un dels llibres que vaig llegir l'estiu passat.\\
(4) & Existeix una vaca tal que, si t\'{e} quatre anys, no t\'{e} taques
blanques.\\
(5) & Per a cada fruita, hi ha una fruita m\'{e}s madura que ella.
\end{tabular}
\ \ \
\]


\item Escriu una negaci\'{o} de cada enunciat. No escriviu la paraula
\textquotedblleft no\textquotedblright\ davant de qualsevol dels objectes que
es quantifiquen (per exemple, no escriviu \textquotedblleft No tots els nois
s\'{o}n bons\textquotedblright\ per a la part (1) d'aquest exercici).%
\[%
\begin{tabular}
[c]{ll}%
(1) & Tots els nois s\'{o}n bons.\\
(2) & Hi ha ratpenats que pesen 3 kg o m\'{e}s.\\
(3) & L'equaci\'{o} $x^{2}-2x>0$ val per a tots els nombres reals $x$.\\
(4) & Hi ha un nombre enter $n$ tal que $n^{2}$ \'{e}s un nombre perfecte.\\
(5) & Cada casa t\'{e} una porta que \'{e}s blanca.
\end{tabular}
\ \ \ \
\]


\item Assumint com a domini de les variables $x$ i $y$ el conjunt $U$,
escriviu una deducci\'{o} per a cadascun dels arguments seg\"{u}ents:

\begin{enumerate}
\item
\begin{tabular}
[c]{ll}%
$P_{1}:$ & $\left(  \forall x\right)  \left(  R(x)\longrightarrow C(x)\right)
$\\
$P_{2}:$ & $\left(  \forall x\right)  \left(  T(x)\longrightarrow R(x)\right)
$\\\hline
$C:$ & $\left(  \forall x\right)  \left(  \lnot C(x)\longrightarrow\lnot
T(x)\right)  $%
\end{tabular}
$\ \ \ \ \ $

\item
\begin{tabular}
[c]{ll}%
$P_{1}:$ & $\left(  \forall x\right)  \left(  \exists y\right)  \left(
E(x)\longrightarrow\left(  M(x)\vee N(x)\right)  \right)  $\\
$P_{2}:$ & $\lnot\left(  \forall x\right)  \ M(x)$\\
$P_{3}:$ & $\left(  \forall x\right)  \ E(x)$\\\hline
$C:$ & $\left(  \exists x\right)  \ N(x)$%
\end{tabular}

\end{enumerate}

\item Escriviu una deducci\'{o} per a cadascun dels arguments seg\"{u}ents:

\begin{enumerate}
\item Tots els peixos amb moltes espines no s\'{o}n agradables de menjar. Tots
els peixos amb moltes espines s\'{o}n peixos blancs. Per tant, tots els peixos
que siguin agradables de menjar s\'{o}n peixos blancs.

\item Tots els estudiants d'un institut de secund\`{a}ria que assisteixen a
classes d'ampliaci\'{o} s\'{o}n genials. Hi ha un estudiant de l'institut que
\'{e}s intel\textperiodcentered ligent i no \'{e}s genial. Per tant, hi ha un
estudiant de l'institut que \'{e}s intel\textperiodcentered ligent i no
assisteix a classes d'ampliaci\'{o}.
\end{enumerate}

\item Constru\"{\i}u definicions de `quadrat perfecte',
`paral\textperiodcentered lela' i `funci\'{o} real de variable real'. Quins
s\'{o}n els dominis de les variables, quins termes s'han d'assumir com a coneguts?

\item Definiu la igualtat de dos nombres racionals, de dues rectes en el pla i
de dues funcions reals de variable real.

\item Considerem dos nombres enters $x$ i $y$, llavors proveu directament que

\begin{enumerate}
\item si $x$ i $y$ s\'{o}n parells, aleshores $x+y$ \'{e}s parell.

\item si $x$ i $y$ s\'{o}n senars, aleshores $x+y$ \'{e}s parell.

\item si un d'ells \'{e}s senar i l'altre \'{e}s parell, aleshores $x+y$
\'{e}s senar.
\end{enumerate}

\item Si $n$ \'{e}s un nombre natural imparell, proveu directament que $n^{2}
$ \'{e}s de la forma $8k+1$, per a algun sencer $k\geq1$.

\item Demostreu directament que les solucions de l'equaci\'{o} $ax^{2}+bx+c=0
$, on $a,b$ i $c$ s\'{o}n nombres reals, s\'{o}n expressades per la
f\'{o}rmula
\[
x=\frac{-b\pm\sqrt{b^{2}-4ac}}{2a}.
\]


\item Proveu cap a enrere:

\begin{enumerate}
\item Que qualsevol enter divideix zero.

\item Que per a tots $x,y\in\mathbb{R}$, $xy=0$ si i nom\'{e}s si $x=0$ o
$y=0$.
\end{enumerate}

\item Proveu per contrarec\'{\i}proc:

\begin{enumerate}
\item Que per a qualssevol nombres reals $a$ i $b$, $\left\vert a\right\vert
<\left\vert b\right\vert $ si i nom\'{e}s si $a^{2}<b^{2}$.

\item Que per a qualssevol nombres reals $x$ i $y$, $x^{2}+y^{2}=0$ si i
nom\'{e}s si $x=y=0$.
\end{enumerate}

\item Considerem que $a$ \'{e}s un nombre racional i $b$ \'{e}s irracional.
Llavors, proveu per reducci\'{o} a l'absurd:

\begin{enumerate}
\item Que$~a+b$ \'{e}s irracional

\item Si $a\neq0$, aleshores $ab$ \'{e}s irracional.
\end{enumerate}

\item Proveu per reducci\'{o} a l'abssurd que hi ha infinits nombres primers.

\item Donats $a,b,c\in\mathbb{Z}$, proveu per reducci\'{o} a l'absurd que si
$a$ no divideix $bc$, llavors $a$ no divideix $b$.

\item \'{E}s pot demostrar que $n^{3}-n$ \'{e}s m\'{u}ltiple de 3, $n^{5}-n$
\'{e}s m\'{u}ltiple de $5$ i $n^{7}-n$ \'{e}s m\'{u}ltiple de 7, per\`{o}
$n^{k}-n$ \'{e}s m\'{u}ltiple de $k$?

\item Proveu per inducci\'{o}:

\begin{enumerate}
\item Si $n\in\mathbb{N}$, llavors $1+3+5+\cdots+\left(  2n-1\right)  =n^{2}$.

\item Si $n\in\mathbb{N}$ i $n\geq5$, $4n>n^{4}$.

\item Si $n$ \'{e}s un nombre enter no negatiu, llavors $5\mid\left(
n^{5}-n\right)  $.
\end{enumerate}

\item (Desigualtat triangular) Si $x,y\in\mathbb{R}$, proveu que $\left\vert
x+y\right\vert \leq\left\vert x\right\vert +\left\vert y\right\vert $.

\item Proveu que no existeixen dos nombres enters $m$ i $n$ tals que
$14m+21n=100$.

\item Considerem que $a,b\in\mathbb{N}$. Llavors hi ha un \'{u}nic
$d\in\mathbb{N}$ tal que: Un enter $m$ \'{e}s m\'{u}ltiple de $d$ si i
nom\'{e}s si $m=ax+by$ per alguns $x,y\in\mathbb{Z}$.

\item Definim per recursi\'{o} per a qualsevol enter no negatiu $k\,:$

\begin{itemize}
\item $u_{0}=0$,

\item $u_{k+1}=3u_{k}+3^{k}$.
\end{itemize}

Proveu que $u_{n}=n\cdot3^{n-1}$ per a tot enter no negatiu $n$.

\item Demostreu que el principi de bona ordenaci\'{o} implica el principi d'inducci\'{o}.

\begin{description}
\item \textbf{Suggeriment}: Considereu una proposici\'{o} $p(n)$ sobre el
nombre natural $n$. Suposeu que $p(1)$ \'{e}s vertadera, i tamb\'{e} que per a
tots els $n$, $p(n)\longrightarrow p(n+1)$. Heu de demostrar que per a tot
$n$, $p(n)$ \'{e}s vertadera. Ara feu la prova per reducci\'{o} a l'absurd.
Suposeu que hi ha algun $n$ pel qual $p(n)$ \'{e}s falsa. Llavors, considereu
el conjunt $K$ que t\'{e} com a elements els nombres naturals $n$ que no
compleixen $p(n)$. Ara heu de concloure que aix\`{o} \'{e}s una contradicci\'{o}.
\end{description}
\end{enumerate}

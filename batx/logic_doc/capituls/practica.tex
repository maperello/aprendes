\chapter{Pràctica}
\label{cap:practica}

\section{L\`{o}gica}

\begin{exercici}
Reformuleu els enunciats seg\"{u}ents fent \'{u}s de variables i constants:
(a) El producte de dos nombres parells \'{e}s parell; (b) Donat un nombre real
no negatiu, busqueu dos nombres reals la difer\`{e}ncia dels seus quadrats no
sigui m\'{e}s gran que el nombre donat.
\end{exercici}

\begin{solucio}
(a) Si simbolitzem per $P$ el predicat \textquotedblleft ser
parell\textquotedblright, aleshores podem escriure:%
\[
\forall x,y\left(  P\left(  x\right)  \wedge P\left(  y\right)
\longrightarrow P\left(  x\cdot y\right)  \right)  \text{.}%
\]
Observa que tamb\'{e} podem escriure:%
\[
\forall x\left(  P\left(  x\right)  \longleftrightarrow\left(  \exists
k\right)  \left(  k\in\mathbb{Z}\wedge x=2k\right)  \right)  \text{.}%
\]


(b) Suposem donat un nombre real $\delta\geq0$, o sigui no negatiu, aleshores
podem escriure:%
\[
\exists x,y\left(  x,y\in\mathbb{R}\wedge x^{2}-y^{2}\leq\delta\right)
\text{.}%
\]

\end{solucio}

\begin{exercici}
Quines de les expressions seg\"{u}ents s\'{o}n proposicions i quines
predicats? (a) $4<3$; (b) $y>7$; (c) $x+y=z$.
\end{exercici}

\begin{solucio}
(a) \'{E}s una proposici\'{o} perqu\`{e} \'{e}s un enunciat vertader; (b)
\'{E}s un predicat perqu\`{e} \'{e}s un enunciat que cont\'{e} una variable
$y$ i si li assignem un valor es t\'{e} una proposici\'{o}; (c) \'{E}s un
predicat de tres variables $x,y$ i $z$. Assignant valors a les tres variables
s'obt\'{e} una proposici\'{o}.
\end{solucio}

\begin{exercici}
Quins dels seg\"{u}ents enunciats s\'{o}n predicats i quins no ho s\'{o}n?
Justifica la teva resposta. (a) $x$ \'{e}s un divisor de 210; (b) El producte
dels nombres $x$ i $y$; (c) La suma de dos nombres es menor que $1$.
\end{exercici}

\begin{solucio}
(a) \'{E}s un predicat de una variable $x$ que \'{e}s vertader quan, per
exemple, $x$ pren el valor $7$, i fals quan, per exemple, \'{e}s $8$; (b) No
\'{e}s predicat perqu\`{e} quan assignem valors a les variables $x$ i $y$ no
s'obt\'{e} una proposici\'{o}. De fet, l'enunciat \'{e}s l'expressi\'{o}
algebraica $x\cdot y$; (c) \'{E}s un predicat que cont\'{e} dues variables.
Podem expressar-lo com $x+y<1$. \'{E}s clar que quan donme valors a $x$ i $y$
s'obt\'{e} una proposici\'{o}.
\end{solucio}

\begin{exercici}
Considereu els enunciats seg\"{u}ents: $p=$ \textquotedblleft Estic
content\textquotedblright, $q=$ \textquotedblleft Estic veient una
pel\textperiodcentered l\'{\i}cula\textquotedblright\ i $r=$ \textquotedblleft
Estic menjant espagueti\textquotedblright. Expresseu en paraules els enunciats
seg\"{u}ents: (a) $\left(  q\vee r\right)  \longrightarrow p$; (b) $\left(
p\wedge\lnot q\right)  \longleftrightarrow\left(  q\vee r\right)  $.
\end{exercici}

\begin{solucio}
(a) Estic content quan veig una pel\textperiodcentered l\'{\i}cula o mengo
espagueti; (b) Veig una pel\textperiodcentered l\'{\i}cula o mengo espagueti
nom\'{e}s si estic content sense veure una pel\textperiodcentered l\'{\i}cula.
\end{solucio}

\begin{exercici}
Analitza les formes l\`{o}giques dels enunciats seg\"{u}ents: (a) El joc es
cancel\textperiodcentered lar\`{a} si plou o neva; (b) Tenir almenys deu
persones \'{e}s una condici\'{o} necess\`{a}ria i suficient per a la
confer\`{e}ncia que s'est\`{a} impartint; (c) Si en Miquel va anar a la
botiga, llavors tenim ous a casa, si no no en tenim.
\end{exercici}

\begin{solucio}
(a) Si $C=$\textquotedblleft El joc ser\`{a} cancel\textperiodcentered
lat\textquotedblright, $P=$\textquotedblleft Plou\textquotedblright\ i
$N=$\textquotedblleft Est\`{a} nevant\textquotedblright. Aleshores l'enunciat
\textquotedblleft El joc es cancel\textperiodcentered lar\`{a} si plou o
neva\textquotedblright\ es la proposici\'{o} $C\leftrightarrow(P\vee N)$.

(b) Si $P=$\textquotedblleft Hi ha almenys deu persones\textquotedblright\ i
$C=$\textquotedblleft La confer\`{e}ncia es donar\`{a}\textquotedblright.
Aleshores l'enunciat \textquotedblleft Tenir almenys deu persones \'{e}s una
condici\'{o} necess\`{a}ria i suficient per a la confer\`{e}ncia que
s'est\`{a} impartint\textquotedblright\ \'{e}s la proposici\'{o}
$P\longleftrightarrow C$.

(c) Si $B=$\textquotedblleft Miquel va anar a la botiga\textquotedblright\ i
$C=$\textquotedblleft Hi ha ous a casa". Aleshores, l'enunciat
\textquotedblleft Si en Miquel va anar a la botiga, llavors tenim ous a casa,
si no no en tenim\textquotedblright\ \'{e}s la proposici\'{o} $(B\rightarrow
C)\wedge(\lnot B\rightarrow\lnot C)$. Aix\`{o} \'{e}s equivalent a
$B\longleftrightarrow C$ perqu\`{e} per la llei del contrarec\'{\i}proc
s'obt\'{e} $(B\rightarrow C)\wedge(C\rightarrow B)$, qu\`{e} \'{e}s alhora
equivalent al que hem dit.
\end{solucio}

\begin{exercici}
Considereu els enunciats seg\"{u}ents: $p=$ \textquotedblleft L'Eduard t\'{e}
els cabells vermells\textquotedblright, $q=$ \textquotedblleft L'Eduard t\'{e}
un nas gran\textquotedblright\ i $r=$ \textquotedblleft A l'Eduard li agrada
menjar crispetes\textquotedblright. Tradueix els enunciats seg\"{u}ents a
s\'{\i}mbols: (a) \textquotedblleft L'Eduard t\'{e} els cabells vermells i no
t\'{e} un nas gran\textquotedblright; (b) \textquotedblleft No \'{e}s el cas
que l'Eduard tingui un nas gran o li agradi menjar crispetes\textquotedblright.
\end{exercici}

\begin{solucio}
(a) $p\wedge\lnot q$; (b) $\lnot(q\vee r)$.
\end{solucio}

\begin{exercici}
Construeix les taules de veritat de les seg\"{u}ents expressions l\`{o}giques:
(a) $(p\wedge q)\vee\lnot p$; (b) $\lnot p\longrightarrow\lnot(q\vee r)$.
\end{exercici}

\begin{solucio}
(a)%
\[%
\begin{tabular}
[c]{ll|l|ll}%
$p$ & $q$ & $\lnot p$ & $p\wedge q$ & $(p\wedge q)\vee\lnot p$\\\hline
\multicolumn{1}{c}{V} & \multicolumn{1}{c|}{V} & F & \multicolumn{1}{|c}{V} &
V\\
\multicolumn{1}{c}{V} & \multicolumn{1}{c|}{F} & F & \multicolumn{1}{|c}{F} &
F\\
\multicolumn{1}{c}{F} & \multicolumn{1}{c|}{V} & V & \multicolumn{1}{|c}{F} &
V\\
\multicolumn{1}{c}{F} & \multicolumn{1}{c|}{F} & V & \multicolumn{1}{|c}{F} &
V
\end{tabular}
\]


(b)%
\[%
\begin{tabular}
[c]{cc|c|c|c|cc}%
$p$ & $q$ & $r$ & $\lnot p$ & $q\wedge r$ & $\lnot(q\vee r)$ & $\lnot
p\longrightarrow\lnot(q\vee r)$\\\hline
V & V & V & F & V & F & V\\
V & V & F & F & F & V & V\\
V & F & V & F & F & V & V\\
V & F & F & F & F & V & V\\
F & V & V & V & V & F & F\\
F & V & F & V & F & V & V\\
F & F & V & V & F & V & V\\
F & F & F & V & F & V & V
\end{tabular}
\]

\end{solucio}

\begin{exercici}
Considerem els predicats $P(x,y)=$ `$x+y=4$' i $Q(x,y)=$ `$x<y$'. Troba els
valors de veritat de les seg\"{u}ents expressions l\`{o}giques: (a) $\lnot
P(x,y)\vee Q(x,y)$ i (b) $\lnot(P(x,y)\longleftrightarrow Q(x,y))$ quan usem
els valos seg\"{u}ents (1) $x=1,y=3$; (2) $x=2,y=1$.
\end{exercici}

\begin{solucio}
(a) Considerem $\lnot P(x,y)\vee Q(x,y)$, aleshores es tenen les proposicions:
(1) $\lnot P(1,3)\vee Q(1,3)$ qu\`{e} \'{e}s vertadera doncs $Q(1,3)$ ho
\'{e}s al complir-se $1<3$; (2) $\lnot(P(1,3)\longleftrightarrow Q(1,3))$
qu\`{e} \'{e}s falsa doncs $P(1,3)\longleftrightarrow Q(1,3)$ es vertadera al
complir-se $1+3=4$ i $1<3$.

(b) Considerem el cas en que $x=2$ i $y=1$. Aleshores (1) $\lnot P(2,1)\vee
Q(2,1)$ \'{e}s vertadera doncs $\lnot P(2,1)$ ho \'{e}s al complir-se
$2+1\neq4$; (2) $\lnot(P(2,1)\longleftrightarrow Q(2,1))$ \'{e}s falsa doncs
$P(2,1)$ i $Q(2,1)$ s\'{o}n ambdues falses.
\end{solucio}

\begin{exercici}
Proveu les implicacions seg\"{u}ents: (a) $(A\longrightarrow B)\wedge\lnot
B\Longrightarrow\lnot A$; (b) $\left(  A\longrightarrow B\right)
\wedge\left(  B\longrightarrow C\right)  \Longrightarrow A\longrightarrow C$.
\end{exercici}

\begin{solucio}
(a) Per veure que $(A\longrightarrow B)\wedge\lnot B\Longrightarrow\lnot A$
hem de comprovar que $\left(  (A\longrightarrow B)\wedge\lnot B\right)  $
$\longrightarrow\lnot A$ \'{e}s tautologia. Constru\"{\i}m la taula de veritat
corresponent:%
\[%
\begin{tabular}
[c]{ccccccc}%
$A$ & $B$ & $\lnot A$ & $\lnot B$ & $A$ $\longrightarrow$ $B$ &
$(A\longrightarrow B)\wedge\lnot B$ & $\left(  (A\longrightarrow B)\wedge\lnot
B\right)  $ $\longrightarrow\lnot A$\\\hline
V & V & F & F & V & F & V\\
V & F & F & V & F & F & V\\
F & V & V & F & V & F & V\\
F & F & V & V & V & V & V
\end{tabular}
\]
i s'obt\'{e} una tautologia doncs veiem que l'\'{u}ltima columna nom\'{e}s
t\'{e} el valor V.

(b) Per veure que $\left(  A\longrightarrow B\right)  \wedge\left(
B\longrightarrow C\right)  \Longrightarrow A\longrightarrow C$ hem de
comprovar que $\left(  \left(  A\longrightarrow B\right)  \wedge\left(
B\longrightarrow C\right)  \right)  $ $\longrightarrow\left(  A\longrightarrow
C\right)  $ \'{e}s tautologia. Constru\"{\i}m la taula de veritat corresponent
prenent $\alpha=A\longrightarrow B$, $\beta=B\longrightarrow C $ i
$\gamma=A\longrightarrow C$:%
\[%
\begin{tabular}
[c]{cccccccc}%
$A$ & $B$ & $C$ & $\alpha$ & $\beta$ & $\gamma$ & $\alpha\wedge\beta$ &
$\left(  \alpha\wedge\beta\right)  \longrightarrow\gamma$\\\hline
V & V & V & V & V & V & V & V\\
V & V & F & V & F & F & F & V\\
V & F & V & F & V & V & F & V\\
V & F & F & F & V & F & F & V\\
F & V & V & V & V & V & V & V\\
F & V & F & V & F & V & F & V\\
F & F & V & V & V & V & V & V\\
F & F & F & V & V & V & V & V
\end{tabular}
\]
i surt que \'{e}s tautologia com era d'esperar.
\end{solucio}

\begin{exercici}
Proveu les equival\`{e}ncies l\`{o}giques seg\"{u}ents: (a) $A\longrightarrow
B\Longleftrightarrow\lnot A\vee B$; (b) $A\longrightarrow\left(  B\wedge
C\right)  \Longleftrightarrow\left(  A\longrightarrow B\right)  \wedge\left(
A\longrightarrow C\right)  $.
\end{exercici}

\begin{solucio}
(a) Per veure que $A\longrightarrow B\Longleftrightarrow\lnot A\vee B$ hem de
comprovar que $(A\longrightarrow B)\longleftrightarrow\left(  \lnot A\vee
B\right)  $ \'{e}s tautologia. Constru\"{\i}m la taula de veritat
corresponent:%
\[%
\begin{tabular}
[c]{cccccc}%
$A$ & $B$ & $\lnot A$ & $A$ $\longrightarrow$ $B$ & $\lnot A\vee B$ &
$(A\longrightarrow B)\longleftrightarrow\left(  \lnot A\vee B\right)
$\\\hline
V & V & F & V & V & V\\
V & F & F & F & F & V\\
F & V & V & V & V & V\\
F & F & V & V & V & V
\end{tabular}
\]
i s'obt\'{e} una tautologia doncs veiem que l'\'{u}ltima columna nom\'{e}s
t\'{e} el valor V.

(b) Per veure que $A\longrightarrow\left(  B\wedge C\right)
\Longleftrightarrow\left(  A\longrightarrow B\right)  \wedge\left(
A\longrightarrow C\right)  $ hem de comprovar que $(A\longrightarrow\left(
B\wedge C\right)  )\longleftrightarrow\left(  \left(  A\longrightarrow
B\right)  \wedge\left(  A\longrightarrow C\right)  \right)  $ \'{e}s
tautologia. Constru\"{\i}m la taula de veritat corresponent, prenent
$\alpha=A\longrightarrow B$, $\beta=A\longrightarrow C$ i $\gamma
=A\longrightarrow\left(  B\wedge C\right)  $:
\[%
\begin{tabular}
[c]{ccccccccc}%
$A$ & $B$ & $C$ & $\alpha$ & $B\wedge C$ & $\beta$ & $\gamma$ & $\delta
=\alpha\wedge\beta$ & $\gamma\longleftrightarrow\delta$\\\hline
V & V & V & V & V & V & V & V & V\\
V & V & F & V & F & F & F & F & V\\
V & F & V & F & F & V & F & F & V\\
V & F & F & F & F & F & F & F & V\\
F & V & V & V & V & V & V & V & V\\
F & V & F & V & F & V & V & V & V\\
F & F & V & V & F & V & V & V & V\\
F & F & F & V & F & V & V & V & V
\end{tabular}
\]
i surt que \'{e}s tautologia com era d'esperar.
\end{solucio}

\begin{exercici}
Simplifiqueu les afirmacions seg\"{u}ents. Podeu fer \'{u}s de les
equival\`{e}ncies de l'exercicicici anterior a m\'{e}s de les equival\`{e}ncies
comentades a teoria i pr\`{a}ctica: (a) $X\longrightarrow\left(  X\wedge
Y\right)  $; (b) $\lnot\left(  X\vee Y\right)  \wedge Y$.
\end{exercici}

\begin{solucio}
(a) Aplicant la primera equival\`{e}ncia de l'apartat anterior es t\'{e}:
$X\longrightarrow\left(  X\wedge Y\right)  \Longleftrightarrow\lnot
X\vee\left(  X\wedge Y\right)  $. Aplicant ara la llei distributiva de $\vee$
respecte de $\wedge$ s'obt\'{e}%
\[
\lnot X\vee\left(  X\wedge Y\right)  \Longleftrightarrow\left(  \lnot X\vee
X\right)  \wedge\left(  \lnot X\vee Y\right)
\]
Ara b\'{e}, $\lnot X\vee X$ \'{e}s tautologia. Per tant, $\left(  \lnot X\vee
X\right)  \wedge\left(  \lnot X\vee Y\right)  \Longleftrightarrow\lnot X\vee Y
$. Com a consequ\`{e}ncia tenim que $X\longrightarrow\left(  X\wedge Y\right)
\Longleftrightarrow X\longrightarrow Y$, doncs $\lnot X\vee
Y\Longleftrightarrow X\longrightarrow Y$.

(b) Aplicant les lleis de De Morgan es t\'{e}:%
\[
\lnot\left(  X\vee Y\right)  \wedge Y\Longleftrightarrow\left(  \lnot
X\wedge\lnot Y\right)  \vee Y.
\]
Ara, aplicant la llei distributiva de $\vee$ respecte de $\wedge$, s'obt\'{e}
\[
\lnot\left(  X\vee Y\right)  \wedge Y\Longleftrightarrow\left(  \lnot X\vee
Y\right)  \wedge\left(  \lnot Y\vee Y\right)  .
\]
Finalment, com $\lnot Y\vee Y$ \'{e}s tautologia i $\lnot X\vee
Y\Longleftrightarrow X\longrightarrow Y$, es t\'{e}%
\[
\lnot\left(  X\vee Y\right)  \wedge Y\Longleftrightarrow X\longrightarrow
Y\text{.}%
\]

\end{solucio}

\begin{exercici}
Escriu la negaci\'{o} dels enunciats seg\"{u}ents: (a) $3>5$ o $7\leq8$; (b)
Si $x=3$, llavors $x^{2}=5$; (c)\ $a-b=c$ si i nom\'{e}s si $a=b+c$.
\end{exercici}

\begin{solucio}
(a) Aplicant la llei de de Morgan, $3\leq5$ o $7>8$.

(b) L'enunciat \'{e}s de la forma $A\longrightarrow B$, aleshores la seva
negaci\'{o} \'{e}s $\lnot\left(  A\longrightarrow B\right)  \,$. Ara b\'{e},
sabem que $A\longrightarrow B\Longleftrightarrow\lnot A\vee B$ i aplicant de
nou la llei de De Morgan es t\'{e} $\lnot\left(  A\longrightarrow B\right)
\Longleftrightarrow A\wedge\lnot B$. Per tant, la negaci\'{o} de l'enunciat
\'{e}s $x=3$ i $x^{2}\neq5$.

(c) Observem primer que $a-b=c$ \'{e}s equivalent a $a=b+c$. Per tant,
l'enunciat \'{e}s tautol\`{o}gic i de la forma $A\longleftrightarrow A$,
aleshores la seva negaci\'{o} \'{e}s contradicci\'{o}: $a-b=c$ i $a\neq b+c$.
\end{solucio}

\section{Raonament l\`{o}gic}

\begin{exercici}
Per a cadascun dels raonaments seg\"{u}ents, si \'{e}s v\`{a}lid, doneu una
deducci\'{o}, i si no \'{e}s v\`{a}lid, mostreu el perqu\`{e}.

\begin{enumerate}
\item Si el fre falla i el cam\'{\i} est\`{a} gelat, llavors el cotxe no
parar\`{a}. Si el cotxe es va revisar, no fallaran els frens. Per\`{o} el
cotxe no es va revisar. Per tant, el cotxe no parar\`{a}.

\item Si el punt d'una recta representa un enter, llavors el n\'{u}mero es pot
definir com un decimal infinit o per un parell de decimals infinits. O el
n\'{u}mero es pot definir per un decimal finit o el n\'{u}mero pot ser definit
o b\'{e} per un decimal infinit o per un parell de decimals infinits. El
n\'{u}mero no pot ser definit per un decimal finit. Per tant, el punt de la
recta representa un enter.

\item Si el rellotge est\`{a} avan\c{c}at, llavors en Joan va arribar abans de
les deu i va veure sortir el cotxe de l'Andreu. Si l'Andreu no diu la veritat,
llavors en Joan no va veure sortir el cotxe de l'Andreu. L'Andreu no diu la
veritat o estava en l'edifici en el moment del crim. El rellotge est\`{a}
avan\c{c}at. Per tant, l'Andreu estava en l'edifici en el moment del crim.
\end{enumerate}
\end{exercici}

\begin{solucio}
(1) Les proposicions simples d'aquest raonament s\'{o}n: $p=$\textquotedblleft
els frens fallen\textquotedblright, $q=$\textquotedblleft el cam\'{\i}
est\`{a} gelat\textquotedblright, $r=$\textquotedblleft el cotxe parar\'{a}" i
$s=$\textquotedblleft el cotxe es va revisar\textquotedblright. Expressem ara
el raonament simb\`{o}licament:%
\[%
\begin{tabular}
[c]{ll}%
$P_{1}:$ & $\left(  p\wedge q\right)  \longrightarrow\lnot r$\\
$P_{2}:$ & $s\longrightarrow\lnot p$\\
$P_{3}:$ & $\lnot s$\\\hline
$C:$ & $\lnot r$%
\end{tabular}
\]


L'argument \'{e}s fals. Prenem com interpretaci\'{o}: $%
\begin{tabular}
[c]{cccc}%
$p$ & $q$ & $r$ & $s$\\\hline
$V$ & $F$ & $V$ & $F$%
\end{tabular}
$, aleshores les premisses s\'{o}n vertaderes
\begin{tabular}
[c]{ccc}%
$\left(  p\wedge q\right)  \longrightarrow\lnot r$ & $s\longrightarrow\lnot p
$ & $\lnot s$\\\hline
$V$ & $V$ & $V$%
\end{tabular}
i la conclusi\'{o}, falsa
\begin{tabular}
[c]{c}%
$\lnot r$\\\hline
$F$%
\end{tabular}
.

(2) Les proposicions simples d'aquest raonament s\'{o}n: $p=$\textquotedblleft
el punt d'una recta representa un enter\textquotedblright, $q=$%
\textquotedblleft el n\'{u}mero es pot definir com un decimal
infinit\textquotedblright, $r=$\textquotedblleft el n\'{u}mero es pot definir
per un parell de decimals infinits" i $s=$\textquotedblleft el n\'{u}mero es
pot definir per un decimal finit\textquotedblright. Expressem ara el raonament
simb\`{o}licament:%
\[%
\begin{tabular}
[c]{ll}%
$P_{1}:$ & $p\longrightarrow\left(  q\vee r\right)  $\\
$P_{2}:$ & $\lnot\left(  s\longleftrightarrow\left(  q\vee r\right)  \right)
$\\
$P_{3}:$ & $\lnot s$\\\hline
$C:$ & $p$%
\end{tabular}
\]


L'argument \'{e}s fals. Prenem com interpretaci\'{o}: $%
\begin{tabular}
[c]{cccc}%
$p$ & $q$ & $r$ & $s$\\\hline
$F$ & $V$ & $V$ & $F$%
\end{tabular}
$, aleshores les premisses s\'{o}n vertaderes
\begin{tabular}
[c]{ccc}%
$p\longrightarrow\left(  q\vee r\right)  $ & $\lnot\left(
s\longleftrightarrow\left(  q\vee r\right)  \right)  $ & $\lnot s$\\\hline
$V$ & $V$ & $V$%
\end{tabular}
i la conclusi\'{o}, falsa
\begin{tabular}
[c]{c}%
$p$\\\hline
$F$%
\end{tabular}
.

(3) Les proposicions simples d'aquest raonament s\'{o}n: $p=$\textquotedblleft
el rellotge est\`{a} avan\c{c}at\textquotedblright, $q=$\textquotedblleft Joan
va arribar abans de les deu\textquotedblright, $r=$\textquotedblleft Joan va
veure sortir el cotxe de l'Andreu", $s=$\textquotedblleft l'Andreu diu la
veritat\textquotedblright\ i $t=$\textquotedblleft Andreu estava en l'edifici
en el moment del crim\textquotedblright. Expressem ara el raonament
simb\`{o}licament:%
\[%
\begin{tabular}
[c]{ll}%
$P_{1}:$ & $p\longrightarrow\left(  q\wedge r\right)  $\\
$P_{2}:$ & $\lnot s\longrightarrow\lnot r$\\
$P_{3}:$ & $\lnot s\vee t$\\
$P_{4}:$ & $p$\\\hline
$C:$ & $t$%
\end{tabular}
\]
Ara provarem la validesa d'aquest argument utilitzant les regles
d'infer\`{e}ncia:%
\[%
\begin{tabular}
[c]{lll}%
1. & $p\longrightarrow\left(  q\wedge r\right)  $ & $P_{1}$\\
2. & $\lnot s\longrightarrow\lnot r$ & $P_{2}$\\
3. & $\lnot s\vee t$ & $P_{3}$\\
4. & $p$ & $P_{4}$\\
5. & $q\wedge r$ & MP(1,4)\\
6. & $r$ & EC 5\\
7. & $\lnot\lnot r$ & DN 6\\
8. & $\lnot\lnot s$ & MT(2,7)\\\hline
9. & $t$ & ED(3,8)
\end{tabular}
\
\]

\end{solucio}

\begin{exercici}
Per a cadascun dels arguments seg\"{u}ents, si \'{e}s v\`{a}lid, doneu una
deducci\'{o}, i si no \'{e}s v\`{a}lid, mostreu el perqu\`{e}.

\begin{enumerate}
\item $%
\begin{tabular}
[c]{ll}%
$P_{1}:$ & $p\longrightarrow q$\\
$P_{2}:$ & $\lnot r\longrightarrow\left(  t\longrightarrow s\right)  $\\
$P_{3}:$ & $r\vee\left(  p\vee t\right)  $\\
$P_{4}:$ & $\lnot r$\\\hline
$C:$ & $q\vee s$%
\end{tabular}
\ \ \ \ \ \ $

\item $%
\begin{tabular}
[c]{ll}%
$P_{1}:$ & $\lnot p\longrightarrow\left(  q\longrightarrow\lnot r\right)  $\\
$P_{2}:$ & $r\longrightarrow\lnot p$\\
$P_{3}:$ & $\left(  \lnot s\vee p\right)  \longrightarrow\lnot\lnot r$\\
$P_{4}:$ & $\lnot s$\\\hline
$C:$ & $\lnot q$%
\end{tabular}
$
\end{enumerate}
\end{exercici}

\begin{solucio}
(1) L'argument \'{e}s v\`{a}lid i una deducci\'{o} \'{e}s:
\[%
\begin{tabular}
[c]{lll}%
1. & $p\longrightarrow q$ & $P_{1}$\\
2. & $\lnot r\longrightarrow\left(  t\longrightarrow s\right)  $ & $P_{2}$\\
3. & $r\vee\left(  p\vee t\right)  $ & $P_{3}$\\
4. & $\lnot r$ & $P_{4}$\\
5. & $p\vee t$ & ED(3,4)\\
6. & $t\longrightarrow s$ & MP(2,4)\\\hline
7. & $q\vee s$ & DL(1,5,6)
\end{tabular}
\]


(2) L'argument tamb\'{e} \'{e}s v\`{a}lid i una deducci\'{o} \'{e}s:%
\[%
\begin{tabular}
[c]{lll}%
1. & $\lnot p\longrightarrow\left(  q\longrightarrow\lnot r\right)  $ &
$P_{1}$\\
2. & $r\longrightarrow\lnot p$ & $P_{2}$\\
3. & $\left(  \lnot s\vee p\right)  \longrightarrow\lnot\lnot r$ & $P_{3}$\\
4. & $\lnot s$ & $P_{4}$\\
5. & $\lnot s\vee p$ & ID 4\\
6. & $\lnot\lnot r$ & MP(3,5)\\
7. & $r$ & DN 6\\
8. & $\lnot p$ & MP(2,7)\\
9. & $q\longrightarrow\lnot r$ & MP(1,8)\\\hline
10. & $\lnot q$ & MT(6,9)
\end{tabular}
\]

\end{solucio}

\begin{exercici}
Escriviu una deducci\'{o} per el seg\"{u}ent raonament. Indiqueu si les
premisses s\'{o}n consistents o inconsistents:\textquotedblleft Els ordinadors
s\'{o}n \'{u}tils i divertits i els ordinadors consumeixen molt de temps. Si
els ordinadors s\'{o}n dif\'{\i}cils d'utilitzar, no s\'{o}n divertits. Si els
ordinadors no estan ben dissenyats, llavors s\'{o}n dif\'{\i}cils d'utilitzar.
Per tant, els ordinadors estan ben dissenyats\textquotedblright.
\end{exercici}

\begin{solucio}
(1) Les proposicions simples d'aquest raonament s\'{o}n: $p=$\textquotedblleft
els ordinadors s\'{o}n \'{u}tils i divertits\textquotedblright, $q=$%
\textquotedblleft els ordinadors consumeixen molt de temps\textquotedblright,
$r=$\textquotedblleft els ordinadors s\'{o}n dif\'{\i}cils d'utilitzar" i
$s=$\textquotedblleft els ordinadors estan ben dissenyats\textquotedblright.
Expressem ara el raonament simb\`{o}licament:%
\[%
\begin{tabular}
[c]{ll}%
$P_{1}:$ & $p\wedge q$\\
$P_{2}:$ & $r\longrightarrow\lnot p$\\
$P_{3}:$ & $\lnot s\longrightarrow r$\\\hline
$C:$ & $s$%
\end{tabular}
\]


Ara provarem la validesa d'aquest argument utilitzant les regles
d'infer\`{e}ncia:%
\[%
\begin{tabular}
[c]{lll}%
1. & $p\wedge q$ & $P_{1}$\\
2. & $r\longrightarrow\lnot p$ & $P_{2}$\\
3. & $\lnot s\longrightarrow r$ & $P_{3}$\\
5. & $p$ & EC(1)\\
6. & $\lnot\lnot p$ & DN 5\\
7. & $\lnot r$ & MT(2,6)\\
8. & $\lnot\lnot s$ & MT(3,7)\\\hline
9. & $s$ & DN 8
\end{tabular}
\]


Podem provar la consist\`{e}ncia de les premisses comprovant que hi ha una
interpretaci\'{o} que faci totes les premisses vertaderes:%
\[%
\begin{tabular}
[c]{lll|llll|llll}%
$p$ & $\wedge$ & $q$ & $r$ & $\longrightarrow$ & $\lnot$ & $p$ & $\lnot$ & $s$
& $\longrightarrow$ & $r$\\
& V &  &  & V &  &  &  &  & V & \\
V & V & V & F & V & F & V & F & V & V & F
\end{tabular}
\
\]

\end{solucio}

\begin{exercici}
\'{E}s una fal\textperiodcentered l\`{a}cia el raonament seg\"{u}ent:
\textquotedblleft Si el poble elegeix un alcalde, s'augmentaran els impostos.
Si no s'augmenten els impostos al poble, llavors no es construir\`{a} un
estadi nou. El poble no tria un alcalde. Per tant, no es construir\`{a} un
estadi nou\textquotedblright.
\end{exercici}

\begin{solucio}
Si formalitzem l'argument prenent: $p=$\textquotedblleft el poble elegeix un
alcalde\textquotedblright, $q=$\textquotedblleft augmentar\`{a} els
impostos\textquotedblright, i $r=$\textquotedblleft es construir\`{a} un
estadi nou\textquotedblright. Aleshores, es t\'{e}%
\[%
\begin{tabular}
[c]{ll}%
$P_{1}:$ & $p\longrightarrow q$\\
$P_{2}:$ & $\lnot q\longrightarrow\lnot r$\\
$P_{3}:$ & $\lnot p$\\\hline
$C:$ & $\lnot r$%
\end{tabular}
\]
Es clar que es tracta d'una fal\textperiodcentered l\`{a}cia perqu\`{e} es
pensa que pel fet de no triar un alcalde no s'augmentaran els impostos i com a
consequ\`{e}ncia no es construir\`{a} un estadi nou. Aix\`{o} \'{e}s fals com
es pot veure a continuaci\'{o}:%
\[%
\begin{tabular}
[c]{lll|lllll|ll|ll}%
$p$ & $\longrightarrow$ & $q$ & $\lnot$ & $q$ & $\longrightarrow$ & $\lnot$ &
$r$ & $\lnot$ & $p$ & $\lnot$ & $r$\\
& V &  &  &  & V! &  &  & V &  & F & \\
F & V & F & V & F & F! & F & V & V & F & F & V
\end{tabular}
\]

\end{solucio}

\begin{exercici}
Formalitzeu els seg\"{u}ents enunciats: (1) \textquotedblleft Tots els enters
que no s\'{o}n senars s\'{o}n parells\textquotedblright; (2) \textquotedblleft
Hi ha un nombre enter que no \'{e}s parell\textquotedblright; (3)
\textquotedblleft Per a cada nombre real $x$, hi ha un nombre real $y$ per al
qual $y^{3}=x$\textquotedblright; i (4) \textquotedblleft Donats dos nombres
racionals $a$ i $b$, el producte $ab$ \'{e}s racional\textquotedblright.
\end{exercici}

\begin{solucio}
Primer identifiquem els predicats i despr\'{e}s escrivim l'enunciat simb\`{o}licament.

(1) Suposem que $P$ \'{e}s el predicat "ser parell" i $S$ \'{e}s el predicat
"ser senar", aleshores es t\'{e}:%
\[
\forall x\in\mathbb{Z}\text{, }\lnot S(x)\longrightarrow P(x)\text{,}%
\]
o tamb\'{e}, m\'{e}s formalment,
\[
\left(  \forall x\right)  \left(  \left(  x\in\mathbb{Z}\wedge\lnot
S(x)\right)  \longrightarrow P\left(  x\right)  \right)  \text{.}%
\]


(2) Usant la mateixa notaci\'{o} que l'apartat anterior, es t\'{e}:%
\[
\exists x\in\mathbb{Z}\text{, }\lnot P\left(  x\right)  \text{,}%
\]
o b\'{e},%
\[
\left(  \exists x\right)  \left(  x\in\mathbb{Z\wedge}\lnot P\left(  x\right)
\right)  \text{.}%
\]


(3) L'enunciat simb\`{o}licament s'escriu aix\'{\i}:%
\[
\forall x\in\mathbb{R}\text{, }\exists y\in\mathbb{R}\text{, }y^{3}=x\text{,}%
\]
o b\'{e}, tamb\'{e} com%
\[
\left(  \forall x\right)  \left(  \exists y\right)  \left(  x\in
\mathbb{R\wedge}\text{ }y\in\mathbb{R\wedge~}y^{3}=x\right)  \text{.}%
\]


(4) L'enunciat s'escriu aix\'{\i} simb\`{o}licament:%
\[
\forall a,b\in\mathbb{Q}\text{, }ab\in\mathbb{Q}\text{,}%
\]
o b\'{e}, com%
\[
\left(  \forall a,b\right)  \left(  a,b\in\mathbb{Q}\longrightarrow
ab\in\mathbb{Q}\right)  \text{.}%
\]

\end{solucio}

\begin{exercici}
Expresseu formalment els enunciats seg\"{u}ents:

\begin{enumerate}
\item Alg\'{u} no va fer els deures.

\item Tothom al dormitori t\'{e} un company de pis que no li agrada.

\item Si alg\'{u} al dormitori t\'{e} un amic que t\'{e} el xarampi\'{o},
llavors tothom el dormitori s'haur\`{a} de posar en quarantena.

\item Tot en aquesta botiga t\'{e} un preu excessiu o est\`{a} mal fet.
\end{enumerate}
\end{exercici}

\begin{solucio}
(1) Si $P(x)=$\textquotedblleft$x$ fa els deures\textquotedblright, aleshores
l'enunciat es formalitza d'aquesta manera: $\exists x\lnot P(x)$.

(2) Necessitem tres predicats per formalitzar l'enunciat: $P(x)=$%
\textquotedblleft$x$ viu en el pis\textquotedblright, $Q(x,y)=$%
\textquotedblleft$x$ i $y$ s\'{o}n company de pis\textquotedblright\ i
$R(x,y)=$\textquotedblleft$x$ li agrada $y$\textquotedblright. Aleshores es
t\'{e}:%
\[
\forall x(P(x)\longrightarrow\exists y\left(  Q\left(  x,y\right)  \wedge\lnot
R\left(  x,y\right)  \right)
\]


(3) Introdu\"{\i}m quatre predicats per formalitzar l'enunciat: $P(x)=$%
\textquotedblleft$x$ viu en el pis\textquotedblright, $Q(x,y)=$%
\textquotedblleft$x$ i $y$ s\'{o}n amics\textquotedblright, $R(x)=$%
\textquotedblleft$x$ t\'{e} xarampi\'{o}\textquotedblright\ i $S(x)=$%
\textquotedblleft$x$ est\`{a} en quarantena\textquotedblright. Aleshores
escribim%
\[
\forall x\left(  P(x)\longrightarrow\exists y\left(  \left(  R(y)\wedge
Q\left(  x,y\right)  \right)  \longrightarrow S(x)\right)  \right)
\]


(4) Si considerem $P(x)=$\textquotedblleft$x$ est\`{a} a la
botiga\textquotedblright, $Q(x)=$\textquotedblleft$x$ t\'{e} el preu
excessiu\textquotedblright\ i $R(x)=$\textquotedblleft$x$ est\`{a} mal
fet\textquotedblright, llavors l'enunciat formalitzat s'escriu aix\'{\i}:%
\[
\forall x\left(  P(x)\longrightarrow\left(  Q(x)\vee R(x)\right)  \right)
\]

\end{solucio}

\begin{exercici}
Tradueix les expressions seg\"{u}ents en paraules:

\begin{enumerate}
\item $\forall x\left(  \left(  H(x)\wedge\lnot\exists yC(x,y)\right)
\longrightarrow\lnot F(x)\right)  $, si $H(x)=$\textquotedblleft x \'{e}s un
home\textquotedblright, $C(x,y)=$\textquotedblleft$x$ est\`{a} casat amb $y
$\textquotedblright\ i $F(x)=$\textquotedblleft$x$ \'{e}s
feli\c{c}\textquotedblright.

\item $\forall x\left(  W(x)\wedge Z(x)\right)  $, si $Z(x)=$\textquotedblleft%
$x$ li agrada les croquetes\textquotedblright\ i $W(x)=$\textquotedblleft$x$
t\'{e} una granota de mascota\textquotedblright.

\item $\exists x\left(  N(x)\wedge\forall y\left(  N(y)\longrightarrow x\leq
y\right)  \right)  $, si $N(x)=$\textquotedblleft$x$ es un nombre
natural\textquotedblright.
\end{enumerate}
\end{exercici}

\begin{solucio}
(1) \textquotedblleft Tot home que no est\`{a} casat no \'{e}s
feli\c{c}\textquotedblright, o dit en unes altres paraules, \textquotedblleft
Tots els homes solters s\'{o}n infeli\c{c}os\textquotedblright.

(2) \textquotedblleft A tots els que tenen una granota com a mascota els
agraden les croquetes\textquotedblright.

(3) \textquotedblleft Existeix un nombre natural m\'{e}s petit o igual que
qualsevol altre nombre natural\textquotedblright.
\end{solucio}

\begin{exercici}
Qu\`{e} signifiquen les afirmacions seg\"{u}ents? S\'{o}n vertaderes o falses?
L'univers del discurs en cada cas \'{e}s $\mathbb{N}$, el conjunt de tots els
nombres naturals.

\begin{enumerate}
\item $\forall x\exists y(x<y)$.

\item $\exists y\forall x(x<y)$.

\item $\exists x\exists y(x<y)$.

\item $\exists x\forall y(x<y)$.

\item $\forall y\forall x(x<y)$.
\end{enumerate}
\end{exercici}

\begin{solucio}
(1) Aix\`{o} vol dir que per a cada nombre natural $x$, l'afirmaci\'{o}
$\exists y(x<y)$ \'{e}s certa. En altres paraules, per a cada nombre natural
$x$, hi ha un nombre natural m\'{e}s gran que $x$. Aix\`{o} \'{e}s cert. Per
exemple, $x+1$ sempre \'{e}s m\'{e}s gran que $x$.

(2) Aix\`{o} vol dir que hi ha algun nombre natural $y$ tal que l'enunciat
$\forall x(x<y)$ \'{e}s cert. En altres paraules, hi ha algun nombre natural
$y$ tal que tots els nombres naturals s\'{o}n m\'{e}s petits que $y$. Es clar
que aix\`{o} \'{e}s fals. No importa el nombre natural $y$ que triem, sempre
hi haur\`{a} nombres naturals m\'{e}s grans.

(3) Aix\`{o} vol dir que hi ha un nombre natural $x$ tal que $\exists y(x<y)$
\'{e}s cert. Per\`{o} com hem vist al primer apartat, aix\`{o} \'{e}s cert per
a tots els nombres naturals $x$, de manera que en particular \'{e}s cert per
almenys un.

(4)\ Aix\`{o} vol dir que hi ha un nombre natural $x$ tal que l'enunciat
$\forall y(x<y)$ \'{e}s cert. Podr\'{\i}eu tenir la temptaci\'{o} de dir que
aquesta afirmaci\'{o} ser\`{a} certa si $x=0$, per\`{o} aix\`{o} no \'{e}s
correcte. Com que $0$ \'{e}s el nombre natural m\'{e}s petit, l'enunciat $0<y
$ \'{e}s cert per a tots els valors de $y$ excepte $y=0$, per\`{o} si $y=0$,
llavors $0<y$ \'{e}s fals i, per tant, $\forall y(0<y)$ \'{e}s fals. Un
raonament similar mostra que per a cada valor de $x$ l'enunciat $\forall
y(x<y)$ \'{e}s fals, per tant $\exists x\forall y(x<y)$ \'{e}s fals.

(5) Aix\`{o} vol dir que per a cada nombre natural $x$, l'enunciat $\forall
y(x<y)$ \'{e}s vertader. Per\`{o} com hem vist a l'apartat anterior, ni tan
sols hi ha un valor de $x$ per al qual aquesta afirmaci\'{o} \'{e}s certa.
Aix\'{\i} doncs $\forall x\forall y(x<y)$ \'{e}s fals.
\end{solucio}

\begin{exercici}
Escriu la negaci\'{o} de l'enunciat seg\"{u}ent: \textquotedblleft per a cada
nombre real $\varepsilon>0$, hi ha un enter positiu $k$ tal que per a tots els
enters positius $n$, es t\'{e} $\left\vert a_{n}-k^{2}\right\vert
<\varepsilon$\textquotedblright.
\end{exercici}

\begin{solucio}
Considerem com univers del discurs el conjunt de tots els nombres reals i
simbolitzem per $\mathbb{Z}^{+}$ el conjunt del enters positius. D'aquesta
manera l'enunciat s'escriu formalment aix\'{\i}:
\[
\forall\varepsilon>0,\exists k\in\mathbb{Z}^{+},\forall n\in\mathbb{Z}%
^{+},\left\vert a_{n}-k^{2}\right\vert <\varepsilon\text{.}%
\]


Aleshores la seva negaci\'{o} \'{e}s:%
\[
\exists\varepsilon>0,\forall k\in\mathbb{Z}^{+},\exists n\in\mathbb{Z}%
^{+},\left\vert a_{n}-k^{2}\right\vert \geq\varepsilon\text{,}%
\]
que s'expressa en paraules com existeix un nombre real $\varepsilon>0$ tal que
per a tot nombre enter positiu $k$ existeix un nombre enter positiu $n$ que fa
que es compleixi $\left\vert a_{n}-k^{2}\right\vert \geq\varepsilon$.
\end{solucio}

\begin{exercici}
Assumint com a domini de les variables $x$ i $y$ el conjunt $U$, escriviu una
deducci\'{o} per el seg\"{u}ent raonament:%
\[%
\begin{tabular}
[c]{ll}%
$P_{1}:$ & $\forall x\left(  \left(  A(x)\longrightarrow R(x)\right)  \vee
T(x)\right)  $\\
$P_{2}:$ & $\exists x\left(  T(x)\longrightarrow P(x)\right)  $\\
$P_{3}:$ & $\forall x\left(  A(x)\wedge\lnot P(x)\right)  $\\\hline
$C:$ & $\exists x\ R(x)$%
\end{tabular}
\]

\end{exercici}

\begin{solucio}
Fem ara la deducci\'{o} per provar la seva validesa:%
\[%
\begin{tabular}
[c]{lll}%
1. & $\forall x\left(  \left(  A(x)\longrightarrow R(x)\right)  \vee
T(x)\right)  $ & $P_{1}$\\
2. & $\exists x\left(  T(x)\longrightarrow P(x)\right)  $ & $P_{2}$\\
3. & $\forall x\left(  A(x)\wedge\lnot P(x)\right)  $ & $P_{3}$\\
4. & $T(a)\longrightarrow P(a)$ & EQE 2\\
5. & $\left(  A(a)\longrightarrow R(a)\right)  \vee T(a)$ & EQU 1\\
6. & $A(a)\wedge\lnot P(a)$ & EQU 3\\
7. & $\lnot P(a)$ & EC 6\\
8. & $\lnot T(a)$ & MT(4,7)\\
9. & $A(a)\longrightarrow R(a)$ & ED(5,8)\\
10. & $A(a)$ & EC 6\\
11. & $R(a)$ & MP(9,10)\\\hline
14. & $\exists x~R(x)$ & IQE 11
\end{tabular}
\]

\end{solucio}

\begin{exercici}
Escriviu una deducci\'{o} del argument seg\"{u}ent: \textquotedblleft Tota
panerola intel\textperiodcentered ligent menga escombraries. Hi ha una
panerola que li agrada la brut\'{\i}cia i no li agrada la pols. Per a cada
panerola, no \'{e}s el cas que no li agradi la brut\'{\i}cia o mengi
escombraries. Per tant, hi ha una panerola tal que no \'{e}s el cas que si no
\'{e}s intel\textperiodcentered ligent, li agradi la pols.\textquotedblright.
\end{exercici}

\begin{solucio}
Formalitzem els enunciats del raonament. Per facilitar l'escriptura,
considerem que la variable $x$ t\'{e} per domini el conjunt de totes les
paneroles. Denotem per $I$, $E$, $B$ i $P$ els predicats \textquotedblleft%
\'{e}s intel\textperiodcentered ligent\textquotedblright, \textquotedblleft
mengen porqueria\textquotedblright, \textquotedblleft agrada la
brut\'{\i}cia\textquotedblright, i \textquotedblleft agrada la
pols\textquotedblright, respectivament. Aleshores, el raonament podem
simbolitzar-lo d'aquesta manera:%
\[%
\begin{tabular}
[c]{ll}%
$P_{1}:$ & $\forall x~\left(  I(x)\longrightarrow E(x)\right)  $\\
$P_{2}:$ & $\exists x~\left(  B(x)\wedge\lnot P(x)\right)  $\\
$P_{3}:$ & $\forall x~\lnot\left(  \lnot B(x)\vee E(x)\right)  $\\\hline
$C:$ & $\exists x~\lnot\left(  \lnot I(x)\longrightarrow P(x)\right)  $%
\end{tabular}
\]
Fem ara la deducci\'{o} per provar la seva validesa:%
\[%
\begin{tabular}
[c]{lll}%
1. & $\forall x~\left(  I(x)\longrightarrow E(x)\right)  $ & $P_{1}$\\
2. & $\exists x~\left(  B(x)\wedge\lnot P(x)\right)  $ & $P_{2}$\\
3. & $\forall x~\lnot\left(  \lnot B(x)\vee E(x)\right)  $ & $P_{3}$\\
4. & $B(a)\wedge\lnot P(a)$ & EQE 2\\
5. & $\lnot\left(  \lnot B(a)\vee E(a)\right)  $ & EQU 3\\
6. & $B(a)\wedge\lnot E(a)$ & DM 5\\
7. & $I(a)\longrightarrow E(a)$ & EQU 1\\
8. & $\lnot E(a)$ & EC 6\\
9. & $\lnot I(a)$ & MT(7,8)\\
10. & $\lnot P(a)$ & EC 4\\
11. & $\lnot I(a)\wedge\lnot P(a)$ & IC(9,10)\\
12. & $\lnot\left(  I(a)\vee P(a)\right)  $ & DM 11\\
13. & $\lnot\left(  \lnot I(a)\longrightarrow P(a)\right)  $ & EQ\\\hline
14. & $\exists x~\lnot\left(  \lnot I(x)\longrightarrow P(x)\right)  $ & IQE
13
\end{tabular}
\]
on EQ significa que hem aplicat l'equivalencia l\`{o}gica: $A\longrightarrow
B\Longleftrightarrow\lnot A\vee B$.
\end{solucio}

\section{Demostraci\'{o}}

\begin{exercici}
Donat un enter positiu $n$, proveu directament que $n^{3}-n$ \'{e}s sempre
m\'{u}ltiple de 3.
\end{exercici}

\begin{solucio}
Observa primer que $n(n^{2}-1)=n(n+1)(n-1)$. Aquest tres factors s\'{o}n tres
nombres naturals consecutius i, per tant, un d'ells ser\`{a} un m\'{u}ltiple
de 3. Com a conseq\"{u}\`{e}ncia $n^{3}-n$ \'{e}s sempre m\'{u}ltiple de 3.
\end{solucio}

\begin{exercici}
Demostreu directament que per a cada terna de nombres reals positius $a,b$ i
$c$ es compleix que%
\[
\frac{a}{b+c}+\frac{b}{a+c}+\frac{c}{a+b}\geq1\text{.}%
\]

\end{exercici}

\begin{solucio}
Considerem $a>0,b>0$ i $c>0$. Aleshores podem escriure les tres expressions
seg\"{u}ents:%
\begin{align*}
M  &  =\frac{a}{b+c}+\frac{b}{a+c}+\frac{c}{a+b}\\
N  &  =\frac{a}{a+c}+\frac{c}{b+c}+\frac{b}{a+b}\\
P  &  =\frac{c}{a+c}+\frac{b}{b+c}+\frac{a}{a+b}%
\end{align*}
Es clar que $N+P=3$ i $3M\geq M+N+P$. Aleshores es t\'{e} $2M\geq N+P$ i, per
tant, $M\geq\frac{3}{2}>1$.

Una altra forma de provar aquesta desigualtat \'{e}s aplicant el fet que la
mitjana aritm\`{e}tica \'{e}s m\'{e}s gran que la hipergeom\`{e}trica. En
efecte, podem escriure ara d'aquesta manera:
\begin{align*}
M  &  =\frac{a}{b+c}+\frac{a}{a+c}+\frac{a}{a+b}\\
N  &  =\frac{b}{b+c}+\frac{b}{a+c}+\frac{b}{a+b}\\
P  &  =\frac{c}{b+c}+\frac{c}{a+c}+\frac{c}{a+b}%
\end{align*}
Sumant tenim:%
\begin{equation}
M+N+P=\frac{a}{b+c}+\frac{b}{a+c}+\frac{c}{a+b}+3 \label{1}%
\end{equation}
Ara b\'{e},%
\[
M+N+P=\frac{a+b+c}{b+c}+\frac{a+b+c}{a+c}+\frac{a+b+c}{a+b}%
\]
Aplicant el que hem dit abans:\qquad%
\[
\frac{1}{2}\cdot\frac{(b+c)+(a+c)+(a+b)}{3}\geq\frac{3}{\dfrac{1}{b+c}%
+\dfrac{1}{a+c}+\dfrac{1}{a+b}}\text{.}%
\]
Per tant, es t\'{e}%
\[
\frac{a+b+c}{b+c}+\frac{a+b+c}{a+c}+\frac{a+b+c}{a+b}=
\]%
\[
\frac{3}{2}\frac{(b+c)+(a+c)+(a+b)}{3}\left(  \dfrac{1}{b+c}+\dfrac{1}%
{a+c}+\dfrac{1}{a+b}\right)  \geq
\]%
\[
\frac{3}{2}\frac{3}{\dfrac{1}{b+c}+\dfrac{1}{a+c}+\dfrac{1}{a+b}}\dfrac
{1}{b+c}+\dfrac{1}{a+c}+\dfrac{1}{a+b}\geq\frac{9}{2}\text{.}%
\]
D'aqu\'{\i} i (\ref{1}), surt%
\[
\frac{a}{b+c}+\frac{b}{a+c}+\frac{c}{a+b}+3\geq\frac{9}{2}\text{,}%
\]
i, per tant,%
\[
\frac{a}{b+c}+\frac{b}{a+c}+\frac{c}{a+b}\geq\frac{3}{2}\text{.}%
\]

\end{solucio}

\begin{exercici}
Proveu cap a enrere que per a qualssevol nombres reals negatius, $a<b$ implica
$a^{2}>b^{2}$.
\end{exercici}

\begin{solucio}
La prova cap enrere surt de la tesi, i, per tant, podem fer es seg\"{u}ent:%
\[%
\begin{tabular}
[c]{ll}%
$a^{2}>b^{2}$ & Tesi\\
$\left\vert a\right\vert >\left\vert b\right\vert $ & Prenent arrels
quadrades\\
$-a>-b$ & $a,b$ s\'{o}n negatius\\
$a<b$ & Multiplicant per $-1$ i surt l'hip\`{o}tesi
\end{tabular}
\]
De fet, aix\`{o} que hem escrit s'havia d'haver pensat mentalment perqu\`{e}
la demostraci\'{o} real \'{e}s una prova directa: Si $a<b$, multiplicant per
$-1$, es t\'{e} $-a>-b$ i $-a,-b$ s\'{o}n positius. Per tant, $\left(
-a\right)  ^{2}>(-b)^{2}$ o sigui, $a^{2}>b^{2}$.
\end{solucio}

\begin{exercici}
Proveu per contrarec\'{\i}proc que per a qualssevol $n,m\in\mathbb{Z}$, si
$mn$ \'{e}s senar, llavors $m$ i $n$ s\'{o}n senars.
\end{exercici}

\begin{solucio}
Si fessim la prova directa seria prendre com hip\`{o}tesi que $mn$ \'{e}s
senar i hem d'arribar a veure que $m$ i $n$ tamb\'{e} ho s\'{o}n. En canvi,
per contrarec\'{\i}proc ser\`{a} prendre com hip\`{o}tesi que no \'{e}s el cas
que $m$ i $n$ siguin senars i hem d'arribar a provar que $mn$ no \'{e}s senar.

Si no \'{e}s el cas que $m$ i $n$ siguin senars, vol dir que $m$ \'{e}s parell
o $n$ \'{e}s parell. Suposem per exemple que $m$ \'{e}s parell. Aleshores
existeix $k\in\mathbb{Z}$ tal que $m=2k$. Per tant, $mn=2kn$ i $kn\in
\mathbb{Z}$. Aleshores $mn$ \'{e}s parell com voliem demostrar.
\end{solucio}

\begin{exercici}
Proveu per reducci\'{o} a l'abssurd que els \'{u}nics enters consecutius no
negatius $a,b$ i $c$ que compleixen $a^{2}+b^{2}=c^{2}$ s\'{o}n $3,4$ i $5$.
\end{exercici}

\begin{solucio}
Suposem que $3,4$ i~$5$ no s\'{o}n els \'{u}nics enters consecutius no
negatius que compleixen $a^{2}+b^{2}=c^{2}$. Aix\`{o} \'{e}s equivalent a
suposar que existeixen enters no negatius $n,n+1$ i $n+2$ i $n\neq3$ tals que%
\[
n^{2}+(n+1)^{2}=(n+2)^{2}%
\]
Ara b\'{e}, fent operacions, s'obt\'{e} l'equaci\'{o} $n^{2}-2n-3=0$, qu\`{e}
t\'{e} com a solucions $3$ i $-1$. Per tant, arribem a un absurd perqu\`{e}
hem suposat que $n\neq3$. Com a conseq\"{u}\`{e}ncia, hem provat el que vol\'{\i}em.
\end{solucio}

\begin{exercici}
Siguin $a,b$ i $c$ enters. Suposem que hi ha un nombre enter $d$ que $d\mid a
$ i $d\mid b$, per\`{o} que $d$ no divideix $c$. Demostreu per reducci\'{o} a
l'absurd que l'equaci\'{o} $ax+by=c$ no t\'{e} cap soluci\'{o} que $x$ i $y $
siguin enters.
\end{exercici}

\begin{solucio}
Suposem que $ax+by=c$ t\'{e} una soluci\'{o} tal que $x$ i $y$ s\'{o}n enters.
Com $d$ divideix $a$ i tamb\'{e} $b$, aleshores existeixen enters $m$ i $n$
tals que $a=md$ i $b=nd$. Aleshores, es t\'{e}%
\[
mdx+ndy=c
\]
Dividint per $d$, s'obt\'{e}%
\[
mx+ny=\frac{c}{d}%
\]
Per\`{o} aix\`{o} \'{e}s absurd perqu\`{e} $mx+ny$ \'{e}s enter i, per tant,
tamb\'{e} $\dfrac{c}{d}$ i aix\`{o} no \'{e}s possible perqu\`{e} $d$ no
divideix $c$.
\end{solucio}

\begin{exercici}
Proveu per inducci\'{o}:

\begin{enumerate}
\item Si $n$ \'{e}s un nombre enter no negatiu, llavors $%
%TCIMACRO{\dsum \limits_{k=0}^{n}}%
%BeginExpansion
{\displaystyle\sum\limits_{k=0}^{n}}
%EndExpansion
k\cdot k!=\left(  n+1\right)  !-1$.

\item Si $n\in\mathbb{N}$, llavors $\left(  1+x\right)  ^{n}\geq1+nx$ per a
tot $x\in\mathbb{R}$ i $x>-1$.
\end{enumerate}
\end{exercici}

\begin{solucio}
(1) Constru\"{\i}m una prova d'inducci\'{o} sobre $n$, sent $P(n)=$
\textquotedblleft$%
%TCIMACRO{\dsum \limits_{k=0}^{n}}%
%BeginExpansion
{\displaystyle\sum\limits_{k=0}^{n}}
%EndExpansion
k\cdot k!=\left(  n+1\right)  !-1$\textquotedblright.

\begin{description}
\item[Cas base:] $P(1)$ \'{e}s $%
%TCIMACRO{\dsum \limits_{k=0}^{1}}%
%BeginExpansion
{\displaystyle\sum\limits_{k=0}^{1}}
%EndExpansion
k\cdot k!=\left(  1+1\right)  !-1$ i, aix\`{o} \'{e}s veritat perqu\`{e}
$0\cdot0!+1\cdot1!=1$.

\item[Hip\`{o}tesi d'inducci\'{o}:] Suposem que $P(n)$ \'{e}s certa, o sigui
que $%
%TCIMACRO{\dsum \limits_{k=0}^{n}}%
%BeginExpansion
{\displaystyle\sum\limits_{k=0}^{n}}
%EndExpansion
k\cdot k!=\left(  n+1\right)  !-1$.

\item[Tesi:] Hem de demostrar que $P(n+1)=$\textquotedblleft$%
%TCIMACRO{\dsum \limits_{k=0}^{n+1}}%
%BeginExpansion
{\displaystyle\sum\limits_{k=0}^{n+1}}
%EndExpansion
k\cdot k!=\left(  n+2\right)  !-1$\textquotedblright\ \'{e}s certa.
\end{description}

En efecte, aplicant l'hip\`{o}tesi d'inducci\'{o}, es t\'{e}%
\begin{align*}%
%TCIMACRO{\dsum \limits_{k=0}^{n+1}}%
%BeginExpansion
{\displaystyle\sum\limits_{k=0}^{n+1}}
%EndExpansion
k\cdot k!  &  =%
%TCIMACRO{\dsum \limits_{k=0}^{n}}%
%BeginExpansion
{\displaystyle\sum\limits_{k=0}^{n}}
%EndExpansion
k\cdot k!+(n+1)(n+1)!\\
&  =\left(  n+1\right)  !-1+(n+1)(n+1)!\\
&  =(1+n+1)(n+1)!-1\\
&  =(n+2)!-1
\end{align*}
que \'{e}s el que vol\'{\i}em veure.

\begin{description}
\item[Conclusi\'{o}:] Pel principi d'inducci\'{o} sobre $n$, dedu\"{\i}m que
$P(n)$ \'{e}s certa per a tot $n\in\mathbb{N}$.
\end{description}

(2) Constru\"{\i}m una prova d'inducci\'{o} sobre $n$, sent $P(n)=$
\textquotedblleft Si $n\in\mathbb{N}$, llavors $\left(  1+x\right)  ^{n}%
\geq1+nx$ per a tot $x\in\mathbb{R}$ i $x>-1$\textquotedblright.

\begin{description}
\item[Cas base:] $P(1)$ \'{e}s evident que \'{e}s vertadera.

\item[Hip\`{o}tesi d'inducci\'{o}:] Suposem que $P(n)$ \'{e}s certa, o sigui
que si $n\in\mathbb{N}$, llavors $\left(  1+x\right)  ^{n}\geq1+nx$ per a tot
$x\in\mathbb{R}$ i $x>-1$.

\item[Tesi:] Hem de demostrar que $P(n+1)=$\textquotedblleft Si $n\in
\mathbb{N}$, llavors $\left(  1+x\right)  ^{n+1}\geq1+(n+1)x$ per a tot
$x\in\mathbb{R}$ i $x>-1$\textquotedblright\ \'{e}s certa.
\end{description}

En efecte, aplicant l'hip\`{o}tesi d'inducci\'{o}, es t\'{e}%
\begin{align*}
\left(  1+x\right)  ^{n+1}  &  =(1+x)^{n}(1+x)\\
&  =\left(  1+nx\right)  (1+x)\\
&  =1+x+nx+nx^{2}\\
&  =1+(n+1)x+nx^{2}\\
&  \geq1+(n+1)x
\end{align*}
que \'{e}s el que vol\'{\i}em veure.

\begin{description}
\item[Conclusi\'{o}:] Pel principi d'inducci\'{o} sobre $n$, dedu\"{\i}m que
$P(n)$ \'{e}s certa per a tot $n\in\mathbb{N}$.
\end{description}
\end{solucio}

\begin{exercici}
Proveu que el residu del quadrat de qualsevol nombre enter quan es divideix
per 4 \'{e}s $0$ o $1$.
\end{exercici}

\begin{solucio}
Tot nombre enter $n$ \'{e}s o parell o b\'{e} senar. Aquesta idea proporciona
l'estrat\`{e}gia de la demostraci\'{o}: constru\"{\i}r una prova per casos:

(1) Si $n$ \'{e}s parell, aleshores $n=2k$, on $k\in\mathbb{Z}$. Llavors,
$n^{2}=4k^{2}$ qu\`{e} \'{e}s m\'{u}ltiple de $4$ i, per tant, quan es
divideix per $4$ el residu val $0$.

(2) si $n$ \'{e}s senar, aleshores $n=2k+1$, on $k\in\mathbb{Z}$. Llavors,
$n^{2}=4k^{2}+4k+1=4\left(  k^{2}+k\right)  +1$ qu\`{e} quan es divideix per
$4$ el residu val $1$.
\end{solucio}

\begin{exercici}
Demostreu que per a cada nombre real $x$, si $\left\vert x-3\right\vert >3$
llavors $x^{2}>6x$.
\end{exercici}

\begin{solucio}
Per definici\'{o} de valor absolut, $\left\vert x-3\right\vert =x-3$ si
$x-3\geq0$, i $\left\vert x-3\right\vert =-\left(  x-3\right)  $ si $x-3<0$.
Constru\"{\i}m una prova per casos:

(1) Si $x-3\geq0$, aleshores $\left\vert x-3\right\vert =x-3>3$ i, per tant,
$x>6$. D'aqu\'{\i}, s'obt\'{e} $x^{2}>6x$.

(2) Si $x-3<0$, aleshores $\left\vert x-3\right\vert =-x+3>3$ i, per tant,
$x<0$. D'aqu\'{\i}, s'obt\'{e} $x^{2}>6x$.
\end{solucio}

\begin{exercici}
\'{E}s cert que per a cada enter positiu $n$ es compleix que $n^{2}-n+17$
\'{e}s un nombre primer?
\end{exercici}

\begin{solucio}
Si pensem que l'enunciat \'{e}s fals hem de trobar un contraexemple per
provar-ho. En efecte, si prenem $n=17$, aleshores $n^{2}-n+17=289$ que no
\'{e}s primer.
\end{solucio}

\begin{exercici}
(Algoritme de la divisi\'{o}) Si $a,b\in\mathbb{N}$, proveu que existeixen
enters $q,r$ i s\'{o}n \'{u}nics per els quals $a=bq+r$, sent $0\leq r<b$.
\end{exercici}

\begin{solucio}
Sabem que $a,b\in\mathbb{N}$. Primer hem de provar l'exist\`{e}ncia, buscant
aquests nombres enters que compleixen la propietat. Considerem el conjunt de
m\'{u}ltiples positius $b$: $B=\left\{  kb:k\in\mathbb{N}\right\}  $. Com el
conjunt dels nombres naturals $\mathbb{N}$ est\`{a} ben ordenat per la
relaci\'{o} $\leq$, tot subconjunt t\'{e} m\'{\i}nim. Per tant, existeix un
natural $n$ de $B$ tal que $nb\leq a<(n+1)b$. Aix\`{o} \'{e}s equivalent a
$0\leq a-nb<b$, i si ara prenem $r=a-nb$ i $q=n$ es compleix la propietat.

Finalment, hem de demostrar la unicitat. Suposem que existeixen $r^{\prime}$ i
$q^{\prime}$ que compleixen $a=bq^{\prime}+r^{\prime}$, sent $0\leq r^{\prime
}<b$. Aleshores, es t\'{e}%
\begin{align*}
bq+r  &  =bq^{\prime}+r^{\prime}\\
r-r^{\prime}  &  =(q^{\prime}-q)b
\end{align*}
i aix\`{o} vol dir que $r-r^{\prime}$ \'{e}s m\'{u}ltiple de $b$. Llavors, com
$0\leq r<b$ i $0\leq r^{\prime}<b$, s'obt\'{e} $0\leq\left\vert r-r^{\prime
}\right\vert <b$, i com a conseq\"{u}\`{e}ncia, $\left\vert r-r^{\prime
}\right\vert =0$ o sigui $r^{\prime}=r$. Aleshores, tamb\'{e} $q^{\prime}=q$.
\end{solucio}

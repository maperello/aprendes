
\section{Demostració}

\begin{enumerate}
\item Considereu l'enunciat \textquotedblleft per a tots els nombres enters
$a$ i $b$, si $a+b$ \'{e}s parell, aleshores $a$ i $b$ s\'{o}n
parells\textquotedblright. Llavors:

\begin{enumerate}
\item L'enunciat rec\'{\i}proc \'{e}s \textquotedblleft per a tots els nombres
enters $a$ i $b$, si $a$ o $b$ no \'{e}s parell, aleshores $a+b$ no \'{e}s
parell\textquotedblright.

\item L'enunciat contrarec\'{\i}proc \'{e}s \textquotedblleft per a tots els
nombres enters $a$ i $b$, si $a$ i $b$ s\'{o}n parells, aleshores $a+b$ \'{e}s
parell\textquotedblright.

\item El contrarec\'{\i}proc \'{e}s fals i un contraexemple \'{e}s, per
exemple, $a=4$ i $b=7$.

\item L'enunciat contrari \'{e}s \textquotedblleft Hi ha nombres $a$ i $b$
tals que $a+b$ \'{e}s parell per\`{o} $a$ i $b$ no s\'{o}n
parells\textquotedblright\ i \'{e}s vertader. (*)
\end{enumerate}

\item Suposem que $a$ i $b$ s\'{o}n nombres reals. Volem provar que si $0<a<b
$ aleshores $a^{2}<b^{2}$. Llavors, quina de les seg\"{u}ents afirmacions
\'{e}s falsa:

\begin{enumerate}
\item La prova directa consisteix en prendre com hip\`{o}tesi $0<a<b$ i com a
tesi $a^{2}<b^{2}$.

\item La prova cap enrere consisteix a prendre com a punt de partida la
conclusi\'{o} $a^{2}<b^{2}$ i arribar a a deduir que $a<b$ suposant que $a$ i
$b$ son positius, i, despr\'{e}s, refer la prova procedint de forma directe.

\item La prova per contrarec\'{\i}proc consisteix en prendre com hip\`{o}tesi
$a^{2}\geq b^{2}$ i com a tesi $a\geq b>0$.

\item La prova per reducci\'{o} a l'absurd consisteix en prendre com hipotesi
$a\geq b>0$ i $a^{2}<b^{2}$ i com a tesi una contradicci\'{o}. (*)
\end{enumerate}

\item Sigui $x=\dfrac{y}{y^{2}+1}$. Llavors
\[
y-x=y-\dfrac{y}{y^{2}+1}=\dfrac{y^{3}}{y^{2}+1}=\dfrac{y}{y^{2}+1}\cdot
y^{2}=xy^{2}\text{.}%
\]
i, per tant, l'enunciat: $\exists x\in\mathbb{R~}\forall y\in\mathbb{R~}%
xy^{2}=y-x$ \'{e}s un teorema.

\begin{enumerate}
\item La prova del teorema \'{e}s correcte.

\item La prova correspon al teorema: $\forall y\in\mathbb{R}$ $\exists
x\in\mathbb{R~}x^{2}y=y-x$

\item La prova del teorema \'{e}s incorrecte perqu\`{e} no es pot definir $x$
abans de $y$. (*)

\item Cap de les respostes \'{e}s vertadera.
\end{enumerate}

\item L'enunciat: \textquotedblleft Suposem que $m$ \'{e}s un nombre enter
parell i $n$ \'{e}s un nombre enter senar. Aleshores $n^{2}-m^{2}%
=n+m$\textquotedblright\ \'{e}s un teorema?

\begin{enumerate}
\item \textquotedblleft$n^{2}-m^{2}=n+m$\textquotedblright\ \'{e}s la
condici\'{o} necess\`{a}ria.

\item \textquotedblleft$m$ \'{e}s un nombre enter parell i $n$ \'{e}s un
nombre enter senar\textquotedblright\ \'{e}s condici\'{o} necess\`{a}ria i suficient.

\item \'{E}s teorema perqu\`{e} $m=2k$ i $n=2k+1$, on $k\in\mathbb{Z}$, i
$n^{2}-m^{2}=\left(  2k+1\right)  +2k=m+n$.

\item No \'{e}s teorema i, $m=2$ i $n=3$ \'{e}s un contraexemple. (*)
\end{enumerate}

\item Quina de les seg\"{u}ents respostes \'{e}s falsa?

\begin{enumerate}
\item Per demostrar que $\forall x\in\mathbb{R}\left(  \exists y\in
\mathbb{R}(x+y=xy)\leftrightarrow x\neq1\right)  $ fem la prova seg\"{u}ent:
Considerem un nombre real arbitrari $x$, aleshores existeix un nombre real
$y=\dfrac{x}{x-1}$ si i nom\'{e}s si $x\neq1$ i tamb\'{e} es compleix%
\[
x+y=x+\dfrac{x}{x-1}=\frac{x^{2}}{x-1}=x\cdot\frac{x}{x-1}=xy\text{.}%
\]


\item Per demostrar que $\forall x\in\mathbb{R}\left(  \exists y\in
\mathbb{R}(x+y=xy)\leftrightarrow y\neq1\right)  $ fem la prova seg\"{u}ent:
Considerem $x=\dfrac{y}{y-1}$ que existeix si i nom\'{e}s si $y\neq1$.
Aleshores es compleix%
\[
x+y=\dfrac{x}{x-1}+y=\frac{xy}{x-1}=\frac{x}{x-1}\cdot y=xy\text{. (*)}%
\]


\item Per provar que $\exists z\in\mathbb{R~}\forall x\in\mathbb{R}^{+}\left(
\exists y\in\mathbb{R~}y-x=y/x\longleftrightarrow x\neq z\right)  $ fem:
Suposem que existeix un nombre real $a$ tal que
\[
y-a=\frac{y}{a}\Longrightarrow y=\dfrac{a}{\dfrac{a-1}{a}}%
\]
aleshores $y$ existir\`{a} si i nom\'{e}s si $a\neq0$ i $a\neq1$. D'aqu\'{\i}
s'obt\'{e} que per a qualsevol $x\in\mathbb{R}^{+}$ existeix $y\in\mathbb{R}$
si i nom\'{e}s si existeix $z=1$ i $x\neq1$.

\item Per demostrar que per a cada nombre enter $n$, $6~|~n$ sii $2~|~n$ i
$3~|~n$ fem la prova seg\"{u}ent: Sigui $n$ qualsevol nombre enter. La
condici\'{o} $2~|~n$ i $3~|~n$ \'{e}s necess\`{a}ria perqu\`{e} $n=2p$ i
$n=3q$ i $p,q\in\mathbb{Z}$. Llavors $n=3\left(  2p\right)  -2\left(
3q\right)  =6\left(  p-q\right)  $ i $6~|~n$. La condici\'{o} $6~|~n$ \'{e}s
suficient perqu\`{e} $n=6k=3(2k)$, $k\in\mathbb{Z}$, i, per tant, $3~|~n$, i
$n=2\left(  3k\right)  $ i aix\'{\i} $2~|~n$.
\end{enumerate}

\item Quina de les seg\"{u}ents respostes \'{e}s correcte?

\begin{enumerate}
\item Sabem que hi han nombres primers; per exemple, $2$ \'{e}s primer.
Suposem ara que nom\'{e}s hi ha un nombre finit de nombres primers
$p_{1},p_{2},...,p_{n}$, $n\in\mathbb{Z}^{+}$. Llavors $q=p_{1}p_{2}\cdots
p_{n}+1$ \'{e}s primer. En efecte, si suposem que $q$ no \'{e}s primer
aleshores la seva descomposici\'{o} factorial en primers segur que existeix
$k$ tal que $p_{k}$ divideix $q$, $1\leq k\leq n$, o sigui%
\[
q=p_{k}r
\]
D'aqu\'{\i}%
\begin{align*}
p_{1}p_{2}\cdots p_{n}+1 &  =p_{k}r\\
p_{k}\left(  r-p_{1}\cdots p_{k-1}p_{k+1}\cdots p_{n}\right)   &  =1
\end{align*}
i, per tant, $p_{k}$ divideix $1$, per\`{o} aix\`{o} \'{e}s una
contradicci\'{o} perqu\`{e} $p_{k}$ \'{e}s primer. Per conseg\"{u}ent, $q$
\'{e}s primer i aix\`{o} torna a ser una contradicci\'{o} perqu\`{e} nom\'{e}s
hi havia $n$ primers. Per tant, hem demostrat per reducci\'{o} a l'absurd que
hi ha infinit nombres primers. (*)

\item Sigui $n$ qualsevol nombre enter. Si $n$ \'{e}s senar, aleshores
$n=2k+1$, $k\in\mathbb{Z}$. Llavors,%
\[
n^{2}=\left(  2k+1\right)  ^{2}=2\left(  2k^{2}+2k\right)  +1
\]
i, per tant, $n^{2}$ \'{e}s senar perqu\`{e} $2k^{2}+2k\in\mathbb{Z}$.
D'aquesta manera hem demostrat per contrarec\'{\i}proc que per qualsevol enter
$n$, si $n$ \'{e}s parell aleshores $n^{2}$ tamb\'{e} ho \'{e}s.

\item Considerem dos nombres reals qualssevol $x$ i $y$. Suposem que $xy$
\'{e}s irracional i $x$ \'{e}s racional. Llavors $x=p/q$, on $p,q\in
\mathbb{Z}$ i $q\neq0$. Suposem ara tamb\'{e} que $y$ \'{e}s racional, llavors
$y=r/s$, on $r,s\in\mathbb{Z}$ i $s\neq0$. Per tant, es t\'{e} $xy=pr/qs$ i
$qs\neq0$. Aleshores $xy$ \'{e}s racional i aix\`{o} \'{e}s una
contradicci\'{o}. Per conseg\"{u}ent, $y$ \'{e}s irracional. D'aquesta manera
hem demostrat per contrarec\'{\i}proc que si $x$ o $y$ \'{e}s racional,
aleshores $xy$ \'{e}s racional.

\item Sigui $n$ un nombre enter senar qualsevol. Aleshores $n=2s+1$,
$s\in\mathbb{Z}$. Suposem que $s$ \'{e}s parell, aleshores $s=2p$,
$p\in\mathbb{Z}$. Llavors $n^{2}=\left(  2s+1\right)  ^{2}=\left(
4p+1\right)  ^{2}=8\left(  2p^{2}+p\right)  +1$ i, per tant, $n^{2}=8k+1$,
$k=2p^{2}+p\in\mathbb{Z}$. D'aquesta manera hem demostrar per casos que si $n$
\'{e}s un nombre enter senar qualsevol, aleshores existeix un enter $k$ tal
que $n^{2}=2k+1$.
\end{enumerate}

\item Quina de les seg\"{u}ents respostes \'{e}s falsa?

\begin{enumerate}
\item Considerem $x$ un nombre real arbitrari, i suposem que $x\neq2$. Ara
considerem $y=\dfrac{x}{2-x}$, que existeix ja que $x\neq2$. Aleshores es
t\'{e}%
\[
\frac{2y}{y+1}=\frac{\dfrac{2x}{2-x}}{\dfrac{x}{2-x}+1}=\frac{\dfrac{2x}{2-x}%
}{\dfrac{2}{2-x}}=\frac{2x}{x}=x\text{.}%
\]
Per veure que aquesta soluci\'{o} \'{e}s \'{u}nica, suposem que existeix $z$
tal que $2z/(z+1)=x$. Aleshores $2z=x\left(  z+1\right)  $, i d'aqu\'{\i} surt
$z\left(  2-x\right)  =x$. Com que $x\neq2$ podem dividir els dos costats per
$2-x$ per obtenir $z=x/\left(  2-x\right)  =y$. D'aquesta manera hem demostrat
que per a cada nombre real $x$, si $x\neq2$, hi ha un nombre real \'{u}nic $y$
tal que $2y/\left(  y+1\right)  =x$.

\item Sabem que $\sqrt{2}$ \'{e}s irracional. Aleshores $\left(  \sqrt
{2}\right)  ^{\sqrt{2}}$ \'{e}s racional o b\'{e} irracional. Si $\left(
\sqrt{2}\right)  ^{\sqrt{2}}$ \'{e}s racional, prenem $a=b=\sqrt{2}$ es t\'{e}
que hi ha dos irracionals tal que $a^{b}=\left(  \sqrt{2}\right)  ^{\sqrt{2}}$
\'{e}s racional. Per altra banda, si $\left(  \sqrt{2}\right)  ^{\sqrt{2}}$
\'{e}s irracional, prenem $a=\left(  \sqrt{2}\right)  ^{\sqrt{2}}$ i
$b=\sqrt{2}$ es t\'{e} que hi ha dos irracionals tals que $a^{b}=\left(
\left(  \sqrt{2}\right)  ^{\sqrt{2}}\right)  ^{\sqrt{2}}=\left(  \sqrt
{2}\right)  ^{2}=2$ \'{e}s racional. D'aquesta manera hem demostrat que hi ha
nombres irracionals $a$ i $b$ tals que $a^{b}$ \'{e}s racional.

\item Suposem que $mn$ \'{e}s m\'{u}ltiple de 3 i que $n$ no \'{e}s
m\'{u}ltiple de 3. Llavors, $mn=3k$, $k\in\mathbb{Z}$, i d'aqu\'{\i}, surt que
$m$ necess\`{a}riament \'{e}s m\'{u}ltiple de 3 perqu\`{e} $n$ no ho \'{e}s.
Aix\'{\i} hem provat que $m$ o $n$ \'{e}s m\'{u}ltiple de 3 \'{e}s
condici\'{o} necess\`{a}ria perqu\`{e} $mn$ \'{e}s m\'{u}ltiple de 3. Suposem
ara que $m$ \'{e}s m\'{u}ltiple de 3; es prova an\`{a}logament si $n$ \'{e}s
m\'{u}ltiple de 3. Llavors, $m=3p$, $p\in\mathbb{Z}$, i d'aqu\'{\i} surt que
$mn=3pn$ i, per tant, $mn$ \'{e}s m\'{u}tiple de 3 perqu\`{e} $pn\in
\mathbb{Z}$. Aix\`{o} prova que $m$ o $n$ \'{e}s m\'{u}ltiple de 3 \'{e}s
condici\'{o} suficient. Per tant, $mn$ \'{e}s m\'{u}ltiple de 3 sii $m$ o $n$
\'{e}s m\'{u}ltiple de 3 \'{e}s un teorema.

\item Sigui $n$ el nombre enter m\'{e}s gran. Aleshores, com que $1$ \'{e}s un
nombre enter, \'{e}s clar que $1\leq n$. D'altra banda, com que $n^{2}$
tamb\'{e} \'{e}s un nombre enter tamb\'{e} es compleix $n^{2}\leq n$ i
d'aqu\'{\i} s'obt\'{e} que $n\leq1$. D'aquesta manera, com que es compleix
$n\leq1$ i $n\geq1$, aleshores es t\'{e} $n=1$, i, per tant, $1$ \'{e}s
l'enter m\'{e}s gran. (*)
\end{enumerate}

\item Examina la seg\"{u}ent fal\textperiodcentered l\`{a}cia:

(i) Considerem l'equaci\'{o} $\frac{x+5}{x-7}-5=\frac{4x-40}{13-x}$, suposant
que $x\neq7$ i $x\neq13$.

(ii) Operant en el terme de l'esquerra, es pot comprovar que $\frac{x+5}%
{x-7}-5=\frac{4x-40}{7-x}$.

(iii) De (i) i (ii) es dedueix que $\frac{4x-40}{13-x}=\frac{4x-40}{7-x}$.

(iv) At\`{e}s que els numeradors s\'{o}n iguals, els denominadors tamb\'{e} ho
han de ser. \'{E}s a dir, de (iii) es dedueix que $7-x=13-x$.

(v) De (iv) es dedueix que 7 = 13. Absurd!

En algun dels passos (i)--(v) hi ha d'haver un error. Quin \'{e}s ? Per
qu\`{e} ?

\begin{enumerate}
\item pas (ii) perqu\`{e} es dedueix $\frac{x+5}{x-7}-5=\frac{40-4x}{x-7}.$

\item pas (iv) perqu\`{e} es dedueix $\left(  4x-40\right)  \left(
7-x\right)  =\left(  4x-40\right)  \left(  13-x\right)  $ i, d'aqu\'{\i} no
s'obt\'{e} $7-x=13-x$ llevat que $x\neq10$ (*)

\item pas (iv) perqu\`{e} es dedueix $13-x=x-7$

\item pas (v) perqu\`{e} \ es dedueix $2x=20$ i, per tant, $x=10$.
\end{enumerate}

\item Una fal\textperiodcentered l\`{a}cia: En qualsevol bossa de bales, totes
les bales s\'{o}n del mateix color.

Demostraci\'{o} per inducci\'{o}: Sigui $n$ el nombre de bales de la bossa. Si
$n=1$, \'{e}s evidentment cert. Suposem que \'{e}s cert per a totes les bosses
de $n$ bales, i considerem una bossa de $n+1$ bales. N'apartem una, i
aix\'{\i} tenim una bossa de $n$ bales, que per la hip\`{o}tesi d'inducci\'{o}
seran totes del mateix color. Ens falta provar que la bala apartada tamb\'{e}
\'{e}s del mateix color. L'afegim a la bossa de $n$ bales que hav\'{\i}em
format, i n'apartem una altra. Tornem a tenir una bossa de $n$ bales, que per
hip\`{o}tesi d'inducci\'{o} seran totes del mateix color, i en particular la
darrera bala ser\`{a} del mateix color que les altres en la bossa. Aix\'{\i}
doncs, totes les $n+1$ bales s\'{o}n del mateix color.

\begin{enumerate}
\item L'error \'{e}s que desconeixem el color de les $n$ bales de la bossa
formada i, per tant, no sabem distingir si la bola escollida per segon cop
\'{e}s o no la mateixa que la que hav\'{\i}em apartat.

\item No podem fer una prova per inducci\'{o} perqu\`{e} la propietat no
est\`{a} relacionada amb objectes matem\`{a}tics.

\item No podem aplicar la hip\`{o}tesi d'inducci\'{o} en el segon cas
perqu\`{e} $n$ no \'{e}s qualsevol sin\'{o} el que ten\'{\i}em al principi. (*)

\item Cap de les anteriors respostes es correcte.
\end{enumerate}

\item Per a tots els $n\geq3$, si $n$ punts diferents d'una circumfer\`{e}ncia
estan connectats de manera consecutiva amb rectes, llavors els angles
interiors del pol\'{\i}gon resultant sumen $\left(  n-2\right)  180%
%TCIMACRO{\U{ba}}%
%BeginExpansion
{{}^o}%
%EndExpansion
$. Quin dels passos seg\"{u}ents en la prova per inducci\'{o} hi ha error?

\begin{enumerate}
\item Cas base: Suposem que $n=3$. Aleshores el pol\'{\i}gon \'{e}s un
triangle i es compleix perqu\`{e} se sap que els angles interiors d'un
triangle sumen $180%
%TCIMACRO{\U{ba}}%
%BeginExpansion
{{}^o}%
%EndExpansion
$.

\item Hip\`{o}tesi d'inducci\'{o}: Donats $n$ punts diferents d'una
circumfer\`{e}ncia estan connectats de manera consecutiva amb rectes, llavors
els angles interiors del pol\'{\i}gon resultant sumen $\left(  n-2\right)  180%
%TCIMACRO{\U{ba}}%
%BeginExpansion
{{}^o}%
%EndExpansion
$. (*)

\item Considerem ara el pol\'{\i}gon P format per la connexi\'{o} de $n+1$
punts diferents $A_{1},A_{2},...,A_{n+1}$ en un cercle, com podeu veure a la
figura 1. Si ens saltem l'\'{u}ltim punt $A_{n+1}$, llavors obtenim un
pol\'{\i}gon P amb nom\'{e}s $n$ v\`{e}rtexs, i per la hip\`{o}tesi
d'inducci\'{o}, els angles interiors d'aquest pol\'{\i}gon sumen $\left(
n-2\right)  180%
%TCIMACRO{\U{ba}}%
%BeginExpansion
{{}^o}%
%EndExpansion
$.%
\FRAME{dtbpFU}{7.2664cm}{6.379cm}{0pt}{\Qcb{Figura 1}}{}{logi3.jpg}{\special{ language "Scientific Word";  type "GRAPHIC";
maintain-aspect-ratio TRUE;  display "USEDEF";  valid_file "F";
width 7.2664cm;  height 6.379cm;  depth 0pt;  original-width 7.1939cm;
original-height 6.3087cm;  cropleft "0.009185";  croptop "1";
cropright "1.009185";  cropbottom "0";
filename 'img/logi3.jpg';file-properties "XNPEU";}}


\item Per\`{o} la suma dels angles interiors de P \'{e}s igual a la suma dels
angles interiors de P m\'{e}s la suma dels angles interiors del triangle
$A_{1}A_{n}A_{n+1}$. Com que la suma dels angles interiors del triangle \'{e}s
$180%
%TCIMACRO{\U{ba}}%
%BeginExpansion
{{}^o}%
%EndExpansion
$, podem concloure que la suma dels angles interiors de P \'{e}s $(n-2)180%
%TCIMACRO{\U{ba}}%
%BeginExpansion
{{}^o}%
%EndExpansion
+180%
%TCIMACRO{\U{ba}}%
%BeginExpansion
{{}^o}%
%EndExpansion
=((n+1)-2)180%
%TCIMACRO{\U{ba}}%
%BeginExpansion
{{}^o}%
%EndExpansion
$.
\end{enumerate}

\item Per a qualsevol nombre enter positiu $n$, una graella quadrada de
$2^{n}\times\ 2^{n}$ amb qualsevol quadrat eliminat es pot cobrir amb rajoles
en forma de L, com
\[%
\begin{tabular}
[c]{|l|l}\cline{1-1}
& \\\hline
& \multicolumn{1}{|l|}{}\\\hline
\end{tabular}
\]
La figura 2 mostra un exemple per al cas $n=2$. En aquest cas $2^{n}=4$, i per
tant tenim una graella de $4\times4$ i el quadrat que s'ha eliminat est\`{a}
ombrejat. Les l\'{\i}nies pesades mostren com es poden cobrir els quadrats
restants amb cinc rajoles en forma de L.%
\FRAME{dtbpFU}{12.0287cm}{4.145cm}{0pt}{\Qcb{Figura 2}}{}{logi4.jpg}{\special{ language "Scientific Word";  type "GRAPHIC";maintain-aspect-ratio TRUE;  display "USEDEF";  valid_file "F";
width 12.0287cm;  height 4.145cm;  depth 0pt;  original-width 6.8029cm;
original-height 2.5898cm;  cropleft "0";  croptop "1";  cropright "1";
cropbottom "0";  filename 'img/logi4.jpg';file-properties "XNPEU";}}

Quin dels passos seg\"{u}ents en la prova per inducci\'{o} hi ha error?

\begin{enumerate}
\item Cas base: Suposem que $n=1$. Aleshores la quadr\'{\i}cula \'{e}s una
quadr\'{\i}cula de $2\times2$ amb un quadrat eliminat, que es pot cobrir
clarament amb una rajola en forma de L.

\item Hip\`{o}tesi d'inducci\'{o}: Sigui $n$ un nombre enter positiu
arbitrari, i suposem que la graella $2^{n}\times2^{n}$ amb qualsevol quadrat
eliminat es pot cobrir amb rajoles en forma de L.

\item Considerem ara una graella $2^{n+1}\times2^{n+1}$ amb un quadrat
eliminat, com es veu a la figura 3. Tallem la graella per la meitat tant
verticalment com horitzontalment, dividint-la en quatre quadr\'{\i}cules
$2^{n}\times2^{n}$.
\FRAME{dtbpFU}{5.6277cm}{5.6014cm}{0pt}{\Qcb{Figura 3}}{}{logi5.jpg}{\special{ language "Scientific Word";  type "GRAPHIC";
maintain-aspect-ratio TRUE;  display "USEDEF";  valid_file "F";
width 5.6277cm;  height 5.6014cm;  depth 0pt;  original-width 3.2159cm;
original-height 3.2159cm;  cropleft "0";  croptop "1";  cropright "1";
cropbottom "0";  filename 'img/logi5.jpg';file-properties "XNPEU";}}

El quadrat que s'ha eliminat est\`{a} dins d'una d'aquestes quadr\'{\i}cules
i, per tant, per la hip\`{o}tesi d'inducci\'{o} la resta d'aquesta
quadr\'{\i}cula es pot cobrir amb rajoles en forma de L.

\item Les altres quadr\'{\i}cules tamb\'{e} es poden cobrir amb rajoles en
forma de L perqu\`{e} s\'{o}n del tipus $2^{n}\times2^{n}$ i podem aplicar la
hip\`{o}tesi d'inducci\'{o}. (*)
\end{enumerate}
\end{enumerate}

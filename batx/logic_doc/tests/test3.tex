\section{Posa't a prova amb un test general senzill!}

\begin{enumerate}
\item Considera l'enunciat: "Si un nombre \'{e}s parell ($p$), llavors el seu
quadrat \'{e}s parell ($q$)." Quina de les seg\"{u}ents afirmacions \'{e}s falsa?

\begin{enumerate}
\item La seva contrarec\'{\i}proca \'{e}s: "Si el quadrat d'un nombre no
\'{e}s parell, llavors el nombre no \'{e}s parell."

\item Si $p$ \'{e}s cert, $q$ tamb\'{e} ha de ser cert.

\item Si $q$ \'{e}s fals, $p$ tamb\'{e} ha de ser fals.

\item L'enunciat \'{e}s vertader, per\`{o} la seva rec\'{\i}proca "Si el
quadrat d'un nombre \'{e}s parell, el nombre \'{e}s parell" \'{e}s falsa.
\end{enumerate}

\item La proposici\'{o} $(p\vee q)\rightarrow(\lnot p\wedge r)$ \'{e}s falsa quan:

\begin{enumerate}
\item $p$ \'{e}s vertadera, $q$ \'{e}s falsa, i $r$ \'{e}s falsa.

\item $p$ \'{e}s falsa, $q$ \'{e}s vertadera, i $r$ \'{e}s vertadera.

\item $p$ \'{e}s vertadera, $q$ \'{e}s vertadera, i $r$ \'{e}s falsa.

\item $p$, $q$ i $r$ s\'{o}n vertaderes.
\end{enumerate}

\item Sigui el conjunt de l'univers \'{e}s els nombres enters ($\mathbb{Z}$).
Si $P(x)$ \'{e}s "$x$ \'{e}s un nombre parell" i $Q(x)$ \'{e}s "$x$ \'{e}s un
nombre senar", quina de les seg\"{u}ents afirmacions \'{e}s vertadera?

\begin{enumerate}
\item $\forall x(P(x)\wedge Q(x))$ \'{e}s vertadera.

\item $\exists x(P(x)\wedge Q(x))$ \'{e}s falsa.

\item $\lnot\exists xP(x)$ \'{e}s equivalent a "Cap nombre enter \'{e}s parell."

\item $\forall x(P(x)\vee Q(x))$ \'{e}s falsa.
\end{enumerate}

\item Si l'afirmaci\'{o} "Si plou ($p$), llavors el carrer es mulla ($q$)"
\'{e}s falsa, quina de les seg\"{u}ents conclusions \'{e}s certa?

\begin{enumerate}
\item Est\`{a} plovent i el carrer est\`{a} mullat.

\item No plou i el carrer no est\`{a} mullat.

\item No plou, per\`{o} el carrer est\`{a} mullat.

\item Est\`{a} plovent, per\`{o} el carrer no est\`{a} mullat.
\end{enumerate}

\item Per demostrar que si $n^{2}$ \'{e}s senar, llavors $n$ \'{e}s senar (per
a qualsevol enter $n$), la demostraci\'{o} per contrarec\'{\i}proc ha de
provar que:

\begin{enumerate}
\item Si $n$ \'{e}s parell, llavors $n^{2}$ \'{e}s parell.

\item Si $n$ \'{e}s senar, llavors $n^{2}$ \'{e}s senar.

\item Si $n^{2}$ \'{e}s parell, llavors $n$ \'{e}s parell.

\item Si $n^{2}$ \'{e}s senar, llavors $n$ \'{e}s senar.
\end{enumerate}

\item Per provar que la suma de dos nombres senars consecutius \'{e}s sempre
un m\'{u}ltiple de 4, el primer pas de la demostraci\'{o} directa seria
expressar els dos nombres com:

\begin{enumerate}
\item $2k+1$ i $2k+3$ per a algun enter $k$.

\item $2k$ i $2k+2$ per a algun enter $k$.

\item $n$ i $n+1$ per a algun enter $n$.

\item $k$ i $k+2$ per a algun enter $k$.
\end{enumerate}

\item La demostraci\'{o} de qu\`{e} $\sqrt{3}$ \'{e}s irracional utilitzant la
reducci\'{o} a l'absurd comen\c{c}a suposant que:

\begin{enumerate}
\item $\sqrt{3}$ \'{e}s un nombre enter.

\item $\sqrt{3{}}$ \'{e}s un nombre racional.

\item $\sqrt{3{}}=p/q$, on $p$ i $q$ s\'{o}n parells.

\item $\sqrt{3{}}=p/q$, on $p$ i $q$ no tenen factors comuns.
\end{enumerate}

\item Quina de les seg\"{u}ents parelles de proposicions s\'{o}n equivalents?

\begin{enumerate}
\item $p\vee q$ i $\lnot(\lnot p\vee\lnot q)$.

\item $p\rightarrow q$ i $\lnot p\vee q$.

\item $p\wedge q$ i $\lnot(\lnot p\wedge\lnot q)$.

\item $p\leftrightarrow q$ i $(p\rightarrow q)\wedge(q\rightarrow p)$.
\end{enumerate}

\item Identifica la fal\textperiodcentered l\`{a}cia en el seg\"{u}ent
raonament: "Si un estudiant estudia ($p$), aprova l'examen ($q$). L'estudiant
ha aprovat l'examen ($q$). Per tant, l'estudiant ha estudiat ($p$)."

\begin{enumerate}
\item Fal\textperiodcentered l\`{a}cia de negaci\'{o} de l'antecedent.

\item Fal\textperiodcentered l\`{a}cia d'afirmaci\'{o} del conseq\"{u}ent.

\item Reducci\'{o} a l'absurd.

\item Sil\textperiodcentered logisme disjuntiu.
\end{enumerate}

\item L'enunciat "No hi ha cap estudiant que sigui bo en matem\`{a}tiques i en
f\'{\i}sica" es pot expressar formalment com:

\begin{enumerate}
\item $\exists x(M(x)\wedge F(x))$

\item $\forall x(\lnot M(x)\vee\lnot F(x))$

\item $\lnot\exists x(M(x)\wedge F(x))$

\item $\forall x(\lnot M(x)\wedge\lnot F(x))$
\end{enumerate}
\end{enumerate}

\bigskip

Les solucions s\'{o}n: 1(d), 2(a), 3(b), 4(d), 5(a), 6(a), 7(b), 8(b), 9(b), 10(c).
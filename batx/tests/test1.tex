

\section{Lògica i raonament}

\begin{enumerate}
\item Considerem l'enunciat seg\"{u}ent sobre una festa: "Si \'{e}s el teu
aniversari o hi haur\`{a} past\'{\i}s, hi haur\`{a} past\'{\i}s". Quina de les
seg\"{u}ents afirmacions \'{e}s falsa?

\begin{enumerate}
\item Si $p=$\textquotedblleft\'{e}s el teu aniversari\textquotedblright\ i
$q=$\textquotedblleft hi haur\`{a} past\'{\i}s\textquotedblright, llavors
l'enunciat s'expressa formalment aix\'{\i}: $\left(  p\vee q\right)
\longrightarrow q$.

\item Suposant que l'enunciat \'{e}s cert, podem concloure (si \'{e}s que hi
ha alguna cosa) afirmant nom\'{e}s que hi haur\`{a} past\'{\i}s.

\item Suposant que l'enunciat \'{e}s cert, si no hi haur\`{a} past\'{\i}s
podem concloure que \'{e}s el teu aniversari. (*)

\item Suposem que s'ha descobert que l'afirmaci\'{o} \'{e}s mentida, aleshores
podem concloure que \'{e}s el teu aniversari, per\`{o} el past\'{\i}s \'{e}s mentida.
\end{enumerate}

\item Si $p=$\textquotedblleft Joan sempre diu la veritat\textquotedblright\ i
$q=$\textquotedblleft Jaume sempre menteix\textquotedblright, llavors quina de
les seg\"{u}ents afirmacions \'{e}s falsa?

\begin{enumerate}
\item L'enunciat \textquotedblleft No \'{e}s cert que en Joan o Jaume diguin
sempre mentides\textquotedblright\ s'expressa formalment d'aquesta manera:
$\lnot\left(  \lnot p\vee q\right)  $.

\item $\lnot p\longrightarrow\lnot q$ \'{e}s l'enunciat \textquotedblleft Si
en Joan menteix, llavors en Jaume diu la veritat\textquotedblright.

\item $p\wedge\lnot q$ \'{e}s l'enunciat \textquotedblleft Joan i Jaume no
menteixen\textquotedblright.

\item L'enunciat \textquotedblleft No \'{e}s el cas que si en Joan menteix,
aleshores en Jaume digui sempre la veritat\textquotedblright\ s'expressa
formalment aix\'{\i} $\lnot\left(  \lnot p\longrightarrow q\right)  $. (*)
\end{enumerate}

\item Prenent com univers el conjunt del nombres enters $\mathbb{Z}$. Si
$p=$\textquotedblleft$x$ \'{e}s m\'{u}ltiple de $3$\textquotedblright\ i
$q=$\textquotedblleft$x$ \'{e}s enter positiu m\'{e}s petit que $10$%
\textquotedblright. Quina de les seg\"{u}ents afirmacions \'{e}s vertadera?

\begin{enumerate}
\item $\lnot\left(  \lnot p\vee\lnot q\right)  =$\textquotedblleft$x$ \'{e}s
m\'{u}tiple de $3$ m\'{e}s petit que $10$\textquotedblright.

\item $p\longrightarrow\lnot q=$\textquotedblleft si $x$ es m\'{u}ltiple de
$3$, aleshores $x$ \'{e}s un enter m\'{e}s petit o igual que $0$ o \'{e}s
m\'{e}s gran o igual que $10$\textquotedblright. (*)

\item $\left(  p\wedge\lnot q\right)  \vee\left(  \lnot p\wedge q\right)
=$\textquotedblleft$x$ \'{e}s m\'{u}ltiple de $3$ o $x$ \'{e}s enter positiu
m\'{e}s petit que $10$\textquotedblright.

\item $\left(  p\longrightarrow q\right)  \vee\lnot p$ \'{e}s una
proposici\'{o} que nom\'{e}s \'{e}s falsa quan assignem a la variable $x$
valors enters m\'{e}s grans o iguals que $10$\textquotedblright.
\end{enumerate}

\item La proposici\'{o} $\left(  \lnot p\longleftrightarrow\lnot q\right)
\vee\left(  r\longrightarrow q\right)  $ \'{e}s falsa quan:

\begin{enumerate}
\item $p$ i $q$ s\'{o}n falses, i $r$ \'{e}s vertadera.

\item $p$ \'{e}s vertadera, i $q$ i~$r$ s\'{o}n vertaderes.

\item $p$ i $r$ s\'{o}n vertaderes, i $q$ \'{e}s falsa. (*)

\item $p,q$ i $r$ s\'{o}n falses.
\end{enumerate}

\item Quin parell de formes proposicionals s\'{o}n l\`{o}gicament equivalents?

\begin{enumerate}
\item $A\wedge B$ i $\lnot\left(  \lnot A\longrightarrow B\right)  $

\item $\left(  A\wedge B\right)  \vee\left(  B\vee\lnot C\right)  $ i $\left(
\lnot A\vee\lnot B\right)  \longrightarrow\lnot\left(  \lnot B\wedge C\right)
$ (*)

\item $\lnot\left(  (A\rightarrow\lnot B)\vee\lnot(C\wedge\lnot C)\right)  $ i
$C\vee\lnot C$

\item $\left(  A\longrightarrow B\right)  \longleftrightarrow B$ i $A$
\end{enumerate}

\item En Pere t'estava explicant el que va menjar ahir a la tarda. Et diu:
\textquotedblleft Vaig prendre crispetes o panses. A m\'{e}s, si tenia
entrepans de cogombre, llavors tenia refresc. Per\`{o} no vaig beure refresc
ni te". Per descomptat, sabeu que en Pere \'{e}s el pitjor mentider del
m\'{o}n, i tot el que diu \'{e}s fals. Qu\`{e} va menjar en Pere?

\begin{enumerate}
\item entrepans de cogombre\thinspace\ i te (*)

\item crispetes o panses i refresc

\item nom\'{e} te

\item entrepans de cogombre\thinspace\ i refresc
\end{enumerate}

\item De les premisses $P_{1}=\lnot A\longrightarrow\left(  B\longrightarrow
\lnot C\right)  ,P_{2}=C\longrightarrow\lnot A,P_{3}=\left(  \lnot D\vee
A\right)  \longrightarrow\lnot\lnot C$ i $P_{4}=\lnot D$, qu\`{e} es pot
dedu\"{\i}r si \'{e}s v\`{a}lid?

\begin{enumerate}
\item Podem deduir qualsevol cosa perqu\`{e} les premisses s\'{o}n inconsistents.

\item $B\wedge\lnot\left(  A\vee D\right)  $

\item $\lnot B\wedge\left(  A\vee D\right)  $

\item $\lnot\left(  A\vee B\vee D\right)  $ (*)
\end{enumerate}

\item Volem simplificar els enunciats seg\"{u}ents de manera que la
negaci\'{o} nom\'{e}s apareix directament al costat dels predicats, quina les
seg\"{u}ents equival\`{e}ncies no \'{e}s correcte?

\begin{enumerate}
\item $\lnot\exists x\forall y(\lnot O(x)\vee E(y))\Longleftrightarrow\forall
x\exists y(O(x)\wedge\lnot E(y))$

\item $\lnot\forall x\lnot\forall y\lnot(x<y\wedge\exists z(x<z\vee
y<z))\Longleftrightarrow\exists x\forall y(x<y\wedge\exists z(x<z\vee y<z)) $ (*)

\item \textquotedblleft Hi ha un nombre $n$ per al qual cap altre nombre
\'{e}s menor o igual a $n$\textquotedblright\ \'{e}s equivalent a
\textquotedblleft Hi ha un nombre $n$ per al qual tots els altres nombres
s\'{o}n estrictament majors que $n$\textquotedblright.

\item \textquotedblleft\'{E}s fals que per a cada nombre $n$ hi hagi dos
nombres m\'{e}s entre els quals es troba $n$\textquotedblright\ \'{e}s
equivalent a \textquotedblleft Hi ha un nombre $n$ que no est\`{a} entre
altres dos nombres\textquotedblright.
\end{enumerate}

\item Es diu que el raonament seg\"{u}ent \'{e}s v\`{a}lid%
\[%
\begin{tabular}
[c]{ll}%
$P_{1}:$ & $\exists x~\left(  P(x)\wedge Q(x)\right)  $\\
$P_{2}:$ & $\exists x$ $\left(  M(x)\wedge\lnot P(x)\right)  $\\\hline
& $\exists x~\left(  M(x)\wedge Q(x)\right)  $%
\end{tabular}
\]
perqu\`{e} es considera com a prova la deducci\'{o} seg\"{u}ent:%
\[%
\begin{tabular}
[c]{lll}%
1. & $\exists x~\left(  P(x)\wedge Q(x)\right)  $ & P 1\\
2. & $\exists x$ $M(x)$ & P 2\\
3. & $P(a)\wedge Q(a)$ & EQE 1\\
4. & $Q(a)$ & EC 3\\
5. & $M(a)\wedge\lnot P(a)$ & EQE 2\\
6. & $M(a)$ & EC 5\\
7. & $M(a)\wedge Q(a)$ & IC (4,5)\\\hline
8. & $\exists x~\left(  M(x)\wedge Q(x)\right)  $ & IQE 6
\end{tabular}
\]


\begin{enumerate}
\item \`{E}s correcte.

\item No \'{e}s v\`{a}lid perqu\`{e} EQE 2 no permet escriure $M(a)\wedge\lnot
P(a)$ sin\'{o}, per exemple, $M(b)\wedge\lnot P(b)$ sent $b$ una inst\`{a}ncia
de $x$ que no apareix abans. (*)

\item El raonament no \'{e}s v\`{a}lid perqu\`{e} \'{e}s inconsistent; de les
premisses es dedueix $P(a)\wedge\lnot P(a)$.

\item Cap de les respostes anteriors \'{e}s certa.
\end{enumerate}

\item Dels seg\"{u}ents raonaments: (i) Qui menysprea a tots els fan\`{a}tics
menysprea tamb\'{e} a tots els pol\'{\i}tics. Alg\'{u} no menysprea a un
determinat pol\'{\i}tic. Per conseg\"{u}ent, hi ha un fan\`{a}tic al qual no
tothom menysprea; i (2) A tots els gats simp\`{a}tics i
intel\textperiodcentered ligents els agrada el fetge picat. Cada siam\`{e}s
\'{e}s simp\`{a}tic. Hi ha un siam\`{e}s que no li agrada el fetge picat. Per
conseg\"{u}ent, hi ha un gat est\'{u}pid.

\begin{enumerate}
\item (i) i (ii) s\'{o}n v\`{a}lids. (*)

\item Cap dels dos \'{e}s v\`{a}lid.

\item (i) \'{e}s v\`{a}lid, i (ii) no.

\item (i) no \'{e}s v\`{a}lid, i (ii) s\'{\i}.
\end{enumerate}
\end{enumerate}

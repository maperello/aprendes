

\begin{enumerate}
\item Donats els conjunts seg\"{u}ents%
\begin{equation*}
\begin{array}{l}
A=\left\{ x:x=2n\text{ \ i \ }n\in \mathbb{N}\right\} \\
B=\left\{ x:x=3n\text{ \ i \ }n\in \mathbb{N}\right\} \\
C=\left\{ x:x=4n-1\text{ \ i \ }n\in \mathbb{N}\right\} \\
D=\left\{ x:x=4n+1\text{ \ i \ }n\in \mathbb{N}\right\}%
\end{array}%
\end{equation*}%
Troba $A\cap B$, $C\cup D$, $C\cap D$, $\complement A$ i $A\cap (C\cup D)$.

Soluci\'{o}: $A\cap B$ \'{e}s el conjunt de m\'{u}ltiples de 6, $C\cup
D=\left\{ x:x=2n+1\text{ \ i \ }n\in \mathbb{N}\right\} $, $C\cap
D=\emptyset $, $\complement A$ \'{e}s el conjunt dels nombres imparells, i $%
A\cap (C\cup D)=\emptyset $.

\item Sigui $U$ l'univers dels conjunts $A,B$ i $C$. Simplifica les seg\"{u}%
ents expressions:

\begin{enumerate}
\item[a)] $(A\cap B)\cup (A\cap \complement B)\cup (\complement A\cap B)\cup
(\complement A\cap \complement B)$

\item[b)] $\left[ (A\cup B)\cap \complement \left( B\cap \complement
A\right) \right] \cup \left[ (A\cap B)\cup \complement \left( B\cup
\complement A\right) \right] $

\item[c)] $(A\cap B)\cup (A\cap C)\cup \complement \left( \complement A\cup
\complement B\right) $

\item[d)] $(A\cap B)\cup (A\cap \complement B)\cup (\complement A\cap C)\cup
(A\cap C)$
\end{enumerate}

Soluci\'{o}: a) $U$, b) $A$, c) $A\cap (B\cup C)$ i d) $A\cup C$

\item Si $\#P\left( A\cup \mathcal{P}(A)\right) =8$, calcula $\#A$.

Soluci\'{o}: $\#A=1$

\item Una empresa ofereix places d'electricista, de mec\`{a}nic i de fuster.
Sabem que 12 persones sol\textperiodcentered liciten pla\c{c}a
d'electricista, 12 de mec\`{a}nic, 15 de fuster, 3 d'electricista i mec\`{a}%
nic, 4 de mec\`{a}nic i fuster, 5 d'electricista i fuster i, finalment, 1 sol%
\^{A}\textperiodcentered licita pla\c{c}a de les tres coses. Calcula quanta
gent ha fet alguna sol\textperiodcentered licitud.

Soluci\'{o}: 28 persones

\item En una reuni\'{o} hi ha 25 persones que s\'{o}n m\`{e}dics, m\'{u}sics
o pol\'{\i}tics. Hi ha 20 metges, 12 m\'{u}sics i 17 pol\'{\i}tics. Hi ha 8
que s\'{o}n m\`{e}dics i m\'{u}sics, 12 que s\'{o}n m\`{e}dics i pol\'{\i}%
tics i 11 que s\'{o}n m\'{u}sics i pol\'{\i}tics. (a) Quants pol\'{\i}tics s%
\'{o}n m\'{u}sics i metges alhora? (b) Quantes persones hi ha amb una sola
professi\'{o}?

Soluci\'{o}: (a) 7 i (b) 8

\item D'un grup de 1000 persones, 950 porten rellotge, 750 porten paraigua,
800 porten corbata i 850 porten barret. Troba el nombre m\'{\i}nim de
persones que porten les quatre coses.

Soluci\'{o}: 350

\item Comprova si les seg\"{u}ents col\textperiodcentered leccions de
conjunts formen una partici\'{o} de $\mathbb{N}$

\begin{enumerate}
\item[a)] $A=\left\{ 2n:n\in \mathbb{N}\right\} $ i $B=\left\{ 2n-1:n\in
\mathbb{N}\right\} $

\item[b)] $A$ \'{e}s el conjunt dels nombres primers entre si, $B=\left\{
2,4,6,8,9\right\} $ i $C=\left\{ x\in \mathbb{N}:x\geq 10\right\} $

\item[c)] $A=\left\{ 2n:n\in \mathbb{N}\right\} $, $B=\left\{ 2n-1:n\in
\mathbb{N}\right\} $ i $C=B=\left\{ 5n:n\in \mathbb{N}\right\} $
\end{enumerate}

Soluci\'{o}: a) S\'{\i}, b) No, c) No

\item En el conjunt $A=\{1,2,3,4,5,6\}$ es defineix la relaci\'{o}%
\begin{equation*}
\begin{array}{ccc}
x~R~y & \Longleftrightarrow & y\text{ \'{e}s m\'{u}ltiple de }x%
\end{array}%
\end{equation*}%
(a) Escriu el graf de la relaci\'{o} i (b) estudia les seves propietats.

Soluci\'{o}: (a) $R=\left\{
\begin{array}{c}
(1,1),(1,2),(1,3),(1,4),(1,5),(1,6),(2,2), \\
(2,4),(2,6),(3,3),(3,6),(4,4),(5,5),(6,6)%
\end{array}%
\right\} $ (b) $R$ \'{e}s reflexiva, antisim\`{e}trica i transitiva.

\item Quina relaci\'{o} bin\`{a}ria sobre un conjunt \'{e}s sim\`{e}trica i
antisim\`{e}trica?

Soluci\'{o}: La relaci\'{o} d'igualtat.

\item De les seg\"{u}ents relacions, quines s\'{o}n d'equival\`{e}ncia? I en
cas de ser-ho, quines s\'{o}n les seves classes?

\begin{enumerate}
\item[a)] \textquotedblleft Tenir la mateixa altura\textquotedblright\ en el
conjunt dels alumnes d'una classe

\item[b)] \textquotedblleft Ser equipol\textperiodcentered
lent\textquotedblright\ en el conjunt dels vectors fixos del pla

\item[c)] \textquotedblleft Equidistar d'un punt fix
donat\textquotedblright\ en el conjunt dels punts del pla

\item[d)] \textquotedblleft Estar alineats amb un punt fix
donat\textquotedblright\ en el conjunt de parells de punts del pla sense el
punt fix donat

Soluci\'{o}: (a) \'{E}s d'equival\`{e}ncia i els alumnes queden classificats
segons les seves altures (b) \'{E}s d'equival\`{e}ncia i les classes s\'{o}n
els vectors lliures del pla (c) \'{E}s d'equival\`{e}ncia i les classes s%
\'{o}n les circumfer\`{e}ncies de centre el punt fix donat (d) \'{E}s
d'equival\`{e}ncia i les classes s\'{o}n les rectes que passen pel punt fix
donat sense contenir aquest punt.
\end{enumerate}

\item En el conjunt dels nombres reals es defineix la relaci\'{o}%
\begin{equation*}
\begin{array}{ccc}
x\sim y & \Longleftrightarrow & \left\lfloor x\right\rfloor =\left\lfloor
y\right\rfloor%
\end{array}%
\end{equation*}%
on $\left\lfloor y\right\rfloor $ significa la part sencera del nombre real $%
x$. \'{E}s una relaci\'{o} d'equival\`{e}ncia? Si ho \'{e}s, quines s\'{o}n
les seves classes?

Soluci\'{o}: Observa que $x\sim y$ si i nom\'{e}s si existeix un nombre
enter $n$ tal que $x,y\in \lbrack n,n+1)$. \'{E}s una relaci\'{o} d'equival%
\`{e}ncia i les classes s\'{o}n els intervals de la forma $[n,n+1)$ amb $%
n\in \mathbb{Z}$.

\item Esbrina si la relaci\'{o} \textquotedblleft $x$ divideix $y$%
\textquotedblright\ \'{e}s d'ordre en cadascun dels conjunts que s'indiquen
a continuaci\'{o}, i, en el cas que ho sigui, \'{e}s parcial o total? Troba
els seus elements maximals i minimals.

\begin{enumerate}
\item[a)] En el conjunt dels nombres naturals $\mathbb{N}$

\item[b)] En el conjunt dels nombres enters $\mathbb{Z}$
\end{enumerate}

Soluci\'{o}: (a) \'{E}s d'ordre parcial, $1$ \'{e}s minimal i no hi ha
elements maximals. (b) No \'{e}s d'ordre

\item En el conjunt dels nombres reals ordenat segons la relaci\'{o} d'ordre
usual $\leq $ es consideren els seg\"{u}ents subconjunts (a) $A=[1,5]$, (b) $%
B=(-2,-1]$, (c) $C=(\pi ,2\pi )$, (d) $D=[2,+\infty )$, (e) $E=(-5,+\infty )$
i (f) $F=(-\infty ,0)$. Calcula, si existeixen, m\'{\i}nim, m\`{a}xim, \'{\i}%
nfim i suprem de cadascun d'ells.

Soluci\'{o}: (a) $\min A=\inf A=1$ i $\max A=\sup A=5$, (b) $\inf B=-2$ i $%
\max B=\sup B=-1$, (c) $\inf C=\pi $ i $\sup C=2\pi $, (d) $\min D=\inf D=2$%
, (e) $\inf E=-5$, (f) $\sup F=0$

\item En el conjunt $A=\{2,3,5,6,15\}$ es defineix la relaci\'{o}%
\begin{equation*}
\begin{array}{ccc}
x~R~y & \Longleftrightarrow & y\text{ \'{e}s m\'{u}ltiple de }x%
\end{array}%
\end{equation*}%
(a) \'{E}s una relaci\'{o} d'ordre? \'{E}s parcial o total? (b) Troba m\`{a}%
xim, m\'{\i}nim, suprem i \'{\i}nfim de $B=\{2,3,6,15\}$. (c) Troba m\`{a}%
xim, m\'{\i}nim, suprem i \'{\i}nfim de $A$. (d) Hi ha elements maximals i
minimals d'$A$?

Soluci\'{o}: (a) \'{E}s d'ordre parcial (b) No existeixen $\max B$, $\min B$%
, $\sup B$, $\inf B$. (c) No existeixen $\max A$, $\min A$, $\sup A$, $\inf
A $. (d) $2,3,5$ s\'{o}n minimals i $6,15$ s\'{o}n maximals.

\item Es donen els conjunts $A=\{2,3,4,7,8\}$ i $B=\{1,2,3,4,5,7,9\}$. Sigui
$f:A~\longrightarrow ~B$ l'aplicaci\'{o} definida per
\begin{equation*}
f(x)=\left\{
\begin{array}{ll}
\frac{x}{2} & \text{si \ }x\text{ \ \'{e}s parell} \\
x & \text{si \ }x\text{ \ \'{e}s senar}%
\end{array}%
\right.
\end{equation*}%
Estudia quina classe d'aplicaci\'{o} s'obt\'{e}.

Soluci\'{o}: \'{E}s una aplicaci\'{o} injectiva

\item Donades les aplicacions $f:\mathbb{R}~\longrightarrow ~\mathbb{R}$
definida per $f(x)=e^{x}$ i $g:[0,+\infty )~\longrightarrow ~\mathbb{R}$
definida per $g(x)=\sqrt{x}$. (a) De quina classe d'aplicacions s\'{o}n $f$
i $g$? (b) Canvia els conjunts de sortida i d'arribada de l'aplicaci\'{o} $f$
perqu\`{e} sigui bijectiva. (c) Canvia els conjunts de sortida i d'arribada
de l'aplicaci\'{o} $g$ perqu\`{e} sigui bijectiva. (d) Calcula les
aplicacions inverses de les aplicacions de l'apartat (c). (e) Calcula si
\'{e}s possible $f\circ g$ i $g\circ f$.

Soluci\'{o}: (a) $f$ \'{e}s injectiva per\`{o} no exhaustiva i $g$ \'{e}s
injectiva per\`{o} no exhaustiva. (b) L'aplicaci\'{o} $f:\mathbb{R}%
~\longrightarrow ~(0,+\infty )$ definida per $f(x)=e^{x}$ \'{e}s bijectiva.
(c) L'aplicaci\'{o} $g:[0,+\infty )~\longrightarrow ~[0,+\infty )$ definida
per $g(x)=\sqrt{x}$ \'{e}s bijectiva. (d) $f^{-1}:(0,+\infty
)~\longrightarrow ~\mathbb{R}$ amb $f^{-1}(x)=\ln x$ i $g^{-1}:[0,+\infty
)~\longrightarrow ~[0,+\infty )$ amb $g^{-1}(x)=x^{2}$. (e) $f\circ
g:[0,+\infty )~\longrightarrow ~\mathbb{R}$ amb $(f\circ g)(x)=e^{\sqrt{x}}$
i $g\circ f:\mathbb{R}~\longrightarrow ~\mathbb{R}$ amb $(g\circ f)(x)=\sqrt{%
e^{x}}$ ja que $\func{Im}f=(0,+\infty )\subset \lbrack 0,+\infty )$.

\item Donada l'aplicaci\'{o} $f:\mathbb{R}~\longrightarrow ~\mathbb{R}$
definida per $f(x)=2x+1$, calcula $f(A)$ i $f^{-1}(A)$ en els casos seg\"{u}%
ents: (a) $A=[1,3]$, (b) $A=[-2,-1)$, (c) $A=[1,+\infty )$ i (d) $A=(-\infty
,-2)$.

Soluci\'{o}: (a) $f(A)=[3,7]$ i $f^{-1}(A)=[0,1]$, (b) $f(A)=[-3-1)$ i $%
f^{-1}(A)=[-3/2,-1)$, (c) $f(A)=[3,+\infty )$ i $f^{-1}(A)=[0,+\infty )$ i
(d) $f(A)=(-\infty ,-3)$ i $f^{-1}(A)=(-\infty ,-3/2)$.

\item Sigui $f:\mathbb{R}-\{1\}~\longrightarrow ~\mathbb{R}-\{3\}$ definida
per%
\begin{equation*}
f(x)=\frac{3x-1}{x-1}
\end{equation*}%
Demostra que \'{e}s bijectiva i calcula l'aplicaci\'{o} inversa.

Soluci\'{o}: Cal demostrar que \'{e}s injectiva i exhaustiva. L'aplicaci\'{o}
inversa \'{e}s%
\begin{equation*}
f^{-1}(x)=\frac{x-1}{x-3}
\end{equation*}
\end{enumerate}

\end{document}

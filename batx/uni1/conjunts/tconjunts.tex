\documentclass[a4paper,12pt]{book}
%%%%%%%%%%%%%%%%%%%%%%%%%%%%%%%%%%%%%%%%%%%%%%%%%%%%%%%%%%%%%%%%%%%%%%%%%%%%%%%%%%%%%%%%%%%%%%%%%%%%%%%%%%%%%%%%%%%%%%%%%%%%%%%%%%%%%%%%%%%%%%%%%%%%%%%%%%%%%%%%%%%%%%%%%%%%%%%%%%%%%%%%%%%%%%%%%%%%%%%%%%%%%%%%%%%%%%%%%%%%%%%%%%%%%%%%%%%%%%%%%%%%%%%%%%%%
%TCIDATA{OutputFilter=LATEX.DLL}
%TCIDATA{Version=5.50.0.2960}
%TCIDATA{<META NAME="SaveForMode" CONTENT="1">}
%TCIDATA{BibliographyScheme=Manual}
%TCIDATA{LastRevised=Sunday, June 30, 2024 17:48:02}
%TCIDATA{<META NAME="GraphicsSave" CONTENT="32">}



%%%%%%%%%%%%%%%%%%%%%%%%%%%%%%%%%%%%%%%%%%%%%%%%%%%%%%%%%%%%%%%%%%%%%%%%%%%%%%%%%%%%%%%%%%%%%%%%%%%%%%%%%%%%%%%%%%%%%%%%%%%%%%%%%%%%%%%%%%%%%%%%%%%%%%%%%%%%%%%%%%%%%%%%%%%%%%%%%%%%%%%%%%%%%%%%%%%%%%%%%%%%%%%%%%%%%%%%%%%%%%%%%%%%%%%%%%%%%%%%%%%%%%%%%%%%

\usepackage[catalan]{babel}
\usepackage[utf8]{inputenc}
\usepackage[T1]{fontenc}

\usepackage[hmargin={3cm,3cm},vmargin={2.5cm,2.5cm},offset=0pt,footskip=1cm,headsep=1cm]{geometry}

\usepackage{amsfonts}
\usepackage{amssymb}
\usepackage{amsmath}
\usepackage{amsthm}
\usepackage{mathtools}
\usepackage[most]{tcolorbox}

\usepackage{xcolor}
\usepackage{sectsty}



\setlength{\parindent}{0pt}
\numberwithin{equation}{section}
\setcounter{MaxMatrixCols}{10}
\graphicspath{./img}

\definecolor{fons}{RGB}{243,231,220}
\definecolor{cdemo}{RGB}{255,115,100}
\definecolor{csolu}{RGB}{110,132,169}
\definecolor{apren}{RGB}{255,64,87}
\definecolor{cobs}{RGB}{250,163,92}
\definecolor{cdefn}{RGB}{101,255,205}
\definecolor{cexer}{RGB}{137,255,133}
\definecolor{cexem}{RGB}{23,205,255}

\sectionfont{\color{apren}}
\subsectionfont{\color{apren}}
\subsubsectionfont{\color{apren}}

\theoremstyle{plain}
\newtheorem{thm}{\color{apren}\textbf{Teorema}}
\newtheorem{cor}[thm]{Coro\l.lari}
\newtheorem{lem}{Lema}
\newtheorem{prop}{Proposici\'{o}}
\theoremstyle{definition}
\newtheorem{defn}{Definici\'{o}}
\newtheorem{exem}{Exemple}

\newcounter{exercici}
\setcounter{exercici}{-1}
\newtheorem{exer}[exercici]{Exercici}

\numberwithin{equation}{section}
\newenvironment{obs}{{\color{apren}\textbf{Observaci\'{o}:}}}{}
\newenvironment{prova}{{\color{apren}\textbf{Demostraci\'{o}:}}}{\quad\hfill $\blacksquare$}
\newenvironment{solucio}{{\color{apren}\textbf{Soluci\'{o}:}}}{\quad\hfill $\square$}


\tcolorboxenvironment{defn}{colback=cdefn!30,colframe=white,breakable,arc=0mm}
\tcolorboxenvironment{thm}{colback=apren!5,colframe=white,breakable,arc=0mm}
\tcolorboxenvironment{lem}{colback=apren!30,colframe=white,breakable,arc=0mm}
\tcolorboxenvironment{cor}{colback=apren!30,colframe=white,breakable,arc=0mm}
\tcolorboxenvironment{exem}{colback=cexem!10,colframe=white,breakable,arc=0mm}
\tcolorboxenvironment{exer}{colback=cexer!10,colframe=white,breakable,arc=0mm}
\tcolorboxenvironment{obs}{blanker,breakable,left=5mm,before skip=10pt,after skip=10pt,borderline west={1mm}{0pt}{cobs!30}}
\tcolorboxenvironment{prova}{blanker,breakable,left=5mm,before skip=10pt,after skip=10pt,borderline west={1mm}{0pt}{cdemo!30}}
\tcolorboxenvironment{solucio}{blanker,breakable,left=5mm,before skip=10pt,after skip=10pt,borderline west={1mm}{0pt}{csolu!30}}



\input{tcilatex}
\renewcommand{\FRAME}[8]{
\begin{center}
\includegraphics[width=#2,height=#3]{img/#7}
\end{center}}

\begin{document}
\pagecolor{fons!20}
\title{{\color{apren}\textbf{\huge{Conjunts, Relacions i Aplicacions}}}}
\author{Miquel Àngel Perelló}
\date{\today}

\let\cleardoublepage\clearpage
\maketitle
\tableofcontents

\frontmatter

\let\cleardoublepage\clearpage

\chapter*{Prefaci}

\markboth{PREFACIO}{PREFACIO}Aquest document \'{e}s la segona part del
llibre que t\'{e} com a objectiu donar les bases per una comprensi\'{o} m%
\'{e}s rigorosa del desenvolupament de les Matem\`{a}tiques des d'un punt de
vista molt elemental. T\'{e} quatre cap\'{\i}tols: El primer s'anomena
"Teoria" i com hom pot imaginar s'hi exposen els conceptes i procediments
que despr\'{e}s s'apliquen al segon cap\'{\i}tol anomenat "Pr\`{a}ctica". En
aquest cap\'{\i}tol es mostren les resolucions d'una certa varietat
d'exercicis representatius per la comprensi\'{o} del contingut tractat al
primer. El tercer cap\'{\i}tol anomenat "Exercicis proposats" \'{e}s una
llista d'exercicis per a completar l'aprenentatge i, finalment, el quart cap%
\'{\i}tol, anomenat "Tests" s\'{o}n ex\`{a}mens per comprovar que els
objectius s'han assolit amb \`{e}xit.

\let\cleardoublepage\clearpage

\chapter{Introducci\'{o}}

En aquest document fem una breu introducci\'{o} a la teoria de conjunts.

\let\cleardoublepage\clearpage

\mainmatter

\chapter{Teoria}

\import{/../../}{conjunts_teo.tex}

\chapter{Pràctica}

\import{/../../}{conjunts_pra.tex}

\chapter{Exercicis proposats}

\import{/../../}{conjunts_pro.tex}

\chapter{Tests}

\section{Conjunts}

\import{/../../}{conjunts_exa1.tex}

\section{Relacions i aplicacions}

\import{/../../}{conjunts_exa2.tex}

\backmatter

\end{document}
